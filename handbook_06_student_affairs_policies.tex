
\section{STUDENT AFFAIRS POLICIES OF INTEREST TO THE FACULTY}
	\subsection{Student Handbook (appended)}
		\change{}{23}{There is no content here in the 9/2020 version }
	\subsection{Student Conduct Code}
		\change{}{23}{There is no content here in the 9/2020 version }
	\subsection{Family Educational Rights and Responsibilities}
		\begin{enumerate}[label=\alph*)]
			\item{Under the ``Family Rights and Privacy Act of 1974'' the faculty are responsible to maintain privacy of academic information for each student.  In agreement with the provisions of this law the College has established the guidelines regarding access to student information.  In particular, the College has decided that only the academic advisor of a given student may see the student's academic record.  Parents have access to that record only if the son or daughter is financially dependent upon them.  Questions regarding the implementation of the law at Westmont College should be directed to the Registrar; the Director of Financial Aid is responsible for assessing the financial dependence of students upon parents.
			}

			\item{Of particular importance to faculty is the proscription under FRPA of the posting of student grades where the students are identified by name, social security number or student identification number, or by any other means which can lead with modest effort to the identification of student with grade.  Students may be asked to provide a coded identification at the beginning of the course which may then be used legally for the posting of grades.
			}
		\end{enumerate}
	\subsection{Student Discipline Code}
		The faculty are responsible for setting and maintaining standards for student conduct in academic work.  All other aspects of student behavior are under the aegis of the Vice President for Student Life and Dean of Students and the Student Life Staff.  Often, disciplinary actions by the student life office has academic consequences; faculty are not permitted to modify its action by, for example, providing opportunities for making up work missed because of suspension.  The established appeal process for the student is first to the Student Life and Development Committee and then to the President of the College.
	\subsection{Rights and Responsibilities}
		By virtue of enrollment at Westmont College the student is entitled to certain rights among which are:
		\begin{enumerate}[label=\alph*)]
			\item{a clear statement of the expectations for each course and the means of evaluation to be used; this is provided in the course syllabus which the instructor distributes at the beginning of the semester, a copy of which is kept in the office of the Provost; the syllabus should be viewed as a contract which may be modified only by the common consent of the faculty member and those students enrolled in the course;
			}
			\item{access to the faculty member outside of class during announced office hours;
			}
			\item{unbiased evaluation;
			}
			\item{free but limited classroom expression of perspectives on topics germane to the course;
			}
			\item{protection against the excessive inclusion of materials which are not related to the course. Judgments of relevance are to be made in light of the mission of Westmont as a Christian liberal arts college.
			}
		\end{enumerate}
	\subsection{Sponsorship/Organizations} \change{}{23}{There is no content here in the 9/2020 version }
