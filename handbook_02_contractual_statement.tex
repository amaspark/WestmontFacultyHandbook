\section{CONTRACTUAL STATEMENT}
	\label{sec:ContractualStatement}
	\subsection{Faculty}
		The Faculty of Westmont College is a community of Christian teacher-scholars who share with others responsibility for institutional governance, and who exercise primary responsibility for curriculum, instruction, faculty status, and any policies for student life which relate to the educational program.  Other persons closely associated with academic programs may be granted faculty status as provided for in
		\ref{sec:SpecialStatusFaculty-Administrators}
		.
		\subsubsection{Types of Faculty Status}
			\paragraph{Ranked Faculty}
				\label{sec:RankedFaculty}
				\subparagraph{Full-Time}
					Full-time faculty are those teaching 24 units, or equivalent as approved by the Department Chair and the Provost, during an academic year.
				\subparagraph{Part-Time}
					Part-time faculty are those teaching fewer than 24 units, or equivalent as approved by the Department Chair and the Provost, during an academic year.
				\subparagraph{Appointment to Rank}
					\label{sec:AppointmentToRank}
					(for complete description see
					section
					\ref{sec:Promotion-Criteria}
					)
					\begin{enumerate}[label=\alph*)]
						\item{\underline{Instructor}:  Normally, faculty with professional preparation but who lack the prerequisites for probationary status (see
							section
							\ref{sec:TypesOfContract-Notice}
							).}
						\item{\underline{Assistant Professor}:  The initial rank for faculty with the academic preparation and experience required for the professorial ranks. In most cases, an Assistant Professorship is a full-time, probationary appointment (notice contract).}
						\item{\underline{Associate Professor}:  An intermediate rank for faculty who possess the academic preparation and experience required of the professorial ranks, including a terminal degree in their discipline, and have demonstrated continuing growth as teachers and scholars. Appointments are probationary (notice contract) or tenured (continuous contract).}
						\item{\underline{Professor}:  The highest recognition in rank given to a faculty member, reserved for those with an appropriate academic degree and experience and who demonstrate maturity in all aspects of scholarship, teaching, and service to the College. Full professors normally have tenured appointments (continuous contract).}
					\end{enumerate}
			\paragraph{Special Status Faculty}
				\subparagraph{Adjunct Faculty}
					Adjunct appointments are temporary relationships with the College provided in term contract agreements of one year or less which specify the benefits granted by the College. These faculty, by special action of the Provost, may be exempted from agreement with the Articles of Faith.
				\subparagraph{Emeritus/Emerita}
					Upon retirement faculty members may be granted emeritus/emerita status by action of the Board of Trustees on the recommendation of the Faculty Personnel Committee.  Normally, the title of ``emeritus'' or ``emerita'' is conferred upon faculty who have served Westmont College with distinction and have held faculty status for a minimum of 10 years.  The title shall include the rank at the time of retirement plus the word ``emeritus'' or ``emerita''.
				\subparagraph{Artist/Scholar-in-Residence}
					An Artist or Scholar-in-Residence is a limited-term faculty appointment given in recognition of the special contributions the individual can make to the academic, spiritual and cultural life of the college community.  Such individuals must affirm the College's Community Life Statement and abide by its expectations.  Normally, appointments would be for one year; additional yearly appointments are possible.  Artists or Scholars-in-Residence would not be eligible for tenure, nor the benefits of continuous contract faculty. Such individuals would, however, have available the full range of collegial opportunities offered other faculty.  An appointment as Artist or Scholar-in-Residence is made by the Provost, in consultation with the Department Chair and Personnel Committee, and with the approval of the Board of Trustees.
				\subparagraph{Administrators}
					\label{sec:SpecialStatusFaculty-Administrators}
					Certain administrative and academic support staff receive faculty status, but not rank, either by position (the Executive Team, Vice Provost, Dean of Curriculum and Educational Effectiveness, Director and Associate Director of the Library and Librarians, Registrar, Associate Registrar, Director of Disability Services, Senior Director of Admissions, Campus Pastor, Associate Dean of Students, Director of Global Education, Directors and Assistant Directors of Westmont Institutes, Director of the Center for Social Entrepreneurship, Director of Westmont in San Francisco, Westmont Off-Campus Programs faculty, Director of Intercultural Programs, Director of Career Development and Calling, Director of Internships, Director of the Westmont Ridley-Tree Museum of Art, Athletic Director) or by action of the Faculty Council.
				\subparagraph{Fellows}
					A fellow of Westmont College is a limited-term appointment made by the Provost to individuals whose work is expected to contribute in some substantial way to the academic, spiritual, and/or cultural life of the college community.  Typically, fellows are appointed in connection with a specific academic department, program, or college institute.  Normally, appointments are for up to one year; additional appointments are possible.  Fellows are granted faculty library privileges and work/study space (as available).  Fellows need not be in residence at Westmont's home campus.
		\subsubsection{Contract}
			\paragraph{Types of Contract}
				See also
				section
				\ref{sec:NonReappointment}
				\nameref{sec:NonReappointment}
				\subparagraph{Term (Non-Tenure Track)}
					\subsubparagraph{Temporary}
					\label{sec:Contract-Temporary}
					Temporary appointments are for one year or less.
					\subsubparagraph{Multi-Year, Non-continuous}
					Non-tenure track faculty who teach a full load may be placed on multi-year term contracts at the discretion of the Provost, in consultation with the Department Chair and the President and with the approval of the Board of Trustees.  Faculty with multi-year term contracts are considered ranked faculty (see
					section
					\ref{sec:RankedFaculty}
					) hired under special circumstances to fill a short-term need.  Multi-year term contracts are made for specific periods of time, usually for two or three years.  There is no expectation that an initial multi-year term contract will be renewed.  However, at the option of the College a new multi-year term contract may be issued, after review by the Provost in consultation with the Department Chair and the President, prior to December 15 of the final year of the contract.  No faculty member will remain on a multi-year, noncontinuous term contract for more than five years.  At the end of that five-year period, the Provost in consultation with the Department Chair and the President, will re-evaluate the special need for which the faculty member was hired, and the faculty member will either be moved to a multi-year, continuous contract, or the College's term contract with the faculty member will be terminated.
					\subsubparagraph{Multi-Year, Continuous}
					Non-tenure-track faculty who teach a full load may be placed on multi-year, continuous term contracts at the discretion of the Provost, in consultation with the Department Chair and the President and with the approval of the Board of Trustees. Faculty with multi-year, continuous term contracts are considered ranked faculty.  (See
					\ref{sec:RankedFaculty}
					)  A multi-year continuous contract recognizes that the College has a long-term need to fill a position, but circumstances (e.g. the faculty member's lack of a terminal degree) may not warrant a notice (probationary/tenure track) contract.  At the option of the College, a new multi-year term contract may be issued, after review by the Provost in consultation with the Department Chair and the President, prior to December 15 of the final year of the contract. Faculty on multi-year continuous term contracts will participate in regular performance reviews, and are eligible for housing benefits and for promotion and sabbatical at appropriate intervals.
				\subparagraph{Notice (Probationary/Tenure Track)}
					\label{sec:TypesOfContract-Notice}
					\begin{enumerate}[label=\alph*)]
						\item{One year notice contracts (probationary appointments) are made for faculty being considered for tenure.  (Criteria and process for granting tenure are discussed in
							section
							\ref{sec:Tenure}
							.)}
						\item{A request to convert a non-tenure track (term contract) appointment to probationary (notice contract) is initiated by the relevant department through its chair to the Provost and the Faculty Personnel Committee.  The Faculty Personnel Committee may waive the requirement to conduct a national search to fill the position.  No more than two years' credit toward final tenure review may be given (see
							section
							\ref{sec:PriorServiceCredit}
							).}
						\item{The principal difference between probationary (notice contract) and tenured (continuous contract) appointment is that persons with probationary status, after timely notice, can be denied reappointment without statement of reason (see
							section
							\ref{sec:NonReappointment-Notification}
							and
							\ref{sec:ReasonsForNonReappointmentOfProbationaryFacultyMember}
							\ref{par:ReasonsForNonReappointmentOfProbationaryFacultyMember-a}
							).}
					\end{enumerate}
				\subparagraph{Continuous (Tenured)}
					Tenure presumes continuous appointment until retirement,
					disability, resignation, death, or termination for cause.
					Under some circumstances reduction in force may affect
					tenured positions (see section \ref{sec:ReductionInForce}).  Tenure signifies a
					mutual relationship of trust and responsibility between the
					faculty member and the College.  The College advocates and
					assures its support in furthering development of the faculty
					as teachers and scholars; faculty affirm their continuing
					commitment to the mission of the College and their
					development as teachers, competence as scholars, and
					continued growth as Christians. Continuous appointment is
					not a guarantee of lifetime employment; it confers formal
					assurance that, thereafter, an individual's membership among
					the faculty of Westmont College will not be placed in
					question without full academic due process (see
					section
					\ref{sec:Discipline} and
					\ref{sec:ReductionInForce}).
				\subparagraph{Requests for Temporary Load Reduction}
					Full-time regular faculty may apply for a temporary reduction in workload for personal, non-professional reasons.  Application should be made to the Provost and should include a rationale for temporary load reduction and a specification of the requested length of term on a reduced-load contract.  In considering the application, the Provost will consult with the chair of the applicant's department to determine the consequences of the applicant's load reduction for his or her department.  Moreover, if a temporary load reduction contract is issued to the applicant, any adjustments made to the applicant's tenure, promotion and/or sabbatical schedule as a result of load reduction will be specified in writing, communicated to the applicant in writing, and placed in the applicant's personnel file when the contract is issued.  Applicants should consult with the Human Resources office for information concerning possible adjustments to benefits that may result from a temporary load reduction.
			\paragraph{Locus of Tenure and Appointment}
				\begin{enumerate}[label=\alph*)]
					\item{Faculty at Westmont College are tenured to the College and appointed to one or more academic departments.}
					\item{Appointment is made by the Board of Trustees upon the recommendation of the Provost and the President, after consultation with the department(s).}
					\item{Although the faculty member is appointed to, and functions primarily as, a professor of a department, faculty responsibilities are not limited to the department.  Since tenure and promotion involve more than departmental considerations, it is the responsibility of the Faculty Personnel Committee to appraise a faculty member's performance and contribution to the College at large, as well as to the discipline, in determining his or her suitability for promotion and tenure (see
						\ref{sec:EvaluationPromotionAndTenure}
						).}
				\end{enumerate}
			\paragraph{Prior Service Credit}
				\label{sec:PriorServiceCredit}
				At the time of the initial appointment of tenure-track faculty, the Provost, in communication with the department chair and the chair of the Faculty Personnel Committee, will determine appropriate credit for prior service toward rank, promotion, tenure and salary step.  Those terms will be put in writing at the time of initial appointment and placed in the faculty member's personnel file.  Final approval is given by the President and the Board of Trustees.
				\begin{enumerate}[label=\alph*)]

					\item{\underline{Rank} Initial rank reflects level of education and
						years of experience.  Ranks are defined in
						section
						\ref{sec:AppointmentToRank}
					}

					\item{\underline{Promotion} Instructional faculty with prior
						college teaching experience or scholarly or artistic
						achievement may be given advanced standing towards
						promotion.  Scholarly or artistic achievement includes
						service in the discipline as a professional employee,
						researcher, practitioner, post-doctoral fellow, or expert
						consultant; publishing through a peer-reviewed process;
						giving performances or exhibits that are publicly reviewed;
						presenting at professional meetings; and/or leading
						workshops or seminars.}

					\item{\underline{Tenure} Faculty who have not held tenure at
						a previous institution will be subject to the normal tenure
						process as specified in the \emph{Faculty Handbook}.
						Faculty with tenure or in a tenure-track position at a
						previous institution may be given up to three year's credit
						toward tenure.

						In exceptional circumstances such as the appointment of an
						endowed chair, faculty with significant experience and
						distinguished careers may be granted ``provisional tenure
						status'' at the time of appointment upon the recommendation of
						the faculty Personnel Committee and the approval of the Provost
						and President.  During the second year of the faculty member's
						appointment at Westmont, a mutual assessment-of-fit would be
						submitted to the Personnel Committee by the department chair and
						the faculty member.  This assessment would be based on the
						\emph{Faculty Handbook} criteria for tenure, but would also take
						into account written expectations at the time of appointment.
						At that time, the Personnel Committee may make one of three
						recommendations:
						\begin{enumerate}[label=(\arabic*)]
							\item{to remove provisional status and recommend
								full tenure:}
							\item{to continue provisional status for an
								additional two years, providing further time for assessment; or}
							\item{
								to schedule a full tenure review in four years as specified
								in
								\ref{sec:EvaluationPromotionAndTenure}
								(\nameref{sec:EvaluationPromotionAndTenure}).}
						\end{enumerate}
					}

					\item{\underline{Salary Step} Advanced credit toward salary
						step may be given for relevant experience that prepares the
						candidate to bring perspective, insight, and/or maturity to
						the position.  Such service includes serving as a
						professional, researcher, practitioner, post-doctoral
						fellow, consultant, or instructor.}

				\end{enumerate}
				\subparagraph{Part Time (Appended)}
				\subparagraph{Full Time (Appended)}
				\subparagraph{Articles of Faith}
					\label{sec:ArticlesOfFaith}
					\begin{enumerate}[label=\alph*)]
						\item{Westmont College is a liberal arts college committed to Jesus Christ and belonging to the worldwide evangelical Protestant tradition.  In that tradition, the college's trustees, administrators, and faculty participate in many different churches and with them confess such historic statements of the church as the Apostles' Creed and the Nicene Creed. In faithfulness to God, who is the source of truth, and under the authority of Scripture, we joyfully and humbly affirm the following articles of faith, which guide our learning, teaching, and living.}
						\item{Articles of Faith:}

						\quad We believe in God

						\quad The Lord our God alone is God, holy and loving, revealing in creation and in Jesus Christ God's own power and glory, grace and mercy. The Lord our God alone is God, just and true, perfect in being and trustworthy in action.

						\quad The Lord our God is infinite and beyond imagination; our minds can never fully know God nor our hearts completely grasp his ways. The Lord our God is faithful and steadfast, unfailing in word and deed.

						\quad The Lord our God is Triune---one being in three persons---Father, Son, and Holy Spirit in co-equal, co-eternal communion. The Lord our God, Creator and Sustainer of all that is, redeems the world from its fallenness and consummates his saving work in a new heaven and a new earth.

						\quad \dots the Father, Son, and Holy Spirit

						\quad God the Father is the source of all that is good. He is Father to his eternal Son, Jesus Christ, and to all who are adopted as his sons and daughters through faith in Jesus Christ. He has sovereignty over us, affection toward us, and glory for us.

						\quad God the Son became incarnate in Jesus Christ---one person in two natures, fully human and fully divine---who was conceived by the Holy Spirit and born of the virgin Mary. In his life and in his death on the cross he conquered the powers of darkness, paid the penalty for our sin, and demonstrated God's love for the world.  In his bodily resurrection his life and death are vindicated, and he is revealed to be the only judge and redeemer of the world. He intercedes for us now before the Father and will return in glory.

						\quad God the Holy Spirit is Lord and Life-Giver, the one who empowered Jesus Christ and who empowers his people to continue God's work today. God the Holy Spirit convicts us of sin, brings us to faith in Jesus Christ, and conforms us to the image of Christ. The Spirit inspired the authors of  Scripture and guides the church in faithful translation and interpretation. The Bible, consisting of the Old and New Testaments, is God-breathed and true, without error in all that it teaches; it is the supreme authority and only infallible guide for Christian faith and conduct---teaching, rebuking, and training us in righteousness.

						\quad \dots	the Author of our salvation

						\quad God created humankind for unbroken relationship with God, one another, and the rest of creation. Through Adam's disobedience, we fell into sin and now suffer alienation and brokenness. The effects of sin are so pervasive that apart from God's grace we are lost and dead. Only by God's grace through faith in Jesus Christ are we saved and made alive.

						\quad In bringing us to faith in Jesus Christ, the Spirit incorporates us into the body of Christ, his church, the community of all believers in heaven and on earth. The church is called to bear witness to Christ among the nations by praising God, preaching the good news, discipling believers, healing the sick, serving the poor, setting free the oppressed, and caring for creation. The gifts and fruit of the Holy Spirit empower the church for this mission.

						\quad Jesus Christ will return one day in his glorified body to judge the living and the dead. Those who do not believe in him will be raised to suffer forever a just punishment. Those who believe in him will be transformed, their bodies raised imperishable and incorruptible, to live and reign with him forever in a new heaven and a new earth in which there will be all that is good and true and beautiful, but no sorrow, no tears, and no evil thing.

						\quad \textit{And so we pray}:  Come, Lord Jesus.

						\item{These declarations of faith do not define in detail what an individual Christian might believe in many important areas of doctrine and theology.  Moreover, as a college seeking to serve evangelical Christians from many denominations, there is less an obligation to decide on these various points in detail than to celebrate not only unity in Jesus Christ but also the freedom to disagree, and to continue grappling in the many non-essential elements of faith.}

					\end{enumerate}
				\subparagraph{Community Life Statement}
					\label{sec:CommunityLifeStatement}
					\begin{enumerate}[label=\alph*)]
						\item{As a matter of conscience and contractual agreement faculty are to affirm and promote the Community Life Statement which provide a common understanding as all members seek to apply biblical principles in daily living.}
						\item{Community Life Statement

							\quad When Jesus Christ summed up the way His followers were to treat each other, He said, ``love one another as I have loved you,'' and ``Love your neighbor as yourself.''  On a college campus, this kind of love must take into consideration the relationship between learning and community.

							\quad Affirming the qualities of this relationship is vital.  As students, staff, and professors learn to live together, we recognize the dual manifestations of love in justice and mercy.  We attempt to work out what it means to live justly and mercifully in common agreements such as this one.  We understand that life in a college will give priority and honor to the wise development of the mind.

							\quad Given this focus, our social and intellectual growth needs freedom for exploration, complemented by a commitment to good will and graciousness.  Personal discipline is also required.  For example, civility is basic to all types of community, while academic honesty and respect for education are fundamental to an instructional environment.

							\quad Learning depends on truth-centered attitudes.  It thrives in an atmosphere of discriminating openness to ideas, a condition that is characterized by a measure of modesty toward one's own views, the desire to affirm the true, and the courage to examine the unfamiliar.  As convictions are expressed, one enters into the ``great conversation'' of collegiate life, a task best approached with a willingness to confront and be confronted with sound thinking.

							\quad Community is built upon other-centered practices.  It flourishes in a place where love for God and neighbor is cultivated and nurtured.  It grows strong when members practice integrity, confession, and forgiveness, attempt to live in reconciled relationships, accept responsibility for their actions, and words, and submit to biblical instructions for communal life.

							\quad Scripture supports these attitudes and principles.  It recognizes that all humans are created in the image of God and that God is the giver and taker of life, from the beginning of life to its end.  It teaches us to value human presence, celebrate human creativity, and promote relationships based on the ideals of trust, compassion, and forbearance and praises actions that manifest sacrificial giving and sincere faith.  Scripture also forbids attitudes such as pride and jealousy and prohibits such actions as drunkenness, sexual promiscuity, and dishonesty.  In keeping with these standards, the Westmont community has agreed to certain guidelines in the Student, Staff, and \textit{Faculty Handbooks}.


							\quad Desiring to implement the teachings of Christ, Westmont encourages true fellowship in the whole body of Christ, including the local church, for when we love each other we imitate Christ's love for us.  As we seek to follow God in truth, certain choices make for greater peace:  a respect for others as they make decisions contrary to ours, a readiness to listen carefully to those who represent situations or cultures unfamiliar to us, and a concern for how our preferences affect the lives of those around us.

							\quad We are committed to inquiry as well as pronouncement, rigorous study as well as kindred friendship, challenging teaching as well as reflective learning.   Sometimes these tensions will lead to conflict.  To live in unity, we must set ourselves to the practical task of discerning daily how to love well, how to inflesh the biblical call to justice and mercy.  As we do so, our life together at Westmont will begin to resemble the community God has envisioned for us.

						}
						\item{Behavioral Expectations:

							\quad The Westmont community chooses freely and willingly to impose upon itself rules for behavior which serve both the long-range interests of the institution and the immediate good of its individual members.  While we do not view these expectations as an index to maturity in Christ, we do regard violations as a serious breach of integrity within the community because each member has voluntarily chosen to associate with it and to accept its standards.

							\quad Consistent with Scripture, the College establishes the following specific expectations for the trustees, administration, faculty, staff, and students of the Westmont community:
							\begin{enumerate}[label=\arabic*)]
								\item{The college does not condone practices that Scripture forbids.  Such activities include but are not limited to occult practices, drunkenness, theft, profanity, dishonesty and sexual relations outside of marriage.  Westmont also recognizes that Scripture condemns ``sins of the spirit'' such as covetousness, jealousy, pride, and lust.  By their very nature, these sins are more difficult to discern.  Because they lie at the heart of the relationship between the individual and God they are of central concern of the Westmont community.}
								\item{The college upholds integrity as a core value of the community.  Members are expected to take responsibility for their own violations of all behavioral guidelines and demonstrate commitment to the value of integrity in word and deed.}
								\item{The college expects our members who choose to marry to abide by the commitment to lifelong heterosexual marriage and whether single or married to strive to maintain healthy family relationships.}
								\item{The college upholds the laws of the local community, the nation, and the state of California that prohibit the possession or use of illegal drugs or drug paraphernalia, against purchasing or consuming alcoholic beverages by persons under the age of 21, drunkenness, and driving under the influence of alcohol.}
								\item{The college recognizes that the use of tobacco products and alcoholic beverages presents a danger to personal health.  It condemns the abuse, and raises questions about the use of tobacco and alcohol.  Under no circumstances shall any member of the community use or posses these products on campus or when attending a college-related student activity.}
							\end{enumerate}
							\quad Westmont will establish other rules and regulations necessary for orderly community life and will list them in appropriate handbooks.
						}
					\end{enumerate}
		\subsubsection{Search, Appointment, Orientation}
			Appointments with notice (probationary) or continuous (tenured) contracts are made by the Board of Trustees upon the recommendation of the Provost and the President. Appointments with term contracts are made by the Provost, subject to review by the President and/or the Board of Trustees.
			\paragraph{Search Procedures}
				\begin{enumerate}[label=\alph*)]
					\item{ Prior to each new faculty search, the Provost will establish the degree requirements for the position in consultation with the Department Chair and Personnel Committee. Except in very rare circumstances, a terminal degree will be required for all faculty positions at Westmont.  The Department Chair will provide a description of the position and an announcement of the opening which will be based on a legally and institutionally acceptable format provided by the Provost.  Normally, a national search is conducted.}
					\item{The search committee will consist of the Provost, the department chair, representatives from the department, and two faculty from outside the department. One of those faculty will also be from outside the division. The Provost chairs the committee. The department chair serves as vice-chair of the search committee, coordinating the early stages of discussing, screening and ranking applications. This may also involve arranging preliminary conference interviews.}
					\item{The Provost may review applications and will expect departments to develop a pool of applications reflecting diversity of ethnicity and sex in addition to commitment both to the traditions of liberal education and to the integration of Christian faith with learning.  A short list of candidates will be developed by the full search committee and the Provost.  The Provost and department chair will determine the strategy for bringing candidates to campus for interview.}
					\item{ The campus visit includes
						\begin{enumerate}[label=(\arabic*)]
							\item{Interviews with
								\begin{enumerate}[label=$\bullet$]
									\item{the search committee}
									\item{departmental members}
									\item{the Provost}
									\item{the President}
									\item{the Vice President for Student Life or representative}
									\item{the non-departmental members of the search committee}
									\item{and students}
								\end{enumerate}
							}
							\item{Informational meetings with
								\begin{enumerate}[label=$\bullet$]
									\item{The Vice President for Finance,}
									\item{The Director of Human Resources, and}
									\item{The Assistant to the Provost for Diversity Initiatives or representatives from these offices.  While these meetings are primarily informational, questions or concerns that arise may be communicated to the search committee.}
								\end{enumerate}
							}
							\item{The candidate teaches at least one but preferably two classes (a lower division class and an upper division class), and offers a research presentation open to the college community.  Information is solicited by the department chair and from all who participated in the interview process.}
							\item{Over the course of the candidate's campus visit, the search committee must conduct significant discussions with the candidate regarding:
								\begin{enumerate}[label=$\bullet$]
									\item{the candidate's commitment to Westmont's mission and the central role that matters of faith play in faculty teaching, community life and scholarship}
									\item{the candidate's commitment to and potential for faith-learning integration}
									\item{criteria for tenure and promotion}
								\end{enumerate}
							}
							\item{The search committee should ensure that all candidates for the same position have comparable interview components.}
						\end{enumerate}
					}
				\end{enumerate}
			\paragraph{Appointment Procedures}

				\begin{enumerate}[label=\alph*)]
					\item{The search committee will make its recommendation for an appointment to the Provost.  If the Provost and the committee are unable to agree on an appointment, they will continue the search.  With the approval of the President, the Provost will forward his or her recommendation and that of the committee to the Academic Committee of the Board of Trustees.  Appointment is not final until the Board of Trustees has approved the contract of the candidate.}
					\item{When a full-time contract is offered, the Provost will specify in writing all conditions and considerations that may be distinctive to the contract in a letter of appointment.  If degree requirements have not been met at the time of appointment, the Provost will specify in writing the amount of time allowed to complete the degree.  Faculty who have not completed the degree by the agreed-upon date are subject to termination.  Attendant to the contract, the Provost will provide a copy of the \emph{Faculty Handbook}, the Articles of Faith, and Community Life Statement.}
				\end{enumerate}
			\paragraph{Orientation}
				It is the responsibility of the departmental chair, during the first semester of a new faculty member's employment, to review with the faculty member departmental procedures and the academic program of the College.  In addition, a faculty mentor will be appointed in accordance with the provisions of section
				\ref{sec:ProfessionalDevelopment}
				.
			\paragraph{Equal Opportunity}
				Westmont College does not unlawfully discriminate on the basis of age, race, color, sex, sexual orientation, national or ethnic origin, marital status, medical condition, genetic information, mental or physical disability in its employment practices, except where physical fitness is a valid occupational qualification. As allowed by federal and state law, Westmont chooses to exercise religious preference in all areas that it deems appropriate for its mission.
			\paragraph{Conflict of Interest}
				Situations involving the evaluation of a family member limit the participation of the faculty member.  When the family member is a student, the faculty member will make a reasonable effort to procure from another colleague additional evaluation of the student's work.  When a family member is an applicant for a position, the faculty member will not participate in the search process; the same criteria are used as in the search for a faculty member in any department.  In addition, it is prudent that neither family member serve as chair or, when this expectation cannot be met, any evaluation of the family member be supervised by a faculty member from another department who will be appointed by the Provost.
			\paragraph{Special Appointment Considerations}
				Full time faculty are required each contract year to indicate their affirmation of the Articles of Faith and the Community Life Statement (see
				sections
				\ref{sec:ArticlesOfFaith}
				and
				\ref{sec:CommunityLifeStatement}).
			\paragraph{Employment Eligibility Verification}
				Prior to appointment, a faculty member is required by federal law to provide verification of eligibility to be employed in the United States.
	\subsection{Evaluation, Promotion and Tenure}
		\label{sec:EvaluationPromotionAndTenure}
		Evaluation of faculty for promotion and tenure benefits the individual and the College. The procedures are similar, but particular criteria are weighted differently depending on the objective of the review.  These differences correspond to the complementary roles promotion and tenure play in accomplishing the educational mission and goals of the College.
		\begin{enumerate}
			\item{The purpose of the intermediate review is to make a judgment
				concerning the reviewee's progress towards tenure and to make a judgment
				as to whether or not the candidate is on a trajectory that would lead to
				tenure.}
			\item{The tenure review, although based in part on the criteria for the
				Associate Professorship and the expectation of sustained professional
				growth (see section
				\ref{sec:AssociateProfessor}
				), places special emphasis on the
				fundamental criterion and on teaching (see
				sections
				\ref{sec:Evaluation}
				and
				\ref{sec:Evaluation-Standards}
				\ref{par:Evaluation-Standards-Teaching}
				).
			}
			\item{ valuation for promotion, while taking into account institutional
				service, emphasizes professional criteria generally associated with the
				faculty member's discipline in the larger academic community and at
				Westmont College.}
		\end{enumerate}
		\subsubsection{Evaluation}
			\label{sec:Evaluation}
			It is understood that within the context of the review process, the Faculty Personnel Committee will determine what constitutes appropriate performance when questions arise about applying criteria (here and in sections
			\ref{sec:EvaluationPromotionAndTenure}
			,
			\ref{sec:Evaluation}
			and
			\ref{sec:Promotion-Criteria}
			).  It is also assumed that these criteria, though not necessarily weighted equally, will enable the Personnel Committee to assess performance as equitably as possible while taking individual strengths and expertise into account.
			\paragraph{Criteria}
				\label{sec:Evaluation-Criteria}
				Fundamental Criteria: The fundamental criterion for any favorable review is that a faculty member display in word and deed an appropriate sense of institutional mission. This means that the faculty member:
				\begin{enumerate}[label=\arabic*)]
					\item{clearly supports the mission of the college as an
						undergraduate, residential, evangelical Christian, liberal arts
						institution, affirming in particulate the Articles of Faith,
						honoring the Community Life Expectations, and displaying a
						commitment to the integration of faith and learning, and to the
						moral and spiritual development of students.}

					\item{cooperates with department colleagues, treats member of
						the large college community with respect and concern and
						approaches conflict constructively.}

				\end{enumerate}
				Failure to meet this fundamental criterion cannot be compensated for by the meeting of other standards.
			\paragraph{Standards}
				\label{sec:Evaluation-Standards}
				It is understood that within the context of the review process, the Personnel Committee will determine what constitutes appropriate performance when questions arise about applying criteria (here and in sections
				\ref{sec:EvaluationPromotionAndTenure}
				,
				\ref{sec:Evaluation}
				and
				\ref{sec:Promotion-Criteria}
				).  It is also assumed that these criteria, though not necessarily weighted equally, will enable the Personnel Committee to assess performance as equitably as possible while taking individual strengths and expertise into account.
				\begin{enumerate}[label=\alph*)]
					\item{\underline{Teaching}:
						\label{par:Evaluation-Standards-Teaching}
						\begin{enumerate}[label=\arabic*)]
							\item{communicates clearly, accurately, engagingly, and respectfully with students}
							\item{works effectively with a wide range of students and sets standards that challenge but do not discourage them}
							\item{explicitly as well as implicitly integrates faith and learning}
							\item{allows time for contact with students outside the classroom in office hours and other venues where conversation may be continued and relationships deepened}
							\item{functions effectively both as student mentor and as advisor (i.e., knows relevant requirements, discusses options with students, takes time to raise and address questions in advising sessions)}
							\item{functions effectively in a variety of teaching environments (lecture, small seminars, labs, etc.); involves students in research where appropriate}
						\end{enumerate}
					}
					\item{\underline{Professional Development}:
						\begin{enumerate}[label=\arabic*)]
							\item{publishes in scholarly journals, through an academic publishing house, or through other appropriate venues, and/or gives performances or exhibits that are publicly reviewed}
							\item{presents at professional meetings, leads workshops or seminars, or serves as an expert consultant in the field}
							\item{conducts scholarly work that reflects Christian commitment and, where appropriate, explicitly brings a Christian perspective to bear}
							\item{keeps abreast of developments in field through professional meetings and literature, and incorporates them into teaching and scholarship}
							\item{when appropriate, serves as a professional resource for the local community}
							\item{where applicable, takes significant initiative to engage in interdisciplinary research and dialogue leading to publication, presentation, or course development}
						\end{enumerate}
					}
					\item{\underline{Institutional Service}:
						\begin{enumerate}[label=\arabic*)]
							\item{takes part in the spiritual life of the community, e.g. through individual mentoring, chapel participation, Bible Study, residence hall events, and/or other activities focused on prayer, communal worship, or spiritual development}
							\item{attends and participates regularly in department, division, and faculty meetings, assuming appropriate leadership in such gatherings when requested}
							\item{serves conscientiously on standing committees when appointed or elected, barring serious reasons for excuse}
							\item{participates in annual assessment reports and multi-year program review reports preparation}
							\item{participates on occasion in task forces, search committees, program development and assessment teams, and other ad hoc working groups.}
							\item{participates in the local community through church involvement, and as appropriate through civic organizations, public causes, lectures, etc.}
							\item{participates on occasion in recruiting events and other public relations efforts of the college}
						\end{enumerate}
					}
				\end{enumerate}

			\paragraph{Evidence}
				Promotion and tenure evaluation will be based on:
				\begin{enumerate}[label=\alph*)]
					\item{student evaluations}
					\item{peer evaluations}
					\item{departmental evaluations}
					\item{administrative evaluations}
					\item{candidate's personal statement (self-assessment and faith-learning portfolio) and \underline{curriculum vitae}

						\begin{enumerate}[label=\arabic*)]
							\item{Self-assessment}
							The 5-10 page document addresses the standards of faculty review in the \emph{Faculty Handbook}, including the fundamental criterion described under
							\ref{sec:Evaluation}
							.  It presents the candidate's own assessment of his or her development and accomplishments in the areas of teaching, professional development, and institutional service, and describes the ways in which the candidate would like to grow in the future.
							\item{Faith-Learning Portfolio}

							This collection is both a record for use in
							reviewing the candidate and a tool for use by the
							faculty member in seeking to develop his or her own
							approach to the integration of faith and learning.
							Considered a work in progress, it may include
							reading lists of relevant material, reflections on
							those readings and/or on the faculty member's other
							efforts to integrate faith and learning in the
							classroom, in scholarship, and in participation in
							the life of the community.  The portfolio should
							include an essay reflecting the candidate's current
							thinking about the philosophy of Christian liberal
							arts education and about the relationship between
							the candidate's Christian faith and academic
							discipline.  At the time of final tenure review, the
							essay should have evolved into a \change{2,500-5,000 word}{30}{faculty approved replacing ``5-10 page''} paper on
							these topics.

						\end{enumerate}
					}
				\end{enumerate}
			\paragraph{Procedure}
				\label{sec:PromotionEvaluationProcedure}
				\underline{Personnel Committee Reviews \& Recommendations.}  Typically, the intermediate tenure review will occur in the spring semester of the third year, and the final tenure review in the spring of the sixth year.  Promotion reviews not connected with the tenure process typically will occur in the fall semester.
				\begin{enumerate}[label=\alph*)]
					\item{For each promotion and tenure review, each faculty member being reviewed is responsible to demonstrate sufficient progress toward tenure and promotion in the form of a portfolio presented to the Personnel Committee.
						\change{Faculty undergoing a review should be prepared to submit the materials listed below by the end of the second week of the semester during which they are to be reviewed.}{30}{faculty approved this addition}
						Required items include:
						\begin{enumerate}[label=\arabic*)]
							\item{a current curriculum vitae;}
							\item{personal statement (self-assessment essay and faith-learning portfolio.)}
							\item{course evaluations for all courses taught during the previous two years, or the results of alternative means of evaluation previously approved by the and the Personnel Committee;}
							\item{the schedule and class rosters for each current course;}
							\item{a list of students for potential interviews;}
							\item{a list of faculty for potential interviews;}
							\item{a list of external references to be contacted (required of candidates for promotion \change{}{30}{faculty approved removing ``to full professor,''} optional in other reviews).}
						\end{enumerate}
					}
					\item{At the time of each review for promotion and tenure, the Personnel Committee will assign one of its members (``reviewer'') to each faculty member undergoing review (``reviewee'').  The reviewer must be at least of the same rank as that for which the faculty member is a candidate, and must be tenured in the case of a final tenure review.  The Personnel Committee shall endeavor to avoid conflict of interest when assigning members to review cases.  Faculty not on the Personnel Committee who believe an assignment may result in a conflict of interest may state their objections to the Personnel Committee through the Provost.  Both in gathering data and in writing reports and summaries, reviewers shall endeavor to ensure confidentiality and, except in the case of student course evaluations, shall disregard comments or evaluations for which authorship is not claimed.  Responsibilities of reviewers are as follows:
						\begin{enumerate}[label=\arabic*)]
							\item{review the faculty member's personnel file, including departmental assessment summaries and any prior reviews;}
							\item{meet with the reviewee at least once toward the beginning of the review process in order to provide an orientation to the process and both ask and answer questions, and again toward the end of the process in order to address any issues or concerns raised during the review process and to ask any follow-up questions;}
							\item{solicit written feedback from and interview each member of a reviewee's department (alternate means of soliciting input may be required if a departmental colleague is out of the area; in such instances telephone interviews or extensive written feedback should suffice; a formal departmental recommendation is not required);}
							\item{solicit evaluations from faculty colleagues and administrators who have relevant information;}
							\item{examine student course evaluations provided by the reviewee;}
							\item{attend two class sessions taught by the reviewee, preferably of two different courses;}
							\item{interview at least eight students from classes taught the previous two years, one-half of the number of students interviewed selected from a list provided by the reviewee and the other half chosen by the reviewer from class rosters;}
							\item{solicit external reviews and evaluations of professional competence in teaching and scholarship from the list compiled by the faculty member under review (required for candidates for promotion\change{}{30}{facutly approved removing `` to full professor,} optional in other reviews);}
							\item{\change{upload all documents relevant fro the review to a secure site}{30}{faculty approved replacing ``prepare and submit a written report''}
								as a basis for deliberation and recommendation of the Personnel Committee, including the reviewee's self-assessments\change{.}{30}{faculty approved deleting ``in an attached appendix.''}}
						\end{enumerate}
					}
					\item{After completion of each promotion and final tenure review, the Personnel Committee will vote whether or not to recommend promotion or tenure.  In cases of intermediate tenure reviews, the Personnel Committee will vote whether nor not a candidate has passed the review.  Each of these decisions will be communicated to the Provost and to the candidate.  Prior to communicating this decision, the Personnel Committee will finalize a comprehensive summary statement (approximately 2-3 pages for the review.  Summary statements will include two sections:
						\begin{enumerate}[label=\arabic*)]
							\item{Formal Recommendation.  An opening paragraph indicating the Personnel Committee's recommendation.}
							\item{Summary.  Several paragraphs summarizing the overall findings of the review, organized according to each of the four criteria for review.  This section should include affirmations as well as suggestions for improvement or further development.}

							In the case of \change{all pre-tenure reviews}{30}{faculty approved replacing ``an intermediate tenure view''},
							the statement \change{must, and in the case of other reviews the statement may,}{30}{faculty approved replacing ``will''} include a third section:

							\item{Expectations.  In this section the Personnel Committee will explicitly state the matters that must be satisfactorily addressed before the candidate's
								\change{next}{30}{faculty approved replacing ``tenure''}
								review.}
						\end{enumerate}
					}
					\item{  The chair of the Personnel Committee (or vice chair, as appropriate) will notify the reviewee of the Committee's recommendation on the day the review is concluded.}
				\end{enumerate}
		\subsubsection{Promotion}
			\paragraph{Eligibility}
				An individual qualifies for promotion review after having served five years in a rank at the College.  Individuals may be reviewed prior to that time either (1) on the basis of written agreements completed at the time of appointment recognizing service in rank at another institution or (2) in the most exceptional cases when a faculty member has served not less than three years in the rank and is recommended by the department chair and Provost for early consideration in recognition of outstanding achievement.
			\paragraph{Criteria}
				\label{sec:Promotion-Criteria}
				\subparagraph{General}
					The criteria used in evaluation for promotion are presented in Sections
					\ref{sec:Evaluation-Criteria}
					and
					\ref{sec:Evaluation-Standards}. In all cases it is understood that the requisite academic degree is in the discipline of a faculty member's appointment.
				\subparagraph{Instructor}
					\begin{enumerate}[label=\alph*)]
						\item{A master's degree; or, in exceptional cases, equivalent professional education and experience directly related to the teaching assignment.}
						\item{Evidence of ability to teach effectively.}
						\item{Evidence of continuing growth in the professional field related to the teaching assignment, such as enrollment in a terminal degree program.}
					\end{enumerate}
				\subparagraph{Assistant Professor}
					\begin{enumerate}[label=\alph*)]
						\item{A terminal degree appropriate to one's discipline; or, in exceptional cases, a master's degree plus one year of advanced graduate study with continuing enrollment in a terminal degree program, or substantial college teaching experience.}
						\item{Reasonable prospect for becoming an excellent teacher.}
						\item{Evidence of professional competence for scholarly work and, where appropriate, creative activity.}
						\item{Participation in professional activities and prospect for continuing professional growth.}
						\item{A demonstrated interest in the integration of faith and learning.}
						\item{Acknowledgment and acceptance of faculty responsibilities for student advising and college governance and to model Christian living in the College community.}
					\end{enumerate}
				\subparagraph{Associate Professor}
					\label{sec:AssociateProfessor}
					\begin{enumerate}[label=\alph*)]
						\item{A terminal degree appropriate to one's discipline and six years of subsequent college teaching. In instances where a doctorate is the terminal degree, and under exceptional circumstances, a master's degree with more than one full year of additional graduate study and eight years of college teaching subsequent to the granting of the master's degree.}
						\item{Evidence of continued growth toward teaching excellence.}
						\item{Recognition of scholarship by peers in the professional field through such means as publication (e.g., original research, interdisciplinary scholarship, textbooks, course software), convention papers, recitals, and showings.}
						\item{Participation in professional activities.}
						\item{Active participation in faculty and departmental responsibilities.}
						\item{Continuing growth in achieving the integration of faith and learning and in the demonstration of Christian living as a model for the College community.}
					\end{enumerate}
				\subparagraph{Full Professor}
					\begin{enumerate}[label=\alph*)]
						\item{A terminal degree appropriate to one's discipline and 12 years of subsequent college teaching. In cases where the expected terminal degree is the doctorate, under the most exceptional circumstances, a master's degree with two (2) or more years of advanced graduate study and 20 years of subsequent college teaching.}
						\item{Evidence of maturity and excellence in teaching that establishes and sets forth the individual as a model for faculty colleagues.}
						\item{Evidence of mature and sustained professional involvements. Evidence should include sustained professional activities and recognition by peers in the professional field.}
						\item{Leadership in faculty roles of teaching, advising and college governance, and in mentoring colleagues.}
						\item{Maturity in Christian living and the integration of faith and learning which serves as a model for other members of the College community.}
					\end{enumerate}
			\paragraph{Procedure}
				The procedure for promotion review is presented in
				section
				\ref{sec:PromotionEvaluationProcedure}
				.  In all cases, a faculty member standing for promotion to a higher rank will receive from the Provost a report of the Personnel Committee's findings including its final recommendation prior to its submission to the President.  The Personnel Committee will either recommend promotion to the next rank or deferral of promotion.
			\paragraph{Appeal of Promotion Decision}
				This section provides recourse when a faculty member wishes to contest a negative decision concerning promotion.
				\begin{enumerate}[label=\alph*)]
					\item{\underline{Procedures}:
						A decision not to recommend promotion may be appealed to the Faculty Council by the same procedures as those provided in
						section
						\ref{sec:AppealOfTenureDecision},
						\nameref{sec:AppealOfTenureDecision}
						. The appeal must be made within \underline{three weeks of notification of the decision not to recommend promotion}. The appeal must be based on procedural and not substantive grounds; the Council will consider only procedural issues.}
					\item{\underline{Outcomes}:
						The decision of the Faculty Council will be rendered as a recommendation made through the Provost to the President. When the Faculty Council upholds a faculty member's appeal of a decision regarding promotion, the effect is not to recommend promotion, but to recommend the voiding of the original decision and the mandating of a new promotion review, to be completed no later than the end of the semester following the original decision.  When such new review results in a recommendation for promotion, the promotion is to be effective retroactively for the aforesaid academic year.}
				\end{enumerate}
		\subsubsection{Tenure}
			\label{sec:Tenure}

			\begin{enumerate}
				\item{ Westmont College affirms the principle of academic tenure.  After the
					expiration of a probationary period, teachers and scholars should have
					continuous tenure, and their service should be terminated only for adequate
					cause, including intentional and substantive denial of the Articles of Faith
					(see
					section
					\ref{sec:ArticlesOfFaith}
					), or under extraordinary circumstances, because of
					financial exigencies or reduction in force (see
					section
					\ref{sec:ReductionInForce} ).}
				\item{ Advancement to tenured status requires positive action by the College;
					tenure cannot be granted through inaction.}
			\end{enumerate}
			\paragraph{Eligibility}
				\begin{enumerate}[label=\alph*)]
					\item{A faculty member is eligible for tenure after six years of full-time service.  He or she must hold the rank of Associate Professor, or will be evaluated for promotion to Associate Professor concurrently with evaluation for tenure.}
					\item{Provisions for credit for prior service are located in
						section
						\ref{sec:PriorServiceCredit}
						.}
				\end{enumerate}
			\paragraph{Criteria}
				Criteria are those listed for Associate Professor (\ref{sec:AssociateProfessor}) with particular consideration given to evidence of:
				\begin{enumerate}[label=\alph*)]
					\item{Continued growth toward teaching excellence (specifically student growth and development).}
					\item{Active participation in faculty and departmental responsibilities.}
					\item{Continued growth in the integration of faith and learning, the demonstration of Christian living as a model for the College community, and the expression of confidence in the authority of Scripture and of commitment to Jesus Christ.}
				\end{enumerate}
			\paragraph{Procedure}
				\begin{enumerate}[label=\alph*)]
					\item{Each person shall be notified of progress toward tenure in accordance with the provisions of
						\ref{sec:PromotionEvaluationProcedure}
						.  Instructional faculty on probationary status must have a final tenure review no later than during the sixth year of full time service on continuous probationary appointment at Westmont College.  A final decision regarding tenure may be twice deferred for one year.  Therefore, not later than the eighth year, the College must grant tenure or a one-year terminal contract.}
					\item{The first review may be delayed for one year if, in extraordinary circumstances, the faculty member, the department chair and the Provost agree to postpone it.}
					\item{The process of probation precludes being removed from tenure track and remaining a faculty member at the College beyond one year.}
					\item{In all cases, a faculty member will receive from the Provost a report of the Personnel Committee including its final recommendation prior to its submission to the President.
						\begin{enumerate}[label=\arabic*)]
							\item{In instances of the first review, the decision of the committee may be:
								\begin{enumerate}[label=(\alph*)]
									\item{\underline{positive}, including acknowledgment of contributions to the College, and specification of areas of development and improvement necessary for a final recommendation for granting of tenure; or,}
									\item{\underline{negative}, a recommendation of non-renewal or that a terminal one-year contract be issued.}
								\end{enumerate}
							}
							\item{In instances of final tenure review, the decision of the Personnel Committee may be:
								\begin{enumerate}[label=(\alph*)]
									\item{\underline{positive}, recommending the \underline{granting of tenure};}
									\item{to recommend \underline{deferral} of tenure; or,}
									\item{\underline{negative}, recommending the \underline{denial of tenure}.}
								\end{enumerate}
							}
						\end{enumerate}
					}
					\item{If the Provost or President does not concur with a positive recommendation of the Personnel Committee, the case will be referred back to the Personnel Committee with suggestions for reconsideration.  If the final recommendation of the Personnel Committee is negative, or if the Provost, President, or Board of Trustees differs with a positive recommendation of the Personnel Committee, then the decision is negative resulting in separation and subject to the provisions of
						section
						\ref{sec:NonReappointment-Notification}
						.}
				\end{enumerate}
			\paragraph{Appeal of Tenure Decision}
				\label{sec:AppealOfTenureDecision}
				This section provides recourse when a faculty member wishes to contest a negative decision concerning tenure.
				\begin{enumerate}[label=\alph*)]

					\item{\underline{Procedures}:

						A negative tenure decision may be appealed to the Faculty Council in
						accordance with the provisions of
						\ref{sec:ReviewOfDecisionNotToIssueANoticeContract},
						.  \underline{The appeal must
							be made within three weeks of notification of the negative tenure
							decision}.  The appeal must be based on the provisions of
						\ref{sec:ReviewOfDecisionNotToIssueANoticeContract},
						\ref{par:ReviewOfDecisionNotToIssueANoticeContract-BasesForAppeal}
						and may not reconsider the substance of a tenure review.}

					\item{\underline{Outcomes}:

						The decision of the Faculty Council will be rendered as a
						recommendation made through the Provost to the President.  When
						the Faculty Council upholds a faculty member's appeal of a
						negative decision regarding tenure, the effect is not to
						recommend tenure, but to recommend the voiding of the original
						decision and the mandating of a new tenure review, to be
						completed no later than the end of the semester following the
						original decision.}

				\end{enumerate}
			\paragraph{Non-Tenure Track Faculty Evaluations}
				Continuing non-tenure track faculty will be reviewed on a regular basis employing the evaluation criteria in
				\ref{sec:Evaluation-Criteria}
				.  The evaluation will be carried out by the Personnel Committee and occur during the third and sixth years after appointment, and every sixth year thereafter.
		\subsubsection{Discipline}
			\label{sec:Discipline}
			Discipline is seldom necessary, and when it is administered, it is to be done in a fair manner which reflects concern for the individual as well as for the community.  The hope is that all discipline will ultimately be remedial.  The following are two procedures for discipline, depending upon the nature of the violation.
			\paragraph{Non-Performance of Contract}
				\label{sec:NonPerformanceOfContract}
				This section provides recourse when a faculty member (
				\ref{sec:NonPerformanceOfContract}
				\ref{par:NonPerformanceOfContract-Procedure}
				\ref{par:NonPerformanceOfContract-Procedure-Concerns}
				) has a grievance against another faculty member involving lapses from contractual obligations associated with teaching, advising, and governance (
				\ref{sec:NonPerformanceOfContract}
				\ref{par:NonPerformanceOfContract-Definition}
				).
				\begin{enumerate}[label=\alph*)]
					\item{\underline{Definitions and Principles}:}
					\label{par:NonPerformanceOfContract-Definition}

					Discipline for ``Non-performance of Contract'' is intended
					to hold faculty accountable for fulfilling contractual
					obligations associated with teaching, advising, and
					governance.  Its primary purpose is not to punish, but
					rather to identify and remedy patterns of non-performance of
					contractual obligations before more severe sanctions are
					required.

					It may not be applied to single instances of non-performance
					of contract or for the purpose of precluding or disparaging
					differences of opinion, for criticizing the content and
					method of a course, or for criticizing a faculty member's
					extra-mural activities.

					\item{\underline{Procedure}:
						\label{par:NonPerformanceOfContract-Procedure}
						\begin{enumerate}[label=\arabic*)]
							\item{
								\label{par:NonPerformanceOfContract-Procedure-Concerns}
								Concerns related to non-performance of contract may be raised by colleagues, by students, or by administrators.  It is usually desirable that the concern be raised first with the faculty member involved.  If there is good reason not to address the concern to the faculty member involved, it should be brought to the attention of the Department Chair or the Provost.}
							\item{The Department Chair (if the accused faculty member is a Department Chair, then the Dean of Curriculum and Educational Effectiveness) and the Provost will meet with the faculty member to discuss the allegation.  The faculty member has the right to give an explanation and/or interpretation, to present evidence, and to have another faculty member present as an advocate.  If the allegation is judged to lack substance, no further action will be taken.  The matter will not be discussed beyond those already involved, nor will a statement be placed in the faculty member's file.  If the allegation is found to have substance the faculty member, Department Chair, and the Provost should endeavor to resolve the problem to their mutual satisfaction.  If an agreement is reached, a statement to this effect will be written and signed by the Provost, co-signed by the Department Chair, and the faculty member, and given to the faculty member.  A copy of the statement will be placed in the faculty member's personnel file and all other communication pertaining to the case will be expunged.}
							\item{
								\label{par:NonPerformanceOfContract-Procedure-Compliance}
								If a resolution is not achieved to the satisfaction of the chair and the Provost, a finding presenting the problem and describing the steps deemed necessary to rectify it will be written by the Provost and co-signed by the Department Chair. The finding will outline the process for ascertaining the faculty member's compliance and the successful resolution of the problem.  This finding will be given to the faculty member and a copy placed in the faculty member's personnel file. In this instance, or for any other action that may follow, the faculty member has the right to append a response.}
							\item{
								\label{par:NonPerformanceOfContract-Procedure-Review}
								At a time determined in the finding (
								\ref{sec:NonPerformanceOfContract}
								\ref{par:NonPerformanceOfContract-Procedure}
								\ref{par:NonPerformanceOfContract-Procedure-Compliance}
								), a meeting among the faculty member, Department Chair, and the Provost will be held to review the faculty member's compliance with the conditions of the finding.  If, in the judgment of the chair and the Provost, the problem has been resolved, a statement written and signed by the Provost, and co-signed by the Department Chair and the faculty member, will be placed in the faculty member's personnel file.  If, however, the chair and the Provost determine that the conditions of the first finding have not been satisfied, a second finding written by the Provost and co-signed by the chair will be placed in the faculty member's personnel file.  This finding will describe the problem and establish the conditions for its resolution.}
							\item{At a time determined in the second finding (
								\ref{sec:NonPerformanceOfContract}
								\ref{par:NonPerformanceOfContract-Procedure}
								\ref{par:NonPerformanceOfContract-Procedure-Review}
								), a meeting among the faculty member, Department Chair and the Provost will be held to review the faculty member's compliance with the conditions of the finding.  If, in the judgment of the chair and the Provost, the problem has been resolved, a statement written by the Provost, and co-signed by the chair and the faculty member, will be placed in the faculty member's personnel file.  If, in their judgment, the conditions of the second finding have not been met, the chair and the Provost will recommend to the Faculty Personnel Committee that a sanction be brought against the faculty member.}
							\item{The Faculty Personnel Committee
								will review the case, relying on the pertinent
								written communications mentioned above.  The
								committee may choose to call the faculty member to
								appear in person.  In any case, the faculty member
								in question has the right to appear before the
								committee.  The Personnel Committee by majority
								vote, excluding the Provost or any other person
								previously involved in the process, may determine
								that: (a) sanctions are unnecessary, dismiss the case, and recommend that all communications pertaining thereto be expunged from the faculty member's personnel file; (b) determine that a sanction is justified and recommend both the sanction and the conditions for its removal (a statement thereof to be placed in the faculty member's personnel file); or (c) determine that the problem is severe enough to warrant suspension (
								\ref{sec:Suspension}
								) or discharge for cause proceedings (
								\ref{sec:DischargeForCauseHearing}
								).}
						\end{enumerate}
					}
					\item{\underline{Sanctions}:
						\label{par:NonPerformanceOfContract-Sanctions}
						Sanctions are limited to the following:
						\begin{enumerate}[label=\arabic*)]
							\item{denial of eligibility for teaching and scholarship awards}
							\item{denial of travel funding}
							\item{denial of faculty development grants}
							\item{delay of sabbatical leave}
							\item{suspension or denial of salary step increase}
						\end{enumerate}
					}
					\item{\underline{Appeal}:
						The application of progressive discipline may be appealed to the Faculty Council on the bases of (1) the severity of the sanction, (2) process and/or (3) the inequitable application of sanctions.}
				\end{enumerate}
			\paragraph{Violation of Contract}
				\label{sec:ViolationOfContract}
				This section provides recourse when a faculty member (
				\ref{sec:ViolationOfContractProcedures}
				\ref{par:ViolationOfContractProcedures-Allegation}
				) has a grievance against another faculty member involving lapses from contractual obligations related to the Articles of Faith, the Community Life Statement, or professional ethics
				(
				\ref{sec:ViolationOfContractDefinitions}
				).
				\subparagraph{Definitions and Principles}
					\label{sec:ViolationOfContractDefinitions}
					Discipline for ``Violation of Contract'' is intended to hold faculty accountable for contractual obligations related to the Articles of Faith, Community Life Statement, and professional ethics.  For the sake of the accused faculty member and the community, allegations about such violations are best kept confidential, if at all possible.
				\subparagraph{Procedures}
					\label{sec:ViolationOfContractProcedures}
					\begin{enumerate}[label=\alph*)]
						\item{
							\label{par:ViolationOfContractProcedures-Allegation}
							A colleague, a student, or an administrator may present an allegation of this sort - usually, in the first instance, to the faculty member involved.  If the faculty member's response fully satisfies the complainant, the matter may be considered as resolved.  Otherwise, or in the case where there is good reason not to address the faculty member directly, the allegation should be presented to the Provost, who will present the charges to the faculty member by way of attempting an informal resolution.}
						\item{The Provost, once involved, will investigate in a manner appropriate to the severity and circumstances of the allegation, and in accordance with the law.  The faculty member, after being apprised of the charges, has the right to a meeting with the Provost to explain, interpret, and present evidence with another faculty member present as advocate.  At the Provost's discretion, the faculty member's Department Chair may also be present (or the Dean of Curriculum and Educational Effectiveness, if the allegation involves the Department Chair), and also the Vice President for Student Life and Dean of Student Life, if the allegation involves students.  At this stage no one else will be made privy to the case.}
						If appropriate investigation shows the allegation to lack substance, no further action will be taken.  The matter will not be discussed beyond the circle of those already privy to the case (see prior paragraph), and no record will be made in the faculty member's file in the Provost's office.
						\item{If the allegation is found to have substance, but no potential for termination of contract, the Provost may resolve the problem administratively without enlarging the circle of those privy to the case.  The terms of such a resolution, including any sanctions, will be put in writing by the Provost, signed by the faculty member, and placed in the faculty member's file in the Provost's office.  Possible sanctions include, but are not limited to, those specified for non-performance of contract (
							section
							\ref{sec:NonPerformanceOfContract}
							\ref{par:NonPerformanceOfContract-Sanctions}
							).  In case of a need for reconciliation between the faculty member and someone else, the Provost will serve as mediator unless the principals agree instead to follow the procedures of the \emph{Faculty Handbook}, section
							\ref{sec:ResolvingGrievances},
							``\nameref{sec:ResolvingGrievances}''.
						}
						\item{If the allegation is found to have substance with a potential for termination of contract, the Provost will immediately refer the case to the Faculty Council - this with a view to preserving for the faculty member both the benefit of doubt and the prospect of full restoration with dignity.  Together the Provost and the Faculty Council will determine whether the Provost or the Faculty Council should pursue the case to a resolution.  Their choice will be guided by consideration for the best interest of the faculty member, the College community, and any other interested parties; but the latter option will be necessary if by this stage agreement cannot be reached as to the nature and severity of the allegations and/or the discipline appropriate to the case.
							\begin{enumerate}[label=\arabic*)]
								\item{On the former option the Provost will make final determinations of discipline in consultation with the Faculty Council.  The written terms of such a resolution, including any sanctions, will be produced by the Provost, signed by the faculty member, and placed in the faculty member's file in the Provost's office.  Possible sanctions include, but are not limited to, those specified for non-performance of contract (
									section
									\ref{sec:NonPerformanceOfContract}
									\ref{par:NonPerformanceOfContract-Sanctions}
									).  In case of a need for reconciliation between the faculty member and anyone else, the Provost will serve as mediator unless the principals agree instead to follow the procedures of the \emph{Faculty Handbook}, Section
									\ref{sec:ResolvingGrievances},
									``\nameref{sec:ResolvingGrievances}''.
									It is to be hoped that most such cases will be resolved without resort to discharge for cause proceedings.}
								\item{On the latter option, the Faculty Council may embark directly on the procedures of discharge for cause (
									section
									\ref{sec:DischargeForCauseHearing}
									) by presenting the charges in writing to the Provost, who will inform the faculty member and the President.  The Faculty Council, however, is not bound to treat the case as one of discharge for cause; if they choose not to do so, they will make final determinations of discipline in consultation with the Provost.}
							\end{enumerate}
						}
					\end{enumerate}
				\subparagraph{Appeal}
					At any stage prior to the Provost's bringing the case to the Faculty Council, the faculty member may do so by personal initiative, according to the formal procedure provided by Section
					\ref{sec:ResolvingGrievances}
					for resolving grievances.
			\paragraph{Suspension}
				\label{sec:Suspension}
				When charges are brought that could result in termination of contract, a faculty member may be suspended, or assigned to other duties in lieu of suspension, but only where there is clear danger of immediate harm to the faculty member or others.  The Provost will consult with Faculty Council in determining the propriety of suspension as well as its length and other conditions.
			\paragraph{Discharge for Cause Hearing}
				\label{sec:DischargeForCauseHearing}

				Confidentiality shall be maintained throughout the hearing process.  When charges include departure from the Articles of Faith, particular and careful attention must be given to Section
				\ref{sec:AcademicFreedom},
				``\nameref{sec:AcademicFreedom}''.
				\subparagraph{Formation and Membership of the Hearing Committee}
					The Hearing Committee will be formed by one of the two following procedures, to be chosen by the faculty member against whom the charges are being brought:
					\subsubparagraph{Hearing Committee Formation, Option 1}
					\label{sec:DischargeForCauseHearing-FormationOfTheHearingCommittee-Option1}
					\begin{enumerate}[label=\alph*)]
						\item{
							\label{sec:DischargeForCauseHearing-FormationOfTheHearingCommittee-Option1-First}
							In consultation with the Provost, Faculty Council will compose the hearing committee of the five most recently elected past members of the Faculty Council who are currently at the College, who are not on sabbatical, and who have completed the term of their subsequent committee release for service on the Faculty Council
							(Section
							\ref{sec:CommitteeRelease}
							\ref{sec:CommitteeRelease-Eligibility}
							\ref{sec:CommitteeRelease-Eligibility-Automatic}
							)
							, subject to the condition that at least three of the five must be tenured.  Of two or more faculty who were elected to the Faculty Council at the same time, the one(s) with greater seniority at the College will be chosen.  Prospective members will be given three days' opportunity to recuse themselves; and the Faculty Council will make the sole determination as to what constitutes adequate cause.}
						\item{After having found five consenting members for the committee, the Faculty Council will identify them to the President and to the faculty member.  All challenges to the committee membership must be submitted in writing to the Faculty Council within the seven days following this notification.  Parties on either side may make unlimited challenges for cause, and each side will be allowed no more than two challenges without cause, within the seven-day period.  The Faculty Council will make the sole determination upon each challenge and the source and identity of any successful challenge will be kept confidential.  If a challenge is filed after the deadline, Faculty Council will decide the case in consultation with the Provost, taking into account the specific circumstances surrounding the late request.  After membership substitutions occasioned by challenges have ended, the Faculty Council identify to each member the other members of the committee, and call them to their first meeting
							\label{sec:DischargeForCauseHearing-FormationOfTheHearingCommittee-Option1-Second}
							(
							\ref{sec:DischargeForCauseHearing-ProceduresOfTheHearingCommittee}
							\ref{sec:DischargeForCauseHearing-ProceduresOfTheHearingCommittee-Preliminary}
							\ref{sec:DischargeForCauseHearing-ProceduresOfTheHearingCommittee-Preliminary-First}
							).}
						\item{Faculty Council will make the sole determination upon members' requests to be excused from the committee at any time subsequent to its final formation as described in
							section
							\ref{sec:DischargeForCauseHearing-FormationOfTheHearingCommittee-Option1}
							\ref{sec:DischargeForCauseHearing-FormationOfTheHearingCommittee-Option1-Second}
							, and will fill all vacancies in accord with the procedures of
							section
							\ref{sec:DischargeForCauseHearing-FormationOfTheHearingCommittee-Option1}
							\ref{sec:DischargeForCauseHearing-FormationOfTheHearingCommittee-Option1-First}
							.}
					\end{enumerate}
					\subsubparagraph{Hearing Committee Formation, Option 2}
					\begin{enumerate}[label=\alph*)]

						\item{Membership will be five faculty members, four of whom must be tenured, and three of whom must be full professors.}

						\item{A slate of nominees is received in an executive session of the faculty convened by the Faculty Council.  Nominees who deem themselves disqualified for reasons of bias or conflict of interest will withdraw their names from nomination, either at the request of one of the parties or on their own initiative.  The number of nominees must be greater than the number of committee members in order to provide Faculty Council a body from which to select alternates should a committee member replacement be necessary.}

						\item{A committee is elected by secret ballot in an executive session of the faculty.}

						\item{After having found five consenting members for the committee, the Faculty Council will identify them to the President and to the faculty member.  All challenges to the committee membership must be submitted in writing to the Faculty Council within the seven days following this notification.  Parties on either side may make unlimited challenges for cause, and each side will be allowed no more than two challenges without cause, within the seven-day period.  The Faculty Council will make the sole determination upon each challenge, and the source and identity of any successful challenge will be kept confidential.  If a challenge is filed after the deadline, Faculty Council will decide the case in consultation with the Provost, taking into account the specific circumstances surrounding the late request.  Vacancies created by challenge will be filled by Faculty Council from among the un-elected nominees in reverse order of votes acquired, within constraints imposed by committee membership requirements.}

					\end{enumerate}
				\subparagraph{Procedures of the Hearing Committee}
					\label{sec:DischargeForCauseHearing-ProceduresOfTheHearingCommittee}

					\begin{enumerate}[label=\alph*)]
						\item{\underline{Preliminary Procedures}:
							\label{sec:DischargeForCauseHearing-ProceduresOfTheHearingCommittee-Preliminary}
							\begin{enumerate}[label=\arabic*)]
								\item{
									\label{sec:DischargeForCauseHearing-ProceduresOfTheHearingCommittee-Preliminary-First}
									At the first meeting of the hearing committee, the Faculty Council and the Provost will inform them for the first of the identity of the accused, and will furnish the committee with a written summary of the specific charges, of the main facts of the case, and of its stages thus far, as well as with all written accusations from aggrieved parties. The Faculty Council will impress upon the committee the importance of the principle that no presumption of guilt is to be inferred from the mere fact that all the procedures undertaken thus far in the case have not achieved a resolution.}
								\item{The hearing committee will choose their chair from among the tenured members.}

								\item{The hearing committee will then furnish the documents described in
									Section
									\ref{sec:DischargeForCauseHearing-ProceduresOfTheHearingCommittee}
									\ref{sec:DischargeForCauseHearing-ProceduresOfTheHearingCommittee-Preliminary}
									\ref{sec:DischargeForCauseHearing-ProceduresOfTheHearingCommittee-Preliminary-First}
									above to the accused faculty member and all other parties involved, with notification of the date for the formal hearing, giving at least thirty days' advance notice, but ensuring also that the date of the hearing follows without undue delay upon the thirty days' interval.  At any time prior to the formal hearing, the accused faculty member may direct to the hearing committee written response(s) to these documents.}

								\item{The faculty member may, by written notice to the hearing committee, waive the hearing at any time before its commencement.  If the faculty member waives the hearing but denies the charges or asserts that the charges do not support a finding of adequate cause for termination, the hearing committee will seek out and evaluate all the available evidence and rest its recommendation
									(see
									Sections
									\ref{sec:DischargeForCauseHearing-ProceduresOfTheHearingCommittee}
									\ref{sec:DischargeForCauseHearing-ProceduresOfTheHearingCommittee-Hearing}
									\ref{sec:DischargeForCauseHearing-ProceduresOfTheHearingCommittee-Hearing-Seventh}
									,
									\ref{sec:DischargeForCauseHearing-ProceduresOfTheHearingCommittee-Hearing-Eighth}
									)
									upon the evidence in the record.}
							\end{enumerate}
						}
						\item{\underline{Procedural Conditions}:
							\label{sec:DischargeForCauseHearing-ProceduresOfTheHearingCommittee-ProceduralConditions}
							\begin{enumerate}[label=\arabic*)]
								\item{
									\label{sec:DischargeForCauseHearing-ProceduresOfTheHearingCommittee-ProceduralConditions-First}
									During all proceedings under
									\ref{sec:DischargeForCauseHearing-ProceduresOfTheHearingCommittee}
									the faculty member will be entitled to have present as an advocate a faculty colleague, or any on- or off-campus advisor who is approved by the committee.  Legal counsel for the faculty member may be present, but only to advise the faculty member, not to address the committee directly.}
								\item{As appropriate to their knowledge of the case, the committee will invite witnesses to appear, and after due consultation with the interested parties will make the final determination as to who qualifies as a witness.}
								\item{Witnesses may be questioned by the members of the hearing committee, by the accused faculty member, by the accused faculty member's designated faculty advocate and on- or off-campus advisor (see subsection b)1) above), and/or by a representative of the College administration.}
								\item{The administration of the College may seek legal guidance from an attorney, but may not be represented in the hearing proceedings by any attorney, including any College administrator who has a law degree.  An attorney for the administration may be present, but only to advise the administration's representative, not to address the committee directly.}
								\item{
									\label{sec:DischargeForCauseHearing-ProceduresOfTheHearingCommittee-ProceduralConditions-Fifth}
									A certified recorder will make a verbatim record of the pre-hearing and hearing.
									(On the faculty member's entitlement to a copy, see Section
									\ref{sec:DischargeForCauseHearing-ProceduresOfTheHearingCommittee}
									\ref{sec:DischargeForCauseHearing-ProceduresOfTheHearingCommittee-Hearing}
									\ref{sec:DischargeForCauseHearing-ProceduresOfTheHearingCommittee-Hearing-Eighth}).
								}
							\end{enumerate}
						}
						\item{\underline{Pre-Hearing}:
							\label{sec:DischargeForCauseHearing-ProceduresOfTheHearingCommittee-PreHearing}
							\begin{enumerate}[label=\arabic*)]
								\item{If the faculty member does not waive the hearing, within 10 days of their serving notice of the formal hearing, the committee will conduct a pre-hearing for all the parties involved, and will give them at least five day's advance notice.  The purposes of the pre-hearing are for the committee to:}
								(a) ensure that all parties understand clearly what is at issue;
								(b) ensure that all parties have access to the same documents and information;
								(c) provide for any further exchange of information that the committee deems necessary as a preliminary to the formal hearing;
								(d) make any other preparations for the hearing that will serve to make it fair, effective, and expeditious;
								(e) apprise the parties involved of the procedures that they have determined should govern the conduct of the formal hearing;
								(f) confirm the date, time, and place for the commencement of the hearing; or, if necessary, to alter the date originally announced, though not to make it earlier.
								\item{
									\label{sec:DischargeForCauseHearing-ProceduresOfTheHearingCommittee-PreHearing-Second}
									In consultation with the committee, the faculty member may decide that the formal hearing will be private, or, otherwise, will decide who is to be admitted to the formal hearing as observer(s).}
								\item{At the request of any of the involved parties or the committee, a representative of a recognized educational association will be permitted to attend the formal hearing as an observer.}
							\end{enumerate}
						}
						\item{\underline{Hearing}:
							\label{sec:DischargeForCauseHearing-ProceduresOfTheHearingCommittee-Hearing}
							\begin{enumerate}[label=\arabic*)]
								\item{The burden of proof that adequate cause for discharge exists will rest with the institution and will be satisfied only by clear and convincing evidence in the record considered as a whole.}
								\item{The committee will grant adjournments to enable either party to investigate evidence about which a valid claim of surprise is made.}
								\item{The faculty member will be allowed time to obtain necessary witnesses and documentary or other evidence.  The administration will cooperate with the committee in securing witnesses and making available documentary and other evidence.}
								\item{The faculty member, the designate faculty advocate and on- or off-campus advisor (
									\ref{sec:DischargeForCauseHearing-ProceduresOfTheHearingCommittee}
									\ref{sec:DischargeForCauseHearing-ProceduresOfTheHearingCommittee-ProceduralConditions}
									\ref{sec:DischargeForCauseHearing-ProceduresOfTheHearingCommittee-ProceduralConditions-First}
									), and the College administration's representative will have the right to confront, question, and cross-examine all witnesses.  If witnesses are unable or unwilling to appear, and yet the committee determines that the interests of justice require their statements, the committee will identify them, disclose their statements, and if possible provide for interrogatories.}
								\item{In the hearing of charges of incompetence, the testimony will include that of qualified faculty members from the College and/or other institutions of higher education as chosen by the Faculty Council.}
								\item{The committee will not be bound by strict rules of legal evidence, and may admit any evidence which is of probative value for the case.}
								\item{
									\label{sec:DischargeForCauseHearing-ProceduresOfTheHearingCommittee-Hearing-Seventh}
									The findings of fact and the recommendation will be based solely on the hearing record.
									If the committee concludes that adequate cause for dismissal has been established, it will recommend either dismissal of the faculty member, or an academic penalty less than dismissal (e.g.,
									\ref{sec:NonPerformanceOfContract}
									\ref{par:NonPerformanceOfContract-Sanctions}
									).  If the committee concludes that adequate cause for dismissal has not been established, it will recommend retention of the faculty member either with no penalties or with an academic penalty less than dismissal.}
								\item{
									\label{sec:DischargeForCauseHearing-ProceduresOfTheHearingCommittee-Hearing-Eighth}
									The committee will convey its findings and recommendation to the President and the faculty member in writing, and will provide them both with a copy of the record of the pre-hearing and the hearing.  [On the pre-hearing copy,
											see section
											\ref{sec:DischargeForCauseHearing-ProceduresOfTheHearingCommittee}
											\ref{sec:DischargeForCauseHearing-ProceduresOfTheHearingCommittee-ProceduralConditions}
											\ref{sec:DischargeForCauseHearing-ProceduresOfTheHearingCommittee-ProceduralConditions-Fifth}
											.
										]}
								\item{The President, if rejecting the committee's recommendation, will convey the reasons in writing to the committee and to the faculty member, and will provide an opportunity for response by the committee, and/or by the faculty member through the committee, before transmitting the case to the Board of Trustees who will make the final decision.}
							\end{enumerate}
						}
						\item{\underline{Confidentiality}:
							\begin{enumerate}[label=\arabic*)]
								\item{Prior to the hearing, except for simple announcements that may be required, covering the time of the hearing and similar matters, public statements and publicity about the case by the faculty member, administrative officers or members of the committee will be avoided.  If the hearing is public, as provided in
									Section
									\ref{sec:DischargeForCauseHearing-ProceduresOfTheHearingCommittee}
									\ref{sec:DischargeForCauseHearing-ProceduresOfTheHearingCommittee-PreHearing}
									\ref{sec:DischargeForCauseHearing-ProceduresOfTheHearingCommittee-PreHearing-Second}
									, no one will make public statements about the case prior to the hearing, except to announce its time and place, and similar matters.}
								\item{If the hearing is private, as provided in
									Section
									\ref{sec:DischargeForCauseHearing-ProceduresOfTheHearingCommittee}
									\ref{sec:DischargeForCauseHearing-ProceduresOfTheHearingCommittee-PreHearing}
									\ref{sec:DischargeForCauseHearing-ProceduresOfTheHearingCommittee-PreHearing-Second}
									, no one besides the faculty member may make any part of the proceedings public.}
								\item{In either case, the committee will not make public the documents of the case or its deliberations}
								\item{The Faculty Council will address infractions of these policies.}
							\end{enumerate}
						}
					\end{enumerate}
		\subsubsection{Personnel Records}
			\paragraph{Purpose and Location}
				For each faculty member a file is maintained in the office of the Provost containing information relative to, and the results of, appointment, promotion, and tenure review.
			\paragraph{Content}
				Personnel files contain:
				\begin{enumerate}[label=\alph*)]
					\item{conditions of appointment}
					\item{reports pertaining to promotion and tenure reviews}
					\item{communications regarding professional performance}
					\item{conditions and reports related to disciplinary action}
					\item{current curriculum vitae}
					\item{current statements of philosophy of education and personal faith}
				\end{enumerate}
			\paragraph{Access}
				Personnel records will be maintained in compliance with pertinent federal and state laws.  Faculty are encouraged to review periodically the contents of their personnel files.
			\paragraph{Submissions}
				A faculty member may submit any material he or she deems pertinent to his or her personnel file.
			\paragraph{Exclusions}
				No communication for which authorship is not given may be placed in a faculty member's personnel file.
			\paragraph{Expungement}
				An individual has the right to challenge and request expungement of information to which he or she has access; at the discretion of the Provost, these materials may be removed.  In the event they are not expunged, the faculty member may place in his or her file information responding to the material in question.
	\subsection{Separation}
		\subsubsection{Resignation}
			\begin{enumerate}[label=\alph*)]
				\item{ All faculty are expected to fulfill the terms of their contracts.  Resignations to take effect during the term of a contract are permissible only under conditions mutually acceptable to the individual and the College.  Contract pay will be prorated based on service rendered.
				}
				\item{ A faculty member with a multi-year, notice or continuous appointment should give written notice of intent not to accept renewal of appointment at the earliest possible opportunity but not later than April 15, or 30 days after receiving notice of the terms of appointment for the coming year, whichever date occurs later. In unusual circumstances, a waiver of this requirement of written notice may be granted.
				}
			\end{enumerate}
		\subsubsection{Retirement}
			Ordinarily, retirement starts at the end of the academic year.  A decision to retire should be communicated in writing to the College as far in advance as possible.  Retired faculty are encouraged to remain active in the life of the College community.  To encourage this, to the extent available and practicable, the retiree may be granted the following privileges:  receipt of mail; access to computers, laboratories, the library, inter-library loans, and a library cubicle; office space; and, as determined by the departmental chair and the Provost, secretarial assistance in the preparation of scholarly manuscripts.
			\paragraph{Early Retirement}
				\label{sec:EarlyRetirement}
				An individual qualifies for early retirement at 62 years of age and 10 consecutive years of faculty service.  Provisions for, policies governing and benefits of early retirement are found in Section 2.8.
		\subsubsection{Non-Reappointment}
			\label{sec:NonReappointment}
			\paragraph{Notification}
				\label{sec:NonReappointment-Notification}
				\subparagraph{Term Contracts}
					\begin{enumerate}[label=\alph*)]
						\item{\underline{Temporary}: Written notification of non-renewal is not given.  Reappointment will not be presumed by faculty with temporary appointments unless a new temporary contract has been issued by the Provost.}
						\item{\underline{Multi-Year}: Written notification of non-renewal will be given no later than December 15 of the last academic year of the contract.}
					\end{enumerate}
				\subparagraph{Notice Contracts (Probationary Appointments)}
					Written notification of non-reappointment will be given to the faculty member in advance of the expiration of the current contract as follows:
					\begin{enumerate}[label=\alph*)]
						\item{not later than March 1 of the first academic year of service;}
						\item{not later than December 15 of the second academic year of service;}
						\item{In cases of non-reappointment after December 15 of the second year of service and during subsequent years of service, the faculty member will be notified of non-reappointment not later than the end of the spring semester and will be given a terminal 12-month contract with duties to be assigned at the discretion of the Provost.}
					\end{enumerate}
			\paragraph{Reasons for Non-Reappointment of Probationary Faculty Member}
				\label{sec:ReasonsForNonReappointmentOfProbationaryFacultyMember}
				\begin{enumerate}[label=\alph*)]
					\item{
						\label{par:ReasonsForNonReappointmentOfProbationaryFacultyMember-a}
						Although there is no presumption of continued employment for a faculty member under notice contract, such a faculty member notified of non-reappointment may reasonably be expected to ask about the reasons for non-reappointment.  Upon written request, and in the spirit of fairness, the Provost may disclose those reasons to the faculty member; however, neither the Provost nor any other agent of the College is obligated to disclose the reasons or to justify them.}
					\item{Though the faculty member may request that the reasons for non-reappointment be stated in writing, it is inappropriate to require that every notice of non-reappointment be accompanied by a written statement of the reasons for non-reappointment.  In some instances it may be difficult for the College to provide such a statement; in others, it may be in the best interest of the faculty member not to have the reasons given in written form.  These considerations will be discussed with the faculty member.}
				\end{enumerate}
			\paragraph{Review of Decision Not to Issue a Notice Contract}
				\label{sec:ReviewOfDecisionNotToIssueANoticeContract}
				This section provides recourse when a faculty member wishes to contest any negative decision concerning renewal of contract, including a negative tenure decision (
				section
				\ref{sec:AppealOfTenureDecision},
				\nameref{sec:AppealOfTenureDecision}
				).
				\begin{enumerate}[label=\alph*)]
					\item{A faculty member who has been notified of a decision not to offer a new notice contract may make written appeal to the Faculty Council.  Such a faculty member bears the responsibility for establishing a prima facie case and assumes the burden of proof in support of the appeal.}
					\item{
						\label{par:ReviewOfDecisionNotToIssueANoticeContract-BasesForAppeal}
						Bases for appeal are limited to: (1) violation of academic freedom, (2) discrimination as defined in the college's policy on unlawful discrimination and harassment, and (3) failure by the College to abide by institutional policies stated in the \emph{Faculty Handbook}.}
					\item{The faculty member may choose to confer with others, not including members of the Faculty Council, in marshaling evidence and writing the appeal.}
					\item{If the Faculty Council is convinced that a \underline{prima facie} case has been established
						\begin{enumerate}[label=\arabic*)]
							\item{Faculty Council will investigate the merits of the appeal
								under
								\ref{sec:ReviewOfDecisionNotToIssueANoticeContract}
								\ref{par:ReviewOfDecisionNotToIssueANoticeContract-BasesForAppeal},
								, collaborating as appropriate with other campus entities with overlapping jurisdiction (e.g. Title IX Officer).}
							\item{It is incumbent on those who made the non-reappointment decision to communicate with Faculty Council to address any alleged violations of
								\ref{sec:ReviewOfDecisionNotToIssueANoticeContract}
								\ref{par:ReviewOfDecisionNotToIssueANoticeContract-BasesForAppeal},
								(1) and (2), and to demonstrate their compliance with the institutional policies stated in the \emph{Faculty Handbook}.}
						\end{enumerate}
					}
					\item{The Faculty Council will render a judgment as to the merits of the appeal under
						section
						\ref{sec:ReviewOfDecisionNotToIssueANoticeContract}
						\ref{par:ReviewOfDecisionNotToIssueANoticeContract-BasesForAppeal},
						.}
					\item{When the Faculty Council is unable to determine that compliance under
						Section
						\ref{sec:ReviewOfDecisionNotToIssueANoticeContract}
						\ref{par:ReviewOfDecisionNotToIssueANoticeContract-BasesForAppeal},
						has occurred, the effect is not to recommend the issuance of a new notice contract, but to recommend a review of the case by those who decided not to issue a new contract.}
					\item{In cases where discrimination is alleged, legal counsel will be sought on behalf of the College prior to making any verbal or written response to the faculty member.}
					\item{The decision of the Faculty Council will be rendered as a recommendation made through the Provost to the President, and will be reported in writing directly to the faculty member.}
				\end{enumerate}
		\subsubsection{Reduction in Force}
			\label{sec:ReductionInForce}
			Reduction in the size of the faculty of the College can have serious consequences for individual faculty members as well as the quality and diversity of the academic program as a whole.  Therefore, when there is reason for a general reduction of personnel, as determined by the Administration and the Board of Trustees, there is also the expectation that proposed alternatives to a reduction will have been explored, that the sacrifices asked of the College will be in consideration of the centrality of the faculty to the life of an academic institution, and that any distribution of reductions within the faculty will be equitable and just.  While it is difficult to be specific about the nature of reductions, the following principles and procedures will apply:
			\begin{enumerate}[label=\alph*)]
				\item{ Principles:
					\begin{enumerate}[label=\arabic*)]
						\item{ Recognition of the faculty as the qualified, principal guardians of the academic program; }
						\item{ Provisions for the participation of the faculty and role of due process in modifications in the academic program, and the right of appropriate appeals by affected individuals; }
						\item{ Consideration of the following: tenure, rank, and seniority in service along with teaching effectiveness of individual faculty, the fit of a particular discipline within the mission of the College, and the balance and quality of the total academic program, in making adjustments. }
					\end{enumerate}
				}
				\item{ Procedures:
					\begin{enumerate}[label=\arabic*)]
						\item{ At the time it is necessary to consider a general reduction in college personnel, the President and Vice Presidents will meet with a faculty body consisting of the Faculty Council plus the two members of the Faculty Budget and Salary Committee who are on the President's Advisory Council.  If it is determined that faculty positions will be affected, this administration-faculty group will assist the President to determine the general priorities and criteria for such reductions. A report of the discussions and reduction priorities and criteria will be presented by the Faculty Council at a meeting of the full faculty. }
						\item{ The statement of priorities and criteria will then be given to the Academic Senate of the faculty.  The Academic Senate will establish a closed ballot of five members selected from the full-time faculty from each division.  The Provost will be an ex-officio member of the reduction-in-force committee thus elected by the faculty.  The statement of priorities and criteria (see 1) above) will be transmitted to the reduction-in-force committee which will consider both programs and individuals and will have the same access to information as the Faculty Personnel Committee.  Faculty members under consideration for layoff will be notified and will have the opportunity to meet with the committee to appeal prior to its final recommendations.  Recommendations of the reduction-in-force committee will be transmitted by the Provost to the President. At the time the President's recommendations are sent to the Board of Trustees, those recommendations will be presented at a meeting of the full faculty. }
						\item{ Academic departments and individual faculty members affected by reductions may appeal the decisions of the College through the Faculty Council on procedural grounds only. }
					\end{enumerate}
				}
			\end{enumerate}
	\subsection{Faculty Rights and Responsibilities}
		\subsubsection{Professional}
			\paragraph{Academic Freedom}
				\label{sec:AcademicFreedom}
				\begin{enumerate}[label=\alph*)]
					\item{As an institution of higher learning, Westmont College exists to advance the work of Christ and His church, to seek the truth, and to promote the common good.  To further these ends, the College affirms the centrality of freedom of thought and expression in liberal education.  Academic freedom is essential to the faculty's primary tasks; it promotes and protects faculty rights of inquiry and expression as they perform their duties as scholars and educators, as well as ensuring students' freedom to learn.  Thus, the faculty, within the framework of and in accord with the Articles of Faith, are entitled to the rights and privileges and bear the obligations of academic freedom.}
					\item{\underline{Westmont College and its faculty accept and
							abide by the following}:
						\begin{enumerate}[label=\arabic*)]
							\item{Faculty are entitled to full freedom in research and in the publication of the results, subject to the adequate performance of their other academic duties; however, research conducted solely for pecuniary return, when the faculty member is under full-time contract with the College, should be based upon a written understanding with the authorities of the College (see also
								Section
								\ref{sec:NonCollegeActivities}
								).}
							\item{Faculty are entitled to freedom in the classroom in discussing their subject, but they should be careful not to make the classroom a vehicle for a personal agenda that is not integral with the discipline or the liberal arts curriculum.}
							\item{Intentional, substantive denial of the Articles of Faith constitutes a violation of contract; in the event that such violations are alleged, the burden of proof rests with those bringing allegations.  Any other qualifications of academic freedom must be clearly stated in writing at the time of the appointment.}
							\item{Faculty are citizens, members of a learned profession, and officers of an educational institution.  When they speak or write as citizens, they should be free from institutional censorship or discipline.  As scholars and educational officers, they should also remember that the public may judge their profession and the College by their utterances.  Hence they should be accurate, exercise appropriate restraint, respect the opinions of others, and exercise care in attributing privately-held views to the College.}
						\end{enumerate}
					}
				\end{enumerate}
			\paragraph{Professional Ethics and Relationships}
				Faculty members should exemplify ethical principles of conduct in living and scholarship, promoting Christian ideals and the common welfare of the College.  These principles have implications for interpersonal relationships, the integrity of one's word, and confidentiality.  Faculty are to hold confidences as appropriate and to the extent allowed under the circumstances and by law.  Furthermore, as a matter of Christian and lawful practice, harassment or discrimination (on the basis of race, age, sex, or other unlawful discrimination) in relationships with students, staff, colleagues or administrators will not be tolerated.
				\begin{enumerate}[label=\alph*)]
					\item{\underline{With Students}:  Faculty members should model for students personal maturity in spiritual, intellectual and social relationships.  Students are to be co-learners with faculty, worthy of courteous, just, and impartial treatment.  Although faculty are called upon to profess, this is done with the understanding that the imposition of personal views on students is contrary to the spirit and process of liberal education; the faculty, therefore, should grant the same freedom of inquiry and conclusion which they presume for themselves.  In faculty-student relationships the well-being of the student is paramount as, for example, in academic counseling where the best interests of students take precedence over obtaining majors in one's discipline or increasing course enrollments.  Information in possession of a faculty member is not necessarily open to the student, but once placed in the student's official College file it is available to him or her in accordance with applicable legal regulations.}
					\item{\underline{With Colleagues}:  In concert with colleagues and others, faculty have a responsibility to participate in the life of the College.  An academic life is nourished and sustained by vigorous discussion of perspectives and methods; thus, all faculty are called to support each member's right to engage in discussion and to honor the privilege of presenting opposing points of view.  Also, recognizing the fragile nature of community relationships, faculty should refrain from undermining or demeaning, directly or by implication, the character, work or academic discipline of a colleague. When personal or collegial differences arise, all faculty are to follow the injunction of first taking their differences to the colleague(s) involved.  Out of responsibility to the College and to the personal and professional development of a colleague, these understandings are not to preclude honest and candid evaluations in the promotion and tenure process.}
					\item{\underline{With Administrators}:  Faculty and administrative relationships grow out of shared stewardship of the College.  Faculty share in governance through their advisory role to administrators.  Faculty perspectives, presented individually and collegially, are important in defining and enacting the mission of the College.  Likewise, administrators provide counsel and encouragement, as well as material support to the faculty.}
					\item{\underline{With Staff}:  Faculty acknowledge the significant contributions made by people in staff positions who also share in the stewardship of the college community.  As expected of all intramural associations, faculty are to be courteous and considerate in their relationships with staff, expressing appreciation and endeavoring to resolve problems through appropriate channels of authority and responsibility.}
					\item{\underline{With the General Community}:  Faculty can render important services to the general community as an expression of Christian social responsibility.  Acting as private citizens but also, when proper, as representatives of the College, faculty should seek opportunities to become involved in the life of the community as teachers and scholars bringing the Gospel to bear on a broken world.}
					\item{\underline{With the Church}:  Westmont College values and encourages a variety of denominational affiliations among its members.  Faculty should not limit their ministry to teaching, research and fellowship in the College and the general community but are expected to be worshipping participants in local congregations and, according to individual gifts, to serve the varied mission of the Church.}
				\end{enumerate}
			\paragraph{Non-College Activities by Full-Time Faculty}
				\label{sec:NonCollegeActivities}
				On-going professional activities or extramural employment that involve substantial time commitments in addition to one's contractual responsibilities require approval of the department chair and the Provost.  Employment or professional activities that create additional burdens for colleagues, detract from one's instructional and other obligations, or result in absence from the campus for more than the equivalent of one day during the school week will not be approved (see Section
				\ref{sec:AcademicFreedom}).
		\subsubsection{Instructional}
			\paragraph{Teaching}
				\begin{enumerate}[label=\alph*)]
					\item{The normal teaching load for full-time faculty is 12 credit hours in each semester.  At the request of the department chair and with the approval of the Provost a faculty member may agree to teach one additional course per semester.  Remuneration for course loads in addition to the normal 12 credit hours is at the same rate as that for part-time instructors.  Specific assignments are made by the department chair in consultation with the faculty member and the Provost.  Directed readings and tutorials, which are not required of a faculty member and for which there is no additional remuneration, are not included when computing the number of credit hours.  Directed readings and tutorials may not exceed three students or six credit hours in any given semester.  Department chairs receive four hours credit per year toward their teaching load.  Occasionally, the teaching assignment for a faculty member may be reduced in a given term to permit completion of a special project.}
					\item{Every faculty member is required to prepare a syllabus for each course and to submit it to the office of the Provost during the first week of classes of each term.  A syllabus should include the topics covered through the semester, required or recommended readings, major assignments, an examination schedule, and any special information regarding mode of evaluation or instruction that may be appropriate.  A course syllabus should be considered as a contract with the class; changes in a syllabus during a semester should be negotiated to the mutual satisfaction of the students and the instructor.}
					\item{Evaluation of student work constitutes a major responsibility of faculty members and should be conducted in a professional and impartial manner.  The faculty member should abide by the grading system and standards of the College.}
					\item{Every faculty member is required to maintain an accurate record of each student's progress within a course.  Course records for the last two semesters must be filed with the registrar's office if the faculty member leaves the employ of the College.}
					\item{Faculty members who are not tenured full professors are required to administer course evaluations for every class.}
				\end{enumerate}
			\paragraph{Advising/Office Hours}
				\begin{enumerate}[label=\alph*)]
					\item{Advising students in relation to their academic programs and professional goals is a principal responsibility of each faculty member.  The advising relationship should be approached as a personal and professional commitment to each advisee for the duration of his or her college experience.  Therefore, each faculty member is expected to be knowledgeable about the academic policies of the College and department including General Education requirements and those for the major.}
					\item{Faculty should be available to students on a regular basis apart from class meetings. Office hours for a minimum of five hours per week should be posted and should take into consideration morning and afternoon class schedules.}
				\end{enumerate}
		\subsubsection{Institutional}
			\paragraph{Participation in Campus Governance}
				\label{sec:ParticipationInCampusGovernance}
				Committee service is a principal means by which faculty participate in the formulation of policies and in the governance of the College.  All faculty members, at the request of the administration or Faculty Council, are expected to serve within the committee system.  The expressed interests of faculty members will be honored whenever possible when committee assignments are made (
				\ref{sec:CommitteesOfTheFaculty-Conditions}
				\ref{sec:CommitteesOfTheFaculty-Conditions-First}
				).  Procedures for selecting committee memberships and descriptions of committee responsibilities are specified in
				Section
				\ref{sec:CommitteesOfTheFaculty}
				.  Normally, no faculty member would serve on more than one major committee at any one time.  Assignment to special committees may be made from time to time by administrative personnel, in consultation with the Faculty Council.  A faculty member who fails to participate or whose participation is counterproductive to the work of the committee may be removed by the Faculty Council at the request of the committee chair.  Likewise, the Faculty Council may request that the Faculty replace the chair in cases of inadequate performance.
			\paragraph{Other Non-Instructional Activities}
				All faculty are required to attend faculty retreat, service of commitment, and commencement activities (Senior Awards Convocation, Baccalaureate, and Commencement).  Regular attendance at faculty meetings is required.  Faculty are encouraged to attend Chapel regularly as an expression of their commitment to the spiritual life of the college community.
			\paragraph{Course Relief for Institutional Service}
				Department Chairs and Vice-Chair of the Faculty will receive a one-course load reduction during each year that they serve in those institutional capacities.  Under extraordinary circumstances, and at the discretion of the Provost, these and other faculty members may also be granted a temporary course-load reduction for institutional service.
	\subsection{Faculty Development}

		\subsubsection{Leave}
			\paragraph{Sabbatical}
				\begin{enumerate}[label=\alph*)]

					\item{\underline{Purpose}: Paid sabbatical leaves for
						scholarly activities are available to faculty members with
						tenure or long-term multi-year contracts.  A sabbatical
						leave is an investment by the College for increasing the
						quality of instruction and scholarship through the
						professional enrichment of the faculty.  A sabbatical leave
						is normally not granted for work toward completion of a
						degree.}

					\item{\underline{Eligibility and General Provisions}:
						\begin{enumerate}[label=\arabic*)]
							\item{For tenured faculty, a minimum of six years (or equivalent) of full-time service since initial appointment or the most recent sabbatical or terminal degree leave is required.  In consultation with the Provost, credit toward a subsequent sabbatical may be granted if more than six years intervened between previous sabbaticals.  As provided for in the letter of appointment, credit up to two years toward a sabbatical may be given to faculty members with previous college or university experience.  Time on leave from the institution does not count toward eligibility.}
							\item{Faculty on multi-year contracts are eligible for a sabbatical during the ninth year of employment, after completing three consecutive multi-year contract periods.  Multi-year contract faculty will be eligible for additional sabbaticals according to conditions specified for tenured faculty.  See 1) above.}
							\item{The faculty member may request sabbatical leave at full salary for one-half of the academic year or half salary for a full academic year.}
							\item{A faculty member on sabbatical leave continues to be eligible to participate in benefit programs of the College.}
							\item{Approval must be received from the Provost whenever professional activities depart from the approved project or before additional employment is accepted during the leave.}
							\item{A faculty member granted sabbatical leave is contractually committed to return to the College for a period of one full year after the academic year in which the sabbatical was taken.  If such service is not completed, upon separation all sabbatical compensation is to be repaid.}
						\end{enumerate}
					}
					\item{\underline{Procedures}:
						\begin{enumerate}[label=\arabic*)]
							\item{Applications should be submitted to the Provost before October 1 for sabbatical leave during the following academic year.  All requests for sabbaticals must be approved by the Professional Development Committee with the concurrence of the Provost, President and the Board of Trustees.  Applicants will be notified of the Professional Development Committee's recommendation before December 1.  Notification of approval by the Provost, President, and the Board of Trustees will come by February 1. If the number of applications exceeds the allotment for a given year priority in awarding sabbaticals will be made on the basis of the significance of the proposed project to the professional development of the individual and to the College as determined by the Professional Development Committee.}
							\item{Within three months of concluding a leave, the recipient will submit a report to the Professional Development Committee on activities and achievements while on leave.  Reports will be posted on the Provost's web page (accessible only to viewers with a Westmont login identification).  Each sabbatical recipient will give an oral report to the faculty in the year following the sabbatical.}
						\end{enumerate}
					}
				\end{enumerate}
			\paragraph{Educational}
				\begin{enumerate}[label=\alph*)]

					\item{\underline{Academic Leave}:  The College encourages occasional
						academic leaves for faculty, especially in cases of faculty exchanges or
						academic fellowships.  A faculty member, with the support of the
						departmental chair, submits a proposal to the Professional Development
						Committee through the Provost.  Preference is given to proposals which
						reflect values consonant with the mission of the College, such as the
						relationship of faith to the discipline, interdisciplinary connections,
						cross-cultural dimensions, the position of Christian higher education in
						the context of higher education in general.  In the rare cases where
						faculty are hired without the completion of the terminal degree required
						for their teaching position, the Provost's office may make leave
						available to full-time faculty on notice contracts.

						\begin{enumerate}[label=\arabic*)]

							\item{Normally, such leaves do not involve compensation.  Because there is no
								salary from the College, no contributions are made to retirement plans or to
								F.I.C.A., nor are faculty eligible for unemployment compensation or state
								disability insurance.}

							\item{The person, however, is considered to be a continuing member of the
								faculty. Normal progress in rank is maintained, although the time away does not
								count toward sabbatical leave.  To the extent that institutional benefits,
								policies and providers allow, medical, dental, life, long-term disability, and
								travel accident coverage will be provided under the normal conditions
								established for all faculty members.}

						\end{enumerate}
					}
				\end{enumerate}
		\subsubsection{Conferences and Travel}
			Funds are available from the Provost for travel related to scholarly work and participation in professional programs.  See Section 5.7 for current policies.
		\subsubsection{Professional Development}
			\label{sec:ProfessionalDevelopment}
			\begin{enumerate}[label=\alph*)]

				\item{\underline{ Faculty Mentoring}:  As a resource for
					newly hired full-time faculty, a mentor will be
					appointed from outside the department (selected by the
					Provost in consultation with the department chair) to
					help the faculty member adjust to institutional
					practices and expectations, to answer questions as they
					arise, and to create a safe space for asking questions
					and addressing difficulties.  While the nature of the
					mentoring relationship is largely informal, a
					probationary faculty member may request that the mentor
					be involved in early probationary assessment meetings
					and processes.  Guidelines for mentors will be provided
					by the Provost's office. }

				\item{\underline{ Initial Departmental Assessment}:  To develop newly hired
					full-time faculty, a departmental assessment shall be conducted in the second
					year of employment.

					\begin{enumerate}[label=\arabic*)]

						\item{ The assessment shall occur in the fall semester of the
							second year and shall consist of both a written summary and a
							meeting among the probationary faculty member, the department
							chair, and the Provost.  Such meeting shall be initiated and
							scheduled by the Provost's office.  In the event that the
							department chair having the most relevant information is
							off-campus or no longer serving as chair when the assessment
							must take place, the Provost (in consultation with the
							department and faculty member) shall designate the most
							appropriate person to submit the written assessment and attend
							the required meeting.  The probationary faculty member may
							request that the mentor be included in the assessment process
							and/or the meeting. }

						\item{ The probationary faculty member will provide the chair
							with a progress portfolio not later than the first day of class
							of the faculty member's second year of service. The progress
							portfolio shall include:  an up-to-date curriculum vitae, a
							2-3-page self-assessment addressing his or her performance in
							the first year according to the full range of responsibilities
							outlined in the \emph{Faculty Handbook}
							section
							\ref{sec:Evaluation}
							, and any
							evidence which seems appropriate to the self-assessment. }

						\item{ Following receipt of the progress portfolio, the department chair shall
							provide the Provost's office and the probationary faculty member a written
							assessment, identifying both strengths and areas for improvement, relative to
							section
							\ref{sec:Evaluation}
							in particular, and requirements for promotion and tenure in
							general.  In preparation for the written assessment, the department chair shall:
							attend no fewer than two of the probationary faculty member's classes,
							preferably in two different courses and in two different semesters, during the
							faculty member's first year of service, paying particular attention both to
							professional competence and general pedagogical effectiveness; read the entire
							set of the faculty member's first-year teaching evaluations; solicit feedback
							from departmental colleagues and students, formally or informally; and meet with
							the probationary faculty member at least twice during the first year both to
							provide feedback and to solicit questions or concerns. }

						\item{ Within two weeks of the joint assessment meeting, any participant (chair,
							mentor, or probationary faculty member) may provide additional written response
							to the Provost.  Such responses will be placed in the probationary faculty
							member's personnel file, together with the rest of the department assessment
							documents and a summary statement written by the Provost. }

					\end{enumerate}
				}

				\item{\underline{ Fifth-Year Departmental Assessment}:  The department chair
					and a probationary faculty member shall meet during the fall semester of the
					fifth year of probationary service in order to discuss progress toward
					tenure and promotion.  The summary statement arising out of the intermediate
					tenure review (conducted by the Personnel Committee) will serve as a
					catalyst and benchmark for discussion and goal-setting.  Following this
					meeting, the department chair will notify the Provost's office that such
					meeting has taken place. }

				\item{\underline{ Curricular and Professional Projects}:  As an
					encouragement for faculty development the College provides funds for
					curricular and professional projects.  Each year, faculty may submit
					requests to the chair of the Professional Development Committee. The
					Professional Development Committee will award proposals according to their
					merits and the availability of funds. }

				\item{\underline{ Mayterm/Summer Session Salary and Policies}:  Prior to each Mayterm/summer
					session a salary schedule for summer teaching is established.  Mayterm/summer
					classes may be offered by faculty either for additional salary or as partial
					completion of their regular academic year contract. Arrangements for summer
					session teaching are made through the office of the Provost. }

				\item{\underline{ Loans for Completion of a Terminal Degree}:  In the rare
					cases where faculty are hired without the completion of the terminal degree
					required for their teaching position, the Provost's office may make loans
					available to full-time faculty members on notice contracts. Subject to
					availability, funds are provided for such expenses as tuition, fees, books,
					required travel, and research costs.

					\begin{enumerate}[label=\arabic*)]

						\item{ Applications are submitted to the Provost.  The
							application should summarize the degree program with
							rationale and anticipated expenses.  The Provost will
							authorize funds prior to each term in which the faculty
							member is enrolled. }

						\item{ For each year of full-time service at Westmont
							College subsequent to the year in which the loan is
							received, a percentage of the loan is forgiven.  Prior to
							the sixth year of full-time service since initial
							appointment, the rate is 15\%; for the sixth and following
							years the rate is 20\%. }

						\item{ All outstanding loan balances are immediately payable
							if the faculty member leaves the employ of Westmont College.}

					\end{enumerate}
				}

				\item{Accountability of Full Professors:  After a faculty member becomes a
					Full Professor, he or she will participate every six years in a structured
					process of discussion, reflection, evaluation, and planning future goals. The
					purpose of this structured process is to encourage ongoing personal and
					professional development in all areas of service to the college. One part of
					this process will involve meeting with a mutual mentoring group. Other parts
					will involve written reflection, student evaluations, class observation, and
					meeting with the department chair and Provost and to provide feedback to the
					faculty member on strengths and on areas where growth and improvement may be
					called for. In more detail:

					\begin{enumerate}[label=\arabic*)]

						\item{ During the fall semester, the faculty member will
							conduct standard class evaluations in all courses and
							discuss these with the department chair (or delegate chosen
							by the Provost). The faculty member is free to use
							personally designed class evaluations in addition to---but
							not in place of---standard class evaluations.}

						\item{ In addition, the department chair (or
							delegate) will observe one or more of the faculty
							member's classes and discuss observations with the
							faculty member. }

						\item{ The Provost will invite input on the faculty member's work (addressing
							the areas outlined in
							wction
							\ref{sec:Evaluation-Standards}:~Teaching, Professional Development, and
							Institutional Service) from all faculty. Such input should be submitted in
							writing to the Provost. }

						\item{ By October 1 of the year following, the department chair (or delegate
							chosen by the Provost) will submit a written assessment of the faculty member to
							the Provost, addressing the areas of accountability outlined in
							section
							\ref{sec:Evaluation-Standards}~:~Teaching, Professional Development, and Institutional Service. }

						\item{ The faculty member will participate in a mutual mentoring group. Each
							mutual mentoring group will be composed of 3-5 full professors, chosen to work
							together by the Professional Development Committee. The group will meet several
							times during the year. At each meeting, one faculty member will share with the
							group about his or her development during the previous six years in the areas of
							accountability outlined in
							section
							\ref{sec:Evaluation-Standards}: Teaching, Professional Development,
							and Institutional Service of pedagogy, scholarship, and philosophy of education.
							This time can be used for exploring goals, sharing insights, and obtaining
							advice from other members of the group. Group members are encouraged---though
							not required---to observe each other's classes. }

						\item{ Out of this experience with the mutual mentoring group, the faculty
							member will reflect in a three-page paper on his or her past development and
							planned future goals in the three areas of accountability outlined in
							section
							\ref{sec:Evaluation-Standards}: Teaching, Professional Development, and Institutional Service. This
							paper is submitted to the Provost by October 1 of the year following the mutual
							mentoring year. Both the department chair (or delegate chosen by the Provost)
							and the Provost will subsequently interact with the faculty member about this
							paper, input from faculty colleagues, and the assessment submitted by the
							department chair. }
					\end{enumerate}
				}
			\end{enumerate}
	\subsection{Working Conditions}
		\subsubsection{Policy on Harassment}
			\label{sec:PolicyOnHarassment}

			This section provides recourse when anyone within the campus community of
			faculty, staff, students, and guests suffers unlawful harassment.

			Westmont College is committed to providing a learning and work environment free
			of unlawful harassment.  In keeping with this commitment, the College prohibits
			and will not tolerate unlawful harassment because of sex (which includes sexual
			harassment\footnote{
				Both men and women are protected from sexual harassment, whether that harassment
				is perpetrated by a member of the same or opposite sex.  Sexual harassment may be commiteed by a
				male or a female toward either a male or a female.

				There are three kinds of sexual harassment:
				\begin{itemize}
					\item{Unwelcome sexual conduct determined by a reasonable person to be so severe, pervasive,
						and objectively offensive that it effectively denies a person equal access to the insitution's
						education program, activity, education or working environment (hostile environment)}
					\item{Sexual assualt, dating violence, domestic violence, or stalking (Clery Act/VAWA offenses)}
					\item{Employee conditions aid, benefit, or service of the institution on an individual's participation
						in unwelcome sexual conduct (quid pro quo)}
				\end{itemize}

			}
			, gender harassment and harassment due to pregnancy, childbirth or
			related medical condition) and harassment because of race, religious creed,
			color, national origin or ancestry, physical or mental disability, medical
			condition, marital status, age, sexual orientation or any other basis protected
			by federal, state, or local law, ordinance or regulation.  All such harassment
			is unlawful.

			Prohibited unlawful harassment includes, but is not limited to, the
			following behavior:

			\begin{enumerate}[label=\alph*)]

				\item{ Verbal conduct such as epithets, derogatory jokes or
					comments, slurs or unwanted sexual advances, invitations or
					comments; }

				\item{ Visual conduct such as derogatory and/or sexually-oriented
					posters, photography, cartoons, drawings or gestures; }

				\item{ Physical conduct such as sexually-oriented gestures, assault,
					unwanted touching, blocking normal movement, or interfering with
					work because of sex, race or any other protected basis; }

				\item{ Threats and demands to submit to sexual requests as a
					condition of continued employment or academic advancement, or to
					avoid some other loss, and offers of employment benefits in return
					for sexual favors; and }

				\item{ Retaliation for having reported or threatened to report
					harassment. }
			\end{enumerate}

			\quad It is Westmont College's policy to prohibit unlawful
			harassment within the campus community of faculty, staff, students
			and guests by any person and in any form, and to apply the
			procedures outlined in Unlawful Sexual Misconduct Policies and Procedures
			in responding to any complaints of
			harassment.  (The procedures outlined there shall preempt any other
			procedures set forth in various handbooks that may be deemed
			inconsistent with the policy.)  Westmont is committed to investigate
			promptly any complaints of harassment.  Where unlawful harassment is
			found to have occurred, the College will take appropriate
			disciplinary action reasonably calculated to end the harassment, up
			to and including termination of employment or expulsion from the
			College.

			\quad A complaint of harassment may be reported to any of the
			report receipients indicated in the college's Unlawful Sexual
			Misconduct Policies and Procedures.

			\quad A report recipient will notify the College's Title IX Coordinator
			whenever a report of harassment has been received.  The Title IX
			Coordinator shall ensure that the complainant promptly receives
			a copy of the Unlawful Sexual Misconduct Policies and Procedures and
			is clearly informed or his or her rights to assistance.


			\footnote{

				If the accused is the Title IX Coordinator, the Provost, or the Vice
				President of Student Life, then the President will be
				notified and help determine the appropriated investigation
				strategy. If the accused is the President or a member of the
				Board of Trustees, then the Chair of the Board will be notified
				and help determine the apporpirate investigation strategy.

			}


			\quad The Title IX Coordinator will ensure that:
			\begin{enumerate}[label=\alph*)]
				\item{ A report of harassment is promptly, fully and effectively
					investigated; }

				\item{ Whatever action is deemed necessary to end the unlawful
					harassment will be taken; and }

				\item{ The determination and imposition of any sanctions is handled
					in accordance with the Unlawful Sexual Misconduct Policies and Procedures.
				}

			\end{enumerate}

			\quad Confidentiality for both the complainant and the accused shall
			be encouraged and maintained as appropriate and to the extent
			allowed under the circumstances and by law.

			\quad The Title IX Coordinator will keep the complainant and the
			respondent informed about the process of investigating and
			responding to the complaint.  The College will not tolerate any
			reprisal or retaliation against someone who has submitted (or
			indicated an intent to submit) a complaint in good faith.

			\quad Westmont encourages all members of the community to report any
			incidents of unlawful harassment immediately so that complaints can
			be resolved quickly.  In addition, any member of the community who
			believes that he or she has been harassed or retaliated against for
			resisting or complaining about harassment, may file a complaint with
			appropriate government agencies.  The nearest offices are listed online.
			The U.S. Department of Education's Office
			of Civil Rights, the Federal Equal Employment Opportunity Commission
			and the California Department of Fair Employment and Housing
			investigate and prosecute complaints of prohibited harassment;
			currently, the statute of limitations for filing a claim with these
			agencies is 180 days, 300 days, and one year, respectively.

		\subsubsection{Policy on Faculty / Student Romantic Relationships}
			The college prohibits any full or part-time faculty member from engaging in a romantic and/or sexual relationship or activity with any current student to whom he or she is not married.
		\subsubsection{Policy on Drug-Free Campus}
			You can find the college's Safe and Drug-Free Schools and Communities Act policy in full at
			\href{
				https://www.westmont.edu/physical-plant/campus-safety/clery-report/safe-and-drug-free-schools-and-communities-act
			
			}.
		\subsubsection{Hazardous Materials}

			\begin{enumerate}[label=\alph*)]

				\item{ Flammable liquids (such as gasoline) may not be stored inside
					any buildings except laboratories and maintenance shops designed for
					this purpose. }

				\item{ No open flames (candles, lanterns, etc.) are permitted in any
					buildings except in laboratories and maintenance shops designed for
					their use. }

				\item{ Any spill of hazardous materials must be reported to Campus
					Security immediately. }

				\item{ No firearms or fireworks shall be carried, used or stored on
					campus. }

				\item{ In the event of a hazardous waste or utility problem, such as
					a gas leak or elevator failure, or other questions pertaining to
					safety policies, contact the Physical Plant Department or Campus
					Security. }

			\end{enumerate}
		\subsubsection{Human Subjects in Research}

			\begin{enumerate}[label=\alph*)]

				\item{ Westmont College, recognizing the responsibility to safeguard
					the rights and welfare of human subjects involved in research,
					complies with the guidelines of the Department of Health and Human
					Services, American Psychological Association, and other guidelines
					appropriate to the academic discipline.  These principles are
					applicable to research conducted at or sponsored by the College,
					regardless of source of funding.  For these purposes, ``research''
					means a systematic investigation designed to develop or contribute
					to generalizable knowledge. }

				\item{ The College maintains an Institutional Review Board (IRB)
					which reviews all non-exempt research projects conducted by College
					faculty, students, and staff.  The IRB has four members of the
					faculty, one elected annually to a four-year term, representing
					several disciplinary areas including both natural and social
					sciences and at least one non-scientific field.  One member will be
					appointed from the community at large (e.g., attorney, clergy,
					ethicist, etc.) in conformity with federal guidelines. }

				\item{ For non-exempt projects involving the use of human subjects,
					it is the responsibility of the project director to submit to the
					Dean of Curriculum and Educational Effectiveness (or another
					representative appointed by the Provost), a copy of WC Form C as
					well as copies of the protocol and consent forms to be used in the
					project.  These items and other relevant publications specifying
					exempt and non-exempt criteria and ethical guidelines are available
					from the IRB through the office of the Provost. }

			\end{enumerate}
	\subsection{Leaves (Non-Professional)}
		\subsubsection{Bereavement Leave}
			In the event of the death of an immediate family member a faculty member may take bereavement leave with pay as arranged with the Department Chair and the Provost.
		\subsubsection{Extended Medical Leave}
			Faculty work is seldom interrupted by illness, and coverage is usually possible with no real cost to the College by colleagues substituting for one another or by scheduling other times to meet with students.  However, paid extended medical leave is available to full-time regular faculty when non-work related personal illness or injury prevents them from fulfilling their responsibilities for more than three consecutive days.  In addition, faculty members may use up to five accrued extended medical leave days per year to care for a sick or injured member of the faculty member's household if that person is unable to care for himself/herself.
			\begin{enumerate}[label=\alph*)]

				\item{ Faculty begin their service with a reserve of 20 days of paid
					medical leave.  After two years of service, paid medical leave will
					accrue at the rate of 1.25 days for each month of completed academic
					service, with a maximum accrual of 10 days per year.  Extended
					medical leave is used at the rate of five days per week for the
					duration of one's illness. The maximum number of accrued days that
					would be charged for medical reasons in a 12 month period is 160 (32
					weeks times five days). }

				\item{ Additional leave of up to one semester with full pay may be
					extended to the faculty member at the discretion of the Provost in
					consultation with the Department Chair.  In such a case the faculty
					member will cover that portion of the semester's teaching
					responsibility or other duties through one or more of the following
					options with no additional compensation:

					\begin{enumerate}[label=\arabic*)]
						\item{ teach an additional course in a semester prior and/or
							following the leave; }

						\item{teach one or two courses in one or more Mayterms;}


						\item{ undertake other projects or work as assigned by the
							department chair and/or the Provost. }
					\end{enumerate}

				}

				\item{ Any advanced extended medical leave remaining upon
					termination of employment will be repaid at a rate based on the
					current salary for an overload course. }

				The number of full courses or the amount of equivalent work for which the faculty member will be responsible will be determined according to the number of days of accrued paid leave used during the leave and the amount of State Disability Insurance (SDI) available to the individual.

				\item{ Personal medical leave is coordinated with State Disability
					Insurance (SDI) payments for non-work related conditions. In order
					to receive short-term disability benefits, the disabled employee
					must file for California State Disability Insurance (SDI) benefits.
					In cases of non-work related leaves exceeding 90 days, Long Term
					Disability Insurance (LTD) may apply.  The Human Resources
					Department should be consulted prior to the leave for more
					information concerning eligibility for and use of SDI and LTD. }

				\item{ The maximum time that a faculty member may be granted medical
					leave, paid or unpaid, is two semesters beyond the semester in which
					the leave began.  Continuing employment beyond that time will depend
					on the ability of the faculty member to teach and on the conditions
					of the previous contract. }

				\item{ Unused medical leave may be carried forward for future use
					with no limit to the number of days in reserve but will be forfeited
					upon termination of employment. }

			\end{enumerate}
		\subsubsection{Parental Benefit and Leave Policy}

			Employees are entitled to certain benefits in the case of birth, adoption, or fostering of a child under the federal Family Medical Leave Act (FMLA), California Family Rights Act (CFRA), and other law. Information on federal and state family leave laws can be obtained from Westmont's \emph{Employee Handbook} and from the Human Resources office. FMLA and CFRA entitle an employee to up to 12 weeks of unpaid job protected leave per twelve-month period within one year of the child's birth, adoption, or start of foster care, and up to 12 weeks combined leave where both parents are Westmont employees. Pregnancy disability leave entitles employees to up to four months of job protected leave per pregnancy that is unpaid but potentially funded by State Disability Insurance (SDI).

			Westmont College provides qualifying faculty with certain options for taking these leaves and with additional benefits. These run concurrently with applicable federal and state leave taken for the same purposes, where the faculty member is also eligible for leave under those laws.

			\paragraph{Parental Leave for Care and Bonding}

				FMLA and CFRA leave provides for the care of and bonding with an infant or newly adopted or fostered child. Faculty may take leave for these purposes (and only these) in the form of a paid four-unit reduction of teaching load. If a reduction is not required in the semester of birth or adoption or placement (because it occurs late in the semester), it may be taken during the next semester.

				Eligibility. Leave benefits under this policy are available to returning multi-year and tenure-track faculty.

			\paragraph{Pregnancy Disability Leave}
				Pregnancy disability leave runs concurrently with FMLA leave taken because of a pregnancy-related medical condition. These employees are legally entitled to up to four months of unpaid job protected leave. Leave does not need to be taken at one time but can be taken as needed.

				All pregnant faculty may take these medical and disability leaves as a reduced workload in the form of a paid four-unit load reduction. Faculty taking paid pregnancy disability leave must apply for the allowable SDI benefit with the state Employment Development Division. The amount of SDI received will be deducted from any other pay the faculty member receives.

				Eligibility. Leave benefits under this policy are available to returning multi-year and tenure-track faculty. Provost approval may require a physician's recommendation for the modified or reduced workload in connection with a pregnancy disability or pregnancy related medical condition.

			\paragraph{Additional Benefits}
				Distribution of course reductions. Leave in the form of load reductions under the above policies may generally be taken in a single semester or split between two semesters or may occur as otherwise medically certified. The department chair in consultation with the provost will consider such requests, which will be reasonably accommodated unless to do so presents the College with an undue burden.

				If a total of eight units of load reduction are taken in a single semester, the remaining four units of that semester's load may be added to another semester's load as an overload or may be earned in Mayterm. Alternately, the faculty member may request a reduction in pay in the amount required to cover the remaining four units at the adjunct salary schedule rate. Either action requires approval of the department chair and provost.

				Advising, Committee Work and Tenure Clock.  The faculty member taking course load reductions under the above policies will be released from advising and committee responsibilities for a period of one semester, in consultation with and subject to the approval of the provost and the department chair. In addition, the faculty member may delay any tenure review by a year, up to a total of two years for all children.  The decision to delay tenure review must be made in advance of the course load reduction in consultation with the provost.

		\subsubsection{Jury and Witness Duty}
			The College recognizes and supports the civic responsibility of faculty members to participate in the judicial process.  Faculty called to serve on jury duty, however, are encouraged to seek deferment until the summer months when teaching would be unaffected.

		\subsubsection{Military Service}
			Westmont College complies with state and federal statutes which provide reemployment rights for inductees, enlistees, re- enlistees, and reservists, who enter active or inactive duty training in the Armed Forces of the United States or the Public Health Service while employed by the College in other than a temporary position.  These statutes also apply to short-term absences for military duty required of reservists and members of the National Guard.  Therefore, certain employee protections are provided for both extended tours of duty, as well as emergency call-up or annual military training duty.  Upon completion of military service, employees are entitled to reinstatement of employment at the College if they apply within 90 days of discharge or one year if hospitalized, and are still qualified to perform the duties of the position.

	\subsection{Benefits}
		\subsubsection{Government Mandated Benefits}
			\label{sec:GovernmentMandatedBenefits}
			\paragraph{Worker's Compensation Insurance}
				\begin{enumerate}[label=\alph*)]
					\item{Work related injuries and illnesses are covered by the College's Workers' Compensation program.  It is the responsibility of the faculty member to report all work related injuries immediately to his or her Department Chair and the Human Resources Department so that the necessary medical treatment may be determined and the required injury reports may be completed. Injuries that require medical attention other than basic first aid are handled as follows:
						\begin{enumerate}[label=\arabic*)]
							\item{Contact the Human Resources Department immediately.}
							\item{The injured faculty member may choose to be treated by his/her regular physician if the faculty member had notified the College of that preference, in writing, prior to the date of injury.}
							\item{If 2) above does not apply, Westmont or its insurance carrier has the right to determine the physician who will provide medical treatment for the first 30 days for all compensable injuries sustained by the faculty member, and to obtain, at reasonable intervals, medical diagnoses, medical progress reports and/or medical opinions as to the fitness of the faculty member for return to instructional and other duties.  The cost of such treatment will be paid by the College.}
						\end{enumerate}
					}
					\item{For injuries requiring time lost from instructional and other duties, Workers' Compensation requires a three day unpaid waiting period before salary replacement benefits begin.  It is the College's policy to pay regular salary to faculty members teaching at least 16 units, or the equivalent, for the three day waiting period.}
				\end{enumerate}
			\paragraph{Social Security}
				Participation in Social Security is required of all faculty and includes equal contributions made by the individual faculty member and the College.
			\paragraph{Unemployment Compensation}
				The California Unemployment Compensation Insurance Program is administered by the State Employment Development Department.  The program is funded solely by the College. The College reimburses the State for the partial salary replacement benefits to qualified unemployed faculty.

			\paragraph{Health Insurance Continuation}
				The Consolidated Budget Reconciliation Act (COBRA) is a federal law requiring employers to provide former employees and their dependents with the opportunity to maintain health benefits for a limited time following separation from employment in instances where such coverage would otherwise end.  Such instances include:  1) termination of employment or reduction in hours, 2) death of employee, 3) dependent ceasing to qualify as a ``dependent child,'' 4) divorce or separation from the employee, 5) employee becomes eligible for Medicare, and 6) termination of employment or lost eligibility due to disability.  It is the employee's responsibility to notify the Human Resource department within 60 days of a divorce, legal separation, or that a child has lost dependent status.  Notification to employees of their COBRA benefits at termination is the responsibility of the Human Resources Department.
			\paragraph{State Disability Insurance}
				\begin{enumerate}[label=\alph*)]
					\item{During periods of unpaid medical leave, including maternity leave, certified by a physician as medically necessary, faculty are eligible to apply for disability income benefits paid by the State of California.}
					\item{It is the faculty member's responsibility to obtain a claim form which must be signed by the faculty member and the attending physician.  Claim forms are available from the Human Resources Department or the State Employee Development Department (EDD) by telephone, letter, or in person.  Physicians or hospitals may also have claim forms.}
					\item{The College will coordinate paid medical leave with state disability payments, making up the difference between the partial salary replacement benefit from the State, and regular net salary, to the extent that the faculty member has paid medical leave available from the College.}
				\end{enumerate}
		\subsubsection{Discretionary Benefits}
			\underline{For purposes of eligibility for discretionary benefits}, ``full-time faculty'' are defined as those teaching at least 16 units or equivalent per academic year or 12 units in a single semester; ``part-time faculty'' are those teaching 11 to 15 units or equivalent over two consecutive semesters, with a minimum of four units per semester.  The benefits descriptions contained in the \emph{Faculty Handbook} are summaries of key features of each benefit.  The Plan Documents or insurance policies for each plan represent the complete and authoritative descriptions of benefits.  Detailed information is available from the Human Resources Office.  The College in consultation with the Faculty Budget and Salary Committee may modify or eliminate discretionary benefits.
			\paragraph{Medical Care Plan}

				Full-time faculty (defined above) and their dependents are
				eligible to participate in one of the College's group medical
				insurance plans.  The HMO premium for full-time faculty members'
				personal coverage is paid 100\% by the College; the PPO premium
				for personal coverage is shared by the faculty member and the
				College.  The cost of dependent coverage for full-time
				faculty is shared by the faculty member and the College.
				Part-time faculty (defined above) are able to participate by
				paying one-half of the cost of their own coverage plus the
				full amount of the employee contribution where applicable,
				and the full cost of dependent coverage through payroll
				deduction.  Faculty may participate on the first day of
				employment.

			\paragraph{Dental Care Plan}

				The College offers group dental insurance plans in which faculty
				and their dependents may participate.  Eligibility criteria for
				participation in a group dental plan by full-time faculty and
				part-time faculty are the same as for participation in a group
				medical plan.  Depending upon the plan chosen, the first day of
				coverage may not coincide with the first day of employment, and
				the faculty member's premium will vary.

			\paragraph{Health Insurance Transition Benefit}
				The College provides a health insurance transition benefit to ensure continuity of medical and dental insurance coverage for new benefits-eligible faculty members and their families during their transition to the College.  Even though a new faculty member's salary does not begin until the initial contract date, a new faculty member whose health insurance coverage expires prior to that date may elect to begin medical and dental insurance coverage at the College at any time between July 1 and the initial contract date, provided that, if enrolling in an HMO, the faculty member and covered dependents reside in the HMO service area.  The College will pay 100% of the premium for medical and dental insurance for such faculty members and their families prior to the start of their initial contract date.  Thereafter, the faculty member will contribute the current employee portion of medical and dental premiums.
			\paragraph{Retirement Medical Plan}
				\begin{enumerate}[label=\alph*)]
					\item{A faculty member taking early retirement (see
						section
						\ref{sec:EarlyRetirement}
						) may elect to continue participation in the College group medical plan until age 65 or as long as permitted by the medical plan provider.  The College will contribute toward the cost of that coverage an amount equal to that which would be contributed if the person were to remain an active faculty member.  Dependents of the retiree are not eligible for medical plan coverage after the faculty member retires.}
					\item{For a retiree who is 65 or older, the College will reimburse an amount that is the lesser of the above amount and the actual annual cost of a medical plan chosen by the retiree or the current specified annual retiree medical plan contribution, whichever is less.}
				\end{enumerate}


			\paragraph{Disability Plan}

				\begin{enumerate}[label=\alph*)]
					\item{\underline{Short-Term Disability Plan}:  The College will pay up to one-half salary, in coordination with State Disability payments, for a maximum of 90 days in any 12-month period, to faculty disabled by illness or injury that is not employment related, when disabled faculty have no available extended medical leave, and the absence is longer than seven consecutive calendar days.  (The waiting period will be waived if the faculty member is hospitalized.)
						\begin{enumerate}[label=\arabic*)]

							\item{The disabled faculty member must file for State Disability Insurance (SDI) benefits.}
							\item{The College will coordinate benefits by paying the difference between the faculty member's net salary and SDI after available extended medical leave days have been exhausted.  To determine the amount to be paid by the College, the College will contact the State Disability office to verify the weekly benefit amount to be granted to the faculty member.  The College will then pay the coordinated benefit on regularly scheduled paydays.  Once the faculty member receives a disability check from the State, the stub should be sent to the Payroll Department to verify the amount received.}
							\item{The College will pay coordinated benefits not to exceed one-half of regular net salary, for a maximum of 90 days from the first day of unpaid leave, or until Long-Term Disability payments would begin, whichever is shorter.  The College's benefit will apply regardless of the faculty member's eligibility to receive Long-Term Disability benefits.}
						\end{enumerate}
					}

					\item{\underline{Long-Term Disability Insurance}:  Long-Term Disability insurance is provided
						and paid 100\% by the College for full-time faculty members.  This insurance
						provides partial salary continuation should a faculty member become disabled and
						unable to work for more than three months.  The maximum amount of salary
						continuation for total disability is 60\% of monthly earnings, less certain other
						sources of income such as Social Security disability entitlements, up to a
						maximum of \$6500 per month.
						\begin{enumerate}[label=\arabic*)]

							\item{Eligible faculty receive this coverage on the first day of employment.  If a faculty member is less than age 60 at the time of disability, benefits continue during a period of disability until age 65, but for not less than five years.  If the faculty member is age 60 or over at the time of disability, the maximum period of benefits gradually decreases from five years at age 60 to one year for a disability at age 69 or over.}
							\item{As an additional benefit for employees receiving Long Term Disability payments, the College provides a ``retirement income protection'' contribution of up to 10\% of monthly earnings.  The contribution will be deposited into the Retirement Plan on behalf of an eligible faculty member, not to exceed the maximum allowed by law, as long as the faculty member is receiving disability payments and has been a Retirement Plan participant for at least three months prior to the disability.}
						\end{enumerate}
					}

					Additional details about this insurance benefit, including information on partial disability, mental illness, and survivor benefits, are contained in the Plan Document.
				\end{enumerate}
			\paragraph{Life Insurance}
				Life insurance coverage is provided and paid 100\% by the College for
				full-time faculty members.  The amount of insurance coverage for faculty less
				than age 65 is equal to the annual base salary rounded up to the nearest
				\$1,000, and is effective on the first day of employment.  For faculty age 65
				or older the amount of coverage is .67 times annual salary rounded up to the
				nearest \$1,000.
			\paragraph{Travel Accident Insurance}
				Travel insurance is provided by the College at no cost to full-time
				non-temporary faculty.  Eligible faculty will have \$25,000 in coverage which provides for payment of the full amount in case of accidental death, or one-quarter to full payment in cases of loss of sight, or dismemberment, depending on the specific injury.  Coverage provides 24-hour, world-wide protection while traveling on College business, and is effective on the first day of employment.
			\paragraph{Retirement Plan}
				\begin{enumerate}[label=\alph*)]

					\item{\underline{Plan Definition}:  The College offers eligible faculty the
						opportunity to participate in a 403(b) Defined Contribution
						Retirement Plan (the ``Plan'').  The Plan is governed by Section
						403(b) of the Internal Revenue Code.  The ``Plan Document'' is the
						legally required description of the rights, obligations and benefits
						under the Plan.  An abbreviated legally required description is
						contained in the ``Summary Plan Description.''  Both documents are
						available for review from the Human Resources Office.  The following
						is a summary of key elements of the Plan.}

					\item{\underline{Matching Contributions}:

						\begin{enumerate}[label=\arabic*)]

							\item{All faculty will be eligible for employer matching
								contributions (``matching contributions'') to the Plan after
								completing one year of service and having reached age 26.  For
								purposes of the Plan, a faculty member on a full-time contract
								calling for 24 units per year will be regarded as completing
								``one year of service'' after completing six months of full-time
								employment.  Years of service with other higher education
								institutions which employed Westmont faculty immediately prior
								to their employment at Westmont College will be recognized by
								the Plan.}

							\item{For other teaching loads, or in cases where the teaching
								load may vary from one semester to another, eligibility, and the
								waiting period for eligibility for matching contributions, will
								be determined by the Human Resources Office in accordance with
								the relevant provisions of the Plan.}

							\item{Once eligible for matching contributions from
								the College, continued eligibility for matching
								contributions is dependent upon maintaining a
								faculty contract that is the equivalent of at least
								12 units per year.  }

							\item{The maximum matching contribution is 7\% for
								the faculty member's contribution of 3\% or more.  }

							\item{Faculty member and College matching
								contributions are computed as percentages of base
								salary only.  For eligible faculty members, matching
								contributions are calculated as follows:

								\begin{enumerate}[label=(\alph*)]
									\item{3 times the faculty member's contribution up
										to the first 1\%}
									\item{ 2 times the faculty member's
										contribution above 1\% and
										up to 3\%}
								\end{enumerate}
							}

							\item{Matching contributions from the College on
								behalf of a faculty member are fully vested
								immediately. }

							\item{Any faculty member who is ineligible for
								matching contributions may elect to make voluntary
								contributions to the Plan through payroll reduction
								as soon as employment commences. }

						\end{enumerate}
					}



					\item{\underline{Contribution Limits}:  Contributions to the
						Plan are subject to limitations set by the IRS.}

					\item{\underline{Requesting Participation}:  It is the
						faculty member's responsibility to initiate a request to
						begin participation in the Plan after being notified by the
						Human Resources Office of eligibility to participate, or
						after declining or suspending participation.}

				\end{enumerate}
		\subsubsection{Institutional Benefits}
			For purposes of determining eligibility for institutional benefits, ``regular faculty'' refers to all faculty except those on temporary term contracts (see
			Section
			\ref{sec:Contract-Temporary}
			)


			\paragraph{Education Assistance}
				\label{sec:InsitutionalBenefits-Educational}
				\begin{enumerate}[label=\alph*)]
					\item{\underline{For Faculty}:
						\begin{enumerate}[label=\arabic*)]

							\item{Full-time regular faculty, with the approval
								of their department chairs, are eligible to enroll
								in one course per semester at no tuition charge.}

							\item{The Admissions Office will determine space
								availability for a faculty member who desires course
								work for degree credit.  Students paying full
								tuition will have preference for class space.
								Charges other than tuition, health fees, and student
								activity fees will be paid by the faculty member.}

						\end{enumerate}
					}
					\item{\underline{For Eligible Dependent Children}:
						\begin{enumerate}[label=\arabic*)]

							\item{Education Assistance is available to full-time
								regular faculty for their children who meet the IRS
								definition of dependent children.  The benefit is
								available for regular semesters and Mayterm up to
								the bachelor degree or teaching credential. All
								regular admission requirements must be met.  The
								benefit will be available for eight semesters,
								including off-campus Mayterm programs, per eligible
								dependent, plus an unlimited number of on-campus
								Mayterms.  The benefit is not available for private
								lessons, tutoring, and on-campus Mayterm classes
								below minimum enrollment.  However, if the
								difference between the partial and full salary of
								the faculty member is less than the cost of full
								tuition, an eligible dependent need only pay that
								difference (see Section
								\ref{sec:InsitutionalBenefits-Educational}
								\ref{sec:InsitutionalBenefits-Educational-Limitations}
								).}

							\item{When one of the eight semesters of benefit is
								used for an off-campus program, the benefit covers
								the tuition portion of the program.  Tuition for an
								off-campus program will be determined as follows:

								\begin{enumerate}[label=(\alph*)]

									\item{Tuition for an off-campus program
										during the fall or spring semester will be
										set at the tuition rate charged for the
										on-campus semester. }

									\item{Tuition for an off-campus program during
										Mayterm or summer will be set at 70\% of the
										total cost of the program.  This benefit may be
										used for any Westmont sponsored off-campus
										program.  However, one semester of benefit will
										be assessed regardless of the units available on
										the program. }

								\end{enumerate}
							}

							\item{The Education Assistance
								benefit for dependents for regular semesters and
								off-campus May Term programs is equal to:

								\begin{center}
									\begin{tabular}{ |l|l| }
										\hline

										Completed service* (prior to beginning of semester) & Benefit          \\
										\hline
										Less than 4 years                                   & None             \\
										\hline
										4 years                                             & 25\% of tuition  \\
										\hline
										5 years                                             & 50\% of tuition  \\
										\hline
										6 years                                             & 75\% of tuition  \\
										\hline
										7 years                                             & 100\% of tuition \\
										\hline
									\end{tabular}
								\end{center}


								*Full-time employment at other institutions of
								higher education immediately prior to beginning
								full-time employment at Westmont will be
								credited as service in qualifying for the
								Education Assistance benefits for dependent
								children.  Employees who began full-time
								employment prior to January 1, 2002 will be
								credited with an additional four years of
								service eligibility for this benefit.
							}

							\item{Full-time employment at other institutions of
								higher education will be credited as service at
								Westmont in qualifying for the Education Assistance
								benefit.}

							\item{Charges other than tuition will be paid by the
								faculty member.  However, the health fees may be waived
								if the dependent is covered by one of the Westmont group
								medical plans.}

						\end{enumerate}
					}

					\item{\underline{For Eligible Spouses}:

						Spouses of full-time regular faculty are eligible to enroll
						in one on-campus course per regular semester and course
						during on-campus Mayterm with no tuition charge.  Students
						paying full tuition and dependent children receiving this
						benefit will have preference for enrollment.  Charges other
						than tuition, health fees, and student activity fees will be
						paid by the faculty member.  No fee will be charged for
						auditing classes.  The benefit is not available for private
						lessons, tutoring and on-campus Mayterm courses below
						minimum enrollment (see d) below). }


					\item{\underline{Limitations}:
						\label{sec:InsitutionalBenefits-Educational-Limitations}
						\begin{enumerate}[label=\arabic*)]

							\item{The Education Assistance benefit is available
								for on-campus Mayterm classes only if the class has
								enrolled the minimum required number of students
								paying full tuition.  However, if the difference
								between the partial and full salary of the
								Mayterm instructor is less than the cost of full
								tuition, the faculty member need pay only that
								difference.}

							\item{Dependent children are required to complete an
								abbreviated Cal Grant application through the
								Financial Aid Office to determine probable
								eligibility for a Cal Grant in order to be eligible
								for the Education Assistance benefit.  The Education
								Assistance benefit will be the difference
								between any Cal Grant and the scheduled
								Education Assistance benefit.  Failure to make
								application will result in loss of eligibility
								for the Education Assistance benefit.  All other
								scholarships, grants, and aid may be
								retained by the student with no reduction of
								the Education Assistance benefit as long as
								the total amount does not exceed the
								student's expense budget established by the
								Financial Aid Office.  All financial aid
								must be reported to the Financial Aid
								Office.}

						\end{enumerate}
					}

					\item{\underline{Paid Leaves}:

						Faculty members who are on sabbaticals or other paid
						leaves of absence will continue to be eligible for the
						benefit. }

					\item{\underline{Retirees' Benefits}:

						The Education Assistance benefit for faculty and faculty
						spouses is available to retirees.

					}


					\item{\underline{Death or Disability of a Faculty Member}:
						\begin{enumerate}[label=\arabic*)]

							\item{If a dependent of a faculty member is
								receiving the Education Assistance benefit at the
								time the faculty member dies or becomes totally or
								permanently disabled, the dependent will continue to
								receive the benefit for the remainder of the current
								semester plus the following three semesters.}

							\item{If a faculty member with 10 or more years of
								continuous service dies or becomes totally disabled
								while a dependent is receiving the Education
								Assistance benefit, the dependent will remain
								eligible to receive eight semesters of
								assistance.}

							\item{If a currently employed faculty member with 10
								or more years of continuous service dies or becomes
								totally disabled, and the Education Assistance
								benefit is not being used at the time, each
								dependent child of the faculty member will remain
								eligible for Education Assistance at the rate of one
								semester of assistance for each two years of
								full-time service of the faculty member according to
								the following scale:

								\begin{center}
									\begin{tabular}{ ll }
										Years of Service & Semesters of Education Assistance \\
										10 years         & 5 semesters                       \\
										12 years         & 6 semesters                       \\
										14 years         & 7 semesters                       \\
										16 years         & 8 semesters                       \\
									\end{tabular}
								\end{center}
								The dependent child must begin use of the benefit within three years of the date of death or disability for a faculty member with 10 years of continuous service or within the eligibility period indicated in the following scale:
								\begin{center}
									\begin{tabular}{ ll }
										Years of Service & Eligibility Period \\
										10 years         & 3 years            \\
										12 years         & 4 years            \\
										14 years         & 5 years            \\
										16 years         & 6 years            \\
									\end{tabular}
								\end{center}
								The child of a deceased or disabled faculty member must continue to meet the IRS definition of a dependent child in order to remain eligible for Education Assistance.
							}
						\end{enumerate}
					}
					\item{\underline{Other Sources of Education Assistance}:

						In addition to the Education Assistance program, the
						College participates in programs which offer tuition
						assistance at many other colleges and universities.  The
						following programs are available to dependent children
						of faculty members who would otherwise be eligible for
						100\% of the Westmont College Education Assistance
						benefit and to dependent children who qualify for
						Education Assistance following the death or disability
						of a faculty member.  The number of available
						opportunities may vary from year to year.  Dependents
						are considered on a first-come, first-served basis, and
						the admission requirements of other schools must be met.
						Further information is available from the Office of
						Admissions.

						\begin{enumerate}[label=\arabic*)]

							\item{\underline{Christian College Consortium
									Program}:  At participating Consortium institutions,
								tuition differentials are ignored and the exchange
								student is treated exactly as those schools treat
								children of their own faculty in such matters as
								tuition, fees, etc.  The college enrolling the
								exchange student reserves the right to restrict the
								student's participation in certain programs.
								Available slots are usually limited by Westmont's
								record of participation in receiving and sending
								students.}

							\item{\underline{Christian College Coalition
									Program}:  At participating Coalition institutions,
								tuition will be waived for eligible students.}

							\item{\underline{Tuition Exchange Program}:
								Hundreds of institutions participate in this program
								which offers tuition waiver based on the
								participation history of the institutions. Some
								schools offer scholarships for graduate study, law
								school, junior college, two-year nursing programs,
								etc., as well as for four-year undergraduate
								education.}

						\end{enumerate}
					}
					\item{\underline{Administration}:

						The Education Assistance benefit is administered by the
						Human Resources Office with assistance from the Office of
						the Provost and the Financial Aid Office.  The Human
						Resources Office determines benefits eligibility.  The
						Financial Aid Office coordinates this benefit with other
						financial aid, as well as providing counsel to faculty on
						available financial assistance.

					}
				\end{enumerate}
			\paragraph{Housing Assistance Program}
				\begin{enumerate}[label=\alph*)]

					\item{All full-time faculty with notice (tenure-track) or continuous
						(tenured) contracts may be eligible to receive financial assistance from the
						College for the first purchase of a residence in the Santa Barbara area.
						The amount of assistance is based on the financial resources of the faculty
						member and the current cost of modest housing as determined by the Board of
						Trustees.}

					\item{Financial assistance is a one-time contribution toward the down
						payment and/or monthly payments toward the mortgage.  In return, the
						College receives a share of the appreciation in the value of the
						property.  Although the College can be repaid at any time, no repayment
						is required until the property is sold, or the faculty member ceases to
						be a full-time faculty member of the College, or ceases to occupy the
						housing as his or her principal residence.}

					\item{Specific provisions and other policies regarding the Housing
						Assistance Program are available from the office of the Vice
						President for Finance.}

				\end{enumerate}

	\subsection{Compensation}
		\subsubsection{Contract Period and Method of Payment}
			A standard contract between Westmont College and a faculty member is for a nine-month academic year.  Salary payments extend over a 12-month period unless the appointment is temporary or for one year, or the faculty member applies for payment over nine months rather than 12; in these cases, the nine-month contract will be paid fully in the contract period.  The payroll is issued to faculty members in 24 equal payments over 12 months.
		\subsubsection{Salary Schedule}
			\begin{enumerate}[label=\alph*)]
				\item{Prior to issuing contracts each year, the Board of Trustees adopts a revised Salary Schedule effective for the following academic year, listing salaries by rank and step.  Ranks are defined in
					section
					\ref{sec:AppointmentToRank}
				}

				\item{Faculty members receive step increases for each year of full-time service except in certain cases of non-performance of contractual obligations
					(see section \ref{sec:Discipline} ).

				}
			\end{enumerate}
		\subsubsection{Health Insurance Premiums and Dependent Care Expenses}

			Faculty members should contact the Human Resources Department for information
			regarding payment with pre-tax dollars of dependent medical insurance premiums,
			child care expenses, and unreimbursed medical expenses.

	\subsection{Procedures for Resolving Grievances}
		\label{sec:ResolvingGrievances}


		No matter how good our intentions may be, conflicts will arise.
		Complaints and minor conflicts can often be worked out between
		parties (with or without intervention).  Faculty members are
		encouraged to try to resolve conflicts informally (for helpful
		guidelines in resolving conflict, see Faculty Advices and Queries).

		Sometimes, however, a conflict or complaint actually constitutes a
		grievance against individuals or the institution (e.g., an
		interpersonal conflict that has polarized colleagues, or the belief
		that a policy has been unjustly applied).  For some grievances, the
		\emph{Faculty Handbook} provides recourse in other sections (see,
		for example,
		sections
		\ref{sec:PolicyOnHarassment}
		for procedures regarding harassment;
		\ref{sec:NonPerformanceOfContract}
		for non-performance of contract; and
		\ref{sec:ViolationOfContract}
		for violation
		of contract.).

		If the grievance is one for which there is no other/particular
		provision for recourse in the \emph{Faculty Handbook}, a faculty
		member may file a petition the Faculty Council.  The petition will
		identify the issue and will set forth in detail the nature of the
		grievance, presenting any factual or other data the petitioner deems
		pertinent to the case.  In acting upon a petition, the Faculty
		Council may:

		\begin{enumerate}[label=\alph*)]
			\item{refuse to review the case until other efforts at reconciliation and resolution have been attempted;
			}
			\item{review the case and declare the grievance to be without merit;
			}
			\item{review the case, declare the grievance to have merit and seek resolution; if resolution is not forthcoming then the Council may bring the matter to an executive session of the faculty;
			}
			\item{for reasons of conflict of interest, decide that an ad hoc committee should be constituted to review the case.
			}
		\end{enumerate}

