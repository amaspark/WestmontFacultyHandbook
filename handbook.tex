\documentclass[letterpaper, 11pt]{article}

\usepackage{amsmath, amsthm, latexsym, graphicx, graphics,tabularx,xcolor, multirow, pdfpages, float} 
\usepackage[charter]{mathdesign} % This font looks better on a screen -DJH
\usepackage{array}

\date{}
\author{}

\title{\vspace*{2in} {\bfseries

Faculty Handbook

 Westmont College}
 
   
   This Faculty Handbook is not a contract of employment.
   Section 2. Contractual Statement contains 
certain terms of employment and procedures 
that apply to faculty members.
  
  
}

% Flexibility for headers and footers
\usepackage{fancyhdr}
\pagestyle{fancyplain}
\headsep 0.2in
\setlength{\headheight}{14pt}

\setcounter{tocdepth}{2}

\usepackage[pdftex, breaklinks=true]{hyperref}
\hypersetup{
    unicode=false,					% non-Latin characters in Acrobat's bookmarks
    pdftoolbar=true,				% show Acrobat's toolbar?
    pdfmenubar=true,				% show Acrobat's menu?
    pdffitwindow=true,				% page fit to window when opened
    pdfnewwindow=true,				% links in new window
    colorlinks=true,				% false: boxed links; true: colored links
    linkcolor=blue,					% color of internal links
    citecolor=blue,					% color of links to bibliography
    filecolor=magenta,				% color of file links
    urlcolor=blue,					% color of external links
	linktoc=all
}

\parindent 0in
\parskip 12pt

\begin{document}

\fancyhf{} %clear all header and footer fields
\cfoot{\bf -- \thepage\ --}

\renewcommand{\contentsname}{Table of Contents}

\maketitle

\thispagestyle{empty}

\clearpage

\tableofcontents

\pagenumbering{roman}

\setcounter{page}{1}

\clearpage

%---------------------------------------------------

\phantomsection

\setcounter{page}{1} 
\pagenumbering{arabic}

\section{HISTORY, MISSION, GENERAL ORGANIZATION AND GOVERNANCE}
\subsection{History}
   Early evangelical Christian institutions of higher learning were conceived to witness to and to preserve the viability of a world-view understood in light of Christian faith and tradition. Many were founded primarily for the training of ministers, missionaries, and teachers.
   The Bible Missionary Institute, established in 1937 in Los Angeles, was the direct antecedent to Westmont. It was founded by Mrs. Alexander Kerr who envisioned a college for training men and women for full-time Christian service, especially young people without the resources to attend existing institutions.
   In 1940 Westmont College was incorporated as a four-year degree-granting Christian liberal arts college. Having outgrown its Los Angeles facilities, the campus was relocated to the Dwight Murphy estate in Montecito in 1945. Forty acres of the adjacent Deane School were added in 1967. Westmont was accredited by WASC in 1958. (For a discussion of the early history of Westmont College, see Nancy PhinneyÕs article in Noticias: Journal of the Santa Barbara Historical Museum, Vol LIV, No. 1 http://www.santabarbaramuseum.com/noticias-article.pdf written for WestmontÕs 75th anniversary.)
\subsection{Mission Statement}
   Westmont College is an undergraduate, residential, Christian, liberal arts community serving God's kingdom by cultivating thoughtful scholars, grateful servants, and faithful leaders for global engagement with the academy, church, and world.
\subsubsection{Philosophy of Education}
   Liberal Arts.  As a liberal arts college, Westmont seeks to help its students become certain kinds of people, not mere repositories of information or mere possessors of professional skills.  Where such information and competencies are acquired, it is to be done in an intellectual and social context that nourishes a larger spiritual vision and is integrated with it.  Crucially, as a liberal arts college, Westmont seeks to help inculcate those skills that contribute to leading a successful and satisfying life.  For just as one must be trained in the skills that enable one to engage in a trade, so one must be trained in those skills that enable one to engage in the distinctively human activities of reasoning, communicating, thoughtfully choosing oneÕs moral and spiritual ends, building political, economic and spiritual communities, and entering into those Òappreciative pleasuresÓ that require knowledge, experience, and trained discrimination. Herein lies the relationship between liberal learning and life, for these are the very skills that translate into performing well oneÕs role as citizen of the state, servant of the church, member of a family, worker or professional and participant in the cultural world.
   Christian.  Westmont College is committed to the universal truths of the Christian faith, to a high view of biblical authority and an orthodox doctrinal vision, and to the central importance of a personal relationship with God through Jesus Christ.  It is this Christian faith that the college seeks to integrate fully into its life as a liberal arts institution.  For the pursuit of a liberal arts education, with its emphasis on producing certain kinds of people and inculcating certain basic human skills essential for living a satisfactory life, cannot take place in isolation from oneÕs most basic commitments and beliefs.  For the Christian, then, this means bringing oneÕs biblical and theological heritage to this educational enterprise. Indeed, to have basic values and commitments that one cannot explicitly and systematically bring to this task is to have an education that is severely truncated, severed, as it were, from oneÕs most important beliefs and values.  To isolate oneÕs worldview in this way, while pursuing an education, will only result in a worldview uninformed by sustained intellectual reflection.  Such an approach will yield persons who are not fully educated, indeed not educated at the core of their being.  For the Christian, therefore, higher education must be Christian education, if it is to be education for the whole person.  At Westmont, then, Christian faith is to inform the academic enterprise and the academic enterprise is to inform oneÕs Christian faith and thus yield a Christian worldview that is biblically based and intellectually sound. 
   Undergraduate.  Westmont is an undergraduate college and as such directs its attention, focuses its resources and devises its pedagogical strategies to facilitate the development of students who are beginning their post-secondary education.  It follows that the primary emphasis at Westmont is on teaching.  But teaching often involves helping students to acquire research skills and to become themselves producers of knowledge.  This can be done effectively only as faculty model research skills for students, and mentor them in the acquisition of those skills.  Moreover, to create a vital intellectual environment profitable for undergraduate students, Westmont must be an institution where knowledge is generated as well as transmitted.  But producing such knowledge is to be largely (though not exclusively) evaluated and appreciated in terms of the benefits that accrue, directly or indirectly, to those undergraduates who have come to Westmont to receive their education. For it is those students that the educational programs at Westmont are dedicated.
   Residential.  The educational programs of Westmont College are residential in character and reflect a commitment to facilitate and exploit the ways in which education occurs within community.  Indeed, ever since the monastic tradition, learning has been cultivated and transmitted within residential communities, enabling learning to be promoted by the joys of shared exploration and the sustenance of spiritual kinship.  Moreover, both the Christian and liberal arts traditions remind us of the integrity of human wholeness; we cannot be neatly compartmentalized into rational, spiritual and affective components.  The residential character allows and encourages expression of this wholeness as we live, learn and worship together.  Further, the residential character of the college reflects the conviction that the goal of all meaningful learning, and of biblical education in particular, is to inform the way we live.  The residential character of the college invites students to apply their studies to the daily task of creating a community in which individuals can grow and mature together.  Students are able to cultivate these patterns of adulthood and redemptive living in the presence of role models and mentors who can help them in this process.
   Global.  Westmont is to be a college with global concerns.  For the earth and all its peoples are GodÕs good creation.  As such, they must be appropriately valued and respected.  We are called in scripture to be stewards of the earth, to be faithful caretakers of the physical creation.  We are also called to appreciate the rich diversity of human cultures Ð cultures shaped by people who bear the mark of GodÕs image in creation.  We are, then to be a community informed and enriched by thoughtful and intentional study of and interaction with cultures other than our own.  Ours is, however, a fallen world, and the earth, its peoples, and their institutions stand in need of the redemptive, reconciling word of the gospel.  We are called, therefore, not only to appreciate and preserve the creation and human cultures, but also to participate in the work of the Kingdom in response to the Great Commission to make disciples of all nations Ð to bring all creation and human institutions under the Lordship of Christ.  This task involves grappling with the full range of ways in which the fall has introduced blindness, disintegration, conflict, and injustice into the world.  Finally, the emphasis on the global nature of education is recognition that our world has increasingly become interconnected and interdependent.  To prepare people to function intelligently, effectively and for the good in a world of global politics, global economics, and global communications must be one of the aims of a Westmont education.
\subsubsection{Accreditation and Memberships}
   Westmont College is accredited by the WASC Senior Commission of Universities and Colleges (WSCUC) (985 Atlantic Ave., Suite 100, Alameda, CA, 94501; phone 510-748-9001), an institutional accrediting body recognized by U.S. Department of Education.  Its teaching programs, with specialization in elementary and secondary teaching, are accredited by the California State Board of Education. Approval has been granted for the training of veterans and war orphans.
   Westmont is a member of the Independent Colleges of Southern California; the Association of Independent California Colleges and Universities; the Christian College Consortium; the Council of Christian Colleges and Universities, Council of Independent Colleges, and National Association of Intercollegiate Athletics and the Annapolis Group of national liberal arts colleges.  Its financial policies are in accordance with those of the Evangelical Council for Financial Accountability (ECFA).
\subsection{College Organization}
\subsubsection{Board of Trustees}
a) The Westmont College Board of Trustees holds in trust the physical and financial assets of the College and has power of review and veto in relation to policy and programs.
b) The Board of Trustees is independent and self-perpetuating. Its members are elected to three-year terms, and one-third of the members are elected annually.
c) Westmont College trustees are knowledgeable of, and committed to, the central and strategic role education plays in contemporary society and the crucial contributions of Christian higher education to the church and its work. The trustees affirm the CollegeÕs Christian liberal arts philosophy and programs, and support the framework of academic governance shared with the faculty and administration.
d) It is essential that trustees be persons of genuine Christian faith, who accept without reservation the Articles of Faith, and seek to uphold the college motto on the corporate seal, Christus Primatum Tenens (ÒChrist holding PreeminenceÓ).
1.3.2 Administration
a) The Board assigns certain powers and authority to the administration of the College and to the faculty. The faculty, meeting in regular session, has authority to recommend changes in policy and programs to the administration and to the Board.
b) Organization Chart (See Appendices)
1.3.2.1 President
a) Function: To serve as chief executive officer of the College and to exercise all executive powers not explicitly reserved to the Trustees.
b) Term: Appointed annually by the Board of Trustees.
c) Accountability: Accountable directly to the Board of Trustees.
d) Salary: Remunerated as recommended and approved by the Board of Trustees.
e) Responsibilities and Duties:
1) Related to the Trustees:
(a) To make regular reports to the Board of Trustees to enable them to arrive at informed judgments.
(b) To engage the Trustees in periodic review of institutional goals, policies and programs.
(c) To recommend to the Board of Trustees for appointment members of the Executive Team and all full-time faculty members with suggested rank and compensation.
2) Related to College Management:
(a) To act on behalf of the Board of Trustees on all College matters subject to its direction.
(b) To provide vision and leadership for the faculty and administration in formulating educational and support programs.
(c) To oversee the operation and development of the institution as a whole.
(d) To supervise the Vice Presidents in the performance of their duties.
(e) To make final review and give final administrative approval of College plans, budgets, and policies and to submit appropriate items to the Board of Trustees for adoption.
3) Related to College Constituencies:
(a) To represent and interpret the College and its programs to internal and external constituencies.
(b) To seek support for the College necessary to achieve the institutionÕs mission and goals.
1.3.2.2 Executive Team
   The administration of the College is the responsibility of the President, who serves at the pleasure of the Board of Trustees as WestmontÕs chief executive officer.  The Provost is the senior administrator of the CollegeÕs educational program, with primary responsibility for the academic program, and, with leadership from the Vice President for Student life, for the out-of-classroom life of the student body.  The President, the Provost, the Vice President for Student Life, the Vice President for Finance, the Vice President for Advancement, and the Vice President for Administration and Planning constitute the Executive Team.
1.3.2.2.1 Provost
a) Function:  To coordinate the entire educational program of the College to achieve the college mission and goals; to build a strong and comprehensive liberal arts curriculum, and to ensure that all academic instruction, student development activities, and spiritual life programs, are effective means to nurture a strong and mature Christian commitment, consistent with the values and beliefs of the evangelical Christian community; to articulate the vision of Christian liberal arts in a manner that is compelling both internally and to the various constituencies of the college.
b) Term:  Appointed annually by the President with the approval of the Board of Trustees.
c) Accountability:  Directly accountable to the President.
d) Salary:  As recommended by the President and approved by the Board of Trustees.
e) Responsibilities and Duties:  The Provost is the number two administrative officer of the college and senior administrator of the educational program.  The Provost is the dean of faculty, with primary responsibility for the academic program.
1) Related to the President:
(a) To advise the President in matters of College policy in general, and educational programs in particular.
(b) To report regularly to the President on the status of all aspects of the College under his or her administrative supervision.
(c) To assume, in the absence of the President, administrative responsibility for the College.
(d) To assist the President in presenting to the Board of Trustees and its committees matters relating to educational programs and personnel.
2) Related to Supervision, Guidance, and Advisory Roles:
(a) To provide administrative supervision over:
(1) Dean of Curriculum and Educational Effectiveness 
(2) Vice Provost
(3) Dean of Admissions
(4) Director of the Library and Information Services
(5) Director of Off-Campus Programs
(6) Athletic Director
(7) Registrar
(8) Institute Directors
(9) Department Chairpersons
(10) Individual Faculty members
(b) As dean of faculty,
(1) To serve as a liaison between the Faculty and the President and his or her administrative officers particularly in academic matters.
(2) To have responsibility, with the involvement of department chairs, for the recruitment of faculty personnel.
(3) To recommend, in consultation with the Faculty Personnel Committee, appointments of faculty personnel.
(4) To implement, with the assistance of department chairs, a program for the evaluation of faculty performance and instructional improvement.
(5) To maintain a program for professional development and in-service education.
(6) To promote, with the assistance of the Faculty Budget and Salary Committee, a program for the enhancement of faculty welfare.
(7) In conjunction with the Faculty Council, to ensure the proper functioning of faculty committees within the governance pattern.
(8) To maintain, in consultation with the Faculty Council, a complete and accurate Faculty Handbook.
(c) To serve ex officio on College committees and to chair the following:
(1) Faculty meeting as a committee of the whole
(2) Academic Senate
(3) and to establish ad hoc committees as may be appropriate
3) Related to Planning and Budgeting:
(a) To provide leadership in the formulation and implementation of educational programs consistent with the goals of the College.
(b) To promote the wise adoption of effective new alternatives in educational methodologies and technologies.
(c) To formulate, in conjunction with appropriate committees, educational policies and regulations for recommendation to the Faculty, President, and Trustees.
(d) To oversee the continued development of a comprehensive undergraduate curriculum strongly built upon the classic liberal arts traditions and informed by a Christian worldview.
(e) To develop and integrate all aspects of college life including academic/instructional, student development and campus ministries, to promote the intellectual development, spiritual growth, and character formation of students.
(f) To prepare and manage the budget for all educational programs of the College.
1.3.2.2.1.1 Dean of Curriculum and Educational Effectiveness
a) Function:  Within the academic program, the Dean of Curriculum and Educational Effectiveness has primary responsibility over curriculum development, program review and assessment.
b) Term:  Appointed annually by the Provost with the approval of the President.
c) Accountability:  Directly accountable to the Provost.
d) Salary:  As recommended by the Provost and approved by the President.
e) Responsibilities and Duties:
4) Facilitate the continued development of a comprehensive undergraduate curriculum strongly built upon the classical liberal arts traditions and informed by a Christian worldview.
5) Promote adoption of effective educational methodologies.
6) Work with the Provost and Academic Senate, developing and implementing a strategic plan for curricular enhancement.
7) Provide administrative support for the Program Review and General Education Committees
8) Serve on the Academic Senate
9) Serve on the Academic Senate Review Committee
10) Retain membership in the PresidentÕs Council, communicating the budget implications of curricular planning decisions
11) Coordinate and provide assessment-related professional development for faculty and staff
12) Assist the Provost in the mentoring of department chairs
1.3.2.2.1.2 	Vice Provost
a) Function:  To coordinate and manage the academic planning and budget process, and to oversee institutional research.
b) Term:  Appointed annually by the Provost with approval of the President.
c) Accountability:  Directly accountable to the Provost.
d) Salary:  As recommended by the Provost and approved by the President
e) Responsibilities and Duties:
1) Related to academic planning and budget
(a) To work closely with academic department chairs and cost-center managers that report to the Provost in soliciting annual budget and CIP requests.
(b) To work with the Provost in developing prioritizing budget and CIP requests for the academic area.
(c) Monitor academic budgets throughout the year and approve spending requests
(d) Work with the Provost on matters related to faculty compensation.
(e) Represent the Provost on the Faculty Salary and Benefits Committee, and other committees as designated.
2) Related to institutional research
(a) Collect data from various campus sources.
(b) Maintain appropriate data from other institutions for comparative and planning purposes.
(c) Develop and maintain a College Factbook, working with college vice presidents to ensure accuracy and thoroughness.
(d) Complete and/or develop surveys and reports required by external agencies or for internal planning.
(e) Develop special reports requested or authorized by members of the Executive Team.
1.3.2.2.1.3 	Director of Off-Campus Programs
a) Function:  the Director of Off-Campus Programs has primary responsibility over the planning, development and management of WestmontÕs off-campus programs, including faculty-led semester-long and Mayterm programs.
b) Term:  Appointed annually by the Provost with the approval of the President.
c) Accountability:  Directly accountable to the Provost.
d) Salary:  As recommended by the Provost and approved by the President.
e) Responsibilities and Duties:
1) Related to Westmont Sponsored and Approved Travel Programs
(a) Chair the Off-Campus Programs Committee, overseeing those issues for which the committee has primary responsibility.
(b) Prepare and supervise expenditures of the budgets for the office, and any semester-long Westmont-sponsored travel program.
(c) Formulate, in conjunction with the appropriate committees, policy regarding risk and safety management for students involved in Off-Campus Programs.
(d) In consultation with the Provost, to make final decisions regarding cost-setting and itineraries for faculty-led Off-Campus Programs.
(e) To report to the Provost periodically on the status of all aspects of the College under his or her administrative supervision.
2) Related to Supervision, Guidance and Advisory Roles
(a) To provide supervision and guidance to the Coordinator of Off-Campus Programs.
(b) To provide supervision and guidance to the Director of WestmontÕs Urban Program.
1.3.2.2.1.4 Department Chair
a) Function:  To give leadership for and coordinate the activities of the academic department.
b) Appointment Procedures:  Department chairs are appointed by the Provost after consultation with all faculty members within the department.  Preferably department chairs should be tenured and hold the rank of associate professor or higher.
c) Term:  Appointments are normally for three-year terms with the possibility of renewal.
d) Responsibilities and Duties:
1) Provide leadership and immediate oversight of the educational program of the department
2) Supervise and coordinate ongoing program review and assessment which includes the timely submission of all required reports to ensure continued departmental health and progress.
3) Coordinate the administrative affairs of the department and serve as liaison to other departments of the College (e.g., working with the RegistrarÕs Office and Director of Advising in the assignment of advisees, facilitating new student recruitment with the Admissions Office).
4) Prepare and supervise expenditures of the department budget.
5) Develop the departmental curriculum and teaching assignments including the recruitment and orientation of part-time faculty to support the curriculum.
6) Represent the department to the Provost or Vice Provost in financial matters, course offerings, teaching load, scheduling of courses, and catalog copy.
7) Conduct department business through regular meetings.
8) Work with the library staff in the ordering of books and other instructional materials.
9) Work with the Provost in the recruitment of full-time faculty members.
10) Oversee and implement probational faculty development procedures as outlined in the Faculty Handbook (2.2.1.4).
11) Be a vehicle of communication for departmental faculty concerning faculty rights and responsibilities.
12) Assist the Provost in the mentoring of new department chairs.
13) Recruit, orient, supervise and evaluate departmental secretaries and support staff.

1.3.2.2.2 	Vice President for Administration and Planning
a) Function:  To coordinate and facilitate administrative services and oversee the offices of information technology, human resources and auxiliary services.
b) Term:  Appointed annually by the President with the approval of the Board of Trustees.
c) Accountability:  Directly accountable to the President.
d) Salary:  As recommended by the President and approved by the Board of Trustees.
e) Responsibilities and Duties:  Under the overall direction of the President, the Vice President for Administration  
1) Related to the President
(a) To work closely with the President on matters of College policy and direction, in general, and advise specifically on the status of all aspects of the College under his supervision.
(b) To serve as College liaison with legal counsel.
2) Related to Supervision, Guidance, and Advisory Roles:
(a) To provide direct administrative supervision over the following:
(1) Director of Information Technology
(2) Director of Human Resources
(3) Director of Auxiliary Services
(b) Chair the Diversity Committee as well as other College committees and task forces as determined by the President.
1.3.2.2.3 Vice President for Finance
a) Function: To manage the financial and business operations of the College in a manner which adequately supports the educational programs.
b) Term: Appointed annually by the President with the approval of the Board of Trustees.
c) Accountability: Directly accountable to the President.
d) Salary: As recommended by the President and approved by the Board of Trustees.
e) Responsibilities and Duties:
1) Related to the President:
(a) To advise the President in matters of College policy in general, and business operations in particular.
(b) To report periodically to the President on the status of all aspects of the College under this Vice PresidentÕs supervision.
(c) To formulate financial and business policies and regulations for recommendations to the President.
(d) To assist the President in presenting to the Finance Committee of the Board of Trustees matters relating to finance, business, and personnel, and to the Building and Grounds Committee of the Board of Trustees matters relating to campus facilities.
(e) To serve between the President and other administers in financial and business matters.
(f) To perform such other related duties and responsibilities as may be assigned by the President.
2) Related to Supervision, Guidance, and Advisory Roles:
(a) Fiscal Affairs
(b) Property and Facilities
(c) Management Systems
(d) Financial Aid
1.3.2.2.4 	Vice President for Advancement
a) Function:  To relate Westmont College and its objectives to its constituent publics, and relate those publics and their resources to the College and its objectives, and to secure the favorable public opinion and resources necessary to adequately support Westmont in its program of Christian higher education.
b) Term:  Appointed annually by the President with the approval of the Board of Trustees.
c) Accountability:  Directly accountable to the President.
d) Salary:  As recommended by the President and approved by the Board of Trustees.
e) Responsibilities and Duties:
1) Related to the President:
(a) To advise the President in matters of College policy, and general affairs in particular.
(b) To report periodically to the President on the status of all aspects of the College under his or her administrative supervision.
2) Related to Supervision, Guidance and Advisory Roles:
(a) Public Affairs Department
(b) Development Programs
(c) Resource Development Programs
(d) Publications
(e) News Bureau
(f) Promotional Programs
1.3.2.2.5 Vice President for Student Life
a) Function: To coordinate the student life programs in a manner conducive to maximum human growth and development.
b) Term: Appointed annually by the President with the approval of the Board of Trustees.
c) Accountability: Directly accountable to the President.
d) Salary: As recommended by the President and approved by the Board of Trustees.
e) Responsibilities and Duties:
1) Related to the President:
(a) To serve as a member of the Executive Team, providing advice on matters of College policy, especially in the area of student life.
(b) To assist the President in representing student needs and programs to the Student Life Committee of the Board of Trustees.
2) Related to Supervision, Guidance, and Advisory Roles:
(a) To provide administrative supervision over:
(1) Residence Life
(2) Counseling Services
(3) Health Services
(4) Career Development and Calling
(5) Campus Life
(6) Campus PastorÕs Office
(7) Intercultural Programs
3) Related to Student Life:
(a) To provide leadership in the formulation and implementation of policies and programs which provide for the orderly development of campus student life.
(b) To prepare and administer on-going policies for student life as developed in the Student Handbook.
(c) To provide liaison with the Faculty in efforts to develop cooperative programs to enhance the mission of the College.
(d) To provide guidance and direction to the student government (WCSA).
(e) To develop conduct guidelines in alignment with college policies.
(f) To develop and direct student social, educational, and cultural programs.
(g) To conduct assessment in Student Life

1.3.3 Faculty
   University and college faculty have a significant role in the governance of the academy. At Westmont College, the faculty, under the final authority of the Board of Trustees, exercise primary authority over instruction and curriculum and share authority for many standards and policies.  While specific administrative functions have been assigned to individual faculty committees, the authority to recommend major changes in policy or to advise the administration or Board of Trustees on central issues of College-wide concerns rests with the faculty as a whole.
1.3.3.1 Final Authority
   The Board of Trustees has final authority to approve substantive changes in institutional purposes, policies, and programs.  The Board normally exercises its responsibility on campus through the President, who is expected to provide leadership within the entire educational program of the College.
1.3.3.2 Primary Authority
a) The faculty, together with the Provost, govern curriculum and formulate academic policies through formal action in faculty meetings and the committee structure; the latter is used to implement established policy, to develop and recommend changes, and to interpret policy as necessary.
b) Curricular and academic policy changes are processed through the Academic Senate. The following items require submission by the Faculty Council for action by the full faculty for implementation:
1) the addition or deletion of a major;
2) the addition or deletion of a complete field of study, including any programs which do not fall within the direct supervision of an existing academic department or any Westmont-operated semester-long off-campus study program;
3) the addition or deletion of a graduation requirement;
4) the addition or deletion of a general education requirement;
5) an alteration in the structure of the grading policy;
6) an alteration in the institutional academic calendar (e.g., quarter, semester);
7) an alteration in the daily class schedule.
8) an addition, deletion, or substantial change to documents that describe the goals of the college-wide curricular program, including those drafted for assessment purposes (e.g., Program Review at Westmont College: Mission-Driven, Meaningful, & Manageable; educational vision documents including ÒPhilosophy of EducationÓ and ÒWhat Do We Want for Our Graduates?Ó). 
1.3.3.3 Shared Authority
a) The faculty share with administrative officers authority for developing standards and policies for the recruitment and admission of students. In addition, they are authorized and expected to be involved in the formulation of other College policies including student life, public relations, institutional budgeting, and long-range planning.
b) Formal relationships with the President and other administrators are to be effected through the Provost who has authority for the instructional program of the College.  The Provost is also the executive officer of the faculty, representing their will and interests consistent with the parameters of the College policies and mission.  As Chair of the Faculty, the Provost represents faculty interests within the Executive Team and to the Board of Trustees in consultation with the Chair of the Faculty Council.  As Vice President of the College, the Provost represents administrative interests and decisions to the faculty.
1.4 Committee Structure and Responsibilities
1.4.1 Committees of the Board of Trustees
   Committees are delegated certain responsibilities for more effective handling of the BoardÕs work.  Committees study proposals, hear constituents before recommendations are formed, receive reports, and formulate recommendations to present to the full Board.
1.4.1.1 Executive Committee
   The Executive Committee is composed of elected Board officers plus the chair of each regular committee.  The Executive Committee may transact business for the Board between regular Board meetings.  All such transactions are subject to the approval of the full Board at their next meeting.
1.4.1.2 Academic Committee
a) Keep abreast of curricular and program changes and be aware of, and able to explain, the rationale behind the changes.
b) Consult with and advise the Provost concerning academic matters.
c) Understand the policies relating to promotion, tenure, and sabbatical leaves and evaluate College recommendations concerning the granting of promotion, tenure, and sabbaticals in order to make recommendations to the Board of Trustees concerning action on these matters.
d) Review materials of faculty candidates recommended by the Provost and faculty with the approval of the President and resolve any questions before recommending candidates for Board approval.  When the timing of an offer of employment is crucial and the full Board cannot act, the Academic Committee may act on their behalf to authorize a contract although full Board action is required at their next regular meeting relative to employment beyond the contract year.
e) Review candidates for professor emeritus/emerita status or honorary degrees and make recommendations to the Board.
f) Approve policies that provide grievance procedures for faculty appeal of any perceived injustices.
1.4.1.3 Student Life Committee
a) Review all student life programs and plans and make any appropriate recommendations to the Board.
b) Review student life expectations policies and conduct procedures and be able to defend and explain them.
c) Consult with and advise the Vice President for Student Life concerning student life matters.
d) Approve policies that provide adequate services for studentsÕ health, safety, and personal growth.
e) Establish communication with students and approve policies for student appeal of any perceived injustices.
1.4.1.4 Membership Committee
a) Serve as Nominating Committee for Board officers.
b) Serve as Nominating Committee for Board members, both for new and reelected.
c) Assess the BoardÕs needs for new members and maintain a roster of prospective members.
d) Ensure that new trustees are properly enlisted and oriented to the institution and to their role as trustees.
e) Recommend any change in membership policies to the Board with respect to Board composition, length of terms, and number of successive terms.
f) Monitor the Board activities of members and ensure that all members have the opportunity to be actively involved as Trustees.
g) Oversee assessment of trustee performance both individual and corporate.
1.4.1.5 Finance Committee
a) Review annual operating budget and recommend action to the Board or Executive Committee.
b) Review and approve revisions to the operating budget which do not result in a deficit.
c) Review and authorize allocations of the contingency budget.
d) Review and authorize interfund transfers and loans.
e) Monitor monthly revenues and expenditures, year-end reports, and the annual financial audit and management letter and recommendations to the Board or Executive Committee.
f) Review and approve employee housing assistance transactions.
g) Review and accept deferred gift agreements.
h) Evaluate and authorize sales of real property and other property received as gifts.
i) Evaluate, engage, and direct investment advisors.
j) Evaluate and approve short-term borrowing.
k) Review and approve terms of annual participation in student loan programs.
l) Review and advise concerning long-range fiscal plans.
m) Consult with and advise the Vice President for Finance concerning fiscal matters.
1.4.1.6 Development Committee
a) Develop an understanding of the role fund raising plays in the life of the College.
b) Educate Board members of their role in acquiring gifts.
c) Review the resource development program to ensure that there are sufficient resources to meet operating costs and maintain institutional integrity as to facilities and programs.
d) Secure financial commitment on the part of individual Board members to personal giving and influencing other persons and organizations to support Westmont College.
e) Recommend institutional policies and guidelines for fund raising.
f) Identify, evaluate, cultivate, and solicit major gift prospects.
g) Monitor the quality of communications with various constituencies.
h) Consult with and advise the Vice President for Development concerning development matters.
1.4.1.7 Buildings and Grounds Committee
a) Review and recommend approval of a Master Plan for the physical campus which includes both present and anticipated needs.
b) Review reports on physical plant utilization, maintenance, and upkeep.
c) Review requests for new construction or remodeling and approve architects, plans, and building schedules.
1.4.1.8 Personnel Committee
a) Oversee the search and screening process and make the final recommendation in selection of the College President, generally working through a representative committee including faculty, students, and other constituents.
b) Periodically review the work of the College President with special sensitivity to the fragile status of college presidents and the need to offer strong Board support as well as clearly-stated performance standards.
c) Periodically review with the President the work of the members of his or her staff.
1.4.1.9 Planning Committee
a) Review all planning documents produced by the College, suggest planning needs if these are not being addressed by campus, and recommend planning policies to the Board.
b) Provide an on-going strategic planning process for the Board of Trustees.
c) Review planning assumptions, statements of institutional mission and goals, faculty and financial needs, and evaluate the success of planning procedures.
1.4.1.10 Diversity Committee
a) Advise the Board Chair on the leadership of the President and the Executive Team in implementing the CollegeÕs long-range priority on diversity.
b) Monitor and report to the Board on campus efforts to achieve the CollegeÕs diversity goals.
c) Educate the Board about the issues, challenges, and opportunities related to providing our students with a more multicultural education.
d) Recommend ways in which the Board and its members can assist the College in reaching its diversity goals.
1.4.1.11 Historic Preservation Committee
a) Ensure that the traditions, culture and historical significance of the College are preserved and where appropriate, improved and enhanced.
b) Promote these standards and expectations to College committees that plan and design physical improvements, capital construction and capital maintenance.
c) Consult with and advise the administration about appropriate historic preservation policies and about promoting public awareness and celebration of the CollegeÕs traditions and heritage.
d) Monitor the work of the College Archivist to collect and store materials of historic significance.
1.4.2 Committees of College Administration
1.4.2.1 Strategic Planning Committee
a)	Membership:
1) President
2) Chair, Board of Trustees
3) Chair, Trustee Planning Committee
4) Member, Trustee Planning Committee
5) Provost
6) Associate Provost for Planning and Research
7) Vice President for Administration and Planning
8) Vice President for Advancement and CIO
9) Vice President for Finance
10) Vice President for Student Life 
11) Two Vice Chairs of the Faculty
12) Two Academic Senators, specifically the Chair of the Senate and one additional senator selected with an eye toward divisional representation and continuity
13) Dean of Admissions
14) Dean of Curriculum and Educational Effectiveness
15) Dean for Student Engagement
16) Dean of Students
17) Director of Library and Information Services
18) WCSA representative
b)  Officers
The President is the convener of the Strategic Planning Committee.  Committee meetings are facilitated by a strategic planning consultant.  The President and the Chair of the Trustee Planning Committee serve as primary liaisons with the Board of Trustees.
c) Responsibilities
1) Represent and interact with college constituencies to:
* Assess WestmontÕs strengths, weaknesses, opportunities and threats
* Identify strategic priorities for possible inclusion on a three-year strategic map
2) Recommend a three-year strategic map, and annual strategic priorities, to the Board of Trustees.
3) Regularly review progress on implementing strategic priorities
4) When appropriate, make needed adjustments to the process of implementing strategic priorities; and when appropriate, recommend strategic map adjustments to the Board of Trustees.

1.4.3 Committees of the Faculty
   See also 2.4.3.1 Participation in Faculty Governance.
1.4.3.1 Faculty Meetings
a) Membership:
1) The privilege of voice with vote at faculty meetings is limited to the following:
(a) faculty with rank of Instructor, Assistant, Associate, and Full Professor with one-half time or more excluding non-credit courses;
(b) certain persons with faculty status whose duties are closely allied with the instructional faculty:  Director of the Library and Information Services, Associate Director of the Library, Director of Admissions, Registrar, and full-time personnel supervising activities for which students receive credit toward graduation (e.g., varsity coaches);
2) The privilege of voice without vote at faculty meetings is extended to the following:
(a) persons with faculty status but not rank;
(b) faculty with rank who teach less than one-half time, administrators with faculty rank, visiting faculty, part-time faculty;
(c) members of the Executive Team.
3) The privilege of observer status without voice or vote may be extended to selected staff who have extensive involvement in student life.  The W.C.S.A President and the Horizon Editor may attend faculty meetings without voice or vote; the faculty reserves the right to meet without student observers.
4) In all cases the Faculty Personnel Committee shall determine disputes involving voting rights.
b) Officers
1) The Provost shall serve as Chair of the Faculty.
2) The Provost will appoint annually a secretary for faculty meetings.
3) The Vice-Chair of the Faculty will preside for formal actions at all faculty meetings, and will chair faculty meetings in the absence of the Provost.  In the absence of the Vice-Chair these duties will be assumed by the member of the Faculty Council in attendance who has the longest seniority at the College.
c) Procedures
1) Regularly scheduled meetings will be held at least once per month during the school year.
2) Special meetings may be called by the President, Provost, the Faculty Council, or by 15 percent of the full-time faculty upon petition to the Faculty Council.
3) A quorum for the conduct of business will be one-half plus one of the current voting membership of the faculty.
4) Formal action by the faculty requires a majority vote of the eligible faculty present, except as otherwise provided in the Faculty Handbook (e.g., 1.5.4).
5) Faculty who must be absent from a faculty meeting may arrange with the chair of the Faculty Council for an absentee ballot to be counted on their behalf provided that:
(a) The vote is related to an election.
(b) The ballot has been published prior to the meeting and has not been changed.
6) In absentia voting is not allowed on action items other than elections.
7) The agenda will be determined by the Faculty Council.  Items requiring faculty action will be submitted to the Faculty Council and distributed to faculty members at least one week in advance of the meeting.
8) The Professional Development Committee will arrange programs relating to faculty growth for certain faculty meetings.
9) Meetings will be governed by RobertÕs Rules of Order, Newly Revised, 9th ed (1990).
10) At least once per semester the full-time teaching faculty with the rank of Instructor, Assistant, Associate, and Full Professor, will caucus in executive session without the attendance of administrators.  Additionally, executive sessions will be called by the Faculty Council at the request of one-third of the teaching faculty.  In executive sessions of the faculty the Vice-Chair, or his or her designee, will preside.
d) Responsibilities
1) To articulate and promote institutional objectives.
2) To establish and govern the academic life and curricular structure of the College.
3) To implement, through instruction, research, committee work, and counsel, instructional programs.
4) To establish and maintain academic standards, criteria for admission and retention of students, degree requirements, and policies relating to financial aid.
5) To recommend to the Board of Trustees for approval all candidates for degrees.
6) To establish and to elect faculty members to standing and ad hoc committees for the governance of the College.
7) To communicate through the Provost and the Faculty Council, the opinions and counsel of the Faculty to the Board of Trustees, the administration, and committees on any issue affecting faculty, student life and other aspects of the College.
8) To foster the professional development and personal well-being of individual faculty members and collegial relationships.
1.4.3.2 	Nomination and Election to Faculty Committees
   Elections to Faculty committees will be held during the second semester of each academic year.
1.4.3.2.1 Election to Faculty Council
1.4.3.2.1.1 Nomination
   No later than three days before the nomination, Faculty Council will present to the Faculty a nominating ballot containing the names of all faculty eligible to be nominated, and indicating which are tenured.  The Faculty will nominate their individual choices of two candidates for each tenured vacancy, and two for each additional vacancy.
1.4.3.2.1.2 Election
a) No later than three days before the election, Faculty Council will present to the Faculty an election ballot, which will include:
1) the names of the tenured faculty who have received the most nominations:  two for each tenured vacancy, or as many more as may be tied with them;
2) the names of the remaining faculty, tenured or not, who have received the most nominations:  two for each additional vacancy, and as many more as may be tied with them.
b) Faculty will cast one vote for each vacancy.
c) Any tenured vacancy will be filled by the tenured faculty who have received the most votes; any additional vacancy will be filled by the remaining faculty, tenured or not, who have received the most votes.
d) In case of a tie:
1) on a ballot of one vacancy only, the Vice-Chair will cast the deciding vote;
2) on a ballot for two or more vacancies, a run-off election will decide the outcome.
1.4.3.2.2 Election to the Faculty Personnel Committee and Academic Senate
   No later than three days before the election, Faculty Council will present to the Faculty a ballot containing two nominees for each seat open on the Faculty Personnel Committee and Academic Senate.  Faculty will cast one vote for each vacancy.
1.4.3.2.3 Election to Other Committees
   After the election of faculty to the Faculty Council, Academic Senate, and the Faculty Personnel Committee, and the Academic Senate, Faculty Council will present to the Faculty, no later than three days before the election to other committees, a slate of single nominees for all the elected seats open on the remaining committees, with the possibility of additional nominations from the floor, upon the prior consent of the nominee(s).  When eligible candidates are limited, Faculty Council can nominate a faculty member from one division to serve in a position designated for a different division, making this substitution clear on the nomination ballot.
1.4.3.2.4 Election of the Vice-Chair of the Faculty
   Once the new member(s) of Faculty Council shall have been elected, Faculty Council will present to the Faculty, no later than three days before the election, a slate containing the names of all tenured members of the Faculty Council as constituted for the academic year following.  Faculty will cast one vote; the candidate who secures a plurality of votes will be the Vice-Chair; the Faculty will break any tie by a run-off election.
1.4.3.2.5 Conditions for Nomination and Election to Faculty Committees
   Each faculty member has the opportunity:
a) To express preferences for committee memberships as well as recommend specific individuals for committee vacancies;
b) To nominate eligible faculty from the floor, their prior consent having been secured, for open committee seats (except for Faculty Council);
c) To limit service to one committee at a time, while the available faculty suffice; or failing that, to limit service to one standing committee at a time;
d) To limit service to one committee per year (assuming a sufficient number of faculty are available);
e) To serve voluntarily on more than one committee of whatever status;
f) To be exempt from service on any committee for one year following completion of a full term on either Faculty Council or Faculty Personnel Committee.
1.4.3.2.6 Officers of Faculty Committees
a) Chairs, secretaries, and other officers as may be necessary will be designated in accordance with the particular provisions of the Faculty Handbook for each committee.
b) Chairs will vote only to break a tie.
1.4.3.2.7 Ex-Officio Members
   The President, Provost, Vice Provost and Associate Dean of the Faculty and Dean of Curriculum and Educational Effectiveness are ex-officio members without vote on all faculty committees, except as provided for otherwise.
1.4.3.2.8 Student Members
   Student members serve on faculty committees as established by the specific provisions for student membership for the committee to which a student is appointed. Appointments are made by the W.C.S.A. student council.  Student copies of committee minutes are retained in the office of the committee chair where student committee members have access to them.
1.4.3.2.9 Replacement of Faculty Committee Members

   In the case of an inactive faculty committee member, the committee chair will report to the Provost any committee member who does not regularly participate.  If the matter is not resolved, the Provost will ask the Faculty Council to replace the member for the remainder of the elected term.
   Faculty Council is responsible for filling temporary committee vacancies (e.g. sabbaticals, committee release, abroad programs and medical or professional leaves).  Replacements may be provided through the normal election process, by a special election, or by appointment, at the discretion of Faculty Council. 
1.4.3.3 Curricular
   
1.4.3.3.1 Academic Senate
a) Membership:
1) Eight elected faculty, two from each division plus two additional faculty at large
(a) Only faculty who have served at the college for a minimum of 6 years are eligible for election to Academic Senate.
(b) For divisional representatives, only faculty who currently serve or have previously served as department chair are eligible for election to Academic Senate.
(c) Faculty are typically elected to 3 year terms, such that a maximum of one new senator is elected each year within each division.
2) Provost and Dean of Curriculum and Educational Effectiveness (both ex officio without vote)
3) Registrar
4) W.C.S.A. President (or representative) (voice without vote)

b) Observers:  Additional ex officio personnel including the Director of Off-Campus Programs, the Associate Provost for Planning and Research, and the Director of the Library and Information Services are welcome as regular participants in Academic Senate meetings.  Academic Senate meetings are also open to other faculty.  All such participants have voice without vote. 
c) Officers:  
1) The Provost shall serve as chair.
2) Each spring, the voting members of the following yearÕs senate shall elect a vice-chair from among its faculty members.  The vice-chair shall meet regularly with the Provost to set agendas and will chair Academic Senate in the ProvostÕs absence.
3) The Registrar shall serve as secretary.
d) Responsibilities:
1) To steward and maintain the integrity of the academic program, including curriculum, general education, majors, support programs, and off-campus and special programs.
(a) To engage in academic planning, to assist the Provost with the crafting and implementation of the strategic plan for the academic programs and to recommend changes in academic staffing resources.
(b) To review, approve, and instate changes in academic programs, and to submit these changes through Faculty Council for action by the full faculty in cases where said changes are designated in section 1.3.3.2 b as requiring a vote of the full faculty.
(c) To read and discuss annual reports from the Program Review, General Education, Academic Resources, and Off-Campus Programs committees, and to ensure that academic evaluation effectively takes place
(d) To accept and consider proposals from students, faculty, staff, or other constituencies and respond with recommendations
(e) To recommend changes in admissions policies and practices
(f) To recommend changes in the area of learning resources, including the library and educational media
(g) To initiate changes in grading practices and student evaluations
2) To meet regularly and to distribute minutes to faculty in a timely fashion.
3) To report to the full faculty any changes within existing curricular structures at the next regularly scheduled faculty meeting following such a decision. A representative of the Academic Senate shall report to the full faculty any changes within existing curricular structures at the next regularly scheduled faculty meeting following such a decision.
e) Means of Appeal:  Appeal of actions by the Academic Senate shall be brought to the full faculty upon the request of seven persons with faculty status who have both voice and vote.
f) Subcommittee Assignment:  Senators, excluding the chair, are appointed annually to 1-year terms on the subcommittees of the Academic Senate (Review Committee and Academic Resources Committee; three senators each) and the Strategic Planning Committee (two senators).  Appointments are made by the Provost in consultation with the Academic Senate Vice-Chair elect.  These appointments will occur in the spring term following Academic Senate elections, with the aim of maximizing continuity on the committee, divisional representation, and fit of expertise and experience.
1.4.3.3.2 Academic Senate:  Review Committee
a) Membership:
4) Provost or representative
5) Registrar
6) Three senators, each appointed by the Provost and Academic Senate Vice Chair to renewable one year terms.
7) One student appointed by W.C.S.A. (voice without vote)
   All Academic Senate members shall have voice without vote at every meeting.
b) Officers:
   The chair shall be elected annually by the members of the Review Committee and the Registrar shall serve as secretary.
c) Responsibilities:
1) Acts for the Academic Senate on student petition appeals, honors and alternative-major proposals.
2) Acts upon minor curricular changes that do not involve, for example, the addition or deletion of a major, a change in units required for a major, or any change that signals a major shift in emphasis of the College.
3) Appeals for action related to student petitions, honors, and alternative-major proposals shall be to the Provost; appeals for action to minor curricular changes shall be to the Academic Senate.
1.4.3.3.3 Academic Senate:  Academic Resources Committee
   The Academic Resources Committee (ARC) is concerned with recommending and implementing resources suitable for faculty and student use in the classroom and in the library.  Additionally, the ARC makes recommendations to the Academic Senate regarding policy on use of instructional resources.
a) Membership: 
1) Provost or representative
2) Director of Library & Information Services (ex-officio)
3) Director of Information Technology (ex-officio)
4) Three senators, each appointed annually by the Provost and Academic Senate chair to renewable 1 year terms.
5) One additional full-time faculty member, selected by the Provost in consultation with Faculty Council, to serve a three-year term as committee chair
6) One student appointed by W.C.S.A.
7) On an ad hoc basis, such members of the library faculty or staff who may be required for a specific purpose, who will have voice without vote
   All Academic Senate members shall have voice without vote at every meeting.
b) Reports to:
1) In matters regarding policy that affect instruction, the ARC reports to the Academic Senate.
2) In matters regarding acquisition and implementation of technology, the ARC makes recommendations to the Director of Information Technology and reports to the Provost.
c) Purposes:
1) To assist the Director of Information Technology in determining what new technologies and products are appropriate for Westmont College, and which should be evaluated for instructional use.
2) To propose the acquisition and implementation of new instructional resources.
3) To work with the Director of Information Technology in educating and enabling faculty in the use of new and existing resources. 
4) To recommend policy relating to technology issues that affect faculty and students.
5) To work with the Director of Library & Information Services in library-related matters, including collection development, building-related matters, staffing, programs, and budgetary issues.
6) To refer items (as appropriate) to the Senate for their recommendation to the faculty.
1.4.3.4 Administrative
1.4.3.4.1 Faculty Council
a) Membership:
7) Five full-time faculty members elected at large by the faculty; at least three must be tenured; the remaining members must either be tenured, on tenure-track or multi-year contract, or library faculty having voice and vote.
(a) Term of office is for three years; no member may be re-elected to the Faculty Council during the year following expiration of term.
8) Any faculty member may attend meetings of the Faculty Council (with voice but without vote, except those closed by majority vote of the Council).
b) Officers:
1) The entire faculty will elect, from among the tenured members of the Council, a Vice-Chair of the Faculty who will preside at meetings of the Council.  The Vice-Chair will also preside over meetings of the full faculty in the absence of the Provost.  Term of office of the Vice-Chair is one year.
2) The Council will elect the secretary from among its membership.
c) Responsibilities:
1) To meet regularly and to distribute minutes to the Faculty in a timely fashion.
2) To provide a forum where the administration can seek, or faculty can volunteer, faculty advice or opinion on matters not on the agenda of other faculty committees.
3) To provide a forum where faculty grievances can be aired and brought to the attention of the administration.
4) To recommend to the Faculty, and to other appropriate bodies, changes in policy (including the triennial review of the Faculty Handbook provided by Section 1.5), and to present for the FacultyÕs approval all proposed major changes in policy.
5) To establish the schedule of regular faculty meetings as stipulated in section 1.4.3.1 1), and to call special faculty meetings as necessary.
6) To determine the agenda of all faculty meetings.
7) To plan faculty retreats.
8) To oversee the faculty standing committee nomination and election process (as stipulated in section 1.4.3.2)  and to appoint replacements as necessary.
1.4.3.4.2   Faculty Budget and Salary Committee
a) Membership:
1) Provost (or representative)
2) Three tenured full professors, one elected annually to a three-year term
3) Two faculty not full professors at time of appointment, one elected annually to a two-year term
b) Officers:
   The full professor serving in the second year of the three-year term shall be the committee chair.  The chair and the most recent past chair shall be the faculty representatives on the PresidentÕs Advisory Council.
c) Responsibilities:
1) To monitor the amount budgeted for the academic program relative to all other areas of the College.
2) To maintain comparisons of the CollegeÕs salary and benefits schedule with those of comparable institutions.
3) To provide a forum for faculty input to Human Resources when changes in Faculty and Staff benefits are proposed; and to initiate changes in benefits at appropriate times.
4) To maintain communication with the faculty regarding salary and benefits.
1.4.3.4.3 Communications Board
a) Membership:
1) Provost (or representative)
2) Vice President for Student Life (or representative)
3) One faculty elected bi-annually to a two-year term
4) One student
5) Editors of student publications
6) Faculty advisors to the student publications
b) Officers:
   The chair will be elected from the faculty and staff not in advisorship and shall appoint a secretary.
c) Responsibilities:
1) To serve as a selection committee for editors.
2) To select advisors.
3) To establish guidelines for advisors to the publications.
4) To receive periodic reports of progress toward publication.
5) To receive and act upon grievances related to the publications from any member of the community.
6) To serve as a review board on questions relating to editorial policy and content.
7) To conduct dismissal proceedings of editors who act in consistent disregard of the stated goals of the College.
1.4.3.4.4 Computer and Technology Advisory Committee
a) Membership:
1) Academic ÐThe chair of the Academic Resources Committee of the Senate and one faculty member elected to a three year term, recommended by Faculty Council and approved by the Faculty.  The Director of the Library and Information Services or designated representative serves as an ex officio member.
2) Administrative Ð Registrar, Vice President for Finance, Director of Admissions, one representative from Advancement and Student Life
3) Two students appointed by WCSA
4) Chief Information Officer or designated representative
5) Ex Officio Ð As needed
b) Officers:
   The Committee shall be chaired by the Director of Information Technology or designated representative.
c) Responsibilities:
1) To review and recommend plans and priorities for the College in the area of information technology.
2) To review and recommend specific policies regarding computer and information technology.
3) To review and recommend staffing and budget.
4) To review and recommend new initiatives and innovations.
5) To review progress toward established goals and objectives.
1.4.3.4.5 Athletic Committee
a) Membership:
1) Provost (or representative)
2) Vice President for Student Life (or representative)
3) Faculty Athletic Representative, who is appointed by the President and must have faculty status.
4) Athletic Director
5) Senior Woman Administrator
6) Two faculty members, one elected annually to a two-year term
b) Officers:
   The Faculty Athletic Representative will serve as chair; the chair shall appoint the secretary. 
c) Responsibilities:
1) To serve as a liaison between the academic and athletic programs, hearing faculty questions and concerns, identifying issues needing to be addressed, and enhancing communication.
2) To provide an annual report (and, if needed, other periodic reports) to the faculty about the athletic program (including information on graduation rates and GPAs for athletes, changes in conference policy and procedures, and other items of relevance to the academic program).
3) To review guidelines regarding scheduling (e.g. length of season, number of games, amount of practice time, policies on class attendance and make-ups, postseason participation).  The Athletic Director shall oversee implementation of these guidelines for each season of play.
4) To approve eligibility regulations for Westmont athletes with respect to those set forth by any conferences and national affiliations to which Westmont may belong.  To review reports presented by the Faculty Athletic Representative and the Athletic Director on academic progress of student athletes.  These reports normally would be presented in early fall.
5) To review proposals for conference and national affiliations or for the addition and deletion of athletic programs.  To make recommendations on these matters to the Provost.
6) To review plans for the collegeÕs compliance with the Title IX mandates and to make recommendations for ensuring gender equity.
7) To assist in the recruitment and hiring of any full-time coaching staff.
1.4.3.4.6 Off-Campus Programs Committee 
a) Membership:
1) Three faculty members with differing disciplinary backgrounds, at least one of whom has significant experience with a Westmont off-campus program, one elected annually to a three-year term.
2)  Registrar (ex officio)
3) Director of Off-Campus Programs
4) Assistant Director of Global Education (voice without vote)
b) Officers:
   The Director of Off-Campus Programs shall serve as chair person.
c) Responsibilities: The Off-Campus Programs committee provides academic oversight to off-campus programs, including all Westmont-operated and Westmont-approved programs for which credit is granted.
1) The committee provides governance for off-campus programs analogous to that normally found in an academic department, including but not limited to the evaluation of programs, determination of qualifications for and recommendations to the Provost regarding staffing of Westmont-operated programs, and oversight of curriculum.
2) The committee is not generally involved in the logistical arrangements for the various off-campus programs.
3) The committee will submit to the Academic Senate any proposed policy or curriculum changes that might require approval by Academic Senate or the full faculty, and to receive from Academic Senate requests for action and information relating to matters of the academic program appropriate to its purpose and function.
4) The committee will submit to academic senate an annual 1- to 2- page summary of its work related to policy, program or curricular changes to the collegeÕs off-campus programs. 
1.4.3.4.7 Program Review Committee 
a) Membership:
1) Dean of Curriculum and Educational Effectiveness
2) Director of Institutional Research
3) Librarian with Faculty Status
4) Vice President for Student Life (or representative) 
5) Three faculty with differing disciplinary backgrounds, one elected annually to a two-year term
6) Accredidation Liaison Officer (ex officio)
b) Officers:
1) Chair, elected each year from among the committee members.
2) Secretary selected by committee.
c) Responsibilities:
The Program Review Committee oversees program review in academic and co-curricular departments and programs, and develops an institutional plan for college-wide program review.  It establishes policies and procedures regarding program review and assessment, and maintains these in a document on file in the ProvostÕs Office.  Specific responsibilities of the Program Review Committee include:.
1) In collaboration with the Dean of Curriculum and Educational Effectiveness
(a) to work with departments as needed in the development of their plan for program review;
(b) assist departments with the generation and completion of the annual assessment reports and multi-year program review reports;
(c) to review the annual reports and provide timely feedback to departments;
(d) to review the results of the multi-year review and provide timely feedback for the department to consider for its meeting with the Provost and the Dean of Curriculum and Educational Effectiveness;
(e) to provide the department with comments or suggestions to assist them in their preparation for the next review cycle;
(f) to encourage a campus conversation that establishes the value of a college-wide program review and addresses concerns as appropriate.
2) To work with and advise the Director of Institutional Research in establishing goals, formats and priorities for the collection and reporting of student data.
3) To work with and advise the Dean of Curriculum and Educational Effectiveness in organizing the data, planning a schedule for Program Review and presenting the data to the campus communities and accrediting organizations.
4) To submit to the Academic Senate any proposed policy or curriculum changes that might require approval by Academic Senate or the full faculty, and to receive from Academic Senate requests for action and information relating to matters of the academic program appropriate to its purpose and function.
5) To report regularly to Academic Senate through the Dean of Curriculum and Educational Effectiveness, and to submit an annual 1- to 2-page summary of its work related to review of the collegeÕs academic program, including a summary of its review of departmental multi-year reports submitted that year.
1.4.3.4.8 General Education Committee 
a) Membership:
1) Dean of Curriculum and Educational Effectiveness.
2) Director of the Library or librarian with faculty status, appointed by the Director of the Library in consultation with Faculty Council.
3) Three tenured faculty, one from each division, one elected annually to a three-year term.
4) Registrar
b) Officers:
1) Chair, elected each year from among the committee members.
2) Secretary, selected by committee.
c) Responsibilities:
1) To oversee the implementation of the general education program of the College according to the criteria approved by the faculty.  Policies, procedures, and criteria for course inclusion in the various general education categories are maintained in documents on file in the ProvostÕs office.
2) To review and approve new course proposals to ensure that they adhere to the general education criteria approved by the faculty. 
3) To facilitate regular program review and assessment of the general education program, as well as course content and methodology of courses previously approved as fulfilling general education criteria.
4) To conduct periodic syllabus review.
5) To recommend change in the general education program to the Academic Senate, with proposals that would significantly alter any general education requirement being forwarded to the full faculty for action.
6) To submit to the Academic Senate any other proposed policy or curriculum changes that might require approval by Academic Senate or the full faculty, and to receive from Academic Senate requests for action and information relating to matters of the academic program appropriate to its purpose and function
7) To report regularly to Academic Senate through the Dean of Curriculum and Educational Effectiveness, and to submit an annual 1- to 2-page summary of its work related to review of the collegeÕs academic program.
1.4.3.5 Institutional
1.4.3.5.1 Admissions and Retention Committee
a) Membership:
1) Vice President for Student Life (or representative)
2) Registrar
3) Dean of Admissions
4) Associate Director of Admissions
5) Coordinator of Academic Advising
6) Three faculty members, one elected annually to a three-year term.  At least one faculty member of the committee or a substitute faculty member must be present at all meetings.
7) Two students, one man and one woman, appointed by the W.C.S.A.
b) Officers:
   The Dean of Admissions shall serve as chair; the chair shall appoint the secretary.  The chair votes only to break a tie.
c) Responsibilities:
1) To review annually and approve policies relative to admissions consistent with the character and objectives of the institution.
2) To serve as a board of interview for applicants when necessary.
3) To review and render decisions on applicants who do not meet the established criteria for admission.
4) To review and interview, if requested, applications from students not admitted and are requesting an appeal of the decision.
5) To review and interview, if requested, students who have been placed on academic suspension and are requesting an appeal of the decision.
6) To communicate with Academic Senate regarding the curricular interests of students.
7) To communicate and work with the Diversity Committee on matters relating to diversity, admissions and retention.
1.4.3.5.2 Student Life Committee
a) Officers:
1) Provost (or representative)
2) Vice President for Student Life (or representative)
3) Two Student Life staff members
4) Three faculty members, one elected annually to a three-year term
5) Two students selected by the W.C.S.A. President in consultation with the Vice President for Student Life.
b) Officers:
   The chair shall be elected from among the faculty or staff members and shall appoint the secretary.
c) Responsibilities:
1) To promote discussion among administration, faculty, and students about issues related to the Student Life Department such as:
(a) Student development philosophy
(b) Student life programming
(c) Student services
(d) Concerns related to Student Life department
(e) Evaluation of the student life program
2) To initiate discussion among administration, faculty, and students on matters relating to the spiritual climate of the college
3) To review periodically judicial procedures and conduct sanctions to ensure a studentÕs access to a fair and impartial hearing and to a consistent application of sanctions.
1.4.3.5.3 Diversity Committee
a) Membership:
1) Provost
2) Vice President for Student Life 
3) Vice President for Administration and Planning
4) Three faculty, one elected annually to a three-year term
5) Special Assistant to the Provost for Diversity Initiatives
6) Dean of Admissions
7) Director of Intercultural Programs

b) Officers:
   The chairperson will be appointed by the President in consultation with the PresidentÕs Executive team and Diversity Committee members; the committee will elect a secretary.
c) Responsibilities:
1) To establish links with others that are working to address diversity-related issues at WestmontÑfor example, Trustee Diversity Committee, Executive Team, Human Resources, Residence Life, Intercultural Programs, Diversity Recruitment Specialists, Off Campus Programs. 
2) To seek to build community-wide awareness and ownership of diversity-related issues and of appropriate ways to address them.
3) To summarize the various challenges and opportunities related to diversity that currently face the College. 
4) To enlist additional faculty, administrators and staff to work with the Committee on special initiatives.
5) To recommend priorities and propose implementation plans to the PresidentÕs Executive Team.
6) To make annual reports on the CommitteeÕs work to the Faculty, the Executive Team and the Trustee Diversity Committee.
1.4.3.5.4 Faculty Personnel Committee
a) Membership:
   All committee members, except the Provost, are eligible to vote.
1) Provost
2) Three tenured full professors and two tenured associate professors, each elected for a two-year term.  The three full professors must come from the three academic divisions of the college, and the associate professors elected must not be scheduled for a review during their term of service.
3) At most three additional full professors, each elected to a one-year term, depending on the anticipated number of faculty being reviewed for a given year.
4) One previously reviewed probationary faculty member to be elected annually for a one-year term and must not be scheduled for a review during their term of service.
b) Officers:
1) Chair, a full professor, to be elected from among the voting committee members.
2) Vice chair, a full professor, to be elected from among the voting committee members, and to serve as chair when the chair is absent, or when the committee is reviewing a faculty member whose review has been assigned to the chair.
3) Committee secretary, to be elected from among the voting committee members.  The officers will have both voice and vote, as do the other faculty members of the committee.
c) Responsibilities:
1) To interview and recommend to the Provost candidates for appointment to the faculty.
2) To make recommendations to the Provost concerning initial status, rank, tenure, termination, and emeritus/emerita status.
3) To solicit faculty opinion in cases of initial appointment which also involve promotion to associate or full professor, or granting of tenure, prior to a recommendation to the Provost.
4) To review cases when progressive discipline (See Section 2.2.4) is invoked and to recommend the nature of sanctions if applicable.
1.4.3.5.5 Professional Development Committee
a) Membership:
1) Provost (or representative)
2) Three faculty members, one from each division, one elected annually to a three year term, at least one of whom must hold the rank of full professor.
b) Officers:
   The chair will be elected from among the faculty members and shall appoint the secretary.
c) Responsibilities:
   To promote the professional development of faculty instruction, research, creative activity, and publication.
1) To administer faculty grants in support of scholarly activities and curricular development.
2) To sponsor activities on campus to stimulate excellence in teaching, including the supervision of a program of student evaluation of courses, special seminars and speakers, etc.
3) To encourage the attendance and participation in professional meetings.
4) To recognize excellence in publication and creative activity.
5) To review the literature pertaining to the evaluation and improvement of instruction and to investigate the practice of other colleges in these areas.
6) To serve as consultants to other faculty in matters of professional improvements.
7) To advise the Provost on a program of in-service training for new and experienced faculty.
8) To make recommendations to the Provost concerning leaves and sabbaticals.
1.4.3.6 Policy of Recusal
   Concerning the participation of faculty on various deliberating committees:  A Faculty member must recuse himself/herself if:
a) The faculty member is on a committee that is addressing an issue directly affecting that faculty member or a close family member of that faculty member.  Examples: A faculty member being reviewed for tenure or promotion who is on the Personnel Committee; a faculty member applying for a faculty development grant or sabbatical who is on the Professional Development Committee; a faculty member involved in a preliminary hearing (see 2.2.4.4.2) who is a member of the Faculty Council; a faculty member on the Off-Campus Programs Committee whose spouse has submitted a proposal for consideration.
b) The faculty member is on an appeals committee reviewing a decision made by an earlier committee of which that faculty member was also a member.
c) The faculty member is on a disciplinary committee that is hearing a case initiated by that faculty member.
   A Faculty member will not typically be recused from a committee merely because:
a) The faculty member believes himself/herself to be biased (due to, for example, a  friendship), or is charged with being biased.
b) The faculty member has personal knowledge of or a close association with the issues being addressed.
   In questionable cases the Faculty Council will determine whether faculty members must/may recuse themselves.  In the event that Council members themselves are involved in such a situation, they will recuse themselves (one at a time, if more than one is involved) and the remaining Council members will make this determination.
1.4.3.7 Committee Release
a) Purpose:  Release from all committee responsibilities is available to faculty who have had exceptionally heavy committee loads or who have an unusual opportunity for professional development.  The purpose is to provide faculty with the opportunity to recover from heavy committee work and to address other areas of development which may have suffered as a result of such work.
b) Eligibility:
1) Automatic:  Faculty members completing full terms on Academic Senate, Faculty Council or the Faculty Personnel committees are automatically granted a one-year release from committee work.
2) By Application:  Faculty can apply to the Professional Development Committee for a one-year release from committee work five years after any previous release.
c) Application Procedure
1) Applications should be submitted to the Provost before December 15 for committee release the following academic year.
2) All requests must be approved by the Professional Development Committee.  Applicants will be notified of decisions before March 1.
3) If the number of applications exceeds the allotment for a given year, a committee release shall be awarded based upon the amount and concentration of institutional service since the last release.
4) If a faculty member is serving a multi-year term on a critical committee, the Professional Development Committee may recommend that the release be postponed for a year.
5) Neither committee release nor load consideration is extended to assignments for which monetary or course load reduction compensation is already provided.

1.5 Protocols for Revision of the Faculty Handbook
1.5.1 Publication
   The current version of the Faculty Handbook will be maintained online at the ProvostÕs website (http://www.westmont.edu/_offices/provost/documents/Faculty-Handbook.pdf).  Every three years the Faculty Council will confer with the Provost to carry out a comprehensive review of the Handbook to ensure its completeness and accuracy.
1.5.2 Sources of Initiative for Revisions
1.5.2.1 Board of Trustees
Revisions proposed by the Board of Trustees do not require the facultyÕs approval, but should be submitted to the faculty for discussion and response before adoption.
1.5.2.2 President or Provost
Revisions proposed by the President or the Provost are subject to the approval either of the Board of Trustees, or of the faculty according to the procedure specified in section 1.5.2.3.
1.5.2.3 Faculty Council
Revisions proposed by the Faculty Council are subject to the approval of the full faculty. Such proposals will be provided in writing to all faculty members at least one week prior to the facultyÕs action. Adoption requires a two-thirds majority vote of the eligible faculty present.
1.5.2.4 Faculty
Revisions proposed by faculty will qualify for consideration by having the endorsement of at least one-fourth of the whole number of faculty eligible to vote. This procedure is an exception to the normal means by which motions may be brought to the faculty by an individual voting member (see Section 1.4.3.1 (c) 8).  Such revisions are subject to the approval of the whole faculty according to the procedure specified in section 1.5.2.3.
   discussion and response before adoption.
1.5.3 Authorization of Revisions
       The Provost, in consultation with Faculty Council, will adjudicate all questions concerning the status of a revision as ÒsubstantiveÓ or Ònon-substantive,Ó and ÒautomaticÓ or Ònon-automatic.Ó
1.5.3.1 Non-Substantive Revisions
The President or the Provost or the Faculty Council, or a designate of any of them, may authorize non-substantive revisions for the sake of correctness (e.g., of spelling, grammar, and mechanics), or of internal consistency (in graphics, e.g., fonts, layouts; in forms of word, e.g., shall/will, chair/chairperson; in formats, e.g., the organization of committee descriptions), or of clarity (e.g., the renumbering of sections).
1.5.3.2 1.5.3.2 Substantive Revisions
1.5.3.2.1 Automatic Revisions
Certain revisions of substance are automatically contingent on events and circumstances beyond the CollegeÕs powers of choice (e.g., 2.8.1 ÒGovernment Mandated Benefits,Ó or the constitution of the Christian College Consortium in section 1.2.2 ÒAccreditations and MembershipsÓ). These revisions will require no additional authorization, and will be treated according to the procedures for non-substantive revisions (section 1.5.3.1).
1.5.3.2.2 Non-Automatic Revisions
Following approval of any non-automatic revision by two-thirds majority vote of the faculty as specified in section 1.5.2.3, the motion will then require the approval of the President and the Board of Trustees.
1.5.4 Emergency Suspension of a Handbook Provision
   The Faculty may choose to suspend a provision of the handbook by a four-fifths vote of the eligible faculty present, upon one weekÕs prior written notice.  Such suspensions shall be limited to issues of faculty governance (e.g., make-up of committees) rather than issues that impact faculty contracts, faculty disciplinary hearings, or trustee governance.
   
2.0 CONTRACTUAL STATEMENT
2.1 Faculty
   The Faculty of Westmont College is a community of Christian teacher-scholars who share with others responsibility for institutional governance, and who exercise primary responsibility for curriculum, instruction, faculty status, and any policies for student life which relate to the educational program.  Other persons closely associated with academic programs may be granted faculty status as provided for in Section 2.1.1.2.4.
2.1.1 Types of Faculty Status
2.1.1.1 Ranked Faculty
2.1.1.1.1 Full-Time
   Full-time faculty are those teaching 24 units, or equivalent as approved by the Department Chair and the Provost, during an academic year.
2.1.1.1.2 Part-Time
   Part-time faculty are those teaching fewer than 24 units, or equivalent as approved by the Department Chair and the Provost, during an academic year.
2.1.1.1.3 Appointment to Rank
	(for complete description see Section 2.2.2.2)
a) Instructor:  Normally, faculty with professional preparation but who lack the prerequisites for probationary status (see Section 2.1.2.1.2).
b) Assistant Professor:  The initial rank for faculty with the academic preparation and experience required for the professorial ranks. In most cases, an Assistant Professorship is a full-time, probationary appointment (notice contract).
c) Associate Professor:  An intermediate rank for faculty who possess the academic preparation and experience required of the professorial ranks, including a terminal degree in their discipline, and have demonstrated continuing growth as teachers and scholars. Appointments are probationary (notice contract) or tenured (continuous contract).
d) Professor:  The highest recognition in rank given to a faculty member, reserved for those with an appropriate academic degree and experience and who demonstrate maturity in all aspects of scholarship, teaching, and service to the College. Full professors normally have tenured appointments (continuous contract).
2.1.1.2 Special Status Faculty
2.1.1.2.1 Adjunct Faculty
   Adjunct appointments are temporary relationships with the College provided in term contract agreements of one year or less which specify the benefits granted by the College. These faculty, by special action of the Provost, may be exempted from agreement with the Articles of Faith.
2.1.1.2.2 Emeritus/Emerita
   Upon retirement faculty members may be granted emeritus/emerita status by action of the Board of Trustees on the recommendation of the Faculty Personnel Committee.  Normally, the title of ÒemeritusÓ or ÒemeritaÓ is conferred upon faculty who have served Westmont College with distinction and have held faculty status for a minimum of 10 years.  The title shall include the rank at the time of retirement plus the word ÒemeritusÓ or Òemerita.Ó
2.1.1.2.3 Artist/Scholar-in-Residence
   An Artist or Scholar-in-Residence is a limited-term faculty appointment given in recognition of the special contributions the individual can make to the academic, spiritual and cultural life of the college community.  Such individuals must affirm the CollegeÕs Community Life Statement and abide by its expectations.  Normally, appointments would be for one year; additional yearly appointments are possible.  Artists or Scholars-in-Residence would not be eligible for tenure, nor the benefits of continuous contract faculty. Such individuals would, however, have available the full range of collegial opportunities offered other faculty.  An appointment as Artist or Scholar-in-Residence is made by the Provost, in consultation with the Department Chair and Personnel Committee, and with the approval of the Board of Trustees.
2.1.1.2.4 Administrators
   Certain administrative and academic support staff receive faculty status, but not rank, either by position (the Executive Team, Associate Provost, Vice Provost, Dean of Curriculum and Educational Effectiveness, Director and Associate Director of the Library and Librarians, Registrar, Associate Registrar, Director of Disability Services, Senior Director of Information Technology, Director of Admissions, Campus Pastor, Associate Dean of Students, Director of Global Education, Directors and Assistant Directors of Westmont Institutes, Director of the Center for Social Entrepreneurship, Director of Westmont in San Francisco, Westmont Off-Campus Programs faculty, Director of Intercultural Programs, Director of the Career Development and Calling Office, Director of Internships, Director of the Westmont Ridley-Tree Museum of Art, Athletic Director) or by action of the Faculty Council.
2.1.1.2.5 Fellows
   A fellow of Westmont College is a limited-term appointment made by the Provost to individuals whose work is expected to contribute in some substantial way to the academic, spiritual, and/or cultural life of the college community.  Typically, fellows are appointed in connection with a specific academic department, program, or college institute.  Normally, appointments are for up to one year; additional appointments are possible.  Fellows are granted faculty library privileges and work/study space (as available).  Fellows need not be in residence at WestmontÕs home campus.
2.1.2 Contract
2.1.2.1 Types of Contract
   See also Section 2.3.3 Non-Reappointment.
2.1.2.1.1 Term (Non-Tenure Track)
2.1.2.1.1.1 Temporary
   Temporary appointments are for one year or less.
2.1.2.1.1.2 Multi-Year, Non-continuous
   Non-tenure track faculty who teach a full load may be placed on multi-year term contracts at the discretion of the Provost, in consultation with the Department Chair and the President and with the approval of the Board of Trustees.  Faculty with multi-year term contracts are considered ranked faculty (see section 2.1.1.1) hired under special circumstances to fill a short-term need.  Multi-year term contracts are made for specific periods of time, usually for two or three years.  There is no expectation that an initial multi-year term contract will be renewed.  However, at the option of the College a new multi-year term contract may be issued, after review by the Provost in consultation with the Department Chair and the President, prior to December 15 of the final year of the contract.  No faculty member will remain on a multi-year, noncontinuous term contract for more than five years.  At the end of that five-year period, the Provost in consultation with the Department Chair and the President, will re-evaluate the special need for which the faculty member was hired, and the faculty member will either be moved to a multi-year, continuous contract, or the CollegeÕs term contract with the faculty member will be terminated.
2.1.2.1.1.3 Multi-Year, Continuous
   Non-tenure-track faculty who teach a full load may be placed on multi-year, continuous term contracts at the discretion of the Provost, in consultation with the Department Chair and the President and with the approval of the Board of Trustees. Faculty with multi-year, continuous term contracts are considered ranked faculty.  (See section 2.1.1.1)  A multi-year continuous contract recognizes that the College has a long-term need to fill a position, but circumstances (e.g. the faculty member's lack of a terminal degree) may not warrant a notice (probationary/tenure track) contract.  At the option of the College, a new multi-year term contract may be issued, after review by the Provost in consultation with the Department Chair and the President, prior to December 15 of the final year of the contract. Faculty on multi-year continuous term contracts will participate in regular performance reviews, and are eligible for housing benefits and for promotion and sabbatical at appropriate intervals.
2.1.2.1.2 Notice (Probationary/Tenure Track)
a) One year notice contracts (probationary appointments) are made for faculty being considered for tenure.  (Criteria and process for granting tenure are discussed in Section 2.2.3.)
b) A request to convert a non-tenure track (term contract) appointment to probationary (notice contract) is initiated by the relevant department through its chair to the Provost and the Faculty Personnel Committee.  The Faculty Personnel Committee may waive the requirement to conduct a national search to fill the position.  No more than two yearsÕ credit toward final tenure review may be given (see Section 2.1.2.3).
c) The principal difference between probationary (notice contract) and tenured (continuous contract) appointment is that persons with probationary status, after timely notice, can be denied reappointment without statement of reason (see Sections 2.3.3.1 and 2.3.3.2.a) ).
2.1.2.1.3 Continuous (Tenured)
   Tenure presumes continuous appointment until retirement, disability, resignation, death, or termination for cause.  Under some circumstances reduction in force may affect tenured positions (see Section 2.3.4).  Tenure signifies a mutual relationship of trust and responsibility between the faculty member and the College.  The College advocates and assures its support in furthering development of the faculty as teachers and scholars; faculty affirm their continuing commitment to the mission of the College and their development as teachers, competence as scholars, and continued growth as Christians. Continuous appointment is not a guarantee of lifetime employment; it confers formal assurance that, thereafter, an individualÕs membership among the faculty of Westmont College will not be placed in question without full academic due process (see Section 2.2.4 and 2.3.4).
2.1.2.1.4 Requests for Temporary Load Reduction
   Full-time regular faculty may apply for a temporary reduction in workload for personal, non-professional reasons.  Application should be made to the Provost and should include a rationale for temporary load reduction and a specification of the requested length of term on a reduced-load contract.  In considering the application, the Provost will consult with the chair of the applicantÕs department to determine the consequences of the applicantÕs load reduction for his or her department.  Moreover, if a temporary load reduction contract is issued to the applicant, any adjustments made to the applicantÕs tenure, promotion and/or sabbatical schedule as a result of load reduction will be specified in writing, communicated to the applicant in writing, and placed in the applicantÕs personnel file when the contract is issued.  Applicants should consult with the Human Resources office for information concerning possible adjustments to benefits that may result from a temporary load reduction.
2.1.2.2 Locus of Tenure and Appointment
a) Faculty at Westmont College are tenured to the College and appointed to one or more academic departments.
b) Appointment is made by the Board of Trustees upon the recommendation of the Provost and the President, after consultation with the department(s).
c) Although the faculty member is appointed to, and functions primarily as, a professor of a department, faculty responsibilities are not limited to the department.  Since tenure and promotion involve more than departmental considerations, it is the responsibility of the Faculty Personnel Committee to appraise a faculty memberÕs performance and contribution to the College at large, as well as to the discipline, in determining his or her suitability for promotion and tenure (see Section 2.2).
2.1.2.3 Prior Service Credit
At the time of the initial appointment of tenure-track faculty, the Provost, in communication with the department chair and the chair of the Faculty Personnel Committee, will determine appropriate credit for prior service toward rank, promotion, tenure and salary step.  Those terms will be put in writing at the time of initial appointment and placed in the faculty memberÕs personnel file.  Final approval is given by the President and the Board of Trustees.
a) Rank Initial rank reflects level of education and years of experience.  Ranks are defined in Section 2.1.1.1.3
b) Promotion Instructional faculty with prior college teaching experience or scholarly or artistic achievement may be given advanced standing towards promotion.  Scholarly or artistic achievement includes service in the discipline as a professional employee, researcher, practitioner, post-doctoral fellow, or expert consultant; publishing through a peer-reviewed process; giving performances or exhibits that are publicly reviewed; presenting at professional meetings; and/or leading workshops or seminars.
c) Tenure Faculty who have not held tenure at a previous institution will be subject to the normal tenure process as specified in the Handbook.  Faculty with tenure or in a tenure-track position at a previous institution may be given up to three yearÕs credit toward tenure.
       In exceptional circumstances such as the appointment of an endowed chair, faculty with significant experience and distinguished careers may be granted Òprovisional tenure statusÓ at the time of appointment upon the recommendation of the faculty Personnel Committee and the approval of the Provost and President.  During the second year of the faculty memberÕs appointment at Westmont, a mutual assessment-of-fit would be submitted to the Personnel Committee by the department chair and the faculty member.  This assessment would be based on the Handbook criteria for tenure, but would also take into account written expectations at the time of appointment.  At that time, the Personnel Committee may make one of three recommendations: (1) to remove provisional status and recommend full tenure: (2) to continue provisional status for an additional two years, providing further time for assessment; or (3) to schedule a full tenure review in four years as specified in section 2.2 (Evaluation, Promotion and Tenure).
d)	Salary Step Advanced credit toward salary step may be given for relevant experience that prepares the candidate to bring perspective, insight, and/or maturity to the position.  Such service includes serving as a professional, researcher, practitioner, post-doctoral fellow, consultant, or instructor.

2.1.2.3.1 Part Time (Appended)
2.1.2.3.2 Full Time (Appended)
2.1.2.3.3 Articles of Faith
a) Westmont College is a liberal arts college committed to Jesus Christ and belonging to the worldwide evangelical Protestant tradition.  In that tradition, the collegeÕs trustees, administrators, and faculty participate in many different churches and with them confess such historic statements of the church as the ApostlesÕ Creed and the Nicene Creed. In faithfulness to God, who is the source of truth, and under the authority of Scripture, we joyfully and humbly affirm the following articles of faith, which guide our learning, teaching, and living.
b) Articles of Faith:
   We believe in God
   The Lord our God alone is God, holy and loving, revealing in creation and in Jesus Christ GodÕs own power and glory, grace and mercy. The Lord our God alone is God, just and true, perfect in being and trustworthy in action.
   The Lord our God is infinite and beyond imagination; our minds can never fully know God nor our hearts completely grasp his ways. The Lord our God is faithful and steadfast, unfailing in word and deed.
   The Lord our God is TriuneÑone being in three personsÑFather, Son, and Holy Spirit in co-equal, co-eternal communion. The Lord our God, Creator and Sustainer of all that is, redeems the world from its fallenness and consummates his saving work in a new heaven and a new earth.
     . . . the Father, Son, and Holy Spirit
   God the Father is the source of all that is good. He is Father to his eternal Son, Jesus Christ, and to all who are adopted as his sons and daughters through faith in Jesus Christ. He has sovereignty over us, affection toward us, and glory for us.
   God the Son became incarnate in Jesus ChristÑone person in two natures, fully human and fully divineÑwho was conceived by the Holy Spirit and born of the virgin Mary. In his life and in his death on the cross he conquered the powers of darkness, paid the penalty for our sin, and demonstrated GodÕs love for the world.  In his bodily resurrection his life and death are vindicated, and he is revealed to be the only judge and redeemer of the world. He intercedes for us now before the Father and will return in glory.
   God the Holy Spirit is Lord and Life-Giver, the one who empowered Jesus Christ and who empowers his people to continue GodÕs work today. God the Holy Spirit convicts us of sin, brings us to faith in Jesus Christ, and conforms us to the image of Christ. The Spirit inspired the authors of  Scripture and guides the church in faithful translation and interpretation. The Bible, consisting of the Old and New Testaments, is God-breathed and true, without error in all that it teaches; it is the supreme authority and only infallible guide for Christian faith and conductÑteaching, rebuking, and training us in righteousness.
      . . . the Author of our salvation 
   God created humankind for unbroken relationship with God, one another, and the rest of creation. Through AdamÕs disobedience, we fell into sin and now suffer alienation and brokenness. The effects of sin are so pervasive that apart from GodÕs grace we are lost and dead. Only by GodÕs grace through faith in Jesus Christ are we saved and made alive.
   In bringing us to faith in Jesus Christ, the Spirit incorporates us into the body of Christ, his church, the community of all believers in heaven and on earth. The church is called to bear witness to Christ among the nations by praising God, preaching the good news, discipling believers, healing the sick, serving the poor, setting free the oppressed, and caring for creation. The gifts and fruit of the Holy Spirit empower the church for this mission.
   Jesus Christ will return one day in his glorified body to judge the living and the dead. Those who do not believe in him will be raised to suffer forever a just punishment. Those who believe in him will be transformed, their bodies raised imperishable and incorruptible, to live and reign with him forever in a new heaven and a new earth in which there will be all that is good and true and beautiful, but no sorrow, no tears, and no evil thing.
   And so we pray:  Come, Lord Jesus.
c) These declarations of faith do not define in detail what an individual Christian might believe in many important areas of doctrine and theology.  Moreover, as a college seeking to serve evangelical Christians from many denominations, there is less an obligation to decide on these various points in detail than to celebrate not only unity in Jesus Christ but also the freedom to disagree, and to continue grappling in the many non-essential elements of faith.
2.1.2.3.4 Community Life Statement
a) As a matter of conscience and contractual agreement faculty are to affirm and promote the Community Life Statement which provide a common understanding as all members seek to apply biblical principles in daily living.
b) Community Life Statement
   When Jesus Christ summed up the way His followers were to treat each other, He said, Òlove one another as I have loved you,Ó and ÒLove your neighbor as yourself.Ó  On a college campus, this kind of love must take into consideration the relationship between learning and community.
   Affirming the qualities of this relationship is vital.  As students, staff, and professors learn to live together, we recognize the dual manifestations of love in justice and mercy.  We attempt to work out what it means to live justly and mercifully in common agreements such as this one.  We understand that life in a college will give priority and honor to the wise development of the mind.  
   Given this focus, our social and intellectual growth needs freedom for exploration, complemented by a commitment to good will and graciousness.  Personal discipline is also required.  For example, civility is basic to all types of community, while academic honesty and respect for education are fundamental to an instructional environment.
   Learning depends on truth-centered attitudes.  It thrives in an atmosphere of discriminating openness to ideas, a condition that is characterized by a measure of modesty toward oneÕs own views, the desire to affirm the true, and the courage to examine the unfamiliar.  As convictions are expressed, one enters into the Ògreat conversationÓ of collegiate life, a task best approached with a willingness to confront and be confronted with sound thinking.
   Community is built upon other-centered practices.  It flourishes in a place where love for God and neighbor is cultivated and nurtured.  It grows strong when members practice integrity, confession, and forgiveness, attempt to live in reconciled relationships, accept responsibility for their actions, and words, and submit to biblical instructions for communal life.
   Scripture supports these attitudes and principles.  It recognizes that all humans are created in the image of God and that God is the giver and taker of life, from the beginning of life to its end.  It teaches us to value human presence, celebrate human creativity, and promote relationships based on the ideals of trust, compassion, and forbearance and praises actions that manifest sacrificial giving and sincere faith.  Scripture also forbids attitudes such as pride and jealousy and prohibits such actions as drunkenness, sexual promiscuity, and dishonesty.  In keeping with these standards, the Westmont community has agreed to certain guidelines in the Student, Staff, and Faculty Handbooks.
   Desiring to implement the teachings of Christ, Westmont encourages true fellowship in the whole body of Christ, including the local church, for when we love each other we imitate ChristÕs love for us.  As we seek to follow God in truth, certain choices make for greater peace:  a respect for others as they make decisions contrary to ours, a readiness to listen carefully to those who represent situations or cultures unfamiliar to us, and a concern for how our preferences affect the lives of those around us.
   We are committed to inquiry as well as pronouncement, rigorous study as well as kindred friendship, challenging teaching as well as reflective learning.   Sometimes these tensions will lead to conflict.  To live in unity, we must set ourselves to the practical task of discerning daily how to love well, how to inflesh the biblical call to justice and mercy.  As we do so, our life together at Westmont will begin to resemble the community God has envisioned for us.
c) Behavioral Expectations:
   The Westmont community chooses freely and willingly to impose upon itself rules for behavior which serve both the long-range interests of the institution and the immediate good of its individual members.  While we do not view these expectations as an index to maturity in Christ, we do regard violations as a serious breach of integrity within the community because each member has voluntarily chosen to associate with it and to accept its standards.
   Consistent with Scripture, the College establishes the following specific expectations for the trustees, administration, faculty, staff, and students of the Westmont community:
1) The college does not condone practices that Scripture forbids.  Such activities include but are not limited to occult practices, drunkenness, theft, profanity, dishonesty and sexual relations outside of marriage.  Westmont also recognizes that Scripture condemns Òsins of the spiritÓ such as covetousness, jealousy, pride, and lust.  By their very nature, these sins are more difficult to discern.  Because they lie at the heart of the relationship between the individual and God they are of central concern of the Westmont community.
2) The college upholds integrity as a core value of the community.  Members are expected to take responsibility for their own violations of all behavioral guidelines and demonstrate commitment to the value of integrity in word and deed.
3) The college expects our members who choose to marry to abide by the commitment to lifelong heterosexual marriage and whether single or married to strive to maintain healthy family relationships.
4) The college upholds the laws of the local community, the nation, and the state of California that prohibit the possession or use of illegal drugs or drug paraphernalia, against purchasing or consuming alcoholic beverages by persons under the age of 21, drunkenness, and driving under the influence of alcohol.
5) The college recognizes that the use of tobacco products and alcoholic beverages presents a danger to personal health.  It condemns the abuse, and raises questions about the use of tobacco and alcohol.  Under no circumstances shall any member of the community use or posses these products on campus or when attending a college-related student activity.
   Westmont will establish other rules and regulations necessary for orderly community life and will list them in appropriate handbooks. 
2.1.3 Search, Appointment, Orientation
   Appointments with notice (probationary) or continuous (tenured) contracts are made by the Board of Trustees upon the recommendation of the Provost and the President. Appointments with term contracts are made by the Provost, subject to review by the President and/or the Board of Trustees.
2.1.3.1 Search Procedures
a)  Prior to each new faculty search, the Provost will establish the degree requirements for the position in consultation with the Department Chair and Personnel Committee. Except in very rare circumstances, a terminal degree will be required for all faculty positions at Westmont.  The Department Chair will provide a description of the position and an announcement of the opening which will be based on a legally and institutionally acceptable format provided by the Provost.  Normally, a national search is conducted.
b) The search committee will consist of the Provost, the department chair, representatives from the department, and two faculty from outside the department. One of those faculty will also be from outside the division. The Provost chairs the committee. The department chair serves as vice-chair of the search committee, coordinating the early stages of discussing, screening and ranking applications. This may also involve arranging preliminary conference interviews.
c) The Provost may review applications and will expect departments to develop a pool of applications reflecting diversity of ethnicity and sex in addition to commitment both to the traditions of liberal education and to the integration of Christian faith with learning.  A short list of candidates will be developed by the full search committee and the Provost.  The Provost and department chair will determine the strategy for bringing candidates to campus for interview.
d)  The campus visit includes
(1) Interviews with
* the search committee
* departmental members
* the Provost
* the President
* the Vice President for Student Life or representative
* the non-departmental members of the search committee
* and students
(2) Informational meetings with
* The Vice President for Finance,
* The Director of Human Resources, and
* The Assistant to the Provost for Diversity Initiatives or representatives from these offices.  While these meetings are primarily informational, questions or concerns that arise may be communicated to the search committee.

(3) The candidate teaches at least one but preferably two classes (a lower division class and an upper division class), and offers a research presentation open to the college community.  Information is solicited by the department chair and from all who participated in the interview process.

(4) Over the course of the candidateÕs campus visit, the search committee must conduct significant discussions with the candidate regarding:
* the candidateÕs commitment to WestmontÕs mission and the central role that matters of faith play in faculty teaching, community life and scholarship
* the candidateÕs commitment to and potential for faith-learning integration
* criteria for tenure and promotion
(5) The search committee should ensure that all candidates for the same position have comparable interview components.

2.1.3.2 Appointment Procedures

a) The search committee will make its recommendation for an appointment to the Provost.  If the Provost and the committee are unable to agree on an appointment, they will continue the search.  With the approval of the President, the Provost will forward his or her recommendation and that of the committee to the Academic Committee of the Board of Trustees.  Appointment is not final until the Board of Trustees has approved the contract of the candidate.
b) When a full-time contract is offered, the Provost will specify in writing all conditions and considerations that may be distinctive to the contract in a letter of appointment.  If degree requirements have not been met at the time of appointment, the Provost will specify in writing the amount of time allowed to complete the degree.  Faculty who have not completed the degree by the agreed-upon date are subject to termination.  Attendant to the contract, the Provost will provide a copy of the Faculty Handbook, the Articles of Faith, and Community Life Statement.
2.1.3.3 Orientation
   It is the responsibility of the departmental chair, during the first semester of a new faculty memberÕs employment, to review with the faculty member departmental procedures and the academic program of the College.  In addition, a faculty mentor will be appointed in accordance with the provisions of section 2.5.3.  
2.1.3.4 Equal Opportunity
   Westmont College does not unlawfully discriminate on the basis of age, race, color, sex, sexual orientation, national or ethnic origin, marital status, medical condition, genetic information, mental or physical disability in its employment practices, except where physical fitness is a valid occupational qualification. As allowed by federal and state law, Westmont chooses to exercise religious preference in all areas that it deems appropriate for its mission.
2.1.3.5 Conflict of Interest
   Situations involving the evaluation of a family member limit the participation of the faculty member.  When the family member is a student, the faculty member will make a reasonable effort to procure from another colleague additional evaluation of the studentÕs work.  When a family member is an applicant for a position, the faculty member will not participate in the search process; the same criteria are used as in the search for a faculty member in any department.  In addition, it is prudent that neither family member serve as chair or, when this expectation cannot be met, any evaluation of the family member be supervised by a faculty member from another department who will be appointed by the Provost.
2.1.3.6 Special Appointment Considerations
   Full time faculty are required each contract year to indicate their affirmation of the Articles of Faith and the Community Life Statement (see Sections 2.1.2.3.3 and 2.1.2.3.4).
2.1.3.7 Employment Eligibility Verification
   Prior to appointment, a faculty member is required by federal law to provide verification of eligibility to be employed in the United States.
2.2 Evaluation, Promotion and Tenure
   Evaluation of faculty for promotion and tenure benefits the individual and the College. The procedures are similar, but particular criteria are weighted differently depending on the objective of the review.  These differences correspond to the complementary roles promotion and tenure play in accomplishing the educational mission and goals of the College.
a) The purpose of the intermediate review is to make a judgment concerning the revieweeÕs progress towards tenure and to make a judgment as to whether or not the candidate is on a trajectory that would lead to tenure.
b) The tenure review, although based in part on the criteria for the Associate Professorship and the expectation of sustained professional growth (see Section 2.2.2.2.4), places special emphasis on the fundamental criterion and on teaching (see Sections 2.2.1 and 2.2.1.2a).
c) Evaluation for promotion, while taking into account institutional service, emphasizes professional criteria generally associated with the faculty memberÕs discipline in the larger academic community and at Westmont College.
2.2.1 Evaluation
   It is understood that within the context of the review process, the Faculty Personnel Committee will determine what constitutes appropriate performance when questions arise about applying criteria (here and in sections 2.2, 2.2.1 and 2.2.2.2).  It is also assumed that these criteria, though not necessarily weighted equally, will enable the Personnel Committee to assess performance as equitably as possible while taking individual strengths and expertise into account.
   
2.2.1.1 Criteria 
   Fundamental Criteria: The fundamental criterion for any favorable review is that a faculty member display in word and deed an appropriate sense of institutional mission. This means that the faculty member:
1) clearly supports the mission of the college as an undergraduate, residential, evangelical Christian, liberal arts institution, affirming in particulate the Articles of Faith, honoring the Community Life Expectations, and displaying a commitment to the integration of faith and learning, and to the moral and spiritual development of students.
2) cooperates with department colleagues, treats member of the large college community with respect and concern and approaches conflict constructively.
Failure to meet this fundamental criterion cannot be compensated for by the meeting of other standards.
2.2.1.2 Standards
   It is understood that within the context of the review process, the Personnel Committee will determine what constitutes appropriate performance when questions arise about applying criteria (here and in sections 2.2, 2.2.1 and 2.2.2.2).  It is also assumed that these criteria, though not necessarily weighted equally, will enable the Personnel Committee to assess performance as equitably as possible while taking individual strengths and expertise into account.
a) Teaching:
1) communicates clearly, accurately, engagingly, and respectfully with students
2) works effectively with a wide range of students and sets standards that challenge but do not discourage them
3) explicitly as well as implicitly integrates faith and learning
4) allows time for contact with students outside the classroom in office hours and other venues where conversation may be continued and relationships deepened
5) functions effectively both as student mentor and as advisor (i.e., knows relevant requirements, discusses options with students, takes time to raise and address questions in advising sessions)
6) functions effectively in a variety of teaching environments (lecture, small seminars, labs, etc.); involves students in research where appropriate
b) Professional Development:
1) publishes in scholarly journals, through an academic publishing house, or through other appropriate venues, and/or gives performances or exhibits that are publicly reviewed
2) presents at professional meetings, leads workshops or seminars, or serves as an expert consultant in the field
3) conducts scholarly work that reflects Christian commitment and, where appropriate, explicitly brings a Christian perspective to bear
4) keeps abreast of developments in field through professional meetings and literature, and incorporates them into teaching and scholarship
5) when appropriate, serves as a professional resource for the local community
6) where applicable, takes significant initiative to engage in interdisciplinary research and dialogue leading to publication, presentation, or course development
c) Institutional Service:
1) takes part in the spiritual life of the community, e.g. through individual mentoring, chapel participation, Bible Study, residence hall events, and/or other activities focused on prayer, communal worship, or spiritual development
2) attends and participates regularly in department, division, and faculty meetings, assuming appropriate leadership in such gatherings when requested
3) serves conscientiously on standing committees when appointed or elected, barring serious reasons for excuse
4) participates in annual assessment reports and multi-year program review reports preparation
5) participates on occasion in task forces, search committees, program development and assessment teams, and other ad hoc working groups.
6) participates in the local community through church involvement, and as appropriate through civic organizations, public causes, lectures, etc.
7) participates on occasion in recruiting events and other public relations efforts of the college

2.2.1.3 Evidence
   Promotion and tenure evaluation will be based on:
a) student evaluations
b) peer evaluations
c) departmental evaluations
d) administrative evaluations
e) candidateÕs personal statement (self-assessment and faith-learning portfolio) and curriculum vitae
1) Self-assessment
The 5-10 page document addresses the standards of faculty review in the Handbook (Error! Reference source not found.), including the fundamental criterion described under 2.2.1.  It presents the candidateÕs own assessment of his or her development and accomplishments in the areas of teaching, professional development, and institutional service, and describes the ways in which the candidate would like to grow in the future.
2) Faith-Learning Portfolio
This collection is both a record for use in reviewing the candidate and a tool for use by the faculty member in seeking to develop his or her own approach to the integration of faith and learning.  Considered a work in progress, it may include reading lists of relevant material, reflections on those readings and/or on the faculty memberÕs other efforts to integrate faith and learning in the classroom, in scholarship, and in participation in the life of the community.  The portfolio should include an essay reflecting the candidateÕs current thinking about the philosophy of Christian liberal arts education and about the relationship between the candidateÕs Christian faith and academic discipline.  At the time of final tenure review, the essay should have evolved into a 5-10 page paper on these topics.
2.2.1.4 Procedure
   Personnel Committee Reviews & Recommendations.  Typically, the intermediate tenure review will occur in the spring semester of the third year, and the final tenure review in the spring of the sixth year.  Promotion reviews not connected with the tenure process typically will occur in the fall semester.  
a) For each promotion and tenure review, each faculty member being reviewed is responsible to demonstrate sufficient progress toward tenure and promotion in the form of a portfolio presented to the Personnel Committee. Required items include:
1) a current curriculum vitae;
2) personal statement (self-assessment essay and faith-learning portfolio as described in section 0 e) );
3) course evaluations for all courses taught during the previous two years, or the results of alternative means of evaluation previously approved by the and the Personnel Committee;
4) the schedule and class rosters for each current course;
5) a list of students for potential interviews; 
6) a list of faculty for potential interviews; 
7) a list of external references to be contacted (required of candidates for promotion to full professor, optional in other reviews);
b) At the time of each review for promotion and tenure, the Personnel Committee will assign one of its members (ÒreviewerÓ) to each faculty member undergoing review (ÒrevieweeÓ).  The reviewer must be at least of the same rank as that for which the faculty member is a candidate, and must be tenured in the case of a final tenure review.  The Personnel Committee shall endeavor to avoid conflict of interest when assigning members to review cases.  Faculty not on the Personnel Committee who believe an assignment may result in a conflict of interest may state their objections to the Personnel Committee through the Provost.  Both in gathering data and in writing reports and summaries, reviewers shall endeavor to ensure confidentiality and, except in the case of student course evaluations, shall disregard comments or evaluations for which authorship is not claimed.  Responsibilities of reviewers are as follows:
1) review the faculty memberÕs personnel file, including departmental assessment summaries and any prior reviews;
2) meet with the reviewee at least once toward the beginning of the review process in order to provide an orientation to the process and both ask and answer questions, and again toward the end of the process in order to address any issues or concerns raised during the review process and to ask any follow-up questions;
3) solicit written feedback from and interview each member of a revieweeÕs department (alternate means of soliciting input may be required if a departmental colleague is out of the area; in such instances telephone interviews or extensive written feedback should suffice; a formal departmental recommendation is not required);
4) solicit evaluations from faculty colleagues and administrators who have relevant information;
5) examine student course evaluations provided by the reviewee;
6) attend two class sessions taught by the reviewee, preferably of two different courses;
7) interview at least eight students from classes taught the previous two years, one-half of the number of students interviewed selected from a list provided by the reviewee and the other half chosen by the reviewer from class rosters; 
8) solicit external reviews and evaluations of professional competence in teaching and scholarship from the list compiled by the faculty member under review (required for candidates for promotion to full professor, optional in other reviews);
9) prepare and submit a written report as a basis for deliberation and recommendation of the Personnel Committee, including the revieweeÕs self-assessments in an attached appendix.
c) After completion of each promotion and final tenure review, the Personnel Committee will vote whether or not to recommend promotion or tenure.  In cases of intermediate tenure reviews, the Personnel Committee will vote whether nor not a candidate has passed the review.  Each of these decisions will be communicated to the Provost and to the candidate.  Prior to communicating this decision, the Personnel Committee will finalize a comprehensive summary statement (approximately 2-3 pages for the review.  Summary statements will include two sections:
1) Formal Recommendation.  An opening paragraph indicating the Personnel CommitteeÕs recommendation.
2) Summary.  Several paragraphs summarizing the overall findings of the review, organized according to each of the four criteria for review.  This section should include affirmations as well as suggestions for improvement or further development.
  In the case of an intermediate tenure view, the statement will include a third section:
3) Expectations.  In this section the Personnel Committee will explicitly state the matters that must be satisfactorily addressed before the candidateÕs tenure review.
d)   The chair of the Personnel Committee (or vice chair, as appropriate) will notify the reviewee of the CommitteeÕs recommendation on the day the review is concluded.
2.2.2 Promotion
2.2.2.1 Eligibility
   An individual qualifies for promotion review after having served five years in a rank at the College.  Individuals may be reviewed prior to that time either (1) on the basis of written agreements completed at the time of appointment recognizing service in rank at another institution or (2) in the most exceptional cases when a faculty member has served not less than three years in the rank and is recommended by the department chair and Provost for early consideration in recognition of outstanding achievement.
2.2.2.2 Criteria
2.2.2.2.1 General
   The criteria used in evaluation for promotion are presented in Sections 2.2.1.1 and 2.2.1.2. In all cases it is understood that the requisite academic degree is in the discipline of a faculty memberÕs appointment.
2.2.2.2.2 Instructor
a) A masterÕs degree; or, in exceptional cases, equivalent professional education and experience directly related to the teaching assignment.
b) Evidence of ability to teach effectively.
c) Evidence of continuing growth in the professional field related to the teaching assignment, such as enrollment in a terminal degree program.
2.2.2.2.3 Assistant Professor
a) A terminal degree appropriate to oneÕs discipline; or, in exceptional cases, a masterÕs degree plus one year of advanced graduate study with continuing enrollment in a terminal degree program, or substantial college teaching experience.
b) Reasonable prospect for becoming an excellent teacher.
c) Evidence of professional competence for scholarly work and, where appropriate, creative activity.
d) Participation in professional activities and prospect for continuing professional growth.
e) A demonstrated interest in the integration of faith and learning.
f) Acknowledgment and acceptance of faculty responsibilities for student advising and college governance and to model Christian living in the College community.
2.2.2.2.4 Associate Professor
a) A terminal degree appropriate to oneÕs discipline and six years of subsequent college teaching. In instances where a doctorate is the terminal degree, and under exceptional circumstances, a masterÕs degree with more than one full year of additional graduate study and eight years of college teaching subsequent to the granting of the masterÕs degree.
b) Evidence of continued growth toward teaching excellence.
c) Recognition of scholarship by peers in the professional field through such means as publication (e.g., original research, interdisciplinary scholarship, textbooks, course software), convention papers, recitals, and showings.
d) Participation in professional activities.
e) Active participation in faculty and departmental responsibilities.
f) Continuing growth in achieving the integration of faith and learning and in the demonstration of Christian living as a model for the College community.
2.2.2.2.5 Full Professor
a) A terminal degree appropriate to oneÕs discipline and 12 years of subsequent college teaching. In cases where the expected terminal degree is the doctorate, under the most exceptional circumstances, a masterÕs degree with two (2) or more years of advanced graduate study and 20 years of subsequent college teaching.
b) Evidence of maturity and excellence in teaching that establishes and sets forth the individual as a model for faculty colleagues.
c) Evidence of mature and sustained professional involvements. Evidence should include sustained professional activities and recognition by peers in the professional field.
d) Leadership in faculty roles of teaching, advising and college governance, and in mentoring colleagues.
e) Maturity in Christian living and the integration of faith and learning which serves as a model for other members of the College community.
2.2.2.3 Procedure
   The procedure for promotion review is presented in Section 2.2.1.4.  In all cases, a faculty member standing for promotion to a higher rank will receive from the Provost a report of the Personnel CommitteeÕs findings including its final recommendation prior to its submission to the President.  The Personnel Committee will either recommend promotion to the next rank or deferral of promotion.
2.2.2.4 Appeal of Promotion Decision
   This section provides recourse when a faculty member wishes to contest a negative decision concerning promotion.
a) Procedures:
   A decision not to recommend promotion may be appealed to the Faculty Council by the same procedures as those provided in Section 2.3.3.3, Appeal of Tenure Decision. The appeal must be made within three weeks of notification of the decision not to recommend promotion. The appeal must be based on procedural and not substantive grounds; the Council will consider only procedural issues.
b) Outcomes:
   The decision of the Faculty Council will be rendered as a recommendation made through the Provost to the President. When the Faculty Council upholds a faculty memberÕs appeal of a decision regarding promotion, the effect is not to recommend promotion, but to recommend the voiding of the original decision and the mandating of a new promotion review, to be completed no later than the end of the semester following the original decision.  When such new review results in a recommendation for promotion, the promotion is to be effective retroactively for the aforesaid academic year.
2.2.3 Tenure
a) Westmont College affirms the principle of academic tenure.  After the expiration of a probationary period, teachers and scholars should have continuous tenure, and their service should be terminated only for adequate cause, including intentional and substantive denial of the Articles of Faith (see Section 2.1.2.3.3), or under extraordinary circumstances, because of financial exigencies or reduction in force (see Section 2.3.4).
b) Advancement to tenured status requires positive action by the College; tenure cannot be granted through inaction.
2.2.3.1 Eligibility
a) A faculty member is eligible for tenure after six years of full-time service.  He or she must hold the rank of Associate Professor, or will be evaluated for promotion to Associate Professor concurrently with evaluation for tenure.
b) Provisions for credit for prior service are located in Section 2.1.2.3.
2.2.3.2 Criteria
   Criteria are those listed for Associate Professor (2.2.2.2.4) with particular consideration given to evidence of:
a) Continued growth toward teaching excellence (specifically student growth and development).
b) Active participation in faculty and departmental responsibilities.
c) Continued growth in the integration of faith and learning, the demonstration of Christian living as a model for the College community, and the expression of confidence in the authority of Scripture and of commitment to Jesus Christ.
2.2.3.3 Procedure
a) Each person shall be notified of progress toward tenure in accordance with the provisions of Section 2.2.1.4.  Instructional faculty on probationary status must have a final tenure review no later than during the sixth year of full time service on continuous probationary appointment at Westmont College.  A final decision regarding tenure may be twice deferred for one year.  Therefore, not later than the eighth year, the College must grant tenure or a one-year terminal contract.
b) The first review may be delayed for one year if, in extraordinary circumstances, the faculty member, the department chair and the Provost agree to postpone it.
c) The process of probation precludes being removed from tenure track and remaining a faculty member at the College beyond one year.
d) In all cases, a faculty member will receive from the Provost a report of the Personnel Committee including its final recommendation prior to its submission to the President.
1) In instances of the first review, the decision of the committee may be:
(a) positive, including acknowledgment of contributions to the College, and specification of areas of development and improvement necessary for a final recommendation for granting of tenure; or,
(b) negative, a recommendation of non-renewal or that a terminal one-year contract be issued.
2) In instances of final tenure review, the decision of the Personnel Committee may be:
(a) positive, recommending the granting of tenure;
(b) to recommend deferral of tenure; or,
(c) negative, recommending the denial of tenure.
e) If the Provost or President does not concur with a positive recommendation of the Personnel Committee, the case will be referred back to the Personnel Committee with suggestions for reconsideration.  If the final recommendation of the Personnel Committee is negative, or if the Provost, President, or Board of Trustees differs with a positive recommendation of the Personnel Committee, then the decision is negative resulting in separation and subject to the provisions of Section 2.3.3.1.
2.2.3.4 Appeal of Tenure Decision
   This section provides recourse when a faculty member wishes to contest a negative decision concerning tenure.
a) Procedures:
   A negative tenure decision may be appealed to the Faculty Council in accordance with the provisions of Section 2.3.3.3.  The appeal must be made within three weeks of notification of the negative tenure decision.  The appeal must be based on the provisions of 2.3.3.3 b) and may not reconsider the substance of a tenure review. 
b) Outcomes:
   The decision of the Faculty Council will be rendered as a recommendation made through the Provost to the President.  When the Faculty Council upholds a faculty memberÕs appeal of a negative decision regarding tenure, the effect is not to recommend tenure, but to recommend the voiding of the original decision and the mandating of a new tenure review, to be completed no later than the end of the semester following the original decision.
2.2.3.5 Non-Tenure Track Faculty Evaluations
   Continuing non-tenure track faculty will be reviewed on a regular basis employing the evaluation criteria in 2.2.1.1.  The evaluation will be carried out by the Personnel Committee and occur during the third and sixth years after appointment, and every sixth year thereafter.
2.2.4 Discipline
   Discipline is seldom necessary, and when it is administered, it is to be done in a fair manner which reflects concern for the individual as well as for the community.  The hope is that all discipline will ultimately be remedial.  The following are two procedures for discipline, depending upon the nature of the violation. 
2.2.4.1 Non-Performance of Contract
   This section provides recourse when a faculty member (2.2.4.1 b)1) ) has a grievance against another faculty member involving lapses from contractual obligations associated with teaching, advising, and governance (2.2.4.1 a) ).
a) Definitions and Principles:
   Discipline for ÒNon-performance of ContractÓ is intended to hold faculty accountable for fulfilling contractual obligations associated with teaching, advising, and governance.  Its primary purpose is not to punish, but rather to identify and remedy patterns of non-performance of contractual obligations before more severe sanctions are required.
   It may not be applied to single instances of non-performance of contract or for the purpose of precluding or disparaging differences of opinion, for criticizing the content and method of a course, or for criticizing a faculty memberÕs extra-mural activities.
b) Procedure:
1) Concerns related to non-performance of contract may be raised by colleagues, by students, or by administrators.  It is usually desirable that the concern be raised first with the faculty member involved.  If there is good reason not to address the concern to the faculty member involved, it should be brought to the attention of the Department Chair or the Provost.
2) The Department Chair (if the accused faculty member is a Department Chair, then the Dean of Curriculum and Educational Effectiveness) and the Provost will meet with the faculty member to discuss the allegation.  The faculty member has the right to give an explanation and/or interpretation, to present evidence, and to have another faculty member present as an advocate.  If the allegation is judged to lack substance, no further action will be taken.  The matter will not be discussed beyond those already involved, nor will a statement be placed in the faculty memberÕs file.  If the allegation is found to have substance the faculty member, Department Chair, and the Provost should endeavor to resolve the problem to their mutual satisfaction.  If an agreement is reached, a statement to this effect will be written and signed by the Provost, co-signed by the Department Chair, and the faculty member, and given to the faculty member.  A copy of the statement will be placed in the faculty memberÕs personnel file and all other communication pertaining to the case will be expunged.
3) If a resolution is not achieved to the satisfaction of the chair and the Provost, a finding presenting the problem and describing the steps deemed necessary to rectify it will be written by the Provost and co-signed by the Department Chair. The finding will outline the process for ascertaining the faculty memberÕs compliance and the successful resolution of the problem.  This finding will be given to the faculty member and a copy placed in the faculty memberÕs personnel file. In this instance, or for any other action that may follow, the faculty member has the right to append a response.
4) At a time determined in the finding (2.2.4.1 b)3) ), a meeting among the faculty member, Department Chair, and the Provost will be held to review the faculty memberÕs compliance with the conditions of the finding.  If, in the judgment of the chair and the Provost, the problem has been resolved, a statement written and signed by the Provost, and co-signed by the Department Chair and the faculty member, will be placed in the faculty memberÕs personnel file.  If, however, the chair and the Provost determine that the conditions of the first finding have not been satisfied, a second finding written by the Provost and co-signed by the chair will be placed in the faculty memberÕs personnel file.  This finding will describe the problem and establish the conditions for its resolution.
5) At a time determined in the second finding (2.2.4.1 b)4) ), a meeting among the faculty member, Department Chair and the Provost will be held to review the faculty memberÕs compliance with the conditions of the finding.  If, in the judgment of the chair and the Provost, the problem has been resolved, a statement written by the Provost, and co-signed by the chair and the faculty member, will be placed in the faculty memberÕs personnel file.  If, in their judgment, the conditions of the second finding have not been met, the chair and the Provost will recommend to the Faculty Personnel Committee that a sanction be brought against the faculty member.
6) The Faculty Personnel Committee will review the case, relying on the pertinent written communications mentioned above.  The committee may choose to call the faculty member to appear in person.  In any case, the faculty member in question has the right to appear before the committee.  The Personnel Committee by majority vote, excluding the Provost or any other person previously involved in the process, may determine that: (a) sanctions are unnecessary, dismiss the case, and recommend that all communications pertaining thereto be expunged from the faculty memberÕs personnel file; (b) determine that a sanction is justified and recommend both the sanction and the conditions for its removal (a statement thereof to be placed in the faculty memberÕs personnel file); or (c) determine that the problem is severe enough to warrant suspension (2.2.4.3) or discharge for cause proceedings (2.2.4.4).
c) Sanctions:
   Sanctions are limited to the following:
1) denial of eligibility for teaching and scholarship awards
2) denial of travel funding
3) denial of faculty development grants
4) delay of sabbatical leave
5) suspension or denial of salary step increase
d) Appeal:
   The application of progressive discipline may be appealed to the Faculty Council on the bases of (1) the severity of the sanction, (2) process and/or (3) the inequitable application of sanctions.
2.2.4.2 Violation of Contract
   This section provides recourse when a faculty member (2.2.4.2.2 a) has a grievance against another faculty member involving lapses from contractual obligations related to the Articles of Faith, the Community Life Statement, or professional ethics (2.2.4.2.1).
2.2.4.2.1 Definitions and Principles
   Discipline for ÒViolation of ContractÓ is intended to hold faculty accountable for contractual obligations related to the Articles of Faith, Community Life Statement, and professional ethics.  For the sake of the accused faculty member and the community, allegations about such violations are best kept confidential, if at all possible.
2.2.4.2.2 Procedures
a) A colleague, a student, or an administrator may present an allegation of this sort Ð usually, in the first instance, to the faculty member involved.  If the faculty memberÕs response fully satisfies the complainant, the matter may be considered as resolved.  Otherwise, or in the case where there is good reason not to address the faculty member directly, the allegation should be presented to the Provost, who will present the charges to the faculty member by way of attempting an informal resolution.
b) The Provost, once involved, will investigate in a manner appropriate to the severity and circumstances of the allegation, and in accordance with the law.  The faculty member, after being apprised of the charges, has the right to a meeting with the Provost to explain, interpret, and present evidence with another faculty member present as advocate.  At the ProvostÕs discretion, the faculty memberÕs Department Chair may also be present (or the Dean of Curriculum and Educational Effectiveness, if the allegation involves the Department Chair), and also the Vice President for Student Life and Dean of Student Life, if the allegation involves students.  At this stage no one else will be made privy to the case.
   If appropriate investigation shows the allegation to lack substance, no further action will be taken.  The matter will not be discussed beyond the circle of those already privy to the case (see prior paragraph), and no record will be made in the faculty memberÕs file in the ProvostÕs office.
c) If the allegation is found to have substance, but no potential for termination of contract, the Provost may resolve the problem administratively without enlarging the circle of those privy to the case.  The terms of such a resolution, including any sanctions, will be put in writing by the Provost, signed by the faculty member, and placed in the faculty memberÕs file in the ProvostÕs office.  Possible sanctions include, but are not limited to, those specified for non-performance of contract (Section 2.2.4.1 c).  In case of a need for reconciliation between the faculty member and someone else, the Provost will serve as mediator unless the principals agree instead to follow the procedures of the Faculty Handbook, section 2.10, ÒProcedures for Resolving Grievances.Ó
d) If the allegation is found to have substance with a potential for termination of contract, the Provost will immediately refer the case to the Faculty Council Ð this with a view to preserving for the faculty member both the benefit of doubt and the prospect of full restoration with dignity.  Together the Provost and the Faculty Council will determine whether the Provost or the Faculty Council should pursue the case to a resolution.  Their choice will be guided by consideration for the best interest of the faculty member, the College community, and any other interested parties; but the latter option will be necessary if by this stage agreement cannot be reached as to the nature and severity of the allegations and/or the discipline appropriate to the case.
1) On the former option the Provost will make final determinations of discipline in consultation with the Faculty Council.  The written terms of such a resolution, including any sanctions, will be produced by the Provost, signed by the faculty member, and placed in the faculty memberÕs file in the ProvostÕs office.  Possible sanctions include, but are not limited to, those specified for non-performance of contract (Section 2.2.4.1 c ).  In case of a need for reconciliation between the faculty member and anyone else, the Provost will serve as mediator unless the principals agree instead to follow the procedures of the Faculty Handbook, Section 2.10, ÒProcedures for Resolving Grievances.Ó  It is to be hoped that most such cases will be resolved without resort to discharge for cause proceedings.
2) On the latter option, the Faculty Council may embark directly on the procedures of discharge for cause (Section 2.2.4.4) by presenting the charges in writing to the Provost, who will inform the faculty member and the President.  The Faculty Council, however, is not bound to treat the case as one of discharge for cause; if they choose not to do so, they will make final determinations of discipline in consultation with the Provost.
2.2.4.2.3 Appeal
   At any stage prior to the ProvostÕs bringing the case to the Faculty Council, the faculty member may do so by personal initiative, according to the formal procedure provided by Section 2.10 for resolving grievances.
2.2.4.3 Suspension
   When charges are brought that could result in termination of contract, a faculty member may be suspended, or assigned to other duties in lieu of suspension, but only where there is clear danger of immediate harm to the faculty member or others.  The Provost will consult with Faculty Council in determining the propriety of suspension as well as its length and other conditions. 
2.2.4.4 Discharge for Cause Hearing
   Confidentiality shall be maintained throughout the hearing process.  When charges include departure from the Articles of Faith, particular and careful attention must be given to Section 2.4.1.1, ÒAcademic Freedom.Ó
2.2.4.4.1 Formation and Membership of the Hearing Committee
   The Hearing Committee will be formed by one of the two following procedures, to be chosen by the faculty member against whom the charges are being brought:
2.2.4.4.1.1 Hearing Committee Formation, Option 1
a) In consultation with the Provost, Faculty Council will compose the hearing committee of the five most recently elected past members of the Faculty Council who are currently at the College, who are not on sabbatical, and who have completed the term of their subsequent committee release for service on the Faculty Council (Section 1.4.3.7 b 1 ), subject to the condition that at least three of the five must be tenured.  Of two or more faculty who were elected to the Faculty Council at the same time, the one(s) with greater seniority at the College will be chosen.  Prospective members will be given three daysÕ opportunity to recuse themselves; and the Faculty Council will make the sole determination as to what constitutes adequate cause.
b) After having found five consenting members for the committee, the Faculty Council will identify them to the President and to the faculty member.  All challenges to the committee membership must be submitted in writing to the Faculty Council within the seven days following this notification.  Parties on either side may make unlimited challenges for cause, and each side will be allowed no more than two challenges without cause, within the seven-day period.  The Faculty Council will make the sole determination upon each challenge and the source and identity of any successful challenge will be kept confidential.  If a challenge is filed after the deadline, Faculty Council will decide the case in consultation with the Provost, taking into account the specific circumstances surrounding the late request.  After membership substitutions occasioned by challenges have ended, the Faculty Council identify to each member the other members of the committee, and call them to their first meeting (Section 2.2.4.4.2 a 1).
c) Faculty Council will make the sole determination upon membersÕ requests to be excused from the committee at any time subsequent to its final formation as described in Section 2.2.4.4.1.1 b), and will fill all vacancies in accord with the procedures of Section 2.2.4.4.1.1 a).
2.2.4.4.1.2 Hearing Committee Formation, Option 2
a) Membership will be five faculty members, four of whom must be tenured, and three of whom must be full professors.
b) A slate of nominees is received in an executive session of the faculty convened by the Faculty Council.  Nominees who deem themselves disqualified for reasons of bias or conflict of interest will withdraw their names from nomination, either at the request of one of the parties or on their own initiative.  The number of nominees must be greater than the number of committee members in order to provide Faculty Council a body from which to select alternates should a committee member replacement be necessary.
c) A committee is elected by secret ballot in an executive session of the faculty.
d) After having found five consenting members for the committee, the Faculty Council will identify them to the President and to the faculty member.  All challenges to the committee membership must be submitted in writing to the Faculty Council within the seven days following this notification.  Parties on either side may make unlimited challenges for cause, and each side will be allowed no more than two challenges without cause, within the seven-day period.  The Faculty Council will make the sole determination upon each challenge, and the source and identity of any successful challenge will be kept confidential.  If a challenge is filed after the deadline, Faculty Council will decide the case in consultation with the Provost, taking into account the specific circumstances surrounding the late request.  Vacancies created by challenge will be filled by Faculty Council from among the un-elected nominees in reverse order of votes acquired, within constraints imposed by committee membership requirements.
2.2.4.4.2 Procedures of the Hearing Committee
a) Preliminary Procedures
1) At the first meeting of the hearing committee, the Faculty Council and the Provost will inform them for the first of the identity of the accused, and will furnish the committee with a written summary of the specific charges, of the main facts of the case, and of its stages thus far, as well as with all written accusations from aggrieved parties. The Faculty Council will impress upon the committee the importance of the principle that no presumption of guilt is to be inferred from the mere fact that all the procedures undertaken thus far in the case have not achieved a resolution.
2) The hearing committee will choose their chair from among the tenured members.
3) The hearing committee will then furnish the documents described in Section 2.2.4.4.2 a 1 above to the accused faculty member and all other parties involved, with notification of the date for the formal hearing, giving at least thirty daysÕ advance notice, but ensuring also that the date of the hearing follows without undue delay upon the thirty daysÕ interval.  At any time prior to the formal hearing, the accused faculty member may direct to the hearing committee written response(s) to these documents.
4) The faculty member may, by written notice to the hearing committee, waive the hearing at any time before its commencement.  If the faculty waives the hearing but denies the charges or asserts that the charges do not support a finding of adequate cause for termination, the hearing committee will seek out and evaluate all the available evidence and rest its recommendation (see Sections 2.2.4.4.2 d)7) and d)8) ) upon the evidence in the record.
b) Procedural Conditions:
1) During all proceedings under 2.2.4.4.2 the faculty member will be entitled to have present as an advocate a faculty colleague, or any on- or off-campus advisor who is approved by the committee.  Legal counsel for the faculty member may be present, but only to advise the faculty member, not to address the committee directly.
2) As appropriate to their knowledge of the case, the committee will invite witnesses to appear, and after due consultation with the interested parties will make the final determination as to who qualifies as a witness.
3) Witnesses may be questioned by the members of the hearing committee, by the accused faculty member, by the accused faculty memberÕs designated faculty advocate and on- or off-campus advisor (see subsection b)1) above), and/or by a representative of the College administration.
4) The administration of the College may seek legal guidance from an attorney, but may not be represented in the hearing proceedings by any attorney, including any College administrator who has a law degree.  An attorney for the administration may be present, but only to advise the administrationÕs representative, not to address the committee directly.
5) A certified recorder will make a verbatim record of the pre-hearing and hearing.
   {On the faculty memberÕs entitlement to a copy, see Section 2.2.4.4.2 d 8. }
c) Pre-Hearing:
1) If the faculty member does not waive the hearing, within 10 days of their serving notice of the formal hearing, the committee will conduct a pre-hearing for all the parties involved, and will give them at least five dayÕs advance notice.  The purposes of the pre-hearing are for the committee to
(a) ensure that all parties understand clearly what is at issue
(b) ensure that all parties have access to the same documents and information
(c) provide for any further exchange of information that the committee deems necessary as a preliminary to the formal hearing
(d) make any other preparations for the hearing that will serve to make it fair, effective, and expeditious
(e) apprise the parties involved of the procedures that they have determined should govern the conduct of the formal hearing
(f) confirm the date, time, and place for the commencement of the hearing; or, if necessary, to alter the date originally announced, though not to make it earlier
2) In consultation with the committee, the faculty member may decide that the formal hearing will be private, or, otherwise, will decide who is to be admitted to the formal hearing as observer(s).
3) At the request of any of the involved parties or the committee, a representative of a recognized educational association will be permitted to attend the formal hearing as an observer.
d) Hearing:
1) The burden of proof that adequate cause for discharge exists will rest with the institution and will be satisfied only by clear and convincing evidence in the record considered as a whole.
2) The committee will grant adjournments to enable either party to investigate evidence about which a valid claim of surprise is made.
3) The faculty member will be allowed time to obtain necessary witnesses and documentary or other evidence.  The administration will cooperate with the committee in securing witnesses and making available documentary and other evidence.
4) The faculty member, the designate faculty advocate and on- or off-campus advisor (2.2.4.4.2  b)1) ), and the College administrationÕs representative will have the right to confront, question, and cross-examine all witnesses.  If witnesses are unable or unwilling to appear, and yet the committee determines that the interests of justice require their statements, the committee will identify them, disclose their statements, and if possible provide for interrogatories.
5) In the hearing of charges of incompetence, the testimony will include that of qualified faculty members from the College and/or other institutions of higher education as chosen by the Faculty Council.
6) The committee will not be bound by strict rules of legal evidence, and may admit any evidence which is of probative value for the case.
7) The findings of fact and the recommendation will be based solely on the hearing record.
If the committee concludes that adequate cause for dismissal has been established, it will recommend either dismissal of the faculty member, or an academic penalty less than dismissal (e.g., Section 2.2.4.1 c) ).  If the committee concludes that adequate cause for dismissal has not been established, it will recommend retention of the faculty member either with no penalties or with an academic penalty less than dismissal.
8) The committee will convey its findings and recommendation to the President and the faculty member in writing, and will provide them both with a copy of the record of the pre-hearing and the hearing.  [On the pre-hearing copy, see section 2.2.4.4.2 b)5).]
9) The President, if rejecting the committeeÕs recommendation, will convey the reasons in writing to the committee and to the faculty member, and will provide an opportunity for response by the committee, and/or by the faculty member through the committee, before transmitting the case to the Board of Trustees who will make the final decision.
e) Confidentiality:
1) Prior to the hearing, except for simple announcements that may be required, covering the time of the hearing and similar matters, public statements and publicity about the case by the faculty member, administrative officers or members of the committee will be avoided.  If the hearing is public, as provided in Section 2.2.4.4.2 c)2) ), no one will make public statements about the case prior to the hearing, except to announce its time and place, and similar matters.
2) If the hearing is private, as provided in Section 2.2.4.4.2 c)2), no one besides the faculty member may make any part of the proceedings public.
3) In either case, the committee will not make public the documents of the case or its deliberations
4) The Faculty Council will address infractions of these policies.
2.2.5 Personnel Records
2.2.5.1 Purpose and Location
   For each faculty member a file is maintained in the office of the Provost containing information relative to, and the results of, appointment, promotion, and tenure review.
2.2.5.2 Content
   Personnel files contain:
a) conditions of appointment
b) reports pertaining to promotion and tenure reviews
c) communications regarding professional performance
d) conditions and reports related to disciplinary action
e) current curriculum vitae
f) current statements of philosophy of education and personal faith
2.2.5.3 Access
   Personnel records will be maintained in compliance with pertinent federal and state laws.  Faculty are encouraged to review periodically the contents of their personnel files.
2.2.5.4 Submissions
   A faculty member may submit any material he or she deems pertinent to his or her personnel file.
2.2.5.5 Exclusions
   No communication for which authorship is not given may be placed in a faculty memberÕs personnel file.
2.2.5.6 Expungement
   An individual has the right to challenge and request expungement of information to which he or she has access; at the discretion of the Provost, these materials may be removed.  In the event they are not expunged, the faculty member may place in his or her file information responding to the material in question.
2.3 Separation
2.3.1 Resignation
a) All faculty are expected to fulfill the terms of their contracts.  Resignations to take effect during the term of a contract are permissible only under conditions mutually acceptable to the individual and the College.  Contract pay will be prorated based on service rendered.
b) A faculty member with a multi-year, notice or continuous appointment should give written notice of intent not to accept renewal of appointment at the earliest possible opportunity but not later than April 15, or 30 days after receiving notice of the terms of appointment for the coming year, whichever date occurs later. In unusual circumstances, a waiver of this requirement of written notice may be granted.
2.3.2 Retirement
   Ordinarily, retirement starts at the end of the academic year.  A decision to retire should be communicated in writing to the College as far in advance as possible.  Retired faculty are encouraged to remain active in the life of the College community.  To encourage this, to the extent available and practicable, the retiree may be granted the following privileges:  receipt of mail; access to computers, laboratories, the library, inter-library loans, and a library cubicle; office space; and, as determined by the departmental chair and the Provost, secretarial assistance in the preparation of scholarly manuscripts.
2.3.2.1 Early Retirement
   An individual qualifies for early retirement at 62 years of age and 10 consecutive years of faculty service.  Provisions for, policies governing and benefits of early retirement are found in Section 2.8.
2.3.3 Non-Reappointment
2.3.3.1 Notification
2.3.3.1.1 Term Contracts
a) Temporary: Written notification of non-renewal is not given.  Reappointment will not be presumed by faculty with temporary appointments unless a new temporary contract has been issued by the Provost.
b) Multi-Year: Written notification of non-renewal will be given no later than December 15 of the last academic year of the contract.
2.3.3.1.2 Notice Contracts (Probationary Appointments)
   Written notification of non-reappointment will be given to the faculty member in advance of the expiration of the current contract as follows:
a) not later than March 1 of the first academic year of service;
b) not later than December 15 of the second academic year of service;
c) In cases of non-reappointment after December 15 of the second year of service and during subsequent years of service, the faculty member will be notified of non-reappointment not later than the end of the spring semester and will be given a terminal 12-month contract with duties to be assigned at the discretion of the Provost. 
2.3.3.2 Reasons for Non-Reappointment of Probationary Faculty Member
a) Although there is no presumption of continued employment for a faculty member under notice contract, such a faculty member notified of non-reappointment may reasonably be expected to ask about the reasons for non-reappointment.  Upon written request, and in the spirit of fairness, the Provost may disclose those reasons to the faculty member; however, neither the Provost nor any other agent of the College is obligated to disclose the reasons or to justify them.
b) Though the faculty member may request that the reasons for non-reappointment be stated in writing, it is inappropriate to require that every notice of non-reappointment be accompanied by a written statement of the reasons for non-reappointment.  In some instances it may be difficult for the College to provide such a statement; in others, it may be in the best interest of the faculty member not to have the reasons given in written form.  These considerations will be discussed with the faculty member.
2.3.3.3 Review of Decision Not to Issue a Notice Contract
   This section provides recourse when a faculty member wishes to contest any negative decision concerning renewal of contract, including a negative tenure decision (Section 2.2.3.4 Appeal of Tenure Decision).
a) A faculty member who has been notified of a decision not to offer a new notice contract may make written appeal to the Faculty Council.  Such a faculty member bears the responsibility for establishing a prima facie case and assumes the burden of proof in support of the appeal.
b) Bases for appeal are limited to: (1) violation of academic freedom, (2) discrimination as defined in the collegeÕs policy on unlawful discrimination and harassment, and (3) failure by the College to abide by institutional policies stated in the Faculty Handbook.
c) The faculty member may choose to confer with others, not including members of the Faculty Council, in marshalling evidence and writing the appeal.
d) If the Faculty Council is convinced that a prima facie case has been established
1) Faculty Council will investigate the merits of the appeal under Section 2.3.3.3 b), collaborating as appropriate with other campus entities with overlapping jurisdiction (e.g. Title IX Officer).
2) It is incumbent on those who made the non-reappointment decision to communicate with Faculty Council to address any alleged violations of 2.3.3.3 b) (1) and (2), and to demonstrate their compliance with the institutional policies stated in the Faculty Handbook.
e) The Faculty Council will render a judgment as to the merits of the appeal under 2.3.3.3 b).
f) When the Faculty Council is unable to determine that compliance under Section 2.3.3.3.b) has occurred, the effect is not to recommend the issuance of a new notice contract, but to recommend a review of the case by those who decided not to issue a new contract.
g) In cases where discrimination is alleged, legal counsel will be sought on behalf of the College prior to making any verbal or written response to the faculty member.
h) The decision of the Faculty Council will be rendered as a recommendation made through the Provost to the President, and will be reported in writing directly to the faculty member
2.3.4 Reduction in Force
   Reduction in the size of the faculty of the College can have serious consequences for individual faculty members as well as the quality and diversity of the academic program as a whole.  Therefore, when there is reason for a general reduction of personnel, as determined by the Administration and the Board of Trustees, there is also the expectation that proposed alternatives to a reduction will have been explored, that the sacrifices asked of the College will be in consideration of the centrality of the faculty to the life of an academic institution, and that any distribution of reductions within the faculty will be equitable and just.  While it is difficult to be specific about the nature of reductions, the following principles and procedures will apply:
a) Principles:
1) Recognition of the faculty as the qualified, principal guardians of the academic program;
2) Provisions for the participation of the faculty and role of due process in modifications in the academic program, and the right of appropriate appeals by affected individuals;
3) Consideration of the following: tenure, rank, and seniority in service along with teaching effectiveness of individual faculty, the fit of a particular discipline within the mission of the College, and the balance and quality of the total academic program, in making adjustments.
b) Procedures:
1) At the time it is necessary to consider a general reduction in college personnel, the President and Vice Presidents will meet with a faculty body consisting of the Faculty Council plus the two members of the Faculty Budget and Salary Committee who are on the PresidentÕs Advisory Council.  If it is determined that faculty positions will be affected, this administration-faculty group will assist the President to determine the general priorities and criteria for such reductions. A report of the discussions and reduction priorities and criteria will be presented by the Faculty Council at a meeting of the full faculty.
2) The statement of priorities and criteria will then be given to the Academic Senate of the faculty.  The Academic Senate will establish a closed ballot of five members selected from the full-time faculty from each division.  The Provost will be an ex-officio member of the reduction-in-force committee thus elected by the faculty.  The statement of priorities and criteria (see 1) above) will be transmitted to the reduction-in-force committee which will consider both programs and individuals and will have the same access to information as the Faculty Personnel Committee.  Faculty members under consideration for layoff will be notified and will have the opportunity to meet with the committee to appeal prior to its final recommendations.  Recommendations of the reduction-in-force committee will be transmitted by the Provost to the President. At the time the PresidentÕs recommendations are sent to the Board of Trustees, those recommendations will be presented at a meeting of the full faculty.
3) Academic departments and individual faculty members affected by reductions may appeal the decisions of the College through the Faculty Council on procedural grounds only.
2.4 Faculty Rights and Responsibilities
2.4.1 Professional
2.4.1.1 Academic Freedom
a) As an institution of higher learning, Westmont College exists to advance the work of Christ and His church, to seek the truth, and to promote the common good.  To further these ends, the College affirms the centrality of freedom of thought and expression in liberal education.  Academic freedom is essential to the facultyÕs primary tasks; it promotes and protects faculty rights of inquiry and expression as they perform their duties as scholars and educators, as well as ensuring studentsÕ freedom to learn.  Thus, the faculty, within the framework of and in accord with the Articles of Faith, are entitled to the rights and privileges and bear the obligations of academic freedom.
b) Westmont College and its faculty accept and abide by the following:
1) Faculty are entitled to full freedom in research and in the publication of the results, subject to the adequate performance of their other academic duties; however, research conducted solely for pecuniary return, when the faculty member is under full-time contract with the College, should be based upon a written understanding with the authorities of the College (see also Section 2.4.1.3).
2) Faculty are entitled to freedom in the classroom in discussing their subject, but they should be careful not to make the classroom a vehicle for a personal agenda that is not integral with the discipline or the liberal arts curriculum.
3) Intentional, substantive denial of the Articles of Faith constitutes a violation of contract; in the event that such violations are alleged, the burden of proof rests with those bringing allegations.  Any other qualifications of academic freedom must be clearly stated in writing at the time of the appointment.
4) Faculty are citizens, members of a learned profession, and officers of an educational institution.  When they speak or write as citizens, they should be free from institutional censorship or discipline.  As scholars and educational officers, they should also remember that the public may judge their profession and the College by their utterances.  Hence they should be accurate, exercise appropriate restraint, respect the opinions of others, and exercise care in attributing privately-held views to the College.
2.4.1.2 Professional Ethics and Relationships
   Faculty members should exemplify ethical principles of conduct in living and scholarship, promoting Christian ideals and the common welfare of the College.  These principles have implications for interpersonal relationships, the integrity of oneÕs word, and confidentiality.  Faculty are to hold confidences as appropriate and to the extent allowed under the circumstances and by law.  Furthermore, as a matter of Christian and lawful practice, harassment or discrimination (on the basis of race, age, sex, or other unlawful discrimination) in relationships with students, staff, colleagues or administrators will not be tolerated.
a) With Students:  Faculty members should model for students personal maturity in spiritual, intellectual and social relationships.  Students are to be co-learners with faculty, worthy of courteous, just, and impartial treatment.  Although faculty are called upon to profess, this is done with the understanding that the imposition of personal views on students is contrary to the spirit and process of liberal education; the faculty, therefore, should grant the same freedom of inquiry and conclusion which they presume for themselves.  In faculty-student relationships the well-being of the student is paramount as, for example, in academic counseling where the best interests of students take precedence over obtaining majors in oneÕs discipline or increasing course enrollments.  Information in possession of a faculty member is not necessarily open to the student, but once placed in the studentÕs official College file it is available to him or her in accordance with applicable legal regulations.
b) With Colleagues:  In concert with colleagues and others, faculty have a responsibility to participate in the life of the College.  An academic life is nourished and sustained by vigorous discussion of perspectives and methods; thus, all faculty are called to support each memberÕs right to engage in discussion and to honor the privilege of presenting opposing points of view.  Also, recognizing the fragile nature of community relationships, faculty should refrain from undermining or demeaning, directly or by implication, the character, work or academic discipline of a colleague. When personal or collegial differences arise, all faculty are to follow the injunction of first taking their differences to the colleague(s) involved.  Out of responsibility to the College and to the personal and professional development of a colleague, these understandings are not to preclude honest and candid evaluations in the promotion and tenure process.
c) With Administrators:  Faculty and administrative relationships grow out of shared stewardship of the College.  Faculty share in governance through their advisory role to administrators.  Faculty perspectives, presented individually and collegially, are important in defining and enacting the mission of the College.  Likewise, administrators provide counsel and encouragement, as well as material support to the faculty.
d) With Staff:  Faculty acknowledge the significant contributions made by people in staff positions who also share in the stewardship of the college community.  As expected of all intramural associations, faculty are to be courteous and considerate in their relationships with staff, expressing appreciation and endeavoring to resolve problems through appropriate channels of authority and responsibility.
e) With the General Community:  Faculty can render important services to the general community as an expression of Christian social responsibility.  Acting as private citizens but also, when proper, as representatives of the College, faculty should seek opportunities to become involved in the life of the community as teachers and scholars bringing the Gospel to bear on a broken world.
f) With the Church:  Westmont College values and encourages a variety of denominational affiliations among its members.  Faculty should not limit their ministry to teaching, research and fellowship in the College and the general community but are expected to be worshipping participants in local congregations and, according to individual gifts, to serve the varied mission of the Church.
2.4.1.3 Non-College Activities by Full-Time Faculty
   On-going professional activities or extramural employment that involve substantial time commitments in addition to oneÕs contractual responsibilities require approval of the department chair and the Provost.  Employment or professional activities that create additional burdens for colleagues, detract from oneÕs instructional and other obligations, or result in absence from the campus for more than the equivalent of one day during the school week will not be approved (see Section 2.4.1.1).
2.4.2 Instructional
2.4.2.1 Teaching
a) The normal teaching load for full-time faculty is 12 credit hours in each semester.  At the request of the department chair and with the approval of the Provost a faculty member may agree to teach one additional course per semester.  Remuneration for course loads in addition to the normal 12 credit hours is at the same rate as that for part-time instructors.  Specific assignments are made by the department chair in consultation with the faculty member and the Provost.  Directed readings and tutorials, which are not required of a faculty member and for which there is no additional remuneration, are not included when computing the number of credit hours.  Directed readings and tutorials may not exceed three students or six credit hours in any given semester.  Department chairs receive four hours credit per year toward their teaching load.  Occasionally, the teaching assignment for a faculty member may be reduced in a given term to permit completion of a special project.
b) Every faculty member is required to prepare a syllabus for each course and to submit it to the office of the Provost during the first week of classes of each term.  A syllabus should include the topics covered through the semester, required or recommended readings, major assignments, an examination schedule, and any special information regarding mode of evaluation or instruction that may be appropriate.  A course syllabus should be considered as a contract with the class; changes in a syllabus during a semester should be negotiated to the mutual satisfaction of the students and the instructor.
c) Evaluation of student work constitutes a major responsibility of faculty members and should be conducted in a professional and impartial manner.  The faculty member should abide by the grading system and standards of the College.
d) Every faculty member is required to maintain an accurate record of each studentÕs progress within a course.  Course records for the last two semesters must be filed with the registrarÕs office if the faculty member leaves the employ of the College.
e) Faculty members who are not tenured full professors are required to administer course evaluations for every class.
2.4.2.2 Advising/Office Hours
a) Advising students in relation to their academic programs and professional goals is a principal responsibility of each faculty member.  The advising relationship should be approached as a personal and professional commitment to each advisee for the duration of his or her college experience.  Therefore, each faculty member is expected to be knowledgeable about the academic policies of the College and department including General Education requirements and those for the major.
b) Faculty should be available to students on a regular basis apart from class meetings. Office hours for a minimum of five hours per week should be posted and should take into consideration morning and afternoon class schedules.
2.4.3 Institutional
2.4.3.1 Participation in Campus Governance
   Committee service is a principal means by which faculty participate in the formulation of policies and in the governance of the College.  All faculty members, at the request of the administration or Faculty Council, are expected to serve within the committee system.  The expressed interests of faculty members will be honored whenever possible when committee assignments are made (1.4.3.2.5 a) ).  Procedures for selecting committee memberships and descriptions of committee responsibilities are specified in Section 1.4.3.  Normally, no faculty member would serve on more than one major committee at any one time.  Assignment to special committees may be made from time to time by administrative personnel, in consultation with the Faculty Council.  A faculty member who fails to participate or whose participation is counterproductive to the work of the committee may be removed by the Faculty Council at the request of the committee chair.  Likewise, the Faculty Council may request that the Faculty replace the chair in cases of inadequate performance.
2.4.3.2 Other Non-Instructional Activities
   All faculty are required to attend faculty retreat, service of commitment, and commencement activities (Senior Awards Convocation, Baccalaureate, and Commencement).  Regular attendance at faculty meetings is required.  Faculty are encouraged to attend Chapel regularly as an expression of their commitment to the spiritual life of the college community.
2.4.3.3 Course Relief for Institutional Service
   Department Chairs and Vice-Chair of the Faculty will receive a one-course load reduction during each year that they serve in those institutional capacities.  Under extraordinary circumstances, and at the discretion of the Provost, these and other faculty members may also be granted a temporary course-load reduction for institutional service.
2.5 Faculty Development
2.5.1 Leave
2.5.1.1 Sabbatical
a) Purpose: Paid sabbatical leaves for scholarly activities are available to faculty members with tenure or long-term multi-year contracts.  A sabbatical leave is an investment by the College for increasing the quality of instruction and scholarship through the professional enrichment of the faculty.  A sabbatical leave is normally not granted for work toward completion of a degree.
b) Eligibility and General Provisions:
1) For tenured faculty, a minimum of six years (or equivalent) of full-time service since initial appointment or the most recent sabbatical or terminal degree leave is required.  In consultation with the Provost, credit toward a subsequent sabbatical may be granted if more than six years intervened between previous sabbaticals.  As provided for in the letter of appointment, credit up to two years toward a sabbatical may be given to faculty members with previous college or university experience.  Time on leave from the institution does not count toward eligibility.
2) Faculty on multi-year contracts are eligible for a sabbatical during the ninth year of employment, after completing three consecutive multi-year contract periods.  Multi-year contract faculty will be eligible for additional sabbaticals according to conditions specified for tenured faculty.  See 1) above.
3) The faculty member may request sabbatical leave at full salary for one-half of the academic year or half salary for a full academic year.
4) A faculty member on sabbatical leave continues to be eligible to participate in benefit programs of the College.
5) Approval must be received from the Provost whenever professional activities depart from the approved project or before additional employment is accepted during the leave.
6) A faculty member granted sabbatical leave is contractually committed to return to the College for a period of one full year after the academic year in which the sabbatical was taken.  If such service is not completed, upon separation all sabbatical compensation is to be repaid.
c) Procedures:
1) Applications should be submitted to the Provost before October 1 for sabbatical leave during the following academic year.  All requests for sabbaticals must be approved by the Professional Development Committee with the concurrence of the Provost, President and the Board of Trustees.  Applicants will be notified of the Professional Development CommitteeÕs recommendation before December 1.  Notification of approval by the Provost, President, and the Board of Trustees will come by February 1. If the number of applications exceeds the allotment for a given year priority in awarding sabbaticals will be made on the basis of the significance of the proposed project to the professional development of the individual and to the College as determined by the Professional Development Committee.
2) Within three months of concluding a leave, the recipient will submit a report to the Professional Development Committee on activities and achievements while on leave.  Reports will be posted on the ProvostÕs web page (accessible only to viewers with a Westmont login identification).  Each sabbatical recipient will give an oral report to the faculty in the year following the sabbatical.  
2.5.1.2 Educational
a) Academic Leave:  The College encourages occasional academic leaves for faculty, especially in cases of faculty exchanges or academic fellowships.  A faculty member, with the support of the departmental chair, submits a proposal to the Professional Development Committee through the Provost.  Preference is given to proposals which reflect values consonant with the mission of the College, such as the relationship of faith to the discipline, interdisciplinary connections, cross-cultural dimensions, the position of Christian higher education in the context of higher education in general.  In the rare cases where faculty are hired without the completion of the terminal degree required for their teaching position, the ProvostÕs office may make leave available to full-time faculty on notice contracts.
1) Normally, such leaves do not involve compensation.  Because there is no salary from the College, no contributions are made to retirement plans or to F.I.C.A., nor are faculty eligible for unemployment compensation or state disability insurance.
2) The person, however, is considered to be a continuing member of the faculty. Normal progress in rank is maintained, although the time away does not count toward sabbatical leave.  To the extent that institutional benefits, policies and providers allow, medical, dental, life, long-term disability, and travel accident coverage will be provided under the normal conditions established for all faculty members. 
2.5.2 Conferences and Travel
   Funds are available from the Provost for travel related to scholarly work and participation in professional programs.  See Section 5.7 for current policies.
2.5.3 Professional Development
a) Faculty Mentoring:  As a resource for newly hired full-time faculty, a mentor will be appointed from outside the department (selected by the Provost in consultation with the department chair) to help the faculty member adjust to institutional practices and expectations, to answer questions as they arise, and to create a safe space for asking questions and addressing difficulties.  While the nature of the mentoring relationship is largely informal, a probationary faculty member may request that the mentor be involved in early probationary assessment meetings and processes.  Guidelines for mentors will be provided by the Provost's office.
b) Initial Departmental Assessment:  To develop newly hired full-time faculty, a departmental assessment shall be conducted in the second year of employment.
1) The assessment shall occur in the fall semester of the second year and shall consist of both a written summary and a meeting among the probationary faculty member, the department chair, and the Provost.  Such meeting shall be initiated and scheduled by the ProvostÕs office.  In the event that the department chair having the most relevant information is off-campus or no longer serving as chair when the assessment must take place, the Provost (in consultation with the department and faculty member) shall designate the most appropriate person to submit the written assessment and attend the required meeting.  The probationary faculty member may request that the mentor be included in the assessment process and/or the meeting.
2) The probationary faculty member will provide the chair with a progress portfolio not later than the first day of class of the faculty memberÕs second year of service. The progress portfolio shall include:  an up-to-date curriculum vitae, a 2-3-page self-assessment addressing his or her performance in the first year according to the full range of responsibilities outlined in the Faculty Handbook section 2.2.1, and any evidence which seems appropriate to the self-assessment.
3) Following receipt of the progress portfolio, the department chair shall provide the ProvostÕs office and the probationary faculty member a written assessment, identifying both strengths and areas for improvement, relative to section 2.2.1 in particular, and requirements for promotion and tenure in general.  In preparation for the written assessment, the department chair shall:  attend no fewer than two of the probationary faculty memberÕs classes, preferably in two different courses and in two different semesters, during the faculty memberÕs first year of service, paying particular attention both to professional competence and general pedagogical effectiveness; read the entire set of the faculty memberÕs first-year teaching evaluations; solicit feedback from departmental colleagues and students, formally or informally; and meet with the probationary faculty member at least twice during the first year both to provide feedback and to solicit questions or concerns.
4) Within two weeks of the joint assessment meeting, any participant (chair, mentor, or probationary faculty member) may provide additional written response to the Provost.  Such responses will be placed in the probationary faculty memberÕs personnel file, together with the rest of the department assessment documents and a summary statement written by the Provost.
c) Fifth-Year Departmental Assessment:  The department chair and a probationary faculty member shall meet during the fall semester of the fifth year of probationary service in order to discuss progress toward tenure and promotion.  The summary statement arising out of the intermediate tenure review (conducted by the Personnel Committee) will serve as a catalyst and benchmark for discussion and goal-setting.  Following this meeting, the department chair will notify the ProvostÕs office that such meeting has taken place.
d) Curricular and Professional Projects:  As an encouragement for faculty development the College provides funds for curricular and professional projects.  Each year, faculty may submit requests to the chair of the Professional Development Committee. The Professional Development Committee will award proposals according to their merits and the availability of funds.
e) Summer Session Salary and Policies:  Prior to each summer session a salary schedule for summer teaching is established.  Summer classes may be offered by faculty either for additional salary or as partial completion of their regular academic year contract. Arrangements for summer session teaching are made through the office of the Provost.
f) Loans for Completion of a Terminal Degree:  In the rare cases where faculty are hired without the completion of the terminal degree required for their teaching position, the ProvostÕs office may make loans available to full-time faculty members on notice contracts. Subject to availability, funds are provided for such expenses as tuition, fees, books, required travel, and research costs.
1) Applications are submitted to the Provost.  The application should summarize the degree program with rationale and anticipated expenses.  The Provost will authorize funds prior to each term in which the faculty member is enrolled.
2) For each year of full-time service at Westmont College subsequent to the year in which the loan is received, a percentage of the loan is forgiven.  Prior to the sixth year of full-time service since initial appointment, the rate is 15%; for the sixth and following years the rate is 20%.
3) All outstanding loan balances are immediately payable if the faculty member leaves the employ of Westmont College.
g) Accountability of Full Professors:  After a faculty member becomes a Full Professor, he or she will participate every six years in a structured process of discussion, reflection, evaluation, and planning future goals. The purpose of this structured process is to encourage ongoing personal and professional development in all areas of service to the college,. One part of this process will involve meeting with a mutual mentoring group. Other parts will involve written reflection, student evaluations, class observation, and meeting with the department chair and provost and to provide feedback to the faculty member on strengths and on areas where growth and improvement may be called for. In more detail: 
1) During the fall semester, the faculty member will conduct standard class evaluations in all courses and discuss these with the department chair (or delegate chosen by the provost). The faculty member is free to use personally designed class evaluations in addition toÑbut not in place ofÑstandard class evaluations. 
2) In addition, the department chair (or delegate) will observe one or more of the faculty memberÕs classes and discuss observations with the faculty member. 
3) The provost will invite input on the faculty memberÕs work (addressing the areas outlined in section 2.2.1.2: Teaching, Professional Development, and Institutional Service) from all faculty. Such input should be submitted in writing to the provost. 
4) By October 1 of the year following, the department chair (or delegate chosen by the provost) will submit a written assessment of the faculty member to the provost, addressing the areas of accountability outlined in section 2.2.1.2: Teaching, Professional Development, and Institutional Service. 
5) The faculty member will participate in a mutual mentoring group. Each mutual mentoring group will be composed of 3-5 full professors, chosen to work together by the Professional Development Committee. The group will meet several times during the year. At each meeting, one faculty member will share with the group about his or her development during the previous six years in the areas of accountability outlined in section 2.2.1.2: Teaching, Professional Development, and Institutional Service of pedagogy, scholarship, and philosophy of education. This time can be used for exploring goals, sharing insights, and obtaining advice from other members of the group. Group members are encouragedÑthough not requiredÑto observe each otherÕs classes. 
6) Out of this experience with the mutual mentoring group, the faculty member will reflect in a three-page paper on his or her past development and planned future goals in the three areas of accountability outlined in section 2.2.1.2: Teaching, Professional Development, and Institutional Service. This paper is submitted to the Provost by October 1 of the year following the mutual mentoring year. Both the department chair (or delegate chosen by the provost) and the provost will subsequently interact with the faculty member about this paper, input from faculty colleagues, and the assessment submitted by the department chair. 
2.6 Working Conditions
2.6.1 Policy on Harassment
   This section provides recourse when anyone within the campus community of faculty, staff, students, and guests suffers unlawful harassment.
   Westmont College is committed to providing a learning and work environment free of unlawful harassment.  In keeping with this commitment, the College prohibits and will not tolerate unlawful harassment because of sex (which includes sexual harassment1, gender harassment and harassment due to pregnancy, childbirth or related medical condition) and harassment because of race, religious creed, color, national origin or ancestry, physical or mental disability, medical condition, marital status, age, sexual orientation or any other basis protected by federal, state, or local law, ordinance or regulation.  All such harassment is unlawful.
   Prohibited unlawful harassment includes, but is not limited to, the following behavior:
a) Verbal conduct such as epithets, derogatory jokes or comments, slurs or unwanted sexual advances, invitations or comments;
b) Visual conduct such as derogatory and/or sexually-oriented posters, photography, cartoons, drawings or gestures;
c) Physical conduct such as sexually-oriented gestures, assault, unwanted touching, blocking normal movement, or interfering with work because of sex, race or any other protected basis;
d) Threats and demands to submit to sexual requests as a condition of continued employment or academic advancement, or to avoid some other loss, and offers of employment benefits in return for sexual favors; and
e) Retaliation for having reported or threatened to report harassment.
   It is Westmont CollegeÕs policy to prohibit unlawful harassment within the campus community of faculty, staff, students and guests by any person and in any form, and to apply the procedures outlined below in responding to any complaints of harassment.  (The procedures outlined here shall preempt any other procedures set forth in various handbooks that may be deemed inconsistent with these.)  Westmont is committed to investigate promptly any complaints of harassment.  Where unlawful harassment is found to have occurred, the College will take appropriate disciplinary action reasonably calculated to end the harassment, up to and including termination of employment or expulsion from the College.
   A complaint of harassment may be reported to any of the following complaint recipients:  the Director or Associate Director of Human Resources, oneÕs immediate supervisor, the Department Chair, a Resident Director, the CollegeÕs Title IX Officer, any Dean or Vice President, or the President.  (A list of the people currently occupying these positions can be found in the catalog or received from the Office of the President or the Department of Human Resources.)
   A complaint recipient will notify the CollegeÕs Title IX Officer whenever a complaint of harassment has been received.  The Title IX Officer shall ensure that the complainant promptly receives a copy of this Policy and is clearly informed of his or her rights to assistance.
   The Title IX Officer will work with the Provost and the Vice President and Dean of Students2 to ensure that:
a) A complaint of harassment is promptly, fully and effectively investigated;
b) Whatever action is deemed necessary to end the unlawful harassment will be taken; and;
c) The determination and imposition of any sanctions is handled in accordance with existing procedures spelled out in the relevant handbook (e.g., Faculty Handbook provisions on Discharge for Cause).
   Confidentiality for both the complainant and the accused shall be encouraged and maintained as appropriate and to the extent allowed under the circumstances and by law.
   The Title IX Officer will keep the complainant informed about the process of investigating and responding to the complaint.  After action on a complaint is concluded, the complainant will be notified whether or not discipline will be imposed.  The College will not tolerate any reprisal or retaliation against someone who has submitted (or indicated an intent to submit) a complaint in good faith.
   Westmont encourages all members of the community to report any incidents of unlawful harassment immediately so that complaints can be resolved quickly.  In addition, any member of the community who believes that he or she has been harassed or retaliated against for resisting or complaining about harassment, may file a complaint with appropriate government agencies.  The nearest offices are listed in the telephone directory. The U.S. Department of EducationÕs Office of Civil Rights, the Federal Equal Employment Opportunity Commission and the California Department of Fair Employment and Housing investigate and prosecute complaints of prohibited harassment; currently, the statute of limitations for filing a claim with these agencies is 180 days, 300 days, and one year, respectively.
2.6.2 Policy on Faculty / Student Romantic Relationships
The college prohibits any full or part-time faculty member from engaging in a romantic and/or sexual relationship or activity with any current student to whom he or she is not married.
2.6.3 Policy on Drug-Free Campus
   You can find the college's Safe and Drug-Free Schools and Communities Act policy in full at https://www.westmont.edu/_offices/human_resources/on_campus/handbook/2_7drugfree.html.
2.6.4 Hazardous Materials
a) Flammable liquids (such as gasoline) may not be stored inside any buildings except laboratories and maintenance shops designed for this purpose.
b) No open flames (candles, lanterns, etc.) are permitted in any buildings except in laboratories and maintenance shops designed for their use.
c) Any spill of hazardous materials must be reported to Campus Security immediately.
d) No firearms or fireworks shall be carried, used or stored on campus.
e) In the event of a hazardous waste or utility problem, such as a gas leak or elevator failure, or other questions pertaining to safety policies, contact the Physical Plant Department or Campus Security.
2.6.5 Human Subjects in Research
a) Westmont College, recognizing the responsibility to safeguard the rights and welfare of human subjects involved in research, complies with the guidelines of the Department of Health and Human Services, American Psychological Association, and other guidelines appropriate to the academic discipline.  These principles are applicable to research conducted at or sponsored by the College, regardless of source of funding.  For these purposes, ÒresearchÓ means a systematic investigation designed to develop or contribute to generalizable knowledge.
b) The College maintains an Institutional Review Board (IRB) which reviews all non-exempt research projects conducted by College faculty, students, and staff.  The IRB has four members of the faculty, one elected annually to a four-year term, representing several disciplinary areas including both natural and social sciences and at least one non-scientific field.  One member will be appointed from the community at large (e.g., attorney, clergy, ethicist, etc.) in conformity with federal guidelines.
c) For non-exempt projects involving the use of human subjects, it is the responsibility of the project director to submit to the Dean of Curriculum and Educational Effectiveness (or another representative appointed by the Provost), a copy of WC Form C as well as copies of the protocol and consent forms to be used in the project.  These items and other relevant publications specifying exempt and non-exempt criteria and ethical guidelines are available from the IRB through the office of the Provost.
2.7 Leaves (Non-Professional)
2.7.1 Bereavement Leave
   In the event of the death of an immediate family member a faculty member may take bereavement leave with pay as arranged with the Department Chair and the Provost.
2.7.2 Extended Medical Leave
   Faculty work is seldom interrupted by illness, and coverage is usually possible with no real cost to the College by colleagues substituting for one another or by scheduling other times to meet with students.  However, paid extended medical leave is available to full-time regular faculty when non-work related personal illness or injury prevents them from fulfilling their responsibilities for more than three consecutive days.  In addition, faculty members may use up to five accrued extended medical leave days per year to care for a sick or injured member of the faculty memberÕs household if that person is unable to care for himself/herself.
a) Faculty begin their service with a reserve of 20 days of paid medical leave.  After two years of service, paid medical leave will accrue at the rate of 1.25 days for each month of completed academic service, with a maximum accrual of 10 days per year.  Extended medical leave is used at the rate of five days per week for the duration of oneÕs illness. The maximum number of accrued days that would be charged for medical reasons in a 12 month period is 160 (32 weeks times five days).
b) Additional leave of up to one semester with full pay may be extended to the faculty member at the discretion of the Provost in consultation with the Department Chair.  In such a case the faculty member will cover that portion of the semesterÕs teaching responsibility or other duties through one or more of the following options with no additional compensation:
5) teach an additional course in a semester prior and/or following the leave;
6) teach one or two courses in one or more Mayterms;
7) undertake other projects or work as assigned by the department chair and/or the Provost.
c) Any advanced extended medical leave remaining upon termination of employment will be repaid at a rate based on the current salary for an overload course.
   The number of full courses or the amount of equivalent work for which the faculty member will be responsible will be determined according to the number of days of accrued paid leave used during the leave and the amount of State Disability Insurance (SDI) available to the individual.
d) Personal medical leave is coordinated with State Disability Insurance (SDI) payments for non-work related conditions. In order to receive short-term disability benefits, the disabled employee must file for California State Disability Insurance (SDI) benefits.  In cases of non-work related leaves exceeding 90 days, Long Term Disability Insurance (LTD) may apply.  The Human Resources Department should be consulted prior to the leave for more information concerning eligibility for and use of SDI and LTD.  
e) The maximum time that a faculty member may be granted medical leave, paid or unpaid, is two semesters beyond the semester in which the leave began.  Continuing employment beyond that time will depend on the ability of the faculty member to teach and on the conditions of the previous contract.
f) Unused medical leave may be carried forward for future use with no limit to the number of days in reserve but will be forfeited upon termination of employment.
   
2.7.3 Parental Benefit and Leave Policy
   Employees are entitled to certain benefits in the case of birth, adoption, or fostering of a child under the federal Family Medical Leave Act (FMLA), California Family Rights Act (CFRA), and other law. Information on federal and state family leave laws can be obtained from WestmontÕs Employee Handbook and from the Human Resources office. FMLA and CFRA entitle an employee to up to 12 weeks of unpaid job protected leave per twelve-month period within one year of the childÕs birth, adoption, or start of foster care, and up to 12 weeks combined leave where both parents are Westmont employees. Pregnancy disability leave entitles employees to up to four months of job protected leave per pregnancy that is unpaid but potentially funded by State Disability Insurance (SDI).
   Westmont College provides qualifying faculty with certain options for taking these leaves and with additional benefits. These run concurrently with applicable federal and state leave taken for the same purposes, where the faculty member is also eligible for leave under those laws.

2.7.3.1 Parental Leave for Care and Bonding
   FMLA and CFRA leave provides for the care of and bonding with an infant or newly adopted or fostered child. Faculty may take leave for these purposes (and only these) in the form of a paid four-unit reduction of teaching load. If a reduction is not required in the semester of birth or adoption or placement (because it occurs late in the semester), it may be taken during the next semester.
   Eligibility. Leave benefits under this policy are available to returning multi-year and tenure-track faculty.
   
2.7.3.2 Pregnancy Disability Leave
   Pregnancy disability leave runs concurrently with FMLA leave taken because of a pregnancy-related medical condition. These employees are legally entitled to up to four months of unpaid job protected leave. Leave does not need to be taken at one time but can be taken as needed. 
   All pregnant faculty may take these medical and disability leaves as a reduced workload in the form of a paid four-unit load reduction. Faculty taking paid pregnancy disability leave must apply for the allowable SDI benefit with the state Employment Development Division. The amount of SDI received will be deducted from any other pay the faculty member receives.
   Eligibility. Leave benefits under this policy are available to returning multi-year and tenure-track faculty. Provost approval may require a physicianÕs recommendation for the modified or reduced workload in connection with a pregnancy disability or pregnancy related medical condition.  
   
2.7.3.3 Additional Benefits
   Distribution of course reductions. Leave in the form of load reductions under the above policies may generally be taken in a single semester or split between two semesters or may occur as otherwise medically certified. The department chair in consultation with the provost will consider such requests, which will be reasonably accommodated unless to do so presents the College with an undue burden. 
   If a total of eight units of load reduction are taken in a single semester, the remaining four units of that semesterÕs load may be added to another semesterÕs load as an overload or may be earned in Mayterm. Alternately, the faculty member may request a reduction in pay in the amount required to cover the remaining four units at the adjunct salary schedule rate. Either action requires approval of the department chair and provost.
   Advising, Committee Work and Tenure Clock.  The faculty member taking course load reductions under the above policies will be released from advising and committee responsibilities for a period of one semester, in consultation with and subject to the approval of the provost and the department chair. In addition, the faculty member may delay any tenure review by a year, up to a total of two years for all children.  The decision to delay tenure review must be made in advance of the course load reduction in consultation with the provost. 
   
2.7.4 Jury and Witness Duty
   The College recognizes and supports the civic responsibility of faculty members to participate in the judicial process.  Faculty called to serve on jury duty, however, are encouraged to seek deferment until the summer months when teaching would be unaffected.
   
2.7.5 Military Service
   Westmont College complies with state and federal statutes which provide reemployment rights for inductees, enlistees, re- enlistees, and reservists, who enter active or inactive duty training in the Armed Forces of the United States or the Public Health Service while employed by the College in other than a temporary position.  These statutes also apply to short-term absences for military duty required of reservists and members of the National Guard.  Therefore, certain employee protections are provided for both extended tours of duty, as well as emergency call-up or annual military training duty.  Upon completion of military service, employees are entitled to reinstatement of employment at the College if they apply within 90 days of discharge or one year if hospitalized, and are still qualified to perform the duties of the position.
   
2.8 Benefits
2.8.1 Government Mandated Benefits
2.8.1.1 WorkerÕs Compensation Insurance
a) Work related injuries and illnesses are covered by the CollegeÕs WorkersÕ Compensation program.  It is the responsibility of the faculty member to report all work related injuries immediately to his or her Department Chair and the Human Resources Department so that the necessary medical treatment may be determined and the required injury reports may be completed. Injuries that require medical attention other than basic first aid are handled as follows:
1) Contact the Human Resources Department immediately.
2) The injured faculty member may choose to be treated by his/her regular physician if the faculty member had notified the College of that preference, in writing, prior to the date of injury.
3) If 2) above does not apply, Westmont or its insurance carrier has the right to determine the physician who will provide medical treatment for the first 30 days for all compensable injuries sustained by the faculty member, and to obtain, at reasonable intervals, medical diagnoses, medical progress reports and/or medical opinions as to the fitness of the faculty member for return to instructional and other duties.  The cost of such treatment will be paid by the College.
b) For injuries requiring time lost from instructional and other duties, WorkersÕ Compensation requires a three day unpaid waiting period before salary replacement benefits begin.  It is the CollegeÕs policy to pay regular salary to faculty members teaching at least 16 units, or the equivalent, for the three day waiting period.
2.8.1.2 Social Security
   Participation in Social Security is required of all faculty and includes equal contributions made by the individual faculty member and the College.
2.8.1.3 Unemployment Compensation
   The California Unemployment Compensation Insurance Program is administered by the State Employment Development Department.  The program is funded solely by the College. The College reimburses the State for the partial salary replacement benefits to qualified unemployed faculty.
2.8.1.4 Health Insurance Continuation
   The Consolidated Budget Reconciliation Act (COBRA) is a federal law requiring employers to provide former employees and their dependents with the opportunity to maintain health benefits for a limited time following separation from employment in instances where such coverage would otherwise end.  Such instances include:  1) termination of employment or reduction in hours, 2) death of employee, 3) dependent ceasing to qualify as a Òdependent child,Ó 4) divorce or separation from the employee, 5) employee becomes eligible for Medicare, and 6) termination of employment or lost eligibility due to disability.  It is the employeeÕs responsibility to notify the Human Resource department within 60 days of a divorce, legal separation, or that a child has lost dependent status.  Notification to employees of their COBRA benefits at termination is the responsibility of the Human Resources Department.
2.8.1.5 State Disability Insurance
a) During periods of unpaid medical leave, including maternity leave, certified by a physician as medically necessary, faculty are eligible to apply for disability income benefits paid by the State of California.
b) It is the faculty memberÕs responsibility to obtain a claim form which must be signed by the faculty member and the attending physician.  Claim forms are available from the Human Resources Department or the State Employee Development Department (EDD) by telephone, letter, or in person.  Physicians or hospitals may also have claim forms.
c) The College will coordinate paid medical leave with state disability payments, making up the difference between the partial salary replacement benefit from the State, and regular net salary, to the extent that the faculty member has paid medical leave available from the College.
2.8.2 Discretionary Benefits 
   For purposes of eligibility for discretionary benefits, Òfull-time facultyÓ are defined as those teaching at least 16 units or equivalent per academic year or 12 units in a single semester; Òpart-time facultyÓ are those teaching 11 to 15 units or equivalent over two consecutive semesters, with a minimum of four units per semester.  The benefits descriptions contained in the Handbook are summaries of key features of each benefit.  The Plan Documents or insurance policies for each plan represent the complete and authoritative descriptions of benefits.  Detailed information is available from the Human Resources Office.  The College in consultation with the Faculty Budget and Salary Committee may modify or eliminate discretionary benefits.
2.8.2.1 Medical Care Plan
   Full-time faculty (defined above) and their dependents are eligible to participate in one of the CollegeÕs group medical insurance plans.  The HMO premium for full-time faculty membersÕ personal coverage is paid 100% by the College; the PPO premium for personal coverage is shared by the faculty member and the College.  The cost of dependent coverage for full-time faculty is shared by the faculty member and the College.  Part-time faculty (defined above) are able to participate by paying one-half of the cost of their own coverage plus the full amount of the employee contribution where applicable, and the full cost of dependent coverage through payroll deduction.  Faculty may participate on the first day of employment.
2.8.2.2 Dental Care Plan
   The College offers group dental insurance plans in which faculty and their dependents may participate.  Eligibility criteria for participation in a group dental plan by full-time faculty and part-time faculty are the same as for participation in a group medical plan.  Depending upon the plan chosen, the first day of coverage may not coincide with the first day of employment, and the faculty memberÕs premium will vary.
2.8.2.3 Health Insurance Transition Benefit
   The College provides a health insurance transition benefit to ensure continuity of medical and dental insurance coverage for new benefits-eligible faculty members and their families during their transition to the College.  Even though a new faculty memberÕs salary does not begin until the initial contract date, a new faculty member whose health insurance coverage expires prior to that date may elect to begin medical and dental insurance coverage at the College at any time between July 1 and the initial contract date, provided that, if enrolling in an HMO, the faculty member and covered dependents reside in the HMO service area.  The College will pay 100% of the premium for medical and dental insurance for such faculty members and their families prior to the start of their initial contract date.  Thereafter, the faculty member will contribute the current employee portion of medical and dental premiums.
2.8.2.4 Retirement Medical Plan
a) A faculty member taking early retirement (see Section 2.3.2.1) may elect to continue participation in the College group medical plan until age 65 or as long as permitted by the medical plan provider.  The College will contribute toward the cost of that coverage an amount equal to that which would be contributed if the person were to remain an active faculty member.  Dependents of the retiree are not eligible for medical plan coverage after the faculty member retires.
b) For a retiree who is 65 or older, the College will reimburse an amount that is the lesser of the above amount and the actual annual cost of a medical plan chosen by the retiree or the current specified annual retiree medical plan contribution, whichever is less.
2.8.2.5 Disability Plan
a) Short-Term Disability Plan:  The College will pay up to one-half salary, in coordination with State Disability payments, for a maximum of 90 days in any 12-month period, to faculty disabled by illness or injury that is not employment related, when disabled faculty have no available extended medical leave, and the absence is longer than seven consecutive calendar days.  (The waiting period will be waived if the faculty member is hospitalized.)
4) The disabled faculty member must file for State Disability Insurance (SDI) benefits.
5) The College will coordinate benefits by paying the difference between the faculty memberÕs net salary and SDI after available extended medical leave days have been exhausted.  To determine the amount to be paid by the College, the College will contact the State Disability office to verify the weekly benefit amount to be granted to the faculty member.  The College will then pay the coordinated benefit on regularly scheduled paydays.  Once the faculty member receives a disability check from the State, the stub should be sent to the Payroll Department to verify the amount received.
6) The College will pay coordinated benefits not to exceed one-half of regular net salary, for a maximum of 90 days from the first day of unpaid leave, or until Long-Term Disability payments would begin, whichever is shorter.  The CollegeÕs benefit will apply regardless of the faculty memberÕs eligibility to receive Long-Term Disability benefits.
b) Long-Term Disability Insurance:  Long-Term Disability insurance is provided and paid 100% by the College for full-time faculty members.  This insurance provides partial salary continuation should a faculty member become disabled and unable to work for more than three months.  The maximum amount of salary continuation for total disability is 60% of monthly earnings, less certain other sources of income such as Social Security disability entitlements, up to a maximum of $6500 per month.
7) Eligible faculty receive this coverage on the first day of employment.  If a faculty member is less than age 60 at the time of disability, benefits continue during a period of disability until age 65, but for not less than five years.  If the faculty member is age 60 or over at the time of disability, the maximum period of benefits gradually decreases from five years at age 60 to one year for a disability at age 69 or over.
8) As an additional benefit for employees receiving Long Term Disability payments, the College provides a Òretirement income protectionÓ contribution of up to 10% of monthly earnings.  The contribution will be deposited into the Retirement Plan on behalf of an eligible faculty member, not to exceed the maximum allowed by law, as long as the faculty member is receiving disability payments and has been a Retirement Plan participant for at least three months prior to the disability.
   Additional details about this insurance benefit, including information on partial disability, mental illness, and survivor benefits, are contained in the Plan Document.
2.8.2.6 Life Insurance
   Life insurance coverage is provided and paid 100% by the College for full-time faculty members.  The amount of insurance coverage for faculty less than age 65 is equal to the annual base salary rounded up to the nearest $1,000, and is effective on the first day of employment.  For faculty age 65 or older the amount of coverage is .67 times annual salary rounded up to the nearest $1,000.
2.8.2.7 Travel Accident Insurance
   Travel insurance is provided by the College at no cost to full-time non-temporary faculty.  Eligible faculty will have $25,000 in coverage which provides for payment of the full amount in case of accidental death, or one-quarter to full payment in cases of loss of sight, or dismemberment, depending on the specific injury.  Coverage provides 24-hour, world-wide protection while traveling on College business, and is effective on the first day of employment.
2.8.2.8 Retirement Plan 
a) Plan Definition:  The College offers eligible faculty the opportunity to participate in a 403(b) Defined Contribution Retirement Plan (the ÒPlanÓ).  The Plan is governed by Section 403(b) of the Internal Revenue Code.  The ÒPlan DocumentÓ is the legally required description of the rights, obligations and benefits under the Plan.  An abbreviated legally required description is contained in the ÒSummary Plan Description.Ó  Both documents are available for review from the Human Resources Office.  The following is a summary of key elements of the Plan.
b) Matching Contributions
9) All faculty will be eligible for employer matching contributions (Òmatching contributionsÓ) to the Plan after completing one year of service and having reached age 26.  For purposes of the Plan, a faculty member on a full-time contract calling for 24 units per year will be regarded as completing Òone year of serviceÓ after completing six months of full-time employment.  Years of service with other higher education institutions which employed Westmont faculty immediately prior to their employment at Westmont College will be recognized by the Plan.
10) For other teaching loads, or in cases where the teaching load may vary from one semester to another, eligibility, and the waiting period for eligibility for matching contributions, will be determined by the Human Resources Office in accordance with the relevant provisions of the Plan.
11) Once eligible for matching contributions from the College, continued eligibility for matching contributions is dependent upon maintaining a faculty contract that is the equivalent of at least 12 units per year.
12) The maximum matching contribution is 7% for the faculty memberÕs contribution of 3% or more.
13) Faculty member and College matching contributions are computed as percentages of base salary only.  For eligible faculty members, matching contributions are calculated as follows:
(a) 3 times the faculty memberÕs contribution up to the first 1%
(b) 2 times the faculty memberÕs contribution above 1% and up to 3%
14) Matching contributions from the College on behalf of a faculty member are fully vested immediately.
15) Any faculty member who is ineligible for matching contributions may elect to make voluntary contributions to the Plan through payroll reduction as soon as employment commences.
c) Contribution Limits:  Contributions to the Plan are subject to limitations set by the IRS.
d) Requesting Participation:  It is the faculty memberÕs responsibility to initiate a request to begin participation in the Plan after being notified by the Human Resources Office of eligibility to participate, or after declining or suspending participation.
2.8.3 Institutional Benefits
   For purposes of determining eligibility for institutional benefits, Òregular facultyÓ refers to all faculty except those on temporary term contracts (see Section 2.1.2.1.1.1)
2.8.3.1 Education Assistance
a) For Faculty:
1) Full-time regular faculty, with the approval of their department chairs, are eligible to enroll in one course per semester at no tuition charge.
2) The Admissions Office will determine space availability for a faculty member who desires course work for degree credit.  Students paying full tuition will have preference for class space.  Charges other than tuition, health fees, and student activity fees will be paid by the faculty member.
b) For Eligible Dependent Children:
1) Education Assistance is available to full-time regular faculty for their children who meet the IRS definition of dependent children.  The benefit is available for regular semesters and Mayterm up to the bachelor degree or teaching credential. All regular admission requirements must be met.  The benefit will be available for eight semesters, including off-campus Mayterm programs, per eligible dependent, plus an unlimited number of on-campus Mayterms.  The benefit is not available for private lessons, tutoring, and on-campus Mayterm classes below minimum enrollment.  However, if the difference between the partial and full salary of the faculty member is less than the cost of full tuition, an eligible dependent need only pay that difference (see Section 2.8.3.1 d) ).
2) When one of the eight semesters of benefit is used for an off-campus program, the benefit covers the tuition portion of the program.  Tuition for an off-campus program will be determined as follows:
(a) Tuition for an off-campus program during the fall or spring semester will be set at the tuition rate charged for the on-campus semester.
(b) Tuition for an off-campus program during Mayterm or summer will be set at 70% of the total cost of the program.  This benefit may be used for any Westmont sponsored off-campus program.  However, one semester of benefit will be assessed regardless of the units available on the program.
3) The Education Assistance benefit for dependents for regular semesters and off-campus May Term programs is equal to:

Completed service* (prior to beginning of semester)
Benefit
Less than 4 years
None
4 years
25% of tuition
5 years
50% of tuition
6 years
75% of tuition
7 years
100% of tuition

   *Full-time employment at other institutions of higher education immediately prior to beginning full-time employment at Westmont will be credited as service in qualifying for the Education Assistance benefits for dependent children.  Employees who began full-time employment prior to January 1, 2002 will be credited with an additional four years of service eligibility for this benefit.
4) Full-time employment at other institutions of higher education will be credited as service at Westmont in qualifying for the Education Assistance benefit.
5) Charges other than tuition will be paid by the faculty member.  However, the health fees may be waived if the dependent is covered by one of the Westmont group medical plans.
c) For Eligible Spouses:
   Spouses of full-time regular faculty are eligible to enroll in one on-campus course per regular semester and course during on-campus Mayterm with no tuition charge.  Students paying full tuition and dependent children receiving this benefit will have preference for enrollment.  Charges other than tuition, health fees, and student activity fees will be paid by the faculty member.  No fee will be charged for auditing classes.  The benefit is not available for private lessons, tutoring and on-campus Mayterm courses below minimum enrollment (see d) below).
d) Limitations:
1) The Education Assistance benefit is available for on-campus Mayterm classes only if the class has enrolled the minimum required number of students paying full tuition.  However, if the difference between the partial and full salary of the Mayterm instructor is less than the cost of full tuition, the faculty member need pay only that difference.
2) Dependent children are required to complete an abbreviated Cal Grant application through the Financial Aid Office to determine probable eligibility for a Cal Grant in order to be eligible for the Education Assistance benefit.  The Education Assistance benefit will be the difference between any Cal Grant and the scheduled Education Assistance benefit.  Failure to make application will result in loss of eligibility for the Education Assistance benefit.  All other scholarships, grants, and aid may be retained by the student with no reduction of the Education Assistance benefit as long as the total amount does not exceed the studentÕs expense budget established by the Financial Aid Office.  All financial aid must be reported to the Financial Aid Office.
e) Paid Leaves:
   Faculty members who are on sabbaticals or other paid leaves of absence will continue to be eligible for the benefit.
f) RetireesÕ Benefits:
   The Education Assistance benefit for faculty and faculty spouses is available to retirees.
g) Death or Disability of a Faculty Member:
1) If a dependent of a faculty member is receiving the Education Assistance benefit at the time the faculty member dies or becomes totally or permanently disabled, the dependent will continue to receive the benefit for the remainder of the current semester plus the following three semesters.
2) If a faculty member with 10 or more years of continuous service dies or becomes totally disabled while a dependent is receiving the Education Assistance benefit, the dependent will remain eligible to receive eight semesters of assistance.
3) If a currently employed faculty member with 10 or more years of continuous service dies or becomes totally disabled, and the Education Assistance benefit is not being used at the time, each dependent child of the faculty member will remain eligible for Education Assistance at the rate of one semester of assistance for each two years of full-time service of the faculty member according to the following scale:
   Years of Service			Semesters of Education Assistance
	10 years				5 semesters
	12 years				6 semesters
	14 years				7 semesters
	16 years				8 semesters
   The dependent child must begin use of the benefit within three years of the date of death or disability for a faculty member with 10 years of continuous service or within the eligibility period indicated in the following scale:
   Years of Service			Eligibility Period
	10 years				3 years
	12 years				4 years
	14 years				5 years
	16 years				6 years
   The child of a deceased or disabled faculty member must continue to meet the IRS definition of a dependent child in order to remain eligible for Education Assistance.
h) Other Sources of Education Assistance:
   In addition to the Education Assistance program, the College participates in programs which offer tuition assistance at many other colleges and universities.  The following programs are available to dependent children of faculty members who would otherwise be eligible for 100% of the Westmont College Education Assistance benefit and to dependent children who qualify for Education Assistance following the death or disability of a faculty member.  The number of available opportunities may vary from year to year.  Dependents are considered on a first-come, first-served basis, and the admission requirements of other schools must be met. Further information is available from the Office of Admissions.
1) Christian College Consortium Program:  At participating Consortium institutions, tuition differentials are ignored and the exchange student is treated exactly as those schools treat children of their own faculty in such matters as tuition, fees, etc.  The college enrolling the exchange student reserves the right to restrict the studentÕs participation in certain programs.  Available slots are usually limited by WestmontÕs record of participation in receiving and sending students.
2) Christian College Coalition Program:  At participating Coalition institutions, tuition will be waived for eligible students.
3) Tuition Exchange Program:  Hundreds of institutions participate in this program which offers tuition waiver based on the participation history of the institutions. Some schools offer scholarships for graduate study, law school, junior college, two-year nursing programs, etc., as well as for four-year undergraduate education.
i) Administration:
   The Education Assistance benefit is administered by the Human Resources Office with assistance from the Office of the Provost and the Financial Aid Office.  The Human Resources Office determines benefits eligibility.  The Financial Aid Office coordinates this benefit with other financial aid, as well as providing counsel to faculty on available financial assistance.
2.8.3.2 Housing Assistance Program
a) All full-time faculty with notice (tenure-track) or continuous (tenured) contracts may be eligible to receive financial assistance from the College for the first purchase of a residence in the Santa Barbara area.  The amount of assistance is based on the financial resources of the faculty member and the current cost of modest housing as determined by the Board of Trustees.
b) Financial assistance is a one-time contribution toward the down payment and/or monthly payments toward the mortgage.  In return, the College receives a share of the appreciation in the value of the property.  Although the College can be repaid at any time, no repayment is required until the property is sold, or the faculty member ceases to be a full-time faculty member of the College, or ceases to occupy the housing as his or her principal residence.
c) Specific provisions and other policies regarding the Housing Assistance Program are available from the office of the Vice President for Finance.
2.9 Compensation
2.9.1 Contract Period and Method of Payment
   A standard contract between Westmont College and a faculty member is for a nine-month academic year.  Salary payments extend over a 12-month period unless the appointment is temporary or for one year, or the faculty member applies for payment over nine months rather than 12; in these cases, the nine-month contract will be paid fully in the contract period.  The payroll is issued to faculty members in 24 equal payments over 12 months.
2.9.2 Salary Schedule
a) Prior to issuing contracts each year, the Board of Trustees adopts a revised Salary Schedule effective for the following academic year, listing salaries by rank and step.  Ranks are defined in Section 2.1.1.1.3.  
b) Faculty members receive step increases for each year of full-time service except in certain cases of non-performance of contractual obligations (see Section 2.2.4).
2.9.3 Health Insurance Premiums and Dependent Care Expenses
   Faculty members should contact the Human Resources Department for information regarding payment with pre-tax dollars of dependent medical insurance premiums, child care expenses, and unreimbursed medical expenses.
2.10 Procedures for Resolving Grievances
   No matter how good our intentions may be, conflicts will arise.  Complaints and minor conflicts can often be worked out between parties (with or without intervention).  Faculty members are encouraged to try to resolve conflicts informally (for helpful guidelines in resolving conflict, see Faculty Advices and Queries).
   Sometimes, however, a conflict or complaint actually constitutes a grievance against individuals or the institution (e.g., an interpersonal conflict that has polarized colleagues, or the belief that a policy has been unjustly applied).  For some grievances, the Faculty Handbook provides recourse in other sections (see, for example, sections 2.6.1 for procedures regarding harassment; 2.2.4.1 for non-performance of contract; and 2.2.4.2 for violation of contract.).
   If the grievance is one for which there is no other/particular provision for recourse in the Handbook, a faculty member may file a petition the Faculty Council.  The petition will identify the issue and will set forth in detail the nature of the grievance, presenting any factual or other data the petitioner deems pertinent to the case.  In acting upon a petition, the Faculty Council may:
a) refuse to review the case until other efforts at reconciliation and resolution have been attempted;
b) review the case and declare the grievance to be without merit;
c) review the case, declare the grievance to have merit and seek resolution; if resolution is not forthcoming then the Council may bring the matter to an executive session of the faculty;
d) for reasons of conflict of interest, decide that an ad hoc committee should be constituted to review the case.

3.0 ACADEMIC POLICIES AND PROCEDURES OF INTEREST TO THE FACULTY
3.1 Academic Policies and Procedures Handbook
   The Academic Policies and Procedures Handbook contains polices approved by the Academic Senate and the Faculty.  The handbook, whose contents are listed below, can be accessed in its entirety at http://cgi2.westmont.edu/publications/academic_policies/ . 
   If you are reading the electronic version of the Faculty Handbook, the individual topics listed below are linked to the appropriate handbook section.
Academic Dishonesty
Academic Forgiveness Policy
Academic Standing 
Alternative Majors Policy 
Attendance Policies
Commencement Participation
Confidentiality of Student Records 
Credit by Exam 
Credit Limitations 
Language Placement 
Examinations 
Grades 
Graduation Requirements 
Honors Programs/Courses 
Internships/Practica 
Majors and Minors 
Off-Campus Programs 
Petitions 
Registration 
Second Baccalaureate Degree 
Serving Society 
Transfer Credit Policies 
Tutorials 
Withdrawing from Westmont
3.2 Copyright Policy
   Policies related to the reproduction and classroom use of copyrighted material are found at http://www.westmont.edu/_faculty_staff/pages/policies/copyright_policy/index.html.
   This document also includes directions for securing permission for classroom use of copyrighted material.
3.3 Plagiarism Policy
   The policies regarding the reporting of student plagiarism, the definitions of the various levels of plagiarism and the corresponding consequences can be found at http://www.westmont.edu/_academics/pages/provost/curriculum/plagiarism/ 
   This site also includes sample plagiarism statements to be included in syllabi and an electronic version of the form used to report plagiarism as well as strategies instructors can use to identify and reduce the incidence of plagiarism.
3.4 Internships and Fellowships for Recent Graduates Policy
   Recently graduated Westmont students are eligible for Westmont internships and fellowships, which must be selected through a defined process that uses established criteria consistent with practices in the field and which must be completedÊwithin one year. The Provost is responsible for establishing such selection protocols to ensure that there is integrity in the process and to avoid conflicts of interest. Awards must be approved by faculty department chairs and the Provost.
   

4.0 ACADEMIC SERVICES
4.1 Library & Information Services
   Roger John Voskuyl Library, named for WestmontÕs third president, provides access to information resources and services to support the research and information needs of faculty, staff, students and the surrounding community.  The library collections include books and e-books, media items, music scores, microforms, print periodicals, and online resources including periodicals, databases and other online tools.  The Westmont College Archives collection is located on the lower level of the library and is comprised of items relating to its history, 1940 to present, and that of its predecessors, Western Bible College, 1939-1940, and the Bible Missionary Institute, 1937-1939. 
a) Library Facility:  The Learning Commons is located on the main level of Voskuyl Library.  This space brings together library, technology and other campus services in an environment designed to foster collaborative and creative work, and social interaction.  The learning spaces in the library are designed for different types of learning styles, including quiet area on the upper and lower levels, conference rooms, an open computer center, a technology teaching lab, group study rooms, individual study stations, group study tables and comfortable lounge furniture.
b) Services:  Library Services available for faculty include interlibrary loan, course reserves, the department liaison program, and the information literacy instruction program.  Detailed information about the libraryÕs services and programs may be found on the library website: http://library.westmont.edu
c) Library Instruction:  Guidance in the use of the library resources is provided by subject and course-related research guides available on the library website.  Information literacy instruction sessions, taught by Westmont librarians, are offered in the library technology teaching lab.  Librarians are also available for in-class instruction sessions as well as one-on-one research consultations.
d) Loans:  Faculty may borrow materials from the VoskuylÕs collections for a period determined by and subject to the libraryÕs circulation policy.  Faculty have access to additional resources through interlibrary loan.
e) Collection Development:  Collection development is seen as a cooperative effort between library professionals and teaching faculty in various academic departments. 
f) Cubicles:  A limited number of library cubicles are available to faculty who are involved in special projects.  Applications for cubicles are made through the Professional Development Committee.

4.2 Computer Services
   Information Technology Computing and Network Services:  Information technology provides a wide variety of computing and network services in support of the collegeÕs mission.  These services include:  local network infrastructure; connection to the Internet; servers for email, web pages, file serving and applications; workstations for faculty and staff offices; data projection systems and workstations in classrooms; campus-wide licensing for general-use software; a course management system; training and assistance in developing resources such as the ones mentioned above.
   Campus network users are expected to abide by our Acceptable Use Policy found at http://acweb.westmont.edu/policy/.  Problems involving any service provided by IT may be reported via the web at http://mayday.westmont.edu, or via email at ithelp@westmont.edu, or by phone at 565-7211.  Further information about these services may be found at http://eureka.westmont.edu and click on ÒIT Wiki.Ó

5.0 ADMINISTRATIVE AND FINANCIAL POLICIES OF INTEREST TO THE FACULTY
5.1 Emergency Closing
5.2 Keys
   Keys to oneÕs office and the building in which it is located are distributed through the Physical Plant office.  Requests for keys are made through the departmental secretary; acknowledgment of receipt of keys must be signed by the faculty member on a form provided by the Physical Plant Department.  When one leaves the employ of Westmont College keys must be returned to the Physical Plant Department.  Keys may not be duplicated without the express consent of the Physical Plant Department.
5.3 Post Office/Mail
   The campus Post Office is operated for the convenience of students, faculty and staff. The Post Office sells stamps and other mailing supplies. They also offer a wide range of mailing services, including campus delivery, outbound shipping and express delivery (FedEx, UPS and others), and bulk mail (special rules and procedures apply to bulk mail; contact the Document Services Manager x6077 for assistance.)
   Mail for students must bear the MS# of each student. This data can be obtained from the students themselves, from the Student Directory, from the campus switchboard (dial "0"), or by calling the Post Office (x6078).
   Mail to be metered for outbound US mail must be brought to the Post Office by 3pm daily. Already-stamped mail can be placed in the mail slots as late as 4pm. During the school year, each department will receive a daily delivery of incoming mail in the late morning, and a daily pickup of outgoing mail in the early to mid-afternoon.
   Hours:  Monday Ð Friday, 10:00 am - 4:00 pm; Saturday 10:00 am Ð 12 noon (package pick-up only)
5.4 Bulletin Boards
5.5 Telephones
   Telephone service, including voice mail, is provided for each office and department for conducting College and professional business.  Long distance calling is available through the use of assigned access codes.  The department chair reviews long distance statements for accuracy and identification of personal long-distance calls placed with access codes.  Reimbursement for personal long-distance calls are made directly to the Business Office and are credited to the department telephone account.  A Faculty and Staff Directory is published once per year with a supplemental quick reference card published shortly after the beginning of each regular semester.  Directories include personal information and are solely for the business and personal use of the Westmont community.  They are not to be used for personal or commercial solicitation and their transfer or sale is prohibited.  Prior year directories should be shredded when they are no longer needed.
5.6 Purchase Orders/Requisitions
   Purchase Orders and Requisitions are made on forms available from the Department Chair.  Purchases must be approved by the Department Chair and the ProvostÕs office.
5.7 Professional Development Funds
a) Principle:  Professional Development Funds encourage scholarly work and professional involvement by guaranteeing an annual allotment of money for membership in professional organizations, journal subscriptions and travel to local or regional meetings.  In addition, supplementary funds are made available once a year for active participation in oneÕs scholarly organizations.
b) Policy:
1) Annual Allotment:
(a) A fixed amount will be allocated at the beginning of each year, to each full-time faculty member.
(b) A fraction of that amount will be allocated to faculty members with less than a full-tim
(c) e load but more than half-time, in proportion to their loads.
(d) An expense report must document transportation, registration fees, and any other incurred expense.
(e) Faculty may accumulate up to three times the annual allocation.
2) Additional Stipends:
(a) Faculty will be allocated an additional stipend each year for reading a paper, leading a session, or participating as a member of an organizational ruling body.  Requests for additional stipends are made to the Provost.
(b) No more than one such additional stipend will be granted each year except under unusual circumstances.
3) Requests for all payment should be submitted to the ProvostÕs office.
5.8 Motor Vehicles on Campus
   Parking permits are required and are available free of charge from Campus Security.  County regulations prohibit faculty members from parking motor vehicles on campus without valid parking permits.  Parking permits must be affixed to oneÕs vehicle in accordance with instructions.  If faculty bring substitute vehicles to campus without permits (e.g., rental car), they are to obtain temporary permits from the physical plant or the housing office.  Faculty operating vehicles found in violation of these regulations will receive an initial warning and then face fines for each occurrence, vehicle immobilization, and/or towing of the vehicle at the ownerÕs expense.  Faculty may park in any parking lot; the Kerr Student Center lot is specifically reserved for faculty and staff during regular office hours.  Faculty should respect ÒReservedÓ areas and short interval time zones during regular hours.
   Faculty members are expected to abide by all parking and traffic regulations described in the Vehicle regulations brochure and web site.  While violations will be subject to fines as described in the Regulations, fines may be appealed to the Office of Public Safety, and then to Faculty Council.  Faculty members who find that a parking regulation interferes with their teaching duties should address their concerns to Faculty Council, who will work with them and Public Safety to mitigate the difficulty.  Public Safety may also convey concerns about excessive faculty violations to Faculty Council.
5.9 Check-Cashing
   Personal checks up to $25.00 may be cashed at the Bookstore during regular hours.
5.10 Athletic Events and Facilities
   Upon presenting a College identification card, a faculty member and immediate family members have free admittance to all home athletic events except play-off games.  Contact the Athletic Department for a schedule of sporting events.  Lockers are available by contacting the Athletic TrainerÕs office.  Faculty and their immediate families may use the swimming pool, track, tennis courts, racquetball courts, exercise room, and gym when those facilities are open and not being used for scheduled events.
5.11 Dining Commons/The Study
   Meals may be purchased at the Dining Commons (the ÒDCÓ).  Snacks and meals are also available from ÒThe StudyÓ located in the same building as the DC.  The DC and ÒThe StudyÓ accept cash or prepaid Òmunch moneyÓ which may be purchased at a discount from the food services contractor at the DC.  The Study (not the DC) accepts Visa and MasterCard, which can be used to purchase ÒMunch Money.Ó  Additional information can be found at www.westmont.eduauxsvcs and www.westmontdining.com.
5.12 Bookstore
   All faculty purchases, except textbooks and special-order books, are eligible for a 10% discount.  Departmental charges are discounted 15%.

6.0 STUDENT AFFAIRS POLICIES OF INTEREST TO THE FACULTY
6.1 Student Handbook (appended)
6.2 Student Conduct Code
6.3 Family Educational Rights and Responsibilities
a) Under the ÒFamily Rights and Privacy Act of 1974Ó the faculty are responsible to maintain privacy of academic information for each student.  In agreement with the provisions of this law the College has established the guidelines regarding access to student information.  In particular, the College has decided that only the academic advisor of a given student may see the studentÕs academic record.  Parents have access to that record only if the son or daughter is financially dependent upon them.  Questions regarding the implementation of the law at Westmont College should be directed to the Registrar; the Director of Financial Aid is responsible for assessing the financial dependence of students upon parents.
b) Of particular importance to faculty is the proscription under FRPA of the posting of student grades where the students are identified by name, social security number or student identification number, or by any other means which can lead with modest effort to the identification of student with grade.  Students may be asked to provide a coded identification at the beginning of the course which may then be used legally for the posting of grades.
6.4 Student Discipline Code
   The faculty are responsible for setting and maintaining standards for student conduct in academic work.  All other aspects of student behavior are under the aegis of the Vice President for Student Life and Dean of Students and the Student Life Staff.  Often, disciplinary actions by the student life office has academic consequences; faculty are not permitted to modify its action by, for example, providing opportunities for making up work missed because of suspension.  The established appeal process for the student is first to the Student Life and Development Committee and then to the President of the College.
6.5 Rights and Responsibilities
   By virtue of enrollment at Westmont College the student is entitled to certain rights among which are:
a) a clear statement of the expectations for each course and the means of evaluation to be used; this is provided in the course syllabus which the instructor distributes at the beginning of the semester, a copy of which is kept in the office of the Provost; the syllabus should be viewed as a contract which may be modified only by the common consent of the faculty member and those students enrolled in the course;
b) access to the faculty member outside of class during announced office hours;
c) unbiased evaluation;
d) free but limited classroom expression of perspectives on topics germane to the course;
e) protection against the excessive inclusion of materials which are not related to the course. Judgments of relevance are to be made in light of the mission of Westmont as a Christian liberal arts college.
6.6 Sponsorship/Organizations

7.0 EXTERNAL RELATIONS POLICIES OF INTEREST TO THE FACULTY
7.1 Communications and Publications
   The Public Affairs office provides assistance in developing brochures, distributing press releases, advertising events, and ordering stationery and business cards. Staff members also produce the Westmont magazine, We-Mail, the Faculty Guide, and the fall and spring events calendars.  They serve as a liaison between journalists participating in ProfNet searches and Westmont faculty. Contact the office at pubaffairs@westmont.edu.
7.2 Grants Policy
a) Grants from Private Foundations:  All grant applications made to private foundations must be processed through Westmont College.  This requirement, and the steps outlined below, is necessitated by the need by the College to determine priorities as it makes requests and to ensure oversight of approved faculty projects by the Provost, in consultation with the Office of College Advancement.
1) Approval to make a grant application must be secured first from the office of the Provost by completing the Intent to Apply for Extramural Funding form accessed online through the ProvostÕs webpage under Faculty Development.
2)  The faculty member will, in most cases, draft and submit the grant proposal.
b) Grants from Governmental Agencies:  Grant applications seeking funding from governmental and quasi-governmental agencies (e.g., the National Science Foundation) are governed by regulations and policies specific to the funding sources. These applications are processed through the office of the Provost.
7.3 Patent/Copyright Policy
   Unauthorized use of trademarked names or symbols, including WestmontÕs, is prohibited.  Where College resources are used, the College retains ownership of all faculty, staff and student inventions and other intellectual property that may be patented, copyrighted, trademarked or licensed for commercial purposes.
   Faculty use of copyrighted material is governed by applicable U.S. copyright law.  Federal copyright law applies to all forms of information, including electronic communications.  Violations of copyright laws include, but are not limited to, making unauthorized copies of any copyrighted material (including software, text, images, audio, and video), and displaying or distributing copyrighted materials over computer networks without the authorÕs permission except as provided in limited form by copyright fair use restrictions.  The Òfair useÓ provision of the copyright law allows for limited reproduction and distribution of published works without permission for such purposes as criticism, news reporting, teaching (including multiple copies for classroom use), scholarship, or research.  Copyright guidelines, available from the College Bookstore, are outlined in ÒQuestions and Answers on Copyright for the Campus CommunityÓ (National Association of College Stores and the Association of American Publishers, 1997).
7.4 Use of Institutional Letterhead, Trademarks, Tradenames
   The logo, letterhead, and other insignia of the College are approved by the President.  College insignia or other identifying symbols should be used only for official business on behalf of the College.  Private use of College symbols and stationery that might imply institutional endorsement of a faculty memberÕs activities must first receive approval of the Provost.
7.5 Political Activity
   As an academic institution Westmont College maintains a non-partisan position and ensures the freedom of faculty as citizens to engage in political speech and activities so far as they are able to do so consistent with their obligations as teachers and scholars.  However, when speaking or acting as private citizens, faculty must avoid creating the impression they are doing so for the College.
7.6 Fundraising
   College fundraising activities are conducted through the Office of College Advancement.  Individual fundraising efforts by faculty may be conducted only with the express approval of the Vice President for Development.
   
8.0 EXCEPTIONS IN APPLICATION OF POLICY
   It is understood that the provisions of this Handbook cannot take into account all possible circumstances that might suggest alternatives in the application of policies related to the faculty, the Provost may grant an exception for a policy at his or her discretion based on a request from a faculty member, with the approval of the faculty member affected.  Faculty members may talk with the Provost at any point regarding the application of a policy when they feel that personal circumstance may call for an exception.





ÊÊÊÊ1Sexual harassment means unwelcome sexual advances, requests for sexual favors, and other verbal, visual or physical conduct of a sexual nature, made by someone from or in the work of educational setting, under any of the following conditions:
(1)  Submission to the conduct is explicitly or implicitly made a term or a condition of an individual's employment, academic status, or progress.
(2)  Submission to, or rejection of, the conduct by the individual is used as a basis of employment or academic decisions affecting the individual.
(3)  The conduct has the purpose or effect of having a negative impact upon the individual's work or academic performance, or of creating an intimidating, hostile, or offensive work or educational environment.
(4)  Submission to, or rejection of, the conduct by the individual is used as the basis of any decision affecting the individual regarding benefits and services, honors, programs, or activities available at or through the education institution.
ÊÊÊÊ2If the accused is the Title IX Officer, the Provost, or the Vice President and Dean of Students, then the President will be notified and help determine the appropriate investigation strategy.  If the accused is the President or a member of the Board of Trustees, then the Chair of the Board will be notified and help determine the appropriate investigation strategy.
---------------

------------------------------------------------------------

---------------

------------------------------------------------------------

[Type text]0[Type text]0[Type text]





    

ii
    

Faculty Handbook00September 2020

v
    



\bibliographystyle{abbrv} 
\bibliography{references} 
\end{document}