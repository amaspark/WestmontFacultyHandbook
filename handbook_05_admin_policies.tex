\section{ADMINISTRATIVE AND FINANCIAL POLICIES OF INTEREST TO THE FACULTY}
	\subsection{Emergency Closing}
	\subsection{Keys}
		Keys to one's office and the building in which it is located are distributed through the Physical Plant office.  Requests for keys are made through the departmental secretary; acknowledgment of receipt of keys must be signed by the faculty member on a form provided by the Physical Plant Department.  When one leaves the employ of Westmont College keys must be returned to the Physical Plant Department.  Keys may not be duplicated without the express consent of the Physical Plant Department.
	\subsection{Post Office/Mail}
		The campus Post Office is operated for the convenience of students, faculty and staff. The Post Office sells stamps and other mailing supplies. They also offer a wide range of mailing services, including campus delivery, outbound shipping and express delivery (FedEx, UPS and others), and bulk mail (special rules and procedures apply to bulk mail; contact the Document Services Manager x6077 for assistance.)

		Mail for students must bear the MS\# of each student. This data can be obtained from the students themselves, from the Student Directory, from the campus switchboard (dial "0"), or by calling the Post Office (x6078).

		Mail to be metered for outbound US mail must be brought to the Post Office by 3pm daily. Already-stamped mail can be placed in the mail slots as late as 4pm. During the school year, each department will receive a daily delivery of incoming mail in the late morning, and a daily pickup of outgoing mail in the early to mid-afternoon.

		Hours:  Monday -- Friday, 10:00 am - 4:00 pm; Saturday 10:00 am -- 12 noon (package pick-up only)
	\subsection{Bulletin Boards}
	\subsection{Telephones}
		Telephone service, including voice mail, is provided for each office and department for conducting College and professional business.  Long distance calling is available through the use of assigned access codes.  The department chair reviews long distance statements for accuracy and identification of personal long-distance calls placed with access codes.  Reimbursement for personal long-distance calls are made directly to the Business Office and are credited to the department telephone account.  A Faculty and Staff Directory is published once per year with a supplemental quick reference card published shortly after the beginning of each regular semester.  Directories include personal information and are solely for the business and personal use of the Westmont community.  They are not to be used for personal or commercial solicitation and their transfer or sale is prohibited.  Prior year directories should be shredded when they are no longer needed.
	\subsection{Purchase Orders/Requisitions}
		Purchase Orders and Requisitions are made on forms available from the Department Chair.  Purchases must be approved by the Department Chair and the Provost's office.
	\subsection{Professional Development Funds}
		\begin{enumerate}[label=\alph*)]
			\item{\underline{Principle}:  Professional Development Funds encourage scholarly work and professional involvement by guaranteeing an annual allotment of money for membership in professional organizations, journal subscriptions and travel to local or regional meetings.  In addition, supplementary funds are made available once a year for active participation in one's scholarly organizations.
			}
			\item{\underline{Policy}:
				\begin{enumerate}[label=\arabic*)]
					\item{Annual Allotment:
						\begin{enumerate}[label=(\alph*)]
							\item{A fixed amount will be allocated at the beginning of each year, to each full-time faculty member.
							}
							\item{A fraction of that amount will be allocated to faculty members with less than a full-tim
							}
							\item{e load but more than half-time, in proportion to their loads.
							}
							\item{An expense report must document transportation, registration fees, and any other incurred expense.
							}
							\item{Faculty may accumulate up to three times the annual allocation.
							}
						\end{enumerate}
					}
					\item{Additional Stipends:
						\begin{enumerate}[label=(\alph*)]
							\item{Faculty will be allocated an additional stipend each year for reading a paper, leading a session, or participating as a member of an organizational ruling body.  Requests for additional stipends are made to the Provost.
							}
							\item{No more than one such additional stipend will be granted each year except under unusual circumstances.
							}
						\end{enumerate}
					}
					\item{Requests for all payment should be submitted to the Provost's office.
					}
				\end{enumerate}
			}
		\end{enumerate}
	\subsection{Motor Vehicles on Campus}

		Parking permits are required and are available free of charge from Campus Security.  County regulations prohibit faculty members from parking motor vehicles on campus without valid parking permits.  Parking permits must be affixed to one's vehicle in accordance with instructions.  If faculty bring substitute vehicles to campus without permits (e.g., rental car), they are to obtain temporary permits from the physical plant or the housing office.  Faculty operating vehicles found in violation of these regulations will receive an initial warning and then face fines for each occurrence, vehicle immobilization, and/or towing of the vehicle at the owner's expense.  Faculty may park in any parking lot; the Kerr Student Center lot is specifically reserved for faculty and staff during regular office hours.  Faculty should respect ``Reserved'' areas and short interval time zones during regular hours.

		Faculty members are expected to abide by all parking and traffic regulations described in the Vehicle regulations brochure and web site.  While violations will be subject to fines as described in the Regulations, fines may be appealed to the Office of Public Safety, and then to Faculty Council.  Faculty members who find that a parking regulation interferes with their teaching duties should address their concerns to Faculty Council, who will work with them and Public Safety to mitigate the difficulty.  Public Safety may also convey concerns about excessive faculty violations to Faculty Council.
	\subsection{Check-Cashing}
		Personal checks up to \$25.00 may be cashed at the Bookstore during regular hours.
	\subsection{ Athletic Events and Facilities}
		Upon presenting a College identification card, a faculty member and immediate family members have free admittance to all home athletic events except play-off games.  Contact the Athletic Department for a schedule of sporting events.  Lockers are available by contacting the Athletic Trainer's office.  Faculty and their immediate families may use the swimming pool, track, tennis courts, racquetball courts, exercise room, and gym when those facilities are open and not being used for scheduled events.
	\subsection{ Dining Commons/The Study}
		Meals may be purchased at the Dining Commons (the ``DC'').  Snacks and meals are also available from ``The Study'' located in the same building as the DC.  The DC and ``The Study'' accept cash or prepaid ``munch money'' which may be purchased at a discount from the food services contractor at the DC.  The Study (not the DC) accepts Visa and MasterCard, which can be used to purchase ``Munch Money.''  Additional information can be found at www.westmont.eduauxsvcs and www.westmontdining.com.
	\subsection{ Bookstore}
		All faculty purchases, except textbooks and special-order books, are eligible
		for a 10\% discount.  Departmental charges are discounted 15\%.
