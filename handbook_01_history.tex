\section{HISTORY, MISSION, GENERAL ORGANIZATION
		AND GOVERNANCE}
	\label{sec:HMGOG}
	\subsection{History}
		Early evangelical Christian institutions of higher learning were conceived to
		witness to and to preserve the viability of a world-view understood in light of
		Christian faith and tradition. Many were founded primarily for the training of
		ministers, missionaries, and teachers.

		The Bible Missionary Institute, established in 1937 in Los Angeles, was the
		direct antecedent to Westmont. It was founded by Mrs. \st{Alexander} \change{Ruth}{24}{change founder's name to Ruth} Kerr who
		envisioned a college for training men and women for full-time Christian service,
		especially young people without the resources to attend existing institutions.

		In 1940 Westmont College was incorporated as a four-year degree-granting
		Christian liberal arts college. Having outgrown its Los Angeles facilities, the
		campus was relocated to the Dwight Murphy estate in Montecito in 1945.  Forty
		acres of the adjacent Deane School were added in 1967. Westmont was accredited
		by WASC in 1958. (For a discussion of the early history of Westmont College, see
		Nancy Phinney's article in the
		Journal of the Santa Barbara Historical Museum
		written for Westmont's 75th anniversary~\cite{Phinney2012}.)

	\subsection{Mission Statement}

		Westmont College is an undergraduate, residential, Christian, liberal arts
		community serving God's kingdom by cultivating thoughtful scholars, grateful
		servants, and faithful leaders for global engagement with the academy, church,
		and world.

		\subsubsection{Philosophy of Education}

			\paragraph {Liberal Arts}
				\change{As a liberal arts college, }{32}{give "Liberal Arts" etc paragraph numbers} Westmont seeks to help its students become certain kinds of people, not mere
				repositories of information or mere possessors of professional skills.  Where such information and
				competencies are acquired, it is to be done in an intellectual and social context that nourishes a
				larger spiritual vision and is integrated with it.  Crucially, as a liberal arts college, Westmont seeks
				to help inculcate those skills that contribute to leading a successful and satisfying life.  For just as
				one must be trained in the skills that enable one to engage in a trade, so one must be trained in those
				skills that enable one to engage in the distinctively human activities of reasoning, communicating,
				thoughtfully choosing one's moral and spiritual ends, building political, economic and spiritual
				communities, and entering into those ``appreciative pleasures" that require knowledge, experience, and
				trained discrimination. Herein lies the relationship between liberal learning and life, for these are
				the very skills that translate into performing well one's role as citizen of the state, servant of the
				church, member of a family, worker or professional and participant in the cultural world.
			\paragraph{Christian}
				Westmont College is committed to the universal truths of the Christian faith, to a high view of biblical
				authority and an orthodox doctrinal vision, and to the central importance of a personal relationship
				with God through Jesus Christ.  It is this Christian faith that the college seeks to integrate fully
				into its life as a liberal arts institution.  For the pursuit of a liberal arts education, with its
				emphasis on producing certain kinds of people and inculcating certain basic human skills essential for
				living a satisfactory life, cannot take place in isolation from one's most basic commitments and
				beliefs.  For the Christian, then, this means bringing one's biblical and theological heritage to this
				educational enterprise. Indeed, to have basic values and commitments that one cannot explicitly and
				systematically bring to this task is to have an education that is severely truncated, severed, as it
				were, from one's most important beliefs and values.  To isolate one's worldview in this way, while
				pursuing an education, will only result in a worldview uninformed by sustained intellectual reflection.
				Such an approach will yield persons who are not fully educated, indeed not educated at the core of their
				being.  For the Christian, therefore, higher education must be Christian education, if it is to be
				education for the whole person.  At Westmont, then, Christian faith is to inform the academic enterprise
				and the academic enterprise is to inform one's Christian faith and thus yield a Christian worldview that
				is biblically based and intellectually sound.
			\paragraph{Undergraduate}
				Westmont is an undergraduate college and as such directs its attention, focuses its resources and devises
				its pedagogical strategies to facilitate the development of students who are beginning their post-secondary
				education.  It follows that the primary emphasis at Westmont is on teaching.  But teaching often involves
				helping students to acquire research skills and to become themselves producers of knowledge.  This can be
				done effectively only as faculty model research skills for students, and mentor them in the acquisition of
				those skills.  Moreover, to create a vital intellectual environment profitable for undergraduate students,
				Westmont must be an institution where knowledge is generated as well as transmitted.  But producing such
				knowledge is to be largely (though not exclusively) evaluated and appreciated in terms of the benefits that
				accrue, directly or indirectly, to those undergraduates who have come to Westmont to receive their
				education. For it is those students that the educational programs at Westmont are dedicated.
			\paragraph{Residential}
				The educational programs of Westmont College are residential in character and reflect a commitment to
				facilitate and exploit the ways in which education occurs within community.  Indeed, ever since the monastic
				tradition, learning has been cultivated and transmitted within residential communities, enabling learning to
				be promoted by the joys of shared exploration and the sustenance of spiritual kinship.  Moreover, both the
				Christian and liberal arts traditions remind us of the integrity of human wholeness; we cannot be neatly
				compartmentalized into rational, spiritual and affective components.  The residential character allows and
				encourages expression of this wholeness as we live, learn and worship together.  Further, the residential
				character of the college reflects the conviction that the goal of all meaningful learning, and of biblical
				education in particular, is to inform the way we live.  The residential character of the college invites
				students to apply their studies to the daily task of creating a community in which individuals can grow and
				mature together.  Students are able to cultivate these patterns of adulthood and redemptive living in the
				presence of role models and mentors who can help them in this process.
			\paragraph{Global}
				Westmont is to be a college with global concerns.  For the earth and all its peoples are God's good
				creation.  As such, they must be appropriately valued and respected.  We are called in scripture to be
				stewards of the earth, to be faithful caretakers of the physical creation.  We are also called to appreciate
				the rich diversity of human cultures --- cultures shaped by people who bear the mark of God's image in
				creation.  We are, then to be a community informed and enriched by thoughtful and intentional study of and
				interaction with cultures other than our own.  Ours is, however, a fallen world, and the earth, its peoples,
				and their institutions stand in need of the redemptive, reconciling word of the gospel.  We are called,
				therefore, not only to appreciate and preserve the creation and human cultures, but also to participate in
				the work of the Kingdom in response to the Great Commission to make disciples of all nations --- to bring
				all creation and human institutions under the Lordship of Christ.  This task involves grappling with the
				full range of ways in which the fall has introduced blindness, disintegration, conflict, and injustice into
				the world.  Finally, the emphasis on the global nature of education is recognition that our world has
				increasingly become interconnected and interdependent.  To prepare people to function intelligently,
				effectively and for the good in a world of global politics, global economics, and global communications must
				be one of the aims of a Westmont education.

		\subsubsection{Accreditation and Memberships}
			\label{sec:AccreditationAndMemberships}

			Westmont College is accredited by the WASC Senior Commission of Universities and Colleges (WSCUC) (985
			Atlantic Ave., Suite 100, Alameda, CA, 94501; phone 510-748-9001), an institutional accrediting body
			recognized by U.S. Department of Education.  Its teaching programs, with specialization in elementary and
			secondary teaching, are accredited by the California State Board of Education. Approval has been granted for
			the training of veterans and war orphans.

			Westmont is a member of the Independent Colleges of Southern California; the Association of Independent
			California Colleges and Universities; the Christian College Consortium; the Council of Christian Colleges
			and Universities, Council of Independent Colleges, and National Association of Intercollegiate Athletics and
			the Annapolis Group of national liberal arts colleges.  Its financial policies are in accordance with those
			of the Evangelical Council for Financial Accountability (ECFA).

	\subsection{College Organization}
		\label{sec:CollegeOrganization}
		\subsubsection{Board of Trustees}
			\begin{enumerate}
				\item{
					The Westmont College Board of Trustees holds in trust the physical and financial assets of the College and has power of review and veto in relation to policy and
					programs.
				}
				\item{
					The Board of Trustees is independent and self-perpetuating. Its members are
					elected to three-year terms, and one-third of the members are elected annually.
				}
				\item{
					Westmont College trustees are knowledgeable of, and committed to, the central and strategic role education plays in contemporary society and the crucial contributions of Christian higher education to the church and its work.  The trustees affirm the College's Christian liberal arts philosophy and programs, and support the framework of academic governance shared with the faculty and administration.
				}
				\item{

					It is essential that trustees be persons of genuine Christian faith, who accept without reservation
					the Articles of Faith, and seek to uphold the college motto on the corporate seal, \emph{Christus Primatum
						Tenens} (``Christ holding Preeminence'').
				}

			\end{enumerate}
		\subsubsection{Administration}
			\begin{enumerate}
				\item{The Board assigns certain powers and authority to the administration of the College and to the faculty. The faculty, meeting in regular session, has authority to recommend changes in policy and programs to the administration and to the Board.}
				\item{Organization Chart (\change{Org chart was not in previous version}{26}{org chart was not in 9/2020 version of handbook})
				See [\autoref{sec:OrgChart}].


				\paragraph{President}
					\begin{enumerate}
						\item{\underline{Function}: To serve as chief executive officer of the College and to exercise all executive powers not explicitly reserved to the Trustees.  }
						\item{\underline{Term}: Appointed annually by the Board of Trustees.  }
						\item{\underline{Accountability}: Accountable directly to the Board of Trustees.  }
						\item{\underline{Salary}: Remunerated as recommended and approved by the Board of Trustees.  }
						\item{\underline{Responsibilities and Duties}:
							\begin{enumerate}
								\item{Related to the Trustees:
									\begin{enumerate}
										\item{To make regular reports to the Board of Trustees to enable them to arrive at informed judgments.}
										\item{To engage the Trustees in periodic review of institutional goals, policies and programs.}
										\item{To recommend to the Board of Trustees for appointment members of the Executive Team and all full-time faculty members with suggested rank and compensation.}
									\end{enumerate}
								}
								\item{Related to College Management:
									\begin{enumerate}
										\item{To act on behalf of the Board of Trustees on all College matters subject to its direction.}
										\item{To provide vision and leadership for the faculty and administration in formulating educational and support programs.}
										\item{To oversee the operation and development of the institution as a whole.}
										\item{To supervise the Vice Presidents in the performance of their duties.}
										\item{To make final review and give final administrative approval of College plans, budgets, and policies and to submit appropriate items to the Board of Trustees for adoption.}
									\end{enumerate}
								}
								\item{Related to College Constituencies:
									\begin{enumerate}
										\item{To represent and interpret the College and its programs to internal and external constituencies.}
										\item{To seek support for the College necessary to achieve the institution's mission and goals.}
									\end{enumerate}
								}
							\end{enumerate}
						}
					\end{enumerate}
					}
			\end{enumerate}
			\paragraph{Executive Team}

				The administration of the College is the responsibility of the President, who serves at the pleasure of the Board of Trustees as Westmont's chief executive officer.  The Provost is the senior administrator of the College's educational program, with primary responsibility for the academic program, and, with leadership from the Vice President for Student life, for the out-of-classroom life of the student body.  The President, the Provost, the Vice President for Student Life, the Vice President for Finance, the Vice President for Advancement, and the Vice President for Administration and Planning constitute the Executive Team.

				\subparagraph{Provost}
					\begin{enumerate}
						\item{\underline{Function}:  To coordinate the entire educational program of the College to achieve the college mission and goals; to build a strong and comprehensive liberal arts curriculum, and to ensure that all academic instruction, student development activities, and spiritual life programs, are effective means to nurture a strong and mature Christian commitment, consistent with the values and beliefs of the evangelical Christian community; to articulate the vision of Christian liberal arts in a manner that is compelling both internally and to the various constituencies of the college.}

						\item{\underline{Term}:  Appointed annually by the President with the approval of the Board of Trustees.}
						\item{\underline{Accountability}:  Directly accountable to the President.}
						\item{\underline{Salary}:  As recommended by the President and approved by the Board of Trustees.}
						\item{\underline{Responsibilities and Duties}:  The Provost is the number two administrative officer of the college and senior administrator of the educational program.  The Provost is the dean of faculty, with primary responsibility for the academic program.
							\begin{enumerate}
								\item{Related to the President:
									\begin{enumerate}
										\item{To advise the President in matters of College policy in general, and educational programs in particular.}
										\item{To report regularly to the President on the status of all aspects of the College under his or her administrative supervision.}
										\item{To assume, in the absence of the President, administrative responsibility for the College.}
										\item{To assist the President in presenting to the Board of Trustees and its committees matters relating to educational programs and personnel.}
									\end{enumerate}
								}
								\item{Related to Supervision, Guidance, and Advisory Roles:
									\begin{enumerate}
										\item{To provide administrative supervision over:
											\begin{enumerate}
												\item{Dean of Curriculum and Educational Effectiveness}
												\item{Vice Provost}
												\item{Dean of Admissions}
												\item{Director of the Library and Information Services}
												\item{Director of Off-Campus Programs}
												\item{Athletic Director}
												\item{Registrar}
												\item{Institute Directors}
												\item{Department Chairpersons}
												\item{ Individual Faculty members}
											\end{enumerate}
										}
										\item{As dean of faculty,
											\begin{enumerate}
												\item{To serve as a liaison between the Faculty and the President and his or her administrative officers particularly in academic matters.}
												\item{To have responsibility, with the involvement of department chairs, for the recruitment of faculty personnel.}
												\item{To recommend, in consultation with the Faculty Personnel Committee, appointments of faculty personnel.}
												\item{To implement, with the assistance of department chairs, a program for the evaluation of faculty performance and instructional improvement.}
												\item{To maintain a program for professional development and in-service education.}
												\item{To promote, with the assistance of the Faculty Budget and Salary Committee, a program for the enhancement of faculty welfare.}
												\item{In conjunction with the Faculty Council, to ensure the proper functioning of faculty committees within the governance pattern.}
												\item{To maintain, in consultation with the Faculty Council, a complete and accurate \emph{Faculty Handbook}.}
											\end{enumerate}
										}
										\item{To serve \emph{ex officio} on College committees and to chair the following:
											\begin{enumerate}
												\item{Faculty meeting as a committee of the whole}
												\item{Academic Senate}
												\item{and to establish ad hoc committees as may be appropriate}
											\end{enumerate}
										}
									\end{enumerate}
								}
								\item{Related to Planning and Budgeting:
									\begin{enumerate}
										\item{To provide leadership in the formulation and implementation of educational programs consistent with the goals of the College.}
										\item{To promote the wise adoption of effective new alternatives in educational methodologies and technologies.}
										\item{To formulate, in conjunction with appropriate committees, educational policies and regulations for recommendation to the Faculty, President, and Trustees.}
										\item{To oversee the continued development of a comprehensive undergraduate curriculum strongly built upon the classic liberal arts traditions and informed by a Christian worldview.}
										\item{To develop and integrate all aspects of college life including academic/instructional, student development and campus ministries, to promote the intellectual development, spiritual growth, and character formation of students.}
										\item{To prepare and manage the budget for all educational programs of the College.}
									\end{enumerate}
								}
							\end{enumerate}
						}
					\end{enumerate}
					\subsubparagraph{Dean of Curriculum and Educational Effectiveness}
					\begin{enumerate}
						\item{\underline{Function}:  Within the academic program, the Dean of Curriculum and Educational Effectiveness has primary responsibility over curriculum development, program review and assessment.}
						\item{\underline{Term}:  Appointed annually by the Provost with the approval of the President.}
						\item{\underline{Accountability}:  Directly accountable to the Provost.}
						\item{\underline{Salary}:  As recommended by the Provost and approved by the President.}
						\item{\underline{Responsibilities and Duties}:
							\begin{enumerate}
								\item{Facilitate the continued development of a comprehensive undergraduate curriculum strongly built upon the classical liberal arts traditions and informed by a Christian worldview.}
								\item{Promote adoption of effective educational methodologies.}
								\item{Work with the Provost and Academic Senate, developing and implementing a strategic plan for curricular enhancement.}
								\item{Provide administrative support for the Program Review and General Education Committees}
								\item{Serve on the Academic Senate}
								\item{Serve on the Academic Senate Review Committee}
								\item{ Retain membership in the President's Council, communicating the budget implications of curricular planning decisions}
								\item{ Coordinate and provide assessment-related professional development for faculty and staff}
								\item{ Assist the Provost in the mentoring of department chairs}
							\end{enumerate}
						}
					\end{enumerate}
					\subsubparagraph{Vice Provost}
					\begin{enumerate}
						\item{\underline{Function}:  To coordinate and manage the academic planning and budget process, and to oversee institutional research.}
						\item{\underline{Term}:  Appointed annually by the Provost with approval of the President.}
						\item{\underline{Accountability}:  Directly accountable to the Provost.}
						\item{\underline{Salary}:  As recommended by the Provost and approved by the President}
						\item{\underline{Responsibilities and Duties}:
							\begin{enumerate}
								\item{Related to academic planning and budget
									\begin{enumerate}
										\item{To work closely with academic department chairs and cost-center managers that report to the Provost in soliciting annual budget and CIP requests.}
										\item{To work with the Provost in developing prioritizing budget and CIP requests for the academic area.}
										\item{Monitor academic budgets throughout the year and approve spending requests.}
										\item{Work with the Provost on matters related to faculty compensation.}
										\item{Represent the Provost on the Faculty Salary and Benefits Committee, and other committees as designated.}
									\end{enumerate}
								}
								\item{Related to institutional research
									\begin{enumerate}
										\item{Collect data from various campus sources.}
										\item{Maintain appropriate data from other institutions for comparative and planning purposes.}
										\item{Develop and maintain a College Factbook, working with college vice presidents to ensure accuracy and thoroughness.}
										\item{Complete and/or develop surveys and reports required by external agencies or for internal planning.}
										\item{Develop special reports requested or authorized by members of the Executive Team.}
									\end{enumerate}
								}
							\end{enumerate}
						}
					\end{enumerate}
					\subsubparagraph{Director of Off-Campus Programs}
					\begin{enumerate}
						\item{\underline{Function}:  the Director of Off-Campus Programs has primary responsibility over the planning, development and management of Westmont's off-campus programs, including faculty-led semester-long and Mayterm programs.}
						\item{\underline{Term}:  Appointed annually by the Provost with the approval of the President.}
						\item{\underline{Accountability}:  Directly accountable to the Provost.}
						\item{\underline{Salary}:  As recommended by the Provost and approved by the President.}
						\item{\underline{Responsibilities and Duties}:
							\begin{enumerate}
								\item{Related to Westmont Sponsored and Approved Travel Programs
									\begin{enumerate}
										\item{Chair the Off-Campus Programs Committee, overseeing those issues for which the committee has primary responsibility.}
										\item{Prepare and supervise expenditures of the budgets for the office, and any semester-long Westmont-sponsored travel program.}
										\item{Formulate, in conjunction with the appropriate committees, policy regarding risk and safety management for students involved in Off-Campus Programs.}
										\item{In consultation with the Provost, to make final decisions regarding cost-setting and itineraries for faculty-led Off-Campus Programs.}
										\item{To report to the Provost periodically on the status of all aspects of the College under his or her administrative supervision.}
									\end{enumerate}
								}
								\item{ Related to Supervision, Guidance and Advisory Roles
									\begin{enumerate}
										\item{To provide supervision and guidance to the Coordinator of Off-Campus Programs.}
										\item{To provide supervision and guidance to the Director of Westmont's Urban Program.}
									\end{enumerate}
								}
							\end{enumerate}
						}
					\end{enumerate}
					\subsubparagraph{Department Chair}
					\begin{enumerate}
						\item{\underline{Function}:  To give leadership for and coordinate the activities of the academic department.}
						\item{\underline{Appointment Procedures}:  Department chairs are appointed by the Provost after consultation with all faculty members within the department.  Preferably department chairs should be tenured and hold the rank of associate professor or higher.}
						\item{\underline{Term}:  Appointments are normally for three-year terms with the possibility of renewal.}
						\item{\underline{Responsibilities and Duties}:
							\begin{enumerate}
								\item{Provide leadership and immediate oversight of the educational program of the department}
								\item{Supervise and coordinate ongoing program review and assessment which includes the timely submission of all required reports to ensure continued departmental health and progress.}
								\item{Coordinate the administrative affairs of the department and serve as liaison to other departments of the College (e.g., working with the Registrar's Office and Director of Advising in the assignment of advisees, facilitating new student recruitment with the Admissions Office).}
								\item{Prepare and supervise expenditures of the department budget.}
								\item{Develop the departmental curriculum and teaching assignments including the recruitment and orientation of part-time faculty to support the curriculum.}
								\item{Represent the department to the Provost or Vice Provost in financial matters, course offerings, teaching load, scheduling of courses, and catalog copy.}
								\item{Conduct department business through regular meetings.}
								\item{Work with the library staff in the ordering of books and other instructional materials.}
								\item{Work with the Provost in the recruitment of full-time faculty members.}
								\item{Oversee and implement probational faculty development procedures as outlined in the \emph{Faculty Handbook}, \ref{sec:PromotionEvaluationProcedure}.}
								\item{Be a vehicle of communication for departmental faculty concerning faculty rights and responsibilities.}
								\item{Assist the Provost in the mentoring of new department chairs.}
								\item{Recruit, orient, supervise and evaluate departmental secretaries and support staff.}
							\end{enumerate}
						}
					\end{enumerate}
				\subparagraph{Vice President for Administration and Planning}
					\begin{enumerate}
						\item{\underline{Function}:  To coordinate and facilitate administrative services and oversee the offices of information technology, human resources and auxiliary services.}
						\item{\underline{Term}:  Appointed annually by the President with the approval of the Board of Trustees.}
						\item{\underline{Accountability}:  Directly accountable to the President.}
						\item{\underline{Salary}:  As recommended by the President and approved by the Board of Trustees.}
						\item{\underline{Responsibilities and Duties}:  Under the overall direction of the President, the Vice President for Administration
							\begin{enumerate}
								\item{Related to the President
									\begin{enumerate}
										\item{ To work closely with the President on matters of College policy and direction, in general, and advise specifically on the status of all aspects of the College under his supervision.}
										\item{To serve as College liaison with legal counsel.}
									\end{enumerate}
								}
								\item{Related to Supervision, Guidance, and Advisory Roles:
									\begin{enumerate}
										\item{To provide direct administrative supervision over the following:
											\begin{enumerate}
												\item{Director of Information Technology}
												\item{Director of Human Resources}
												\item{Director of Auxiliary Services}
											\end{enumerate}
										}
										\item{Chair the Diversity Committee as well as other College committees and task forces as determined by the President.}
									\end{enumerate}
								}
							\end{enumerate}
						}
					\end{enumerate}
				\subparagraph{Vice President for Finance}
					\begin{enumerate}
						\item{\underline{Function}: To manage the financial and business operations of the College in a manner which adequately supports the educational programs.}
						\item{\underline{Term}: Appointed annually by the President with the approval of the Board of Trustees.}
						\item{\underline{Accountability}: Directly accountable to the President.}
						\item{\underline{Salary}: As recommended by the President and approved by the Board of Trustees.}
						\item{\underline{Responsibilities and Duties}:
							\begin{enumerate}
								\item{Related to the President:
									\begin{enumerate}
										\item{To advise the President in matters of College policy in general, and business operations in particular.}
										\item{To report periodically to the President on the status of all aspects of the College under this Vice President's supervision.}
										\item{To formulate financial and business policies and regulations for recommendations to the President.}
										\item{To assist the President in presenting to the Finance Committee of the Board of Trustees matters relating to finance, business, and personnel, and to the Building and Grounds Committee of the Board of Trustees matters relating to campus facilities.}
										\item{To serve between the President and other administers in financial and business matters.}
										\item{To perform such other related duties and responsibilities as may be assigned by the President.}
									\end{enumerate}
								}
								\item{Related to Supervision, Guidance, and Advisory Roles:
									\begin{enumerate}
										\item{Fiscal Affairs}
										\item{Property and Facilities}
										\item{Management Systems}
										\item{Financial Aid}
									\end{enumerate}
								}
							\end{enumerate}
						}
					\end{enumerate}
				\subparagraph{Vice President for Advancement}
					\begin{enumerate}
						\item{\underline{Function}:  To relate Westmont College and its objectives to its constituent publics, and relate those publics and their resources to the College and its objectives, and to secure the favorable public opinion and resources necessary to adequately support Westmont in its program of Christian higher education.}
						\item{\underline{Term}:  Appointed annually by the President with the approval of the Board of Trustees.}
						\item{\underline{Accountability}:  Directly accountable to the President.}
						\item{\underline{Salary}:  As recommended by the President and approved by the Board of Trustees.}
						\item{\underline{Responsibilities and Duties}:
							\begin{enumerate}
								\item{Related to the President:
									\begin{enumerate}
										\item{To advise the President in matters of College policy, and general affairs in particular.}
										\item{To report periodically to the President on the status of all aspects of the College under his or her administrative supervision.}
									\end{enumerate}
								}
								\item{Related to Supervision, Guidance and Advisory Roles:
									\begin{enumerate}
										\item{Public Affairs Department}
										\item{Development Programs}
										\item{Resource Development Programs}
										\item{Publications}
										\item{News Bureau}
										\item{Promotional Programs}
									\end{enumerate}
								}
							\end{enumerate}
						}
					\end{enumerate}
				\subparagraph{Vice President for Student Life}
					\begin{enumerate}
						\item{\underline{Function}: To coordinate the student life programs in a manner conducive to maximum human growth and development.}
						\item{\underline{Term}: Appointed annually by the President with the approval of the Board of Trustees.}
						\item{\underline{Accountability}: Directly accountable to the President.}
						\item{\underline{Salary}: As recommended by the President and approved by the Board of Trustees.}
						\item{Responsibilities and Duties:
							\begin{enumerate}
								\item{Related to the President:
									\begin{enumerate}
										\item{To serve as a member of the Executive Team, providing advice on matters of College policy, especially in the area of student life.}
										\item{To assist the President in representing student needs and programs to the Student Life Committee of the Board of Trustees.}
									\end{enumerate}
								}
								\item{Related to Supervision, Guidance, and Advisory Roles:
									\begin{enumerate}
										\item{To provide administrative supervision over:
											\begin{enumerate}
												\item{Residence Life}
												\item{Counseling Services}
												\item{Health Services}
												\item{Career Development and Calling}
												\item{Campus Life}
												\item{Campus Pastor's Office}
												\item{Intercultural Programs}
											\end{enumerate}
										}
									\end{enumerate}
								}
								\item{Related to Student Life:
									\begin{enumerate}
										\item{To provide leadership in the formulation and implementation of policies and programs which provide for the orderly development of campus student life.}
										\item{To prepare and administer on-going policies for student life as developed in the \emph{Student Handbook}.}
										\item{To provide liaison with the Faculty in efforts to develop cooperative programs to enhance the mission of the College.}
										\item{To provide guidance and direction to the student government (WCSA).}
										\item{To develop conduct guidelines in alignment with college policies.}
										\item{To develop and direct student social, educational, and cultural programs.}
										\item{To conduct assessment in Student Life}
									\end{enumerate}
								}
							\end{enumerate}
						}
					\end{enumerate}

		\subsubsection{Faculty}
			\label{sec:CollegeOrganization-Faculty}

			University and college faculty have a significant role in the
			governance of the academy. At Westmont College, the faculty, under
			the final authority of the Board of Trustees, exercise primary
			authority over instruction and curriculum and share authority for
			many standards and policies.  While specific administrative
			functions have been assigned to individual faculty committees, the
			authority to recommend major changes in policy or to advise the
			administration or Board of Trustees on central issues of
			College-wide concerns rests with the faculty as a whole.

			\paragraph{Final Authority}

				The Board of Trustees has final authority to approve substantive changes in institutional purposes, policies, and programs.  The Board normally exercises its responsibility on campus through the President, who is expected to provide leadership within the entire educational program of the College.

			\paragraph{Primary Authority}
				\label{sec:CollegeOrganization-Faculty-PrimaryAuthority}
				\begin{enumerate}

					\item{ The faculty, together with the Provost, govern curriculum
						and formulate academic policies through formal action in
						faculty meetings and the committee structure; the latter is
						used to implement established policy, to develop and
						recommend changes, and to interpret policy as necessary.}
					\item{
						\label{sec:CollegeOrganization-Faculty-PrimaryAuthority-Curricular}
						Curricular and academic policy changes are processed

						through the Academic Senate. The following items require
						submission by the Faculty Council for action by the full
						faculty for implementation:
						\begin{enumerate}
							\item{ the addition or deletion of a major;}

							\item{ the addition or deletion of a complete field of study, including any programs which
								do not fall within the direct supervision of an existing academic department or any
								Westmont-operated semester-long off-campus study program;}

							\item{ the addition or deletion of a graduation requirement;}

							\item{ the addition or deletion of a general education requirement;}

							\item{ an alteration in the structure of the grading policy;}

							\item{ an alteration in the institutional academic calendar (e.g., quarter, semester);}

							\item{ an alteration in the daily class schedule.}

							\item{ an addition, deletion, or substantial change to documents that describe the goals of
								the college-wide curricular program, including those drafted for assessment purposes
								(e.g., Program Review at Westmont College: Mission-Driven, Meaningful, \& Manageable;
								educational vision documents including ``Philosophy of Education" and ``What Do We Want
								for Our Graduates?").}

						\end{enumerate}
					}
				\end{enumerate}
			\paragraph{Shared Authority}
				\begin{enumerate}
					\item{ The faculty share with administrative officers authority for developing standards and policies
						for the recruitment and admission of students. In addition, they are authorized and expected to be
						involved in the formulation of other College policies including student life, public relations,
						institutional budgeting, and long-range planning.}
					\item{ Formal relationships with the President and other administrators are to be effected through the
						Provost who has authority for the instructional program of the College.  The Provost is also the
						executive officer of the faculty, representing their will and interests consistent with the
						parameters of the College policies and mission.  As Chair of the Faculty, the Provost represents
						faculty interests within the Executive Team and to the Board of Trustees in consultation with the
						Chair of the Faculty Council.  As Vice President of the College, the Provost represents
						administrative interests and decisions to the faculty.}
				\end{enumerate}


	\subsection{Committee Structure and Responsibilities}
		\subsubsection{Committees of the Board of Trustees}
			Committees are delegated certain responsibilities for more effective handling of the Board's work.  Committees study proposals, hear constituents before recommendations are formed, receive reports, and formulate recommendations to present to the full Board.
			\paragraph{Executive Committee}
				The Executive Committee is composed of elected Board officers plus the chair of each regular committee.  The Executive Committee may transact business for the Board between regular Board meetings.  All such transactions are subject to the approval of the full Board at their next meeting.
			\paragraph{Academic Committee}
				\begin{enumerate}
					\item{Keep abreast of curricular and program changes and be aware of, and able to explain, the rationale behind the changes.}
					\item{Consult with and advise the Provost concerning academic matters.}
					\item{Understand the policies relating to promotion, tenure, and sabbatical leaves and evaluate College recommendations concerning the granting of promotion, tenure, and sabbaticals in order to make recommendations to the Board of Trustees concerning action on these matters.}
					\item{Review materials of faculty candidates recommended by the Provost and faculty with the approval of the President and resolve any questions before recommending candidates for Board approval.  When the timing of an offer of employment is crucial and the full Board cannot act, the Academic Committee may act on their behalf to authorize a contract although full Board action is required at their next regular meeting relative to employment beyond the contract year.}
					\item{Review candidates for professor emeritus/emerita status or honorary degrees and make recommendations to the Board.}
					\item{Approve policies that provide grievance procedures for faculty appeal of any perceived injustices.}
				\end{enumerate}
			\paragraph{Student Life Committee}
				\begin{enumerate}
					\item{Review all student life programs and plans and make any appropriate recommendations to the Board.}
					\item{Review student life expectations policies and conduct procedures and be able to defend and explain them.}
					\item{Consult with and advise the Vice President for Student Life concerning student life matters.}
					\item{Approve policies that provide adequate services for students' health, safety, and personal growth.}
					\item{Establish communication with students and approve policies for student appeal of any perceived injustices.}
				\end{enumerate}
			\paragraph{Membership Committee}
				\begin{enumerate}
					\item{Serve as Nominating Committee for Board officers.}
					\item{Serve as Nominating Committee for Board members, both for new and reelected.}
					\item{Assess the Board's needs for new members and maintain a roster of prospective members.}
					\item{Ensure that new trustees are properly enlisted and oriented to the institution and to their role as trustees.}
					\item{Recommend any change in membership policies to the Board with respect to Board composition, length of terms, and number of successive terms.}
					\item{Monitor the Board activities of members and ensure that all members have the opportunity to be actively involved as Trustees.}
					\item{Oversee assessment of trustee performance both individual and corporate.}
				\end{enumerate}
			\paragraph{Finance Committee}
				\begin{enumerate}
					\item{Review annual operating budget and recommend action to the Board or Executive Committee.}
					\item{Review and approve revisions to the operating budget which do not result in a deficit.}
					\item{Review and authorize allocations of the contingency budget.}
					\item{Review and authorize interfund transfers and loans.}
					\item{Monitor monthly revenues and expenditures, year-end reports, and the annual financial audit and management letter and recommendations to the Board or Executive Committee.}
					\item{Review and approve employee housing assistance transactions.}
					\item{Review and accept deferred gift agreements.}
					\item{Evaluate and authorize sales of real property and other property received as gifts.}
					\item{Evaluate, engage, and direct investment advisors.}
					\item{Evaluate and approve short-term borrowing.}
					\item{Review and approve terms of annual participation in student loan programs.}
					\item{Review and advise concerning long-range fiscal plans.}
					\item{Consult with and advise the Vice President for Finance concerning fiscal matters.}
				\end{enumerate}
			\paragraph{Development Committee}
				\begin{enumerate}
					\item{Develop an understanding of the role fund raising plays in the life of the College.}
					\item{Educate Board members of their role in acquiring gifts.}
					\item{Review the resource development program to ensure that there are sufficient resources to meet operating costs and maintain institutional integrity as to facilities and programs.}
					\item{Secure financial commitment on the part of individual Board members to personal giving and influencing other persons and organizations to support Westmont College.}
					\item{Recommend institutional policies and guidelines for fund raising.}
					\item{Identify, evaluate, cultivate, and solicit major gift prospects.}
					\item{Monitor the quality of communications with various constituencies.}
					\item{Consult with and advise the Vice President for Development concerning development matters.}
				\end{enumerate}
			\paragraph{Buildings and Grounds Committee}
				\begin{enumerate}
					\item{Review and recommend approval of a Master Plan for the physical campus which includes both present and anticipated needs.}
					\item{Review reports on physical plant utilization, maintenance, and upkeep.}
					\item{Review requests for new construction or remodeling and approve architects, plans, and building schedules.}
				\end{enumerate}
			\paragraph{Personnel Committee}
				\begin{enumerate}
					\item{Oversee the search and screening process and make the final recommendation in selection of the College President, generally working through a representative committee including faculty, students, and other constituents.}
					\item{Periodically review the work of the College President with special sensitivity to the fragile status of college presidents and the need to offer strong Board support as well as clearly-stated performance standards.}
					\item{Periodically review with the President the work of the members of his or her staff.}
				\end{enumerate}
			\paragraph{Planning Committee}
				\begin{enumerate}
					\item{Review all planning documents produced by the College, suggest planning needs if these are not being addressed by campus, and recommend planning policies to the Board.}
					\item{Provide an on-going strategic planning process for the Board of Trustees.}
					\item{Review planning assumptions, statements of institutional mission and goals, faculty and financial needs, and evaluate the success of planning procedures.}
				\end{enumerate}
			\paragraph{Diversity Committee}
				\begin{enumerate}
					\item{Advise the Board Chair on the leadership of the President and the Executive Team in implementing the College's long-range priority on diversity.}
					\item{Monitor and report to the Board on campus efforts to achieve the College's diversity goals.}
					\item{Educate the Board about the issues, challenges, and opportunities related to providing our students with a more multicultural education.}
					\item{Recommend ways in which the Board and its members can assist the College in reaching its diversity goals.}
				\end{enumerate}
			\paragraph{Historic Preservation Committee}
				\begin{enumerate}
					\item{Ensure that the traditions, culture and historical significance of the College are preserved and where appropriate, improved and enhanced.}
					\item{Promote these standards and expectations to College committees that plan and design physical improvements, capital construction and capital maintenance.}
					\item{Consult with and advise the administration about appropriate historic preservation policies and about promoting public awareness and celebration of the College's traditions and heritage.}
					\item{Monitor the work of the College Archivist to collect and store materials of historic significance.}
				\end{enumerate}
		\subsubsection{Committees of College Administration}
			\paragraph{Strategic Planning Committee}
				\begin{enumerate}
					\item{\underline{Membership}:
						\begin{enumerate}
							\item{President}
							\item{Chair, Board of Trustees}
							\item{Chair, Trustee Planning Committee}
							\item{Member, Trustee Planning Committee}
							\item{Provost}
							\item{Associate Provost for Planning and Research}
							\item{Vice President for Administration and Planning}
							\item{Vice President for Advancement and CIO}
							\item{Vice President for Finance}
							\item{Vice President for Student Life}
							\item{Two Vice Chairs of the Faculty}
							\item{Two Academic Senators, specifically the Chair of the Senate and one additional senator selected with an eye toward divisional representation and continuity}
							\item{Dean of Admissions}
							\item{Dean of Curriculum and Educational Effectiveness}
							\item{Dean for Student Engagement}
							\item{Dean of Students}
							\item{Director of Library and Information Services}
							\item{WCSA representative}
						\end{enumerate}
					}
					\item{\underline{Officers}
						The President is the convener of the Strategic Planning Committee.  Committee meetings are facilitated by a strategic planning consultant.  The President and the Chair of the Trustee Planning Committee serve as primary liaisons with the Board of Trustees.}
					\item{\underline{Responsibilities}
						\begin{enumerate}
							\item{Represent and interact with college constituencies to:
								\begin{itemize}
									\item{ Assess Westmont's strengths, weaknesses,
										opportunities and threats}
									\item{ Identify strategic priorities for
										possible inclusion on a three-year strategic
										map}
								\end{itemize}
							}
							\item{ Recommend a three-year strategic map, and annual
								strategic priorities, to the Board of Trustees.}
							\item{ Regularly review progress on implementing
								strategic priorities}
							\item{ When appropriate, make needed adjustments to the
								process of implementing strategic priorities; and when
								appropriate, recommend strategic map adjustments to the
								Board of Trustees.}
						\end{enumerate}
					}
				\end{enumerate}
		\subsubsection{Committees of the Faculty}
			\label{sec:CommitteesOfTheFaculty}
			See also
			\ref{sec:ParticipationInCampusGovernance}
			\nameref{sec:ParticipationInCampusGovernance}


			\paragraph{Faculty Meetings}
				\label{sec:FacultyMeetings}
				\begin{enumerate}[label=\alph*)]
					\item{Membership:
						\begin{enumerate}[label=\arabic*)]
							\item{The privilege of \underline {voice with vote} at faculty meetings is limited to the following:
								\begin{enumerate}[label=(\alph*)]
									\item{ faculty with rank of Instructor,
										Assistant, Associate, and Full Professor with
										one-half time or more excluding non-credit
										courses;}
									\item{ certain persons with faculty status whose
										duties are closely allied with the instructional
										faculty:  Director of the Library and
										Information Services, Associate Director of the
										Library, Director of Admissions, Registrar, and
										full-time personnel supervising activities for
										which students receive credit toward graduation
										(e.g., varsity coaches);}
								\end{enumerate}
							}
							\item{The privilege of \underline{voice without vote} at faculty meetings is extended to the following:
								\begin{enumerate}[label=(\alph*)]
									\item{ persons with faculty status but not
										rank;}
									\item{ faculty with rank who teach less than
										one-half time, administrators with faculty rank,
										visiting faculty, part-time faculty;}
									\item{ members of the Executive Team.}
								\end{enumerate}
							}
							\item{The privilege of \underline{observer status without voice or vote} may be extended to selected staff who have extensive involvement in student life.  The W.C.S.A President and the Horizon Editor may attend faculty meetings without voice or vote; the faculty reserves the right to meet without student observers.}
							\item{In all cases the Faculty Personnel Committee shall determine disputes involving voting rights.}
						\end{enumerate}
					}
					\item{Officers
						\begin{enumerate}[label=\arabic*)]
							\item{The Provost shall serve as Chair of the Faculty.}
							\item{The Provost will appoint annually a secretary for faculty meetings.}

							\item{The Vice-Chair of the Faculty will preside for formal actions at all faculty meetings, and will chair faculty meetings in the absence of the Provost.  In the absence of the Vice-Chair these duties will be assumed by the member of the Faculty Council in attendance who has the longest seniority at the College.}
						\end{enumerate}
					}
					\item{Procedures
						\label{sec:FacultyMeetings-Procedures}
						\begin{enumerate}[label=\arabic*)]
							\item{Regularly scheduled meetings will be held at least once per month during the school year.}
							\item{Special meetings may be called by the President, Provost, the Faculty Council, or by 15 percent of the full-time faculty upon petition to the Faculty Council.}
							\item{A quorum for the conduct of business will be one-half plus one of the current voting membership of the faculty.}
							\item{Formal action by the faculty requires a
								majority vote of the eligible
								faculty present, except as otherwise
								provided in the \emph{Faculty
									Handbook} (e.g.,
								\nameref{sec:EmergencySuspension},
								\ref{sec:EmergencySuspension}
								).}
							\item{Faculty who must be absent from a faculty meeting may arrange with the chair of the Faculty Council for an absentee ballot to be counted on their behalf provided that:
								\begin{enumerate}[label=(\alph*)]
									\item{The vote is related to an election.}
									\item{The ballot has been published prior to the meeting and has not been changed.}
								\end{enumerate}
							}
							\item{In absentia voting is not allowed on action items other than elections.}
							\item{The agenda will be determined by the Faculty Council.  Items requiring faculty action will be submitted to the Faculty Council and distributed to faculty members at least one week in advance of the meeting.}
							\item{The Professional Development Committee will arrange programs relating to faculty growth for certain faculty meetings.}
							\item{
								\label{sec:FacultyMeetings-Procedures-RobertsRules}
								Meetings will be governed by \emph{Robert's Rules of Order}, Newly Revised, $9^{th}$ ed (1990).}
							\item{At least once per semester the full-time teaching faculty with the rank of Instructor, Assistant, Associate, and Full Professor, will caucus in executive session without the attendance of administrators.  Additionally, executive sessions will be called by the Faculty Council at the request of one-third of the teaching faculty.  In executive sessions of the faculty the Vice-Chair, or his or her designee, will preside.}
						\end{enumerate}
					}
					\item{Responsibilities
						\begin{enumerate}[label=\arabic*)]
							\item{To articulate and promote institutional objectives.}
							\item{To establish and govern the academic life and curricular structure of the College.}
							\item{To implement, through instruction, research, committee work, and counsel, instructional programs.}
							\item{To establish and maintain academic standards, criteria for admission and retention of students, degree requirements, and policies relating to financial aid.}
							\item{To recommend to the Board of Trustees for approval all candidates for degrees.}
							\item{To establish and to elect faculty members to standing and ad hoc committees for the governance of the College.}
							\item{To communicate through the Provost and the Faculty Council, the opinions and counsel of the Faculty to the Board of Trustees, the administration, and committees on any issue affecting faculty, student life and other aspects of the College.}
							\item{To foster the professional development and personal well-being of individual faculty members and collegial relationships.}
						\end{enumerate}
					}
				\end{enumerate}
			\paragraph{	Nomination and Election to Faculty Committees}
				\label{sec:NominationAndElectionToFacultyCommittees}
				Elections to Faculty committees will be held during the second semester of each academic year.
				\subparagraph{Election to Faculty Council}
					\subsubparagraph{Nomination}
					No later than three days before the nomination, Faculty Council will present to the Faculty a nominating ballot containing the names of all faculty eligible to be nominated, and indicating which are tenured.  The Faculty will nominate their individual choices of two candidates for each tenured vacancy, and two for each additional vacancy.
					\subsubparagraph{Election}
					\begin{enumerate}[label=\alph*)]
						\item{No later than three days before the election, Faculty Council will present to the Faculty an election ballot, which will include:
							\begin{enumerate}[label=\arabic*)]
								\item{the names of the tenured faculty who have received the most nominations:  two for each tenured vacancy, or as many more as may be tied with them;}
								\item{the names of the remaining faculty, tenured or not, who have received the most nominations:  two for each additional vacancy, and as many more as may be tied with them.}
							\end{enumerate}
						}
						\item{Faculty will cast one vote for each vacancy.}
						\item{Any tenured vacancy will be filled by the tenured faculty who have received the most votes; any additional vacancy will be filled by the remaining faculty, tenured or not, who have received the most votes.}
						\item{In case of a tie:
							\begin{enumerate}[label=\arabic*)]
								\item{on a ballot of one vacancy only, the Vice-Chair will cast the deciding vote;}
								\item{on a ballot for two or more vacancies, a run-off election will decide the outcome.}
							\end{enumerate}
						}
					\end{enumerate}
				\subparagraph{Election to the Faculty Personnel Committee and Academic Senate}
					No later than three days before the election, Faculty Council will present to the Faculty a ballot containing two nominees for each seat open on the Faculty Personnel Committee and Academic Senate.  Faculty will cast one vote for each vacancy.
				\subparagraph{Election to Other Committees}
					After the election of faculty to the Faculty Council, Academic Senate, and the Faculty Personnel Committee, and the Academic Senate, Faculty Council will present to the Faculty, no later than three days before the election to other committees, a slate of single nominees for all the elected seats open on the remaining committees, with the possibility of additional nominations from the floor, upon the prior consent of the nominee(s).  When eligible candidates are limited, Faculty Council can nominate a faculty member from one division to serve in a position designated for a different division, making this substitution clear on the nomination ballot.
				\subparagraph{Election of the Vice-Chair of the Faculty}
					Once the new member(s) of Faculty Council shall have been elected, Faculty Council will present to the Faculty, no later than three days before the election, a slate containing the names of all tenured members of the Faculty Council as constituted for the academic year following.  Faculty will cast one vote; the candidate who secures a plurality of votes will be the Vice-Chair; the Faculty will break any tie by a run-off election.
				\subparagraph{Conditions for Nomination and Election to Faculty Committees}
					\label{sec:CommitteesOfTheFaculty-Conditions}
					Each faculty member has the opportunity:
					\begin{enumerate}[label=\alph*)]
						\item{
							\label{sec:CommitteesOfTheFaculty-Conditions-First}
							To express preferences for committee memberships as well as recommend specific individuals for committee vacancies;}
						\item{To nominate eligible faculty from the floor, their prior consent having been secured, for open committee seats (except for Faculty Council);}
						\item{To limit service to one committee at a time, while the available faculty suffice; or failing that, to limit service to one standing committee at a time;}
						\item{To limit service to one committee per year (assuming a sufficient number of faculty are available);}
						\item{To serve voluntarily on more than one committee of whatever status;}
						\item{To be exempt from service on any committee for one year following completion of a full term on Faculty Council, Academic Senate, or Faculty Personnel Committee.}
					\end{enumerate}
				\subparagraph{Officers of Faculty Committees}
					\begin{enumerate}[label=\alph*)]
						\item{Chairs, secretaries, and other officers as may be necessary will be designated in accordance with the particular provisions of the \emph{Faculty Handbook} for each committee.}
						\item{Chairs will vote only to break a tie.}
					\end{enumerate}
				\subparagraph{Ex-Officio Members}
					The President, Provost, Vice Provost, Associate Dean of the Faculty, and Dean of Curriculum and Educational Effectiveness are \underline{ex-officio members without vote} on all faculty committees, except as provided for otherwise.
				\subparagraph{Student Members}
					Student members serve on faculty committees as established by the specific provisions for student membership for the committee to which a student is appointed. Appointments are made by the W.C.S.A. student council.  Student copies of committee minutes are retained in the office of the committee chair where student committee members have access to them.
				\subparagraph{Replacement of Faculty Committee Members}

					In the case of an inactive faculty committee member, the committee chair will report to the Provost any committee member who does not regularly participate.  If the matter is not resolved, the Provost will ask the Faculty Council to replace the member for the remainder of the elected term.
					Faculty Council is responsible for filling temporary committee vacancies (e.g. sabbaticals, committee release, abroad programs and medical or professional leaves).  Replacements may be provided through the normal election process, by a special election, or by appointment, at the discretion of Faculty Council.
			\paragraph{Curricular}

				\subparagraph{Academic Senate}
					\begin{enumerate}[label=\alph*)]
						\item{\underline{Membership}:
							\begin{enumerate}[label=\arabic*)]
								\item{Eight elected faculty, two from each division plus two additional faculty at large
									\begin{enumerate}[label=(\alph*)]
										\item{Only faculty who have served at the college for a minimum of 6 years are eligible for election to Academic Senate.}
										\item{For divisional representatives, only faculty who currently serve or have previously served as department chair are eligible for election to Academic Senate.}
										\item{Faculty are typically elected to 3 year terms, such that a maximum of one new senator is elected each year within each division.}
									\end{enumerate}
								}
								\item{Provost and Dean of Curriculum and
									Educational Effectiveness (both \emph{ex officio} without vote)}
								\item{Registrar}
								\item{W.C.S.A. President (or representative) (voice without vote)}
							\end{enumerate}
						}
						\item{\underline{Observers}:  Additional \emph{ex officio} personnel including the Director of Off-Campus Programs, the Associate Provost for Planning and Research, and the Director of the Library and Information Services are welcome as regular participants in Academic Senate meetings.  Academic Senate meetings are also open to other faculty.  All such participants have voice without vote.}
						\item{\underline{Officers}:
							\begin{enumerate}[label=\arabic*)]
								\item{The Provost shall serve as chair.}
								\item{Each spring, the voting members of the following year's senate shall elect a vice-chair from among its faculty members.  The vice-chair shall meet regularly with the Provost to set agendas and will chair Academic Senate in the Provost's absence.}
								\item{The Registrar shall serve as secretary.}
							\end{enumerate}
						}
						\item{\underline{Responsibilities}:
							\begin{enumerate}[label=\arabic*)]
								\item{To steward and maintain the integrity of the academic program, including curriculum, general education, majors, support programs, and off-campus and special programs.
									\begin{enumerate}[label=(\alph*)]
										\item{To engage in academic planning, to assist the Provost with the crafting and implementation of the strategic plan for the academic programs and to recommend changes in academic staffing resources.}
										\item{To review, approve, and instate changes in academic programs, and to submit these changes through Faculty Council for action by the full faculty in cases where said changes are designated in
											section
											\ref{sec:CollegeOrganization-Faculty-PrimaryAuthority}
											\ref{sec:CollegeOrganization-Faculty-PrimaryAuthority-Curricular}
											as requiring a vote of the full faculty.}
										\item{To read and discuss annual reports from the Program Review, General Education, Academic Resources, and Off-Campus Programs committees, and to ensure that academic evaluation effectively takes place.}
										\item{To accept and consider proposals from students, faculty, staff, or other constituencies and respond with recommendations.}
										\item{To recommend changes in admissions policies and practices.}
										\item{To recommend changes in the area of learning resources, including the library and educational media.}
										\item{To initiate changes in grading practices and student evaluations.}
									\end{enumerate}
								}
								\item{To meet regularly and to distribute minutes to faculty in a timely fashion.}
								\item{To report to the full faculty any changes within existing curricular structures at the next regularly scheduled faculty meeting following such a decision. A representative of the Academic Senate shall report to the full faculty any changes within existing curricular structures at the next regularly scheduled faculty meeting following such a decision.}
							\end{enumerate}
						}
						\item{\underline{Means of Appeal}:  Appeal of actions by the Academic Senate shall be brought to the full faculty upon the request of seven persons with faculty status who have both voice and vote.}
						\item{\underline{Subcommittee Assignment}:  Senators, excluding the chair, are appointed annually to 1-year terms on the subcommittee of the Academic Senate (Review Committee; three senators) and the Strategic Planning Committee (two senators).  Appointments are made by the Provost in consultation with the Academic Senate Vice-Chair elect.  These appointments will occur in the spring term following Academic Senate elections, with the aim of maximizing continuity on the committee, divisional representation, and fit of expertise and experience.}
					\end{enumerate}
				\subparagraph{Academic Senate:  Review Committee}
					\begin{enumerate}[label=\alph*)]
						\item{\underline{Membership}:
							\begin{enumerate}[label=\arabic*)]
								\item{Provost or representative}
								\item{Registrar}
								\item{Three senators, each appointed by the Provost and Academic Senate Vice Chair to renewable one year terms}
								\item{One student appointed by W.C.S.A. (voice without vote)}
							\end{enumerate}
							All Academic Senate members shall have voice without vote at every meeting.
						}
						\item{\underline{Officers}:
							The chair shall be elected annually by the members of the Review Committee and the Registrar shall serve as secretary.
						}
						\item{\underline{Responsibilities}:
							\begin{enumerate}[label=\arabic*)]
								\item{Acts for the Academic Senate on student petition appeals, honors and alternative-major proposals.}
								\item{Acts upon minor curricular changes that do not involve, for example, the addition or deletion of a major, a change in units required for a major, or any change that signals a major shift in emphasis of the College.}
								\item{Appeals for action related to student petitions, honors, and alternative-major proposals shall be to the Provost; appeals for action to minor curricular changes shall be to the Academic Senate.}
							\end{enumerate}
						}
					\end{enumerate}
			\paragraph{Administrative}
				\subparagraph{Faculty Council}
					\begin{enumerate}[label=\alph*)]
						\item{\underline{Membership}:
							\begin{enumerate}[label=\arabic*)]
								\item{Five full-time faculty members elected at large by the faculty; at least three must be tenured; the remaining members must either be tenured, on tenure-track or multi-year contract, or library faculty having voice and vote.
									\begin{enumerate}[label=(\alph*)]
										\item{Term of office is for three years; no member may be re-elected to the Faculty Council during the year following expiration of term.}
									\end{enumerate}
								}
								\item{Any faculty member may attend meetings of the Faculty Council (with voice but without vote, except those closed by majority vote of the Council).}
							\end{enumerate}
						}
						\item{\underline{Officers}:
							\begin{enumerate}[label=\arabic*)]
								\item{The entire faculty will elect, from among the tenured members of the Council, a Vice-Chair of the Faculty who will preside at meetings of the Council.  The Vice-Chair will also preside over meetings of the full faculty in the absence of the Provost.  Term of office of the Vice-Chair is one year.}
								\item{The Council will elect the secretary from among its membership.}
							\end{enumerate}
						}
						\item{\underline{Responsibilities}:
							\begin{enumerate}[label=\arabic*)]
								\item{To meet regularly and to distribute minutes to the Faculty in a timely fashion.}
								\item{To provide a forum where the administration can seek, or faculty can volunteer, faculty advice or opinion on matters not on the agenda of other faculty committees.}
								\item{To provide a forum where faculty grievances can be aired and brought to the attention of the administration.}
								\item{To recommend to the Faculty, and to other appropriate bodies, changes in policy (including the triennial review of the \emph{Faculty Handbook} provided by
									section
									\ref{sec:ProtocolsForRevision-Publication}
									), and to present for the Faculty's approval all proposed major changes in policy.}
								\item{To establish the schedule of regular faculty meetings as stipulated in
									section
									\ref{sec:FacultyMeetings}, and to call special faculty meetings as necessary.}
								\item{To determine the agenda of all faculty meetings.}
								\item{To plan faculty retreats.}
								\item{To oversee the faculty standing committee nomination and election process (as stipulated in
									section
									\ref{sec:NominationAndElectionToFacultyCommittees}
									)

									and to appoint replacements as necessary.}
							\end{enumerate}
						}
					\end{enumerate}
				\subparagraph{Faculty Budget and Salary Committee}
					\begin{enumerate}[label=\alph*)]
						\item{\underline{Membership}:
							\begin{enumerate}[label=\arabic*)]
								\item{Provost (or representative)}
								\item{Three tenured full professors, one elected annually to a three-year term}
								\item{Two faculty not full professors at time of appointment, one elected annually to a two-year term}
							\end{enumerate}
						}
						\item{\underline{Officers}:
							The full professor serving in the second year of the three-year term shall be the committee chair.  The chair and the most recent past chair shall be the faculty representatives on the President's Advisory Council.
						}
						\item{\underline{Responsibilities}:
							\begin{enumerate}[label=\arabic*)]
								\item{To monitor the amount budgeted for the academic program relative to all other areas of the College.}
								\item{To maintain comparisons of the College's salary and benefits schedule with those of comparable institutions.}
								\item{To provide a forum for faculty input to Human Resources when changes in Faculty and Staff benefits are proposed; and to initiate changes in benefits at appropriate times.}
								\item{To maintain communication with the faculty regarding salary and benefits.}
							\end{enumerate}
						}
					\end{enumerate}
				\subparagraph{Communications Board}
					\begin{enumerate}[label=\alph*)]
						\item{\underline{Membership}:
							\begin{enumerate}[label=\arabic*)]
								\item{Provost (or representative)}
								\item{Vice President for Student Life (or representative)}
								\item{One faculty elected bi-annually to a two-year term}
								\item{One student}
								\item{Editors of student publications}
								\item{Faculty advisors to the student publications}
							\end{enumerate}
						}
						\item{\underline{Officers}:
							The chair will be elected from the faculty and staff not in advisorship and shall appoint a secretary.
						}
						\item{\underline{Responsibilities}:
							\begin{enumerate}[label=\arabic*)]
								\item{To serve as a selection committee for editors.}
								\item{To select advisors.}
								\item{To establish guidelines for advisors to the publications.}
								\item{To receive periodic reports of progress toward publication.}
								\item{To receive and act upon grievances related to the publications from any member of the community.}
								\item{To serve as a review board on questions relating to editorial policy and content.}
								\item{To conduct dismissal proceedings of editors who act in consistent disregard of the stated goals of the College.}
							\end{enumerate}
						}
					\end{enumerate}
				\subparagraph{Academic Resources Committee}
					The Academic Resources Committee (ARC) is concerned with reviewing, recommending and implementing resources suitable for faculty and student use in the classroom, the library, and in the area of information technology. Additionally, the ARC reviews and makes recommendations to the College regarding budget allocation and prioritization of instructional resources. Such recommendations should account for the sustainability of new resources as well as the maintenance of old resources.
					\begin{enumerate}[label=\alph*)]
						\item{\underline{Membership}:
							\begin{enumerate}[label=\arabic*)]
							\item{Provost or representative (ex-officio)}
							\item{Director of Library \& Information Services or representative (ex-officio)}
							\item{Chief Information Officer or representative (ex-officio)}
							\item{Three faculty, preferably one from each division, one elected annually to a three-year term}
							\item{One student appointed by W.C.S.A.}
							\item{On an ad hoc basis, such members of the library and information technology faculty or staff who may be required for a specific purpose, who will have voice without vote}
							\end{enumerate}
						}
					      \item{\underline{Officers}:
							The chair will be elected from among the tenured faculty members and shall appoint the secretary}	
      						\item{\underline{Reports to}:
							\begin{enumerate}[label=\arabic*)]
							\item{In matters regarding policy that affect instruction, the ARC reports to the Academic Senate and Provost.}
							\item{In matters regarding acquisition and implementation of technology, the ARC makes recommendations to the Chief Information Officer and reports to the Provost.}
							\end{enumerate}
       						}
						\item{\underline{Purposes}:
							\begin{enumerate}[label=\arabic*)]
							\item{To assist the Chief Information Officer in determining what new technologies and products are appropriate for Westmont College, and which should be evaluated for instructional use.
							\item{To propose the acquisition and implementation of new instructional resources, and to advocate for the maintenance and upkeep of existing facilities and instructional resources.}
							\item{To advocate for and recommend policy relating to technology issues that affect faculty and students.}
							\item{To assist the Director of Library & Information Services in library-related matters that affect faculty and students, including collection development, building-related matters, staffing, programs, and budgetary issues.}
							\item{To refer items (as appropriate) to the Academic Senate for their recommendation to the faculty.}
							\item{The Chief Information Officer or a designate will as needed provide updates to the ARC on support or procurement related challenges specific to the classroom and other academic technology areas.}
							\item{To review and make recommendations to the Provost (or representative) and Capital Improvement Project committee regarding budget allocation for items that affect faculty and students.}
					\end{enumerate}
						}     
				\subparagraph{Athletic Committee}
					\begin{enumerate}[label=\alph*)]
						\item{\underline{Membership}:
							\begin{enumerate}[label=\arabic*)]
								\item{Provost (or representative)}
								\item{Vice President for Student Life (or representative)}
								\item{Faculty Athletic Representative, who is appointed by the President and must have faculty status.}
								\item{Athletic Director}
								\item{Senior Woman Administrator}
								\item{Two faculty members, one elected annually to a two-year term}
							\end{enumerate}
						}
						\item{\underline{Officers}:
							The Faculty Athletic Representative will serve as chair; the chair shall appoint the secretary.
						}
						\item{\underline{Responsibilities}:
							\begin{enumerate}[label=\arabic*)]
								\item{To serve as a liaison between the academic and athletic programs, hearing faculty questions and concerns, identifying issues needing to be addressed, and enhancing communication.}
								\item{To provide an annual report (and, if needed, other periodic reports) to the faculty about the athletic program (including information on graduation rates and GPAs for athletes, changes in conference policy and procedures, and other items of relevance to the academic program).}
								\item{To review guidelines regarding scheduling (e.g. length of season, number of games, amount of practice time, policies on class attendance and make-ups, postseason participation).  The Athletic Director shall oversee implementation of these guidelines for each season of play.}
								\item{To approve eligibility regulations for Westmont athletes with respect to those set forth by any conferences and national affiliations to which Westmont may belong.  To review reports presented by the Faculty Athletic Representative and the Athletic Director on academic progress of student athletes.  These reports normally would be presented in early fall.}
								\item{To review proposals for conference and national affiliations or for the addition and deletion of athletic programs.  To make recommendations on these matters to the Provost.}
								\item{To review plans for the college's compliance with the Title IX mandates and to make recommendations for ensuring gender equity.}
								\item{To assist in the recruitment and hiring of any full-time coaching staff.}
							\end{enumerate}
						}
					\end{enumerate}
				\subparagraph{Off-Campus Programs Committee}
					\begin{enumerate}[label=\alph*)]
						\item{\underline{Membership}:
							\begin{enumerate}[label=\arabic*)]
								\item{Three faculty members with differing disciplinary backgrounds, at least one of whom has significant experience with a Westmont off-campus program, one elected annually to a three-year term.}
								\item{ Registrar (\emph{ex officio})}
								\item{Director of Off-Campus Programs}
								\item{Assistant Director of Global Education (voice without vote)}
							\end{enumerate}
						}
						\item{\underline{Officers}:
							The Director of Off-Campus Programs shall serve as chair person.
						}
						\item{\underline{Responsibilities}: The Off-Campus Programs committee provides academic oversight to off-campus programs, including all Westmont-operated and Westmont-approved programs for which credit is granted.
							\begin{enumerate}[label=\arabic*)]
								\item{The committee provides governance for off-campus programs analogous to that normally found in an academic department, including but not limited to the evaluation of programs, determination of qualifications for and recommendations to the Provost regarding staffing of Westmont-operated programs, and oversight of curriculum.}
								\item{The committee is not generally involved in the logistical arrangements for the various off-campus programs.}
								\item{The committee will submit to the Academic Senate any proposed policy or curriculum changes that might require approval by Academic Senate or the full faculty, and to receive from Academic Senate requests for action and information relating to matters of the academic program appropriate to its purpose and function.}
								\item{The committee will submit to academic senate an annual 1- to 2- page summary of its work related to policy, program or curricular changes to the college's off-campus programs.}
							\end{enumerate}
						}
					\end{enumerate}
				\subparagraph{Program Review Committee}
					\begin{enumerate}[label=\alph*)]
						\item{\underline{Membership}:
							\begin{enumerate}[label=\arabic*)]
								\item{Dean of Curriculum and Educational Effectiveness}
								\item{Director of Research, Planning and Implementation}
								\item{Librarian with Faculty Status}
								\item{Vice President for Student Life (or representative)}
								\item{Three faculty with differing disciplinary backgrounds, one elected annually to a two-year term}
							\end{enumerate}
						}
						\item{\underline{Officers}:
							\begin{enumerate}[label=\arabic*)]
								\item{Chair, elected each year from among the committee members.}
								\item{Secretary selected by committee.}
							\end{enumerate}
						}
						\item{\underline{Responsibilities}:
							The Program Review Committee oversees program review in academic and co-curricular departments and programs, and develops an institutional plan for college-wide program review.  It establishes policies and procedures regarding program review and assessment, and maintains these in a document on file in the Provost's Office.  Specific responsibilities of the Program Review Committee include:
							\begin{enumerate}[label=\arabic*)]
								\item{In collaboration with the Dean of Curriculum and Educational Effectiveness
									\begin{enumerate}[label=(\alph*)]
										\item{to work with departments as needed in the development of their plan for program review;}
										\item{assist departments with the generation and completion of the annual assessment reports and multi-year program review reports;}
										\item{to review the annual reports and provide timely feedback to departments;}
										\item{to review the results of the multi-year review and provide timely feedback for the department to consider for its meeting with the Provost and the Dean of Curriculum and Educational Effectiveness;}
										\item{to provide the department with comments or suggestions to assist them in their preparation for the next review cycle;}
										\item{to encourage a campus conversation that establishes the value of a college-wide program review and addresses concerns as appropriate.}
									\end{enumerate}
								}
								\item{To work with and advise the Director of Research, Planning and Implementation in establishing goals, formats and priorities for the collection and reporting of student data.}
								\item{To work with and advise the Dean of Curriculum and Educational Effectiveness in organizing the data, planning a schedule for Program Review and presenting the data to the campus communities and accrediting organizations.}
								\item{To submit to the Academic Senate any proposed policy or curriculum changes that might require approval by Academic Senate or the full faculty, and to receive from Academic Senate requests for action and information relating to matters of the academic program appropriate to its purpose and function.}
								\item{To report regularly to Academic Senate through the Dean of Curriculum and Educational Effectiveness, and to submit an annual 1- to 2-page summary of its work related to review of the college's academic program, including a summary of its review of departmental multi-year reports submitted that year.}
							\end{enumerate}
						}
					\end{enumerate}
				\subparagraph{General Education Committee}
					\begin{enumerate}[label=\alph*)]
						\item{\underline{Membership}:
							\begin{enumerate}[label=\arabic*)]
								\item{Dean of Curriculum and Educational Effectiveness.}
								\item{Director of the Library or librarian with faculty status, appointed by the Director of the Library in consultation with Faculty Council.}
								\item{Three tenured faculty, one from each division, one elected annually to a three-year term.}
								\item{Registrar}
							\end{enumerate}
						}
						\item{\underline{Officers}:
							\begin{enumerate}[label=\arabic*)]
								\item{Chair, elected each year from among the committee members.}
								\item{Secretary, selected by committee.}
							\end{enumerate}
						}
						\item{\underline{Responsibilities}:
							\begin{enumerate}[label=\arabic*)]
								\item{To oversee the implementation of the general education program of the College according to the criteria approved by the faculty.  Policies, procedures, and criteria for course inclusion in the various general education categories are maintained in documents on file in the Provost's office.}
								\item{To review and approve new course proposals to ensure that they adhere to the general education criteria approved by the faculty.}
								\item{To facilitate regular program review and assessment of the general education program, as well as course content and methodology of courses previously approved as fulfilling general education criteria.}
								\item{To conduct periodic syllabus review.}
								\item{To recommend change in the general education program to the Academic Senate, with proposals that would significantly alter any general education requirement being forwarded to the full faculty for action.}
								\item{To submit to the Academic Senate any other proposed policy or curriculum changes that might require approval by Academic Senate or the full faculty, and to receive from Academic Senate requests for action and information relating to matters of the academic program appropriate to its purpose and function.}
								\item{To report regularly to Academic Senate through the Dean of Curriculum and Educational Effectiveness, and to submit an annual 1- to 2-page summary of its work related to review of the college's academic program.}
							\end{enumerate}
						}
					\end{enumerate}
			\paragraph{Institutional}
				\subparagraph{Admissions and Retention Committee}
					\begin{enumerate}[label=\alph*)]
						\item{\underline{Membership}:
							\begin{enumerate}[label=\arabic*)]
								\item{Vice President for Student Life (or representative)}
								\item{Registrar}
								\item{Vice President for Enrollment, Marketing, and Communications (or representative)}
								\item{Senior Director of Admissions}
								\item{Coordinator of Academic Support Services}
								\item{Three faculty members, one elected annually to a three-year term.  At least one faculty member of the committee or a substitute faculty member must be present at all meetings.}
								\item{Two students, one man and one woman, appointed by the W.C.S.A.}
							\end{enumerate}
						}
						\item{\underline{Officers}:
							The Senior Director of Admissions (or representative) shall serve as chair; the chair shall appoint the secretary.  The chair votes only to break a tie.
						}
						\item{\underline{Responsibilities}:
							\begin{enumerate}[label=\arabic*)]
								\item{To review annually and approve policies relative to admissions consistent with the character and objectives of the institution.}
								\item{To serve as a board of interview for applicants when necessary.}
								\item{To review and render decisions on applicants who do not meet the established criteria for admission.}
								\item{To review and interview, if requested, applications from students not admitted and are requesting an appeal of the decision.}
								\item{To review and interview, if requested, students who have been placed on academic suspension and are requesting an appeal of the decision.}
								\item{To communicate with Academic Senate regarding the curricular interests of students.}
								\item{To communicate and work with the Diversity Committee on matters relating to diversity, admissions and retention.}
							\end{enumerate}
						}
					\end{enumerate}
				\subparagraph{Student Life Committee}
					\begin{enumerate}[label=\alph*)]
						\item{\underline{Membership}:
							\begin{enumerate}[label=\arabic*)]
								\item{Provost (or representative)}
								\item{Vice President for Student Life (or representative)}
								\item{Two Student Life staff members}
								\item{Three faculty members, one elected annually to a three-year term}
								\item{Two students selected by the W.C.S.A. President in consultation with the Vice President for Student Life}
							\end{enumerate}
						}
						\item{\underline{Officers}:
							The chair shall be elected from among the faculty or staff members and shall appoint the secretary.
						}
						\item{\underline{Responsibilities}:
							\begin{enumerate}[label=\arabic*)]
								\item{To promote discussion among administration, faculty, and students about issues related to the Student Life Department such as:
									\begin{enumerate}[label=(\alph*)]
										\item{Student development philosophy}
										\item{Student Life programming}
										\item{Student services}
										\item{Concerns related to Student Life department}
										\item{Evaluation of the Student Life program}
									\end{enumerate}
								}
								\item{To initiate discussion among administration, faculty, and students on matters relating to the spiritual climate of the college}
								\item{To review periodically judicial procedures and conduct sanctions to ensure a student's access to a fair and impartial hearing and to a consistent application of sanctions.}
							\end{enumerate}
						}
					\end{enumerate}
				\subparagraph{Diversity Committee}
					\begin{enumerate}[label=\alph*)]
						\item{\underline{Membership}:
							\begin{enumerate}[label=\arabic*)]
								\item{Provost}
								\item{Vice President for Student Life}
								\item{Vice President for Administration and Planning}
								\item{Three faculty, one elected annually to a three-year term}
								\item{Special Assistant to the Provost for Diversity Initiatives}
								\item{Dean of Admissions}
								\item{Director of Intercultural Programs}
							\end{enumerate}
						}
						\item{\underline{Officers}:
							The chairperson will be appointed by the President in consultation with the President's Executive team and Diversity Committee members; the committee will elect a secretary.}
						\item{\underline{Responsibilities}:
							\begin{enumerate}[label=\arabic*)]
								\item{To establish links with others that are working to address diversity-related issues at Westmont---for example, Trustee Diversity and Global Engagement Committee, Executive Team, Human Resources, Residence Life, Intercultural Programs, Diversity Recruitment Specialists, Off Campus Programs.}
								\item{To seek to build community-wide awareness and ownership of diversity-related issues and of appropriate ways to address them.}
								\item{To summarize the various challenges and opportunities related to diversity that currently face the College.}
								\item{To enlist additional faculty, administrators and staff to work with the Committee on special initiatives.}
								\item{To recommend priorities and propose implementation plans to the President's Executive Team.}
								\item{To make annual reports on the Committee's work to the Faculty, the Executive Team and the Trustee Diversity Committee.}
							\end{enumerate}
						}
					\end{enumerate}
				\subparagraph{Faculty Personnel Committee}
					\begin{enumerate}[label=\alph*)]
						\item{\underline{Membership}:
							All committee members, except the Provost, are eligible to vote.
							\begin{enumerate}[label=\arabic*)]
								\item{Provost}
								\item{Three tenured full professors and two tenured associate professors, each elected for a two-year term.  The three full professors must come from the three academic divisions of the college, and the associate professors elected must not be scheduled for a review during their term of service.}
								\item{At most three additional full professors, each elected to a one-year term, depending on the anticipated number of faculty being reviewed for a given year.}
								\item{One previously reviewed probationary faculty member to be elected annually for a one-year term and must not be scheduled for a review during their term of service.}

							\end{enumerate}
						}
						\item{\underline{Officers}:
							\begin{enumerate}[label=\arabic*)]
								\item{Chair, a full professor, to be elected from among the voting committee members.}
								\item{Vice chair, a full professor, to be elected from among the voting committee members, and to serve as chair when the chair is absent, or when the committee is reviewing a faculty member whose review has been assigned to the chair.}
								\item{Committee secretary, to be elected from among the voting committee members.  The officers will have both voice and vote, as do the other faculty members of the committee.}
							\end{enumerate}
						}
						\item{\underline{Responsibilities}:
							\begin{enumerate}[label=\arabic*)]
								\item{\change{}{30}{approved by faculty to strike the following text: To interview and recommend to the Provost candidates for appointment to the faculty.}}
								\item{To make recommendations to the Provost concerning initial status, rank, tenure, termination, and emeritus/emerita status.}
								\item{To solicit faculty opinion in cases of initial appointment which also involve promotion to associate or full professor, or granting of tenure, prior to a recommendation to the Provost.}
								\item{To review cases when progressive discipline (See section \ref{sec:Discipline}) is invoked and to recommend the nature of sanctions if applicable.}
							\end{enumerate}
						}
					\end{enumerate}
				\subparagraph{Professional Development Committee}
					\begin{enumerate}[label=\alph*)]
						\item{\underline{Membership}:
							\begin{enumerate}[label=\arabic*)]
								\item{Provost (or representative)}
								\item{Three faculty members, one from each division, one elected annually to a three year term, at least one of whom must hold the rank of full professor.}
							\end{enumerate}
						}
						\item{\underline{Officers}:
							The chair will be elected from among the faculty members and shall appoint the secretary.}
						\item{\underline{Responsibilities}:
							To promote the professional development of faculty instruction, research, creative activity, and publication.
							\begin{enumerate}[label=\arabic*)]
								\item{To administer faculty grants in support of scholarly activities and curricular development.}
								\item{To sponsor activities on campus to stimulate excellence in teaching, including the supervision of a program of student evaluation of courses, special seminars and speakers, etc.}
								\item{To encourage the attendance and participation in professional meetings.}
								\item{To recognize excellence in publication and creative activity.}
								\item{To review the literature pertaining to the evaluation and improvement of instruction and to investigate the practice of other colleges in these areas.}
								\item{To serve as consultants to other faculty in matters of professional improvements.}
								\item{To advise the Provost on a program of in-service training for new and experienced faculty.}
								\item{To make recommendations to the Provost concerning leaves and sabbaticals.}
							\end{enumerate}
						}
					\end{enumerate}
			\paragraph{Policy of Recusal}
				Concerning the participation of faculty on various deliberating committees:  A Faculty member must recuse himself/herself if:
				\begin{enumerate}[label=\alph*)]
					\item{The faculty member is on a committee that is addressing an issue directly affecting that faculty member or a close family member of that faculty member.  Examples: A faculty member being reviewed for tenure or promotion who is on the Personnel Committee; a faculty member applying for a faculty development grant or sabbatical who is on the Professional Development Committee; a faculty member involved in a preliminary hearing (see
						\ref{sec:DischargeForCauseHearing-ProceduresOfTheHearingCommittee}
						) who is a member of the Faculty Council; a faculty member on the Off-Campus Programs Committee whose spouse has submitted a proposal for consideration.}
					\item{The faculty member is on an appeals committee reviewing a decision made by an earlier committee of which that faculty member was also a member.}
					\item{The faculty member is on a disciplinary committee that is hearing a case initiated by that faculty member.}
				\end{enumerate}
				A Faculty member will not typically be recused from a committee merely because:

				\begin{enumerate}[label=\alph*)]
					\item{The faculty member believes himself/herself to be biased (due to, for example, a  friendship), or is charged with being biased.}
					\item{The faculty member has personal knowledge of or a close association with the issues being addressed.}
				\end{enumerate}

				In questionable cases the Faculty Council will determine whether faculty members must/may recuse themselves.  In the event that Council members themselves are involved in such a situation, they will recuse themselves (one at a time, if more than one is involved) and the remaining Council members will make this determination.

			\paragraph{Committee Release}
				\label{sec:CommitteeRelease}
				\begin{enumerate}[label=\alph*)]
					\item{\underline{Purpose}:  Release from all committee responsibilities is available to faculty who have had exceptionally heavy committee loads or who have an unusual opportunity for professional development.  The purpose is to provide faculty with the opportunity to recover from heavy committee work and to address other areas of development which may have suffered as a result of such work.}
					\item{\underline{Eligibility}:
						\begin{enumerate}[label=\arabic*)]
							\label{sec:CommitteeRelease-Eligibility}
							\item{
								\label{sec:CommitteeRelease-Eligibility-Automatic}
								Automatic:  Faculty members completing full terms on Academic Senate, Faculty Council or the Faculty Personnel committees are automatically granted a one-year release from committee work.}
							\item{By Application:  Faculty can apply to the Professional Development Committee for a one-year release from committee work five years after any previous release.}
						\end{enumerate}
					}
					\item{\underline{Application Procedure}:
						\begin{enumerate}[label=\arabic*)]
							\item{Applications should be submitted to the Provost before December 15 for committee release the following academic year.}
							\item{All requests must be approved by the Professional Development Committee.  Applicants will be notified of decisions before March 1.}
							\item{If the number of applications exceeds the allotment for a given year, a committee release shall be awarded based upon the amount and concentration of institutional service since the last release.}
							\item{If a faculty member is serving a multi-year term on a critical committee, the Professional Development Committee may recommend that the release be postponed for a year.}
							\item{Neither committee release nor load consideration is extended to assignments for which monetary or course load reduction compensation is already provided.}
						\end{enumerate}
					}
				\end{enumerate}

	\subsection{Protocols for Revision of the \emph{Faculty Handbook}}
		\label{sec:ProtocolsForRevision}
		\subsubsection{Publication}
			\label{sec:ProtocolsForRevision-Publication}
			The current version of the \emph{Faculty Handbook} will be maintained online at the \href{http://www.westmont.edu/_offices/provost/documents/Faculty-Handbook.pdf}{Provost's website}.  Every three years the Faculty Council will confer with the Provost to carry out a comprehensive review of the \emph{Faculty Handbook} to ensure its completeness and accuracy.
		\subsubsection{Sources of Initiative for Revisions}
			\label{sec:ProtocolsForRevision-Sources}
			\paragraph{Board of Trustees}
				Revisions proposed by the Board of Trustees do not require the faculty's approval, but should be submitted to the faculty for discussion and response before adoption.
			\paragraph{President or Provost}
				Revisions proposed by the President or the Provost are subject to the approval either of the Board of Trustees, or of the faculty according to the procedure specified in
				section
				\ref{sec:ProtocolsForRevision-Sources-FacultyCouncil}
				.

			\paragraph{Faculty Council}
				\label{sec:ProtocolsForRevision-Sources-FacultyCouncil}
				Revisions proposed by the Faculty Council are subject to the approval of the full faculty. Such proposals will be provided in writing to all faculty members at least one week prior to the faculty's action. Adoption requires a two-thirds majority vote of the eligible faculty present.
			\paragraph{Faculty}
				Revisions proposed by faculty will qualify for consideration by having the endorsement of at least one-fourth of the whole number of faculty eligible to vote. This procedure is an exception to the normal means by which motions may be brought to the faculty by an individual voting member (see
				section
				\ref{sec:FacultyMeetings}~\ref{sec:FacultyMeetings-Procedures}~\ref{sec:FacultyMeetings-Procedures-RobertsRules}).

				Such revisions are subject to the approval of the whole faculty according to the procedure specified in
				section
				\ref{sec:ProtocolsForRevision-Sources-FacultyCouncil}
		\subsubsection{Authorization of Revisions}
			The Provost, in consultation with Faculty Council, will adjudicate all questions concerning the status of a revision as ``substantive'' or ``non-substantive,'' and ``automatic'' or ``non-automatic''.
			\paragraph{Non-Substantive Revisions}
				\label{sec:NonSubstantiveRevisions}
				The President or the Provost or the Faculty Council, or a designate of any of them, may authorize non-substantive revisions for the sake of correctness (e.g., of spelling, grammar, and mechanics), or of internal consistency (in graphics, e.g., fonts, layouts; in forms of word, e.g., shall/will, chair/chairperson; in formats, e.g., the organization of committee descriptions), or of clarity (e.g., the renumbering of sections).
			\paragraph{Substantive Revisions}
				\subparagraph{Automatic Revisions}
					Certain revisions of substance are automatically contingent on events and circumstances beyond the College's powers of choice (e.g.,
					\ref{sec:GovernmentMandatedBenefits}~(``\nameref{sec:GovernmentMandatedBenefits}'')
					or the constitution of the Christian College Consortium in
					section
					\ref{sec:AccreditationAndMemberships}~(``\nameref{sec:AccreditationAndMemberships}'')
					). These revisions will require no additional authorization, and will be treated according to the procedures for non-substantive revisions (
					section
					\ref{sec:NonSubstantiveRevisions}
					).
				\subparagraph{Non-Automatic Revisions}
					Following approval of any non-automatic revision by two-thirds majority vote of the faculty as specified in section
					\ref{sec:ProtocolsForRevision-Sources-FacultyCouncil}
					, the motion will then require the approval of the President and the Board of Trustees.
		\subsubsection{Emergency Suspension of a \emph{Faculty Handbook} Provision}
			\label{sec:EmergencySuspension}
			The Faculty may choose to suspend a provision of the handbook by a four-fifths vote of the eligible faculty present, upon one week's prior written notice.  Such suspensions shall be limited to issues of faculty governance (e.g., make-up of committees) rather than issues that impact faculty contracts, faculty disciplinary hearings, or trustee governance.
